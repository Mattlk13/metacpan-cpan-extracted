% Copyright 2014, 2015, 2016, 2017 Kevin Ryde
%


\documentclass{article}
\usepackage[T1]{fontenc}  % T1 for accents, before babel
\usepackage{amsmath}
\allowdisplaybreaks

\usepackage{float}      % for \begin{figure}[H]
\usepackage{gensymb}    % for \degree
\usepackage{hyphenat}   % for \hyp hyphenation of words with -

\usepackage[pdfusetitle,
            pdflang={en}, % RFC3066 style ISO639
           ]{hyperref}
\renewcommand\figureautorefname{figure}   % lower case
\renewcommand\theoremautorefname{theorem} % lower case
\providecommand*{\lemmaautorefname}{lemma}
\usepackage[all]{hypcap} % figure links to top of figure

\usepackage{mathtools}  % for \mathclap and showonlyrefs
\mathtoolsset{showonlyrefs}

\usepackage{amsthm}
\newtheorem{theorem}{Theorem}
\newtheorem{lemma}{Lemma}

\usepackage{tikz}
\usetikzlibrary{arrows.meta}  % for arrow To[length etc
\usetikzlibrary{bending}      % for arrow [bend]
\usetikzlibrary{calc}         % for ($(...)$) coordinate calculations
\usetikzlibrary{decorations}  % for [decoration=]
\usetikzlibrary{decorations.pathreplacing}  % for decoration=brace
\usetikzlibrary{shapes}       % for shape aspect=1
\tikzset{font=\small,         % same as text
         >=Latex}             % arrowhead style
  % must be capital Latex for [harpoon] half arrows

\hyphenation{Amer-ican}
\hyphenation{Pos-it-ive}
\hyphenation{Calc-ul-at-ions}
\hyphenation{Rep-lic-at-ions}
\hyphenation{Twin-drag-on}
\hyphenation{Pap-er-fold-ing}
\hyphenation{Bit-Above-Low-est-One}
\hyphenation{gBit-Above-Low-est-One}
\hyphenation{Blob-List}
\hyphenation{Blob-Seg-ments}
\hyphenation{Blob-Delta}
\hyphenation{Blob-Delta-Len}
\hyphenation{Blob-Start}
\hyphenation{Drag-on-Bord-er}
\hyphenation{fract-al}
\hyphenation{trig-on-om-et-ric}
\hyphenation{co-eff-ic-ients}

%---------------------------------
% personal preferences

\hypersetup{
  pdfborderstyle={/W 0},  % no border on hyperlinks
}

% these must be after \begin{document} to take effect, hence \AtBeginDocument
\AtBeginDocument{%
  \setlength\abovedisplayskip{.7\baselineskip}
  \setlength\belowdisplayskip{.7\baselineskip}
  \setlength\abovedisplayshortskip{.5\baselineskip}
  \setlength\belowdisplayshortskip{.5\baselineskip}
}

% less space after "plain" style \end{theorem} etc
\makeatletter
\g@addto@macro\th@plain{\thm@postskip=1\baselineskip}
\makeatother

%------------------------------------------------------------------------------
% GP-DEFINE  default(strictargs,1);

\newcommand\MySlash{\slash\hspace{0pt}}
\newcommand\MyTightDots{.\kern.1em.\kern.1em.}

\newbox\MyNegativePhantomBox
\newcommand\MyNegativePhantom[1]{%
  \setbox\MyNegativePhantomBox\hbox{#1}%
  \hbox to -\wd\MyNegativePhantomBox{}}
% -\wd\MyNegativePhantomBox

% % Uncomment this to see the bounding box around each picture.
% \tikzset{
%   every picture/.append style={
%     execute at end picture={
%       \draw (current bounding box.south west)
%         rectangle (current bounding box.north east);
% }}}


%------------------------------------------------------------------------------
\begin{document}

binary beanstalk count 1-bits A179016
  parent = n - number of 1-bits in n

number of runs binary beanstalk A255327=size of subtree, A255056=trunk
  parent = A236840(vertex)   n - number of runs in n



%------------------------------------------------------------------------------
\section{Rook Grid 4,4}

\begin{center}
\begin{tikzpicture}
    [scale=3,
     my box/.style={draw,circle,minimum size=1.5em,inner sep=.1em,outer sep=.3em},
     font=\small,
    ]
  \newcommand\MyW{-.2}

  \node (1) at (0,0) [my box] {1};
  \node (2) at (1,0-\MyW) [my box] {2};
  \node (3) at (2,0-\MyW) [my box] {3};
  \node (4) at (3,0) [my box] {4};
  \node (5) at (0-\MyW,1) [my box] {5};
  \node (6) at (1-\MyW, 1-\MyW) [my box] {6};
  \node (7) at (2+\MyW, 1-\MyW) [my box] {7};
  \node (8) at (3+\MyW,1) [my box] {8};
  \node (9) at (0-\MyW,2) [my box] {9};
  \node (10) at (1-\MyW, 2+\MyW) [my box] {10};
  \node (11) at (2+\MyW, 2+\MyW) [my box] {11};
  \node (12) at (3+\MyW,2) [my box] {12};
  \node (13) at (0,3) [my box] {13};
  \node (14) at (1,3+\MyW) [my box] {14};
  \node (15) at (2,3+\MyW) [my box] {15};
  \node (16) at (3,3) [my box] {16};

  \draw [] (5) to (8);
  \draw [] (10) to (11);
  \draw [] (6) to (7);
  \draw [] (10) to (2);
  \draw [] (5) to (6);
  \draw [] (3) to (7);
  \draw [] (7) to (8);
  \draw [] (4) to (8);
  \draw [] (14) to (6);
  \draw [] (6) to (8);
  \draw [] (12) to (8);
  \draw [] (5) to (13);
  \draw [] (10) to (12);
  \draw [] (2) to (4);
  \draw [] (1) to (4);
  \draw [] (10) to (6);
  \draw [] (12) to (4);
  \draw [] (12) to (9);
  \draw [] (11) to (3);
  \draw [] (9) to (13);
  \draw [] (1) to (9);
  \draw [] (11) to (7);
  \draw [] (2) to (3);
  \draw [] (10) to (9);
  \draw [] (16) to (13);
  \draw [] (15) to (16);
  \draw [] (10) to (14);
  \draw [] (11) to (9);
  \draw [] (3) to (4);
  \draw [] (5) to (7);
  \draw [] (15) to (7);
  \draw [] (1) to (13);
  \draw [] (15) to (3);
  \draw [] (15) to (13);
  \draw [] (11) to (15);
  \draw [] (14) to (13);
  \draw [] (5) to (9);
  \draw [] (16) to (8);
  \draw [] (14) to (15);
  \draw [] (1) to (5);
  \draw [] (1) to (2);
  \draw [] (12) to (16);
  \draw [] (1) to (3);
  \draw [] (16) to (4);
  \draw [] (14) to (2);
  \draw [] (14) to (16);
  \draw [] (11) to (12);
  \draw [] (2) to (6);

\end{tikzpicture}
\end{center}


%------------------------------------------------------------------------------
\section{Regular But Median Size Less than N}

polyhedral of vertices
A000944 0, 0, 0, 1, 2, 7, 34, 257, 2606, ...
n =     1  2  3  4  5  6   7    8     9

n=8 regular 1 but median size 4 is not all vertices
\begin{center}
\begin{tikzpicture}
    [scale=2,
     my box/.style={draw,minimum size=2em,inner sep=.1em,outer sep=.3em},
     font=\small,
    ]

  \node (0) at (1.5, .4) [my box] {0};
  \node (1) at (2, -1) [my box] {1};
  \node (2) at (.5,0) [my box] {2};
  \node (3) at (2,1) [my box] {3};
  \node (4) at (1,-1) [my box] {4};
  \node (5) at (1.5, -.4) [my box] {5};
  \node (6) at (2.5,0) [my box] {6};
  \node (7) at (1,1) [my box] {7};

  \draw [] (0) to (7);
  \draw [] (4) to (5);
  \draw [] (0) to (5);
  \draw [] (1) to (5);
  \draw [] (2) to (4);
  \draw [] (1) to (4);
  \draw [] (0) to (3);
  \draw [] (2) to (7);
  \draw [] (1) to (6);
  \draw [] (2) to (6);
  \draw [] (3) to (7);
  \draw [] (3) to (6);

\end{tikzpicture}
\end{center}

\begin{center}
\begin{tikzpicture}
    [scale=2,
     my box/.style={draw,minimum size=2em,inner sep=.1em,outer sep=.3em},
     font=\small,
    ]

  \node (0) at (0,0) [my box] {0};
  \node (1) at (-1,1) [my box] {1};
  \node (2) at (-2, 2) [my box] {2};
  \node (3) at (1, 1) [my box] {3};
  \node (4) at (-2, 1) [my box] {4};
  \node (5) at (-1, 0) [my box] {5};
  \node (6) at (1, 2) [my box] {6};
  \node (7) at (0,1) [my box] {7};

  \draw [] (0) to (7);
  \draw [] (4) to (5);
  \draw [] (0) to (5);
  \draw [] (1) to (5);
  \draw [] (2) to (4);
  \draw [] (1) to (4);
  \draw [] (0) to (3);
  \draw [] (2) to (7);
  \draw [] (1) to (6);
  \draw [] (2) to (6);
  \draw [] (3) to (7);
  \draw [] (3) to (6);

\end{tikzpicture}
\end{center}

\begin{center}
\begin{tikzpicture}
    [scale=2,
     my box/.style={draw,minimum size=2em,inner sep=.1em,outer sep=.3em},
     font=\small,
    ]

  \node (0) at (0/8*360:1) [my box] {0};
  \node (1) at (3/8*360:1) [my box] {1};
  \node (2) at (6/8*360:1) [my box] {2};
  \node (3) at (1/8*360:1) [my box] {3};
  \node (4) at (5/8*360:1) [my box] {4};
  \node (5) at (4/8*360:1) [my box] {5};
  \node (6) at (2/8*360:1) [my box] {6};
  \node (7) at (7/8*360:1) [my box] {7};

  \draw [] (0) to (7);
  \draw [] (4) to (5);
  \draw [] (0) to (5);
  \draw [] (1) to (5);
  \draw [] (2) to (4);
  \draw [] (1) to (4);
  \draw [] (0) to (3);
  \draw [] (2) to (7);
  \draw [] (1) to (6);
  \draw [] (2) to (6);
  \draw [] (3) to (7);
  \draw [] (3) to (6);

\end{tikzpicture}
\end{center}

\begin{center}
\begin{tikzpicture}
    [scale=2,
     my box/.style={draw,minimum size=2em,inner sep=.1em,outer sep=.3em},
     font=\small,
    ]

  \node (0) at (6/8*360:1) [my box] {0};
  \node (1) at (3/8*360:1) [my box] {1};
  \node (2) at (1/8*360:1) [my box] {2};
  \node (3) at (7/8*360:1) [my box] {3};
  \node (4) at (4/8*360:1) [my box] {4};
  \node (5) at (5/8*360:1) [my box] {5};
  \node (6) at (2/8*360:1) [my box] {6};
  \node (7) at (0/8*360:1) [my box] {7};

  \draw [] (0) to (7);
  \draw [] (4) to (5);
  \draw [] (0) to (5);
  \draw [] (1) to (5);
  \draw [] (2) to (4);
  \draw [] (1) to (4);
  \draw [] (0) to (3);
  \draw [] (2) to (7);
  \draw [] (1) to (6);
  \draw [] (2) to (6);
  \draw [] (3) to (7);
  \draw [] (3) to (6);

\end{tikzpicture}
\end{center}

XXXXXXXX

n=8 regular 1 but median size 4 is not all vertices
\begin{center}
\begin{tikzpicture}
    [scale=1,
     my box/.style={draw,minimum size=2em,inner sep=.1em,outer sep=.3em},
     font=\small,
    ]

  \node (0) at (0,-2) [my box] {0};
  \node (1) at (2,1) [my box] {1};
  \node (2) at (-2,1) [my box] {2};
  \node (3) at (0,-.7) [my box] {3};
  \node (4) at (0,2) [my box] {4};
  \node (5) at (0,.7) [my box] {5};
  \node (6) at (2,-1) [my box] {6};
  \node (7) at (-2,-1) [my box] {7};

  \draw [] (2) to (7);
  \draw [] (0) to (6);
  \draw [] (0) to (3);
  \draw [] (2) to (5);
  \draw [] (4) to (5);
  \draw [] (1) to (5);
  \draw [] (1) to (4);
  \draw [] (2) to (4);
  \draw [] (0) to (7);
  \draw [] (3) to (6);
  \draw [] (3) to (7);
  \draw [] (1) to (6);

\end{tikzpicture}
\end{center}

XXXXXXXX

n=9 not regular but median size = 9

\begin{center}
\begin{tikzpicture}
    [scale=1.2,
     my box/.style={draw,minimum size=2em,inner sep=.1em,outer sep=.3em},
     font=\small,
    ]

  \node (0) at (2.8, 1.9) [my box] {0};
  \node (1) at (-.8, 1.9) [my box] {1};

  \node (3) at (2, 1.3) [my box] {3};
  \node (7) at (3,0) [my box] {7};
  \node (4) at (.2, 1.4) [my box] {4};
  \node (8) at (-.9, 0) [my box] {8};
  \node (6) at (1, 2.8) [my box] {6};
  \node (5) at (1,.5) [my box] {5};
  \node (2) at (1,-.5) [my box] {2};

  \draw [] (0) to (6);
  \draw [] (1) to (4);
  \draw [] (2) to (7);
  \draw [] (1) to (6);
  \draw [] (3) to (7);
  \draw [] (0) to (3);
  \draw [] (2) to (8);
  \draw [] (5) to (8);
  \draw [] (1) to (8);
  \draw [] (2) to (5);
  \draw [] (4) to (6);
  \draw [] (5) to (7);
  \draw [] (3) to (6);
  \draw [] (0) to (7);
  \draw [] (4) to (8);

\end{tikzpicture}
\end{center}

%------------------------------------------------------------------------------
\section{n=8 Pseudosimilar which are Line Graph}

edges=9, is smallest of all n=8 pseudosimilars
\begin{center}
\begin{tikzpicture}
    [scale=1.2,
     my box/.style={draw,minimum size=2em,inner sep=.1em,outer sep=.3em},
     font=\small,
    ]

  \node (0) at (2,1) [my box] {0};
  \node (1) at (0,1) [my box] {1};
  \node (2) at (-2,1) [my box] {2};
  \node (3) at (3,2) [my box] {3};
  \node (4) at (1,2) [my box] {4};
  \node (5) at (3,1) [my box] {5};
  \node (6) at (-1,1) [my box] {6};
  \node (7) at (1,1) [my box] {7};

  % 9 edges
  \draw [] (4) to (7);
  \draw [] (0) to (5);
  \draw [] (1) to (6);
  \draw [] (1) to (7);
  \draw [] (2) to (6);

  \draw [] (0) to (7);
  \draw [] (3) to (5);
  \draw [] (1) to (4);
  \draw [] (0) to (3);

\end{tikzpicture}
\end{center}

XXXXXXXXXXXXX

\url{https://hog.grinvin.org/ViewGraphInfo.action?id=30339}

edges=10, vertices 1,5 pseudosimilar
\begin{center}
\begin{tikzpicture}
    [scale=1.2,
     my box/.style={draw,minimum size=2em,inner sep=.1em,outer sep=.3em},
     my circle box/.style={draw,circle,minimum size=2em,inner sep=.1em,outer sep=.3em},
     font=\small,
    ]

  \node (0) at (1,1) [my box] {0};
  \node (1) at (-1,1) [my circle box] {1};
  \node (2) at (-.5,2) [my box] {2};
  \node (3) at (2,1) [my box] {3};
  \node (4) at (-2,1) [my box] {4};
  \node (5) at (1,2) [my circle box] {5};
  \node (6) at (-2,2) [my box] {6};
  \node (7) at (0,1) [my box] {7};

  % 10 edges
  \draw [] (1) to (6);
  \draw [] (5) to (7);
  \draw [] (0) to (5);
  \draw [] (1) to (7);
  \draw [] (0) to (3);

  \draw [] (1) to (4);
  \draw [] (2) to (5);
  \draw [] (4) to (6);
  \draw [] (2) to (6);
  \draw [] (0) to (7);

\end{tikzpicture}
\end{center}

XXXXXXXXXXXXX

\url{https://hog.grinvin.org/ViewGraphInfo.action?id=30337}

edges=12, vertices 5,6 pseudosimilar

degree 3s 
\begin{center}
\begin{tikzpicture}
    [scale=1.25,
     my box/.style={draw,minimum size=2em,inner sep=.1em,outer sep=.3em},
     my circle box/.style={draw,circle,minimum size=2em,inner sep=.1em,outer sep=.3em},
     font=\small,
    ]

  \node (0) at (1,1) [my box] {0};
  \node (1) at (3,2) [my box] {1};
  \node (2) at (2,0) [my box] {2};
  \node (3) at (0,0) [my box] {3};
  \node (4) at (2,2) [my box] {4};
  \node (5) at (3,1) [my circle box] {5};
  \node (6) at (1,0) [my circle box] {6};
  \node (7) at (2,1) [my box] {7};

  % 12 edges
  \draw [] (0) to (7);
  \draw [] (1) to (4);
  \draw [] (2) to (5);
  \draw [] (5) to (7);
  \draw [] (2) to (7);

  \draw [] (4) to (7);
  \draw [] (3) to (6);
  \draw [] (0) to (6);
  \draw [] (0) to (3);
  \draw [] (2) to (6);

  \draw [] (1) to (5);
  \draw [] (0) to (4);

\end{tikzpicture}
\end{center}

\medskip

XXXXXXXXXXXXX

\url{https://hog.grinvin.org/ViewGraphInfo.action?id=30335}

edges=17

drawn to show contains second biggest as subgraph
\begin{center}
\begin{tikzpicture}
    [scale=1.2,yscale=1.3,xscale=1.5,
     my box/.style={draw,minimum size=2em,inner sep=.1em,outer sep=.3em},
     my circle box/.style={draw,circle,minimum size=2em,inner sep=.1em,outer sep=.3em},
     font=\small,
    ]

  \node (0) at (3,2) [my box] {0};
  \node (1) at (0,0) [my box] {1};
  \node (3) at (1,1) [my box] {3};

  \node (2) at (2,0) [my box] {2};

  \node (4) at (3,1) [my circle box] {4};
  \node (5) at (1,0) [my circle box] {5};
  \node (6) at (2,2) [my box] {6};

  \node (7) at (2,1) [my box] {7};

  % 17 edges
  \draw [] (0) to (3);
  \draw [] (2) to (4);
  \draw [] (3) to (6);
  \draw [] (0) to (4);
  \draw [bend right=25] (2) to (6);

  \draw [] (6) to (7);
  \draw [] (1) to (3);
  \draw [] (3) to (5);
  \draw [] (2) to (5);
  \draw [] (0) to (6);

  \draw [] (4) to (6);
  \draw [] (1) to (7);
  \draw [] (3) to (7);
  \draw [] (1) to (5);
  \draw [] (5) to (7);

  \draw [] (4) to (7);
  \draw [] (2) to (7);

\end{tikzpicture}
\end{center}

vertices 4,5 pseudosimilar
\begin{center}
\begin{tikzpicture}
    [scale=1.2,yscale=1.3,xscale=1.5,
     my box/.style={draw,minimum size=2em,inner sep=.1em,outer sep=.3em},
     my circle box/.style={draw,circle,minimum size=2em,inner sep=.1em,outer sep=.3em},
     font=\small,
    ]

  \node (0) at (-2,0) [my box] {0};
  \node (1) at (300:1) [my box] {1};
  \node (2) at (60:1) [my box] {2};
  \node (3) at (240:1) [my box] {3};
  \node (4) at (120:1) [my circle box] {4};
  \node (5) at (1,0) [my circle box] {5};
  \node (6) at (-1,0) [my box] {6};
  \node (7) at (0, 0) [my box] {7};

  % 17 edges
  \draw [] (0) to (3);
  \draw [] (2) to (4);
  \draw [] (3) to (6);
  \draw [] (0) to (4);
  \draw [] (2) to (6);

  \draw [] (6) to (7);
  \draw [] (1) to (3);
  \draw [] (3) to (5);
  \draw [] (2) to (5);
  \draw [] (0) to (6);

  \draw [] (4) to (6);
  \draw [] (1) to (7);
  \draw [] (3) to (7);
  \draw [] (1) to (5);
  \draw [] (5) to (7);

  \draw [] (4) to (7);
  \draw [] (2) to (7);

\end{tikzpicture}
\end{center}

its root graph
\begin{center}
\begin{tikzpicture}
    [scale=1.2,yscale=1.3,xscale=1.5,
     my box/.style={draw,minimum size=2em,inner sep=.1em,outer sep=.3em},
     font=\small,
    ]
  \node (0) at (1,0) [my box] {0};
  \node (1) at (3,1) [my box] {1};
  \node (2) at (3,0) [my box] {2};
  \node (3) at (1,1) [my box] {3};
  \node (4) at (2,0) [my box] {4};
  \node (5) at (2,1) [my box] {5};

  \draw [] (4) to (1);
  \draw [] (1) to (5);
  \draw [] (3) to (5);
  \draw [] (0) to (5);
  \draw [] (4) to (5);
  \draw [] (0) to (4);
  \draw [] (4) to (2);
  \draw [] (0) to (3);
\end{tikzpicture}
\end{center}




%------------------------------------------------------------------------------
\section{Pseudosimilar n=8 Subgraph Relations}

\begin{center}
\begin{tikzpicture}
    [scale=1.2,yscale=.8,
     my box/.style={draw,inner sep=.1em,outer sep=0em,font=\scriptsize},
     font=\small,
     rotate=90,
    ]
  \node (291-19) at (303,19) [my box] {291,19};

  \node (294-18) at (294,19) [my box] {294,18};
  \node (297-18) at (298,19) [my box] {297,18};
  \node (291-18) at (303,18) [my box] {291,18};

  \node (294-17) at (292,17) [my box] {294,17};
  \node (297-17) at (295,17) [my box] {297,17};
  \node (291-17) at (298,18) [my box] {291,17};

  \node (291-16) at (290,16) [my box] {291,16};
  \node (294-16) at (294,16) [my box] {294,16};
  \node (297-16) at (297,16) [my box] {297,16};
  \node (300-16) at (300,16) [my box] {300,16};
  \node (306-16) at (307,19) [my box] {306,16};
  \node (303-16) at (302,17) [my box] {303,16};

  \node (291-15) at (291,15) [my box] {291,15};
  \node (294-15) at (294,15) [my box] {294,15};
  \node (297-15) at (297,15) [my box] {297,15};
  \node (300-15) at (300,15) [my box] {300,15};
  \node (303-15) at (302,16) [my box] {303,15};
  \node (306-15) at (305,15) [my box] {306,15};
  \node (309-15) at (309,15) [my box] {309,15};




  \node (291-9) at (296,9) [my box] {291,9};

  \node (291-10) at (291, 9) [my box] {291,10};
  \node (294-10) at (294, 9) [my box] {294,10};
  \node (297-10) at (298, 9) [my box] {297,10};

  \node (291-11) at (291,11) [my box] {291,11};
  \node (294-11) at (294,11) [my box] {294,11};
  \node (297-11) at (301,10) [my box] {297,11};

  \node (291-12) at (290,12) [my box] {291,12};
  \node (294-12) at (294,12) [my box] {294,12};
  \node (297-12) at (297,12) [my box] {297,12};
  \node (300-12) at (300,12) [my box] {300,12};
  \node (303-12) at (303,12) [my box] {303,12};
  \node (306-12) at (306,12) [my box] {306,12};

  \node (291-13) at (290,13) [my box] {291,13};
  \node (294-13) at (294,13) [my box] {294,13};
  \node (297-13) at (297,13) [my box] {297,13};
  \node (300-13) at (300,13) [my box] {300,13};
  \node (303-13) at (303,13) [my box] {303,13};
  \node (306-13) at (306,13) [my box] {306,13};
  \node (309-13) at (309,13) [my box] {309,13};

  \node (291-14) at (291,14) [my box] {291,14};
  \node (294-14) at (294,14) [my box] {294,14};
  \node (297-14) at (297,14) [my box] {297,14};
  \node (300-14) at (300,14) [my box] {300,14};


  \draw [<-] (291-10) to (291-11);
  \draw [<-] (294-14) to (303-16);
  \draw [<-] (294-11) to (291-12);
  \draw [<-] (294-10) to (300-14);
  \draw [<-] (291-9) to (297-12);
  \draw [<-] (297-12) to (291-15);
  \draw [<-] (291-14) to (291-17);
  \draw [<-] (297-12) to (294-16);
  \draw [<-] (291-15) to (297-17);
  \draw [<-] (303-15) to (291-18);
  \draw [<-] (300-13) to (306-15);
  \draw [<-] (309-13) to (294-16);
  \draw [<-] (294-12) to (294-14);
  \draw [<-] (300-14) to (291-17);
  \draw [<-] (297-13) to (294-17);
  \draw [<-] (294-13) to (294-15);
  \draw [<-] (297-14) to (309-15);
  \draw [<-] (294-16) to (291-17);
  \draw [<-] (300-15) to (291-18);
  \draw [<-] (291-9) to (294-10);
  \draw [<-] (291-13) to (303-15);
  \draw [<-] (303-12) to (297-14);
  \draw [<-] (303-12) to (306-13);
  \draw [<-] (294-11) to (291-14);
  \draw [<-] (300-12) to (297-13);
  \draw [<-] (297-10) to (297-12);
  \draw [<-] (297-16) to (291-19);
  \draw [<-] (294-14) to (297-17);
  \draw [<-] (303-13) to (294-17);
  \draw [<-] (291-12) to (300-16);
  \draw [<-] (306-15) to (306-16);
  \draw [<-] (291-11) to (297-14);
  \draw [<-] (294-12) to (300-15);
  \draw [<-] (291-11) to (297-13);
  \draw [<-] (294-16) to (294-17);
  \draw [<-] (291-9) to (300-12);
  \draw [<-] (291-14) to (294-17);
  \draw [<-] (291-14) to (303-16);
  \draw [<-] (294-14) to (294-17);
  \draw [<-] (297-11) to (291-12);
  \draw [<-] (297-10) to (294-11);
  \draw [<-] (309-15) to (297-16);
  \draw [<-] (300-14) to (306-16);
  \draw [<-] (291-18) to (291-19);
  \draw [<-] (291-11) to (291-14);
  \draw [<-] (300-15) to (294-18);
  \draw [<-] (291-15) to (291-16);
  \draw [<-] (294-11) to (297-14);
  \draw [<-] (303-16) to (291-19);
  \draw [<-] (306-12) to (300-13);
  \draw [<-] (297-15) to (294-18);
  \draw [<-] (291-12) to (294-17);
  \draw [<-] (294-10) to (297-14);
  \draw [<-] (297-14) to (291-18);
  \draw [<-] (291-16) to (291-18);
  \draw [<-] (297-10) to (291-13);
  \draw [<-] (297-12) to (303-13);
  \draw [<-] (300-13) to (300-14);
  \draw [<-] (291-12) to (297-15);
  \draw [<-] (297-10) to (297-11);
  \draw [<-] (297-13) to (303-16);
  \draw [<-] (294-15) to (297-18);
  \draw [<-] (291-10) to (294-12);
  \draw [<-] (300-16) to (291-19);
  \draw [<-] (291-11) to (303-15);
  \draw [<-] (294-12) to (291-13);
  \draw [<-] (297-13) to (297-17);
  \draw [<-] (303-12) to (306-15);
  \draw [<-] (300-12) to (294-14);
  \draw [<-] (297-12) to (306-15);
  \draw [<-] (306-13) to (300-16);
  \draw [<-] (291-12) to (294-15);
  \draw [<-] (294-10) to (297-13);
  \draw [<-] (303-15) to (297-17);
  \draw [<-] (294-10) to (303-13);
  \draw [<-] (303-12) to (291-16);
  \draw [<-] (297-13) to (306-15);
  \draw [<-] (297-15) to (291-17);
  \draw [<-] (297-11) to (309-15);
  \draw [<-] (306-12) to (309-13);
  \draw [<-] (294-14) to (294-15);
  \draw [<-] (294-14) to (291-17);
  \draw [<-] (306-13) to (306-16);
  \draw [<-] (309-13) to (300-14);
  \draw [<-] (294-17) to (297-18);
  \draw [<-] (291-12) to (303-16);
  \draw [<-] (291-16) to (294-18);
  \draw [<-] (303-13) to (291-16);
  \draw [<-] (303-13) to (303-15);
  \draw [<-] (297-11) to (303-15);
  \draw [<-] (303-15) to (303-16);
  \draw [<-] (294-11) to (294-14);
  \draw [<-] (291-13) to (291-14);
  \draw [<-] (294-13) to (297-17);
  \draw [<-] (294-13) to (294-17);
  \draw [<-] (297-17) to (294-18);
  \draw [<-] (297-11) to (291-15);
  \draw [<-] (297-13) to (300-15);
  \draw [<-] (306-13) to (303-15);
  \draw [<-] (297-14) to (303-16);
  \draw [<-] (300-12) to (291-14);
  \draw [<-] (291-10) to (309-13);
  \draw [<-] (309-15) to (294-17);
  \draw [<-] (291-12) to (300-14);
  \draw [<-] (297-16) to (297-18);
  \draw [<-] (294-15) to (291-16);
  \draw [<-] (294-10) to (294-12);
  \draw [<-] (300-16) to (297-18);
  \draw [<-] (297-11) to (294-16);
  \draw [<-] (297-13) to (294-15);
  \draw [<-] (291-14) to (297-17);
  \draw [<-] (303-16) to (294-18);
  \draw [<-] (303-13) to (300-16);
  \draw [<-] (303-16) to (297-18);
  \draw [<-] (300-12) to (303-15);
  \draw [<-] (297-11) to (300-13);
  \draw [<-] (291-11) to (294-14);
  \draw [<-] (306-13) to (309-15);
  \draw [<-] (297-14) to (294-16);
  \draw [<-] (294-11) to (297-13);
  \draw [<-] (294-12) to (294-13);
  \draw [<-] (297-14) to (297-17);
  \draw [<-] (291-10) to (300-12);
  \draw [<-] (294-13) to (300-16);
  \draw [<-] (300-14) to (291-18);
  \draw [<-] (300-12) to (300-14);
  \draw [<-] (291-14) to (291-16);
  \draw [<-] (291-9) to (306-12);
  \draw [<-] (291-11) to (300-14);
  \draw [<-] (300-15) to (300-16);
  \draw [<-] (297-12) to (300-15);
  \draw [<-] (300-14) to (297-17);
  \draw [<-] (294-12) to (306-16);
  \draw [<-] (291-10) to (303-13);
  \draw [<-] (303-13) to (306-16);
  \draw [<-] (291-11) to (294-13);
  \draw [<-] (303-15) to (291-17);
  \draw [<-] (297-11) to (294-14);
  \draw [<-] (297-10) to (306-12);
  \draw [<-] (291-13) to (291-15);
  \draw [<-] (297-11) to (300-15);
  \draw [<-] (291-13) to (294-16);
  \draw [<-] (303-12) to (300-15);
  \draw [<-] (300-13) to (294-17);
  \draw [<-] (297-15) to (297-16);
  \draw [<-] (291-17) to (297-18);
  \draw [<-] (291-11) to (291-15);
  \draw [<-] (297-11) to (291-14);
  \draw [<-] (297-14) to (297-15);
  \draw [<-] (294-11) to (309-13);
  \draw [<-] (297-10) to (300-12);
  \draw [<-] (300-12) to (294-16);

\end{tikzpicture}
\end{center}




%------------------------------------------------------------------------------
\section{Crown 4}

\begin{center}
\begin{tikzpicture}
    [scale=1.2,
     my box/.style={draw,minimum size=2em,inner sep=.1em,outer sep=.3em},
     font=\small,
    ]

  \node (1) at (0,0) [my box] {1};
  \node (2) at (0,1) [my box] {2};
  \node (3) at (0,2) [my box] {3};
  \node (4) at (0,3) [my box] {4};

  \node (5) at (3,0) [my box] {5};
  \node (6) at (3,1) [my box] {6};
  \node (7) at (3,2) [my box] {7};
  \node (8) at (3,3) [my box] {8};

  \draw [] (2) to (5);
  \draw [] (6) to (3);
  \draw [] (7) to (4);
  \draw [] (1) to (6);
  \draw [] (8) to (3);
  \draw [] (1) to (7);
  \draw [] (8) to (2);
  \draw [] (7) to (2);
  \draw [] (6) to (4);
  \draw [] (5) to (3);
  \draw [] (5) to (4);
  \draw [] (1) to (8);

\end{tikzpicture}
\end{center}



%------------------------------------------------------------------------------
\section{Most Minimum Dominating Sets}

n=5
\begin{center}
\begin{tikzpicture}
    [scale=1.2,
     my box/.style={draw,minimum size=2em,inner sep=.1em,outer sep=.3em},
     font=\small,
    ]

  \node (0) at (0,0) [my box] {0};
  \node (1) at (2,0) [my box] {1};
  \node (2) at (1,0) [my box] {2};
  \node (3) at (1,1) [my box] {3};
  \node (4) at (1,-1) [my box] {4};

  \draw [] (0) to (4);
  \draw [] (0) to (2);
  \draw [] (1) to (3);
  \draw [] (0) to (3);
  \draw [] (1) to (2);
  \draw [] (2) to (4);
  \draw [] (1) to (4);

\end{tikzpicture}
\end{center}

n=6 domnum 3 in 15 ways

complete-5 less 3 non-touching edges
\begin{center}
\begin{tikzpicture}
    [scale=1,
     my box/.style={draw,minimum size=2em,inner sep=.1em,outer sep=.3em},
     font=\small,
    ]

  \node (0) at (3,2) [my box] {0};
  \node (1) at (1,0) [my box] {1};
  \node (2) at (-.5,1) [my box] {2};
  \node (3) at (4.5,1) [my box] {3};
  \node (4) at (3,0) [my box] {4};
  \node (5) at (1,2) [my box] {5};

  \draw [] (0) to (4);
  \draw [] (3) to (4);
  \draw [] (1) to (5);
  \draw [] (0) to (2);
  \draw [] (0) to (5);
  \draw [] (0) to (3);
  \draw [] (2) to (5);
  \draw [] (2) to (4);
  \draw [] (3) to (5);
  \draw [] (1) to (3);
  \draw [] (1) to (2);
  \draw [] (1) to (4);

\end{tikzpicture}
\end{center}

n=7 domnum 3 in 22 ways
\begin{center}
\begin{tikzpicture}
    [scale=1,
     my box/.style={draw,minimum size=2em,inner sep=.1em,outer sep=.3em},
     font=\small,
    ]

  \node (0) at (5,-1) [my box] {0};
  \node (1) at (1,1) [my box] {1};
  \node (2) at (1,-1) [my box] {2};
  \node (3) at (5,1) [my box] {3};
  \node (4) at (-.5,0) [my box] {4};
  \node (5) at (3,-1) [my box] {5};
  \node (6) at (3,1) [my box] {6};

  \draw [] (2) to (6);
  \draw [] (0) to (3);
  \draw [] (5) to (6);
  \draw [] (1) to (5);
  \draw [] (1) to (6);
  \draw [] (3) to (6);
  \draw [] (2) to (4);
  \draw [] (0) to (5);
  \draw [] (1) to (4);
  \draw [] (2) to (5);

\end{tikzpicture}
\end{center}

\pagebreak

n=8 domnum 3 in 36 ways
\begin{center}
\begin{tikzpicture}
    [scale=1.3,
     my box/.style={draw,minimum size=2em,inner sep=.1em,outer sep=.2em},
     font=\small,
    ]

  \node (0) at (0,-2.5) [my box] {0};
  \node (1) at (2,-1.5) [my box] {1};
  \node (2) at (2, 1.5) [my box] {2};
  \node (3) at (4, -2.5) [my box] {3};
  \node (4) at (1, 0) [my box] {4};
  \node (5) at (5, 0) [my box] {5};
  \node (6) at (-1, 0) [my box] {6};
  \node (7) at (3, 0) [my box] {7};

  \draw [] (2) to (5);
  \draw [] (0) to (4);
  \draw [] (2) to (4);
  \draw [] (4) to (7);
  \draw [] (1) to (4);
  \draw [] (3) to (7);
  \draw [] (1) to (6);
  \draw [] (2) to (7);
  \draw [] (1) to (7);
  \draw [] (3) to (5);
  \draw [] (1) to (5);
  \draw [] (0) to (3);
  \draw [] (2) to (6);
  \draw [] (0) to (6);

\end{tikzpicture}
\end{center}

\begin{center}
\begin{tikzpicture}
    [scale=1.3,
     my box/.style={draw,minimum size=2em,inner sep=.1em,outer sep=.2em},
     font=\small,
    ]

  \node (0) at (-3,0) [my box] {0};
  \node (1) at (-4,2) [my box] {1};
  \node (2) at (0,2) [my box] {2};
  \node (3) at (-.5, 0) [my box] {3};
  \node (4) at (-2,1) [my box] {4};
  \node (5) at (-4,-2) [my box] {5};
  \node (6) at (0,-2) [my box] {6};
  \node (7) at (-2,3) [my box] {7};

  \draw [] (2) to (5);
  \draw [] (0) to (4);
  \draw [] (2) to (4);
  \draw [] (4) to (7);
  \draw [] (1) to (4);
  \draw [] (3) to (7);
  \draw [] (1) to (6);
  \draw [] (2) to (7);
  \draw [] (1) to (7);
  \draw [] (3) to (5);
  \draw [] (1) to (5);
  \draw [] (0) to (3);
  \draw [] (2) to (6);
  \draw [] (0) to (6);

\end{tikzpicture}
\end{center}



\end{document}
%------------------------------------------------------------------------------
\section{6 Vertices Mean Distance 2/3 of Diameter}

12 total

with one hanging
\begin{center}
\begin{tikzpicture}
    [scale=1.2,
     my box/.style={draw,minimum size=2em,inner sep=.1em,outer sep=.3em},
     font=\small,
    ]

  \node (2) at (3,0) [my box] {2};
  \node (3) at (-2,0) [my box] {3};
  \node (0) at (0,1) [my box] {0};
  \node (1) at (0,3) [my box] {1};
  \node (4) at (0,2) [my box] {4};
  \node (5) at (2,0) [my box] {5};

  \draw [] (3) to (4);
  \draw [] (0) to (4);
  \draw [] (1) to (5);
  \draw [] (4) to (5);
  \draw [] (3) to (5);
  \draw [] (0) to (5);
  \draw [] (3) to (1);
  \draw [] (1) to (4);
  \draw [] (0) to (3);
  \draw [] (5) to (2);

\end{tikzpicture}
\end{center}




%------------------------------------------------------------------------------
\section{Noughts and Crosses 2x2, 1 Player}

Tesseract induced subgraph.

\url{https://hog.grinvin.org/ViewGraphInfo.action?id=27032}

\begin{center}
\begin{tikzpicture}
    [scale=1.2,
     my box/.style={draw,minimum size=2em,inner sep=.1em,outer sep=.3em},
     font=\small,
    ]

  \node (0000) at (0,0) [my box] {0000};

  \node (0001) at (2,1.5) [my box] {0001};
  \node (0010) at (2,.5) [my box] {0010};
  \node (0100) at (2,-.5) [my box] {0100};
  \node (1000) at (2,-1.5) [my box] {1000};

  \node (0011) at (4,2.5) [my box] {0011};
  \node (0101) at (4,1.5) [my box] {0101};
  \node (1001) at (5.25,0) [my box] {1001};
  \node (0110) at (4,0) [my box] {0110};
  \node (1010) at (4,-1.5) [my box] {1010};
  \node (1100) at (4,-2.5) [my box] {1100};

  \draw [->] (0010) to (0011);
  \draw [->] (1000) to (1100);
  \draw [->] (1000) to (1010);
  \draw [->] (0001) to (0011);
  \draw [->] (0010) to (0110);
  \draw [->] (0100) to (1100);
  \draw [->] (1000) to (1001);
  \draw [->] (0100) to (0110);
  \draw [->] (0001) to (1001);
  \draw [->] (0000) to (0001);
  \draw [->] (0000) to (1000);
  \draw [->] (0000) to (0100);
  \draw [->] (0001) to (0101);
  \draw [->] (0010) to (1010);
  \draw [->] (0000) to (0010);
  \draw [->] (0100) to (0101);

\end{tikzpicture}
\end{center}

\begin{center}
\begin{tikzpicture}
    [scale=1.2,
     my box/.style={draw,minimum size=2em,inner sep=.1em,outer sep=.3em},
     font=\small,
    ]

  \node (0000) at (0,0) [my box] {0000};

  \node (0001) at (2,1.5) [my box] {0001};
  \node (0010) at (2,.5) [my box] {0010};
  \node (0100) at (2,-.5) [my box] {0100};
  \node (1000) at (2,-1.5) [my box] {1000};

  \node (0011) at (4,2.5) [my box] {0011};
  \node (0110) at (4,1.5) [my box] {0110};
  \node (1100) at (4, .5) [my box] {1100};
  \node (1001) at (4,-.5) [my box] {1001};
  \node (0101) at (4,-1.5) [my box] {0101};
  \node (1010) at (4,-2.5) [my box] {1010};

  \draw [->] (0010) to (0011);
  \draw [->] (1000) to (1100);
  \draw [->] (1000) to (1010);
  \draw [->] (0001) to (0011);
  \draw [->] (0010) to (0110);
  \draw [->] (0100) to (1100);
  \draw [->] (1000) to (1001);
  \draw [->] (0100) to (0110);
  \draw [->] (0001) to (1001);
  \draw [->] (0000) to (0001);
  \draw [->] (0000) to (1000);
  \draw [->] (0000) to (0100);
  \draw [->] (0001) to (0101);
  \draw [->] (0010) to (1010);
  \draw [->] (0000) to (0010);
  \draw [->] (0100) to (0101);

\end{tikzpicture}
\end{center}

\begin{center}
\begin{tikzpicture}
    [scale=1.5,
     my box/.style={draw,minimum size=2em,inner sep=.1em,outer sep=.3em},
     font=\small,
    ]

  \node (0000) at (0,0) [my box] {0000};
  \node (0001) at (90*0+90:2) [my box] {0001};
  \node (0010) at (90*1+90:2) [my box] {0010};
  \node (0100) at (90*2+90:2) [my box] {0100};
  \node (1000) at (90*3+90:2) [my box] {1000};

  \node (0011) at (-1,1) [my box] {0011};
  \node (0110) at (-1,-1) [my box] {0110};
  \node (1100) at (1,-1) [my box] {1100};
  \node (1001) at (1,1) [my box] {1001};
  \node (1010) at (0,-4) [my box] {1010};
  \node (0101) at (4,0) [my box] {0101};

  \draw [->] (0010) to (0011);
  \draw [->] (1000) to (1100);
  \draw [->] (1000) to (1010);
  \draw [->] (0001) to (0011);
  \draw [->] (0010) to (0110);
  \draw [->] (0100) to (1100);
  \draw [->] (1000) to (1001);
  \draw [->] (0100) to (0110);
  \draw [->] (0001) to (1001);
  \draw [->] (0000) to (0001);
  \draw [->] (0000) to (1000);
  \draw [->] (0000) to (0100);
  \draw [->] (0001) to (0101);
  \draw [->] (0010) to (1010);
  \draw [->] (0000) to (0010);
  \draw [->] (0100) to (0101);

\end{tikzpicture}
\end{center}

\begin{center}
\begin{tikzpicture}
    [scale=3,
     my box/.style={draw,minimum size=2em,inner sep=.1em,outer sep=.3em},
     font=\small,
    ]

  \node (0000) at (0,0) [my box] {0000};
  \node (0001) at (90*0+90:2) [my box] {0001};
  \node (0010) at (90*1+90:2) [my box] {0010};
  \node (0100) at (90*2+90:2) [my box] {0100};
  \node (1000) at (90*3+90:2) [my box] {1000};

  \node (0011) at (-1,1) [my box] {0011};
  \node (0110) at (-1,-1) [my box] {0110};
  \node (1100) at (1,-1) [my box] {1100};
  \node (1001) at (1,1) [my box] {1001};
  \node (1010) at (.5,-.5) [my box] {1010};
  \node (0101) at (-.5,.5) [my box] {0101};

  \draw [->] (0010) to (0011);
  \draw [->] (1000) to (1100);
  \draw [->] (1000) to (1010);
  \draw [->] (0001) to (0011);
  \draw [->] (0010) to (0110);
  \draw [->] (0100) to (1100);
  \draw [->] (1000) to (1001);
  \draw [->] (0100) to (0110);
  \draw [->] (0001) to (1001);
  \draw [->] (0000) to (0001);
  \draw [->] (0000) to (1000);
  \draw [->] (0000) to (0100);
  \draw [->] (0001) to (0101);
  \draw [->] (0010) to (1010);
  \draw [->] (0000) to (0010);
  \draw [->] (0100) to (0101);

\end{tikzpicture}
\end{center}

\begin{center}
\begin{tikzpicture}
    [scale=4,
     my box/.style={draw,minimum size=2em,inner sep=.1em,outer sep=.3em},
     font=\small,
    ]

  \node (0000) at (0,0) [my box] {0000};
  \node (0001) at (90*0+90:1) [my box] {0001};
  \node (0010) at (90*1+90:1) [my box] {0010};
  \node (0100) at (90*2+90:1) [my box] {0100};
  \node (1000) at (90*3+90:1) [my box] {1000};

  \node (0011) at (-1,1) [my box] {0011};
  \node (0110) at (-1,-1) [my box] {0110};
  \node (1100) at (1,-1) [my box] {1100};
  \node (1001) at (1,1) [my box] {1001};
  \node (1010) at (.5,-.5) [my box] {1010};
  \node (0101) at (-.5,.5) [my box] {0101};

  \draw [->] (0010) to (0011);
  \draw [->] (1000) to (1100);
  \draw [->] (1000) to (1010);
  \draw [->] (0001) to (0011);
  \draw [->] (0010) to (0110);
  \draw [->] (0100) to (1100);
  \draw [->] (1000) to (1001);
  \draw [->] (0100) to (0110);
  \draw [->] (0001) to (1001);
  \draw [->] (0000) to (0001);
  \draw [->] (0000) to (1000);
  \draw [->] (0000) to (0100);
  \draw [->] (0001) to (0101);
  \draw [->] (0010) to (1010);
  \draw [->] (0000) to (0010);
  \draw [->] (0100) to (0101);

\end{tikzpicture}
\end{center}


\begin{center}
\begin{tikzpicture}
    [scale=2,
     my box/.style={draw,minimum size=2em,inner sep=.1em,outer sep=.3em},
     font=\small,
    ]

  \node (0000) at (0,0) [my box] {0000};
  \node (0001) at (90*0+90:1) [my box] {0001};
  \node (0010) at (90*1+90:1) [my box] {0010};
  \node (0100) at (90*2+90:1) [my box] {0100};
  \node (1000) at (90*3+90:1) [my box] {1000};

  \node (0011) at (-1,1) [my box] {0011};
  \node (0110) at (-1,-1) [my box] {0110};
  \node (1100) at (1,-1) [my box] {1100};
  \node (1001) at (1,1) [my box] {1001};
  \node (1010) at (0,-2) [my box] {1010};
  \node (0101) at (2,0) [my box] {0101};

  \draw [->] (0010) to (0011);
  \draw [->] (1000) to (1100);
  \draw [->] (1000) to (1010);
  \draw [->] (0001) to (0011);
  \draw [->] (0010) to (0110);
  \draw [->] (0100) to (1100);
  \draw [->] (1000) to (1001);
  \draw [->] (0100) to (0110);
  \draw [->] (0001) to (1001);
  \draw [->] (0000) to (0001);
  \draw [->] (0000) to (1000);
  \draw [->] (0000) to (0100);
  \draw [->] (0001) to (0101);
  \draw [->] (0010) to (1010);
  \draw [->] (0000) to (0010);
  \draw [->] (0100) to (0101);

\end{tikzpicture}
\end{center}

%------------------------------------------------------------------------------
\section{Noughts and Crosses 2x2}

\url{https://hog.grinvin.org/ViewGraphInfo.action?id=27017}

\begin{center}
\begin{tikzpicture}
    [scale=1.5,
     my box/.style={draw,minimum size=2em,inner sep=.1em,outer sep=.3em},
     font=\small,
    ]

  \node (00-00) at (0,0) [my box] {00-00};

  % 1
  \node (00-01) at (1,0) [my box] {00-01};
  \node (00-10) at (0,1) [my box] {00-10};
  \node (01-00) at (-1,0) [my box] {01-00};
  \node (10-00) at (0,-1) [my box] {10-00};

  % 2
  \node (02-10) at (2,1) [my box] {02-10};
  \node (20-10) at (0,2) [my box] {20-10};
  \node (00-12) at (-2,1) [my box] {00-12};

  \node (21-00) at (-1,2) [my box] {21-00};
  \node (01-02) at (-2,0) [my box] {01-02};
  \node (01-20) at (-1,-2) [my box] {01-20};

  \node (10-02) at (-2,-1) [my box] {10-02};
  \node (10-20) at (0,-2) [my box] {10-20};
  \node (12-00) at (2,-1) [my box] {12-00};

  \node (00-21) at (1,-2) [my box] {00-21};
  \node (02-01) at (2,0) [my box] {02-01};
  \node (20-01) at (1,2) [my box] {20-01};

  % 3
  \node (12-10) at (3.5,0) [my box] {12-10};
  \node (02-11) at (3.5,1) [my box] {02-11};
  \node (12-01) at (3.5,-1) [my box] {12-01};

  \node (20-11) at (1,3.5) [my box] {20-11};
  \node (21-01) at (0,3.5) [my box] {21-01};
  \node (21-10) at (-1,3.5) [my box] {21-10};

  \node (01-12) at (-3.5,1) [my box] {01-12};

  \node (10-12) at (-3.5,0) [my box] {10-12};
  \node (11-02) at (-3.5,-1) [my box] {11-02};

  \node (11-20) at (-1,-3.5) [my box] {11-20};
  \node (01-21) at (0,-3.5) [my box] {01-21};
  \node (10-21) at (1,-3.5) [my box] {10-21};


  \draw [->] (12-00) to (12-10);
  \draw [->] (00-12) to (01-12);
  \draw [->] (00-10) to (02-10);
  \draw [->] (10-20) to (10-21);
  \draw [->] (01-00) to (01-20);
  \draw [->] (00-21) to (10-21);
  \draw [->] (10-00) to (10-20);
  \draw [->] (01-02) to (11-02);
  \draw [->] (21-00) to (21-01);
  \draw [->] (10-02) to (11-02);
  \draw [->] (20-01) to (20-11);
  \draw [->] (00-01) to (20-01);
  \draw [->] (00-00) to (10-00);
  \draw [->] (00-00) to (01-00);
  \draw [->] (00-12) to (10-12);
  \draw [->] (00-01) to (00-21);
  \draw [->] (02-01) to (02-11);
  \draw [->] (12-00) to (12-01);
  \draw [->] (02-10) to (02-11);
  \draw [->] (10-00) to (12-00);
  \draw [->] (20-01) to (21-01);
  \draw [->] (01-20) to (01-21);
  \draw [->] (01-20) to (11-20);
  \draw [->] (20-10) to (21-10);
  \draw [->] (10-02) to (10-12);
  \draw [->] (01-00) to (21-00);
  \draw [->] (00-10) to (20-10);
  \draw [->] (00-10) to (00-12);
  \draw [->] (01-02) to (01-12);
  \draw [->] (10-20) to (11-20);
  \draw [->] (00-00) to (00-10);
  \draw [->] (20-10) to (20-11);
  \draw [->] (02-01) to (12-01);
  \draw [->] (00-00) to (00-01);
  \draw [->] (00-21) to (01-21);
  \draw [->] (01-00) to (01-02);
  \draw [->] (02-10) to (12-10);
  \draw [->] (10-00) to (10-02);
  \draw [->] (21-00) to (21-10);
  \draw [->] (00-01) to (02-01);

\end{tikzpicture}
\end{center}



\begin{center}
\begin{tikzpicture}
    [scale=1.5,
     my box/.style={draw,minimum size=2em,inner sep=.1em,outer sep=.3em},
     font=\small,
    ]

  \node (00-00) at (0,0) [my box] {00-00};

  % 1
  \node (00-01) at (1,0) [my box] {00-01};
  \node (00-10) at (0,1) [my box] {00-10};
  \node (01-00) at (-1,0) [my box] {01-00};
  \node (10-00) at (0,-1) [my box] {10-00};

  % 2
  \node (20-10) at (0,2) [my box] {20-10};
  \node (02-10) at ($(0,1) + (45:1)$) [my box] {02-10};
  \node (00-12) at ($(0,1) + (135:1)$) [my box] {00-12};

  \node (21-00) at ($(-1,0) + (135:1)$) [my box] {21-00};
  \node (01-02) at (-2,0) [my box] {01-02};
  \node (01-20) at ($(-1,0) + (-135:1)$) [my box] {01-20};

  \node (10-02) at ($(0,-1) + (-135:1)$) [my box] {10-02};
  \node (10-20) at (0,-2) [my box] {10-20};
  \node (12-00) at ($(0,-1) + (-45:1)$) [my box] {12-00};

  \node (00-21) at ($(1,0) + (-45:1)$) [my box] {00-21};
  \node (02-01) at (2,0) [my box] {02-01};
  \node (20-01) at ($(1,0) + (45:1)$) [my box] {20-01};

  % 3
  \node (12-10) at (0*30:4) [my box] {12-10};
  \node (02-11) at (1*30:4) [my box] {02-11};

  \node (20-11) at (2*30:4) [my box] {20-11};
  \node (21-01) at (3*30:4) [my box] {21-01};

  \node (21-10) at (4*30:4) [my box] {21-10};
  \node (01-12) at (5*30:4) [my box] {01-12};

  \node (10-12) at (6*30:4) [my box] {10-12};
  \node (11-02) at (7*30:4) [my box] {11-02};

  \node (11-20) at (8*30:4) [my box] {11-20};
  \node (01-21) at (9*30:4) [my box] {01-21};

  \node (10-21) at (10*30:4) [my box] {10-21};
  \node (12-01) at (11*30:4) [my box] {12-01};


  \draw [->] (12-00) to (12-10);
  \draw [->] (00-12) to (01-12);
  \draw [->] (00-10) to (02-10);
  \draw [->] (10-20) to (10-21);
  \draw [->] (01-00) to (01-20);
  \draw [->] (00-21) to (10-21);
  \draw [->] (10-00) to (10-20);
  \draw [->] (01-02) to (11-02);
  \draw [->] (21-00) to (21-01);
  \draw [->] (10-02) to (11-02);
  \draw [->] (20-01) to (20-11);
  \draw [->] (00-01) to (20-01);
  \draw [->] (00-00) to (10-00);
  \draw [->] (00-00) to (01-00);
  \draw [->] (00-12) to (10-12);
  \draw [->] (00-01) to (00-21);
  \draw [->] (02-01) to (02-11);
  \draw [->] (12-00) to (12-01);
  \draw [->] (02-10) to (02-11);
  \draw [->] (10-00) to (12-00);
  \draw [->] (20-01) to (21-01);
  \draw [->] (01-20) to (01-21);
  \draw [->] (01-20) to (11-20);
  \draw [->] (20-10) to (21-10);
  \draw [->] (10-02) to (10-12);
  \draw [->] (01-00) to (21-00);
  \draw [->] (00-10) to (20-10);
  \draw [->] (00-10) to (00-12);
  \draw [->] (01-02) to (01-12);
  \draw [->] (10-20) to (11-20);
  \draw [->] (00-00) to (00-10);
  \draw [->] (20-10) to (20-11);
  \draw [->] (02-01) to (12-01);
  \draw [->] (00-00) to (00-01);
  \draw [->] (00-21) to (01-21);
  \draw [->] (01-00) to (01-02);
  \draw [->] (02-10) to (12-10);
  \draw [->] (10-00) to (10-02);
  \draw [->] (21-00) to (21-10);
  \draw [->] (00-01) to (02-01);

\end{tikzpicture}
\end{center}


%------------------------------------------------------------------------------
\section{Noughts and Crosses 2x2 up to Rotation}

\url{https://hog.grinvin.org/ViewGraphInfo.action?id=27020}

\begin{center}
\begin{tikzpicture}
    [scale=1.5,
     my box/.style={draw,minimum size=2em,inner sep=.1em,outer sep=.3em},
     font=\small,
    ]

  \node (00-00) at (45:1) [my box] {00-00};

  \node (00-01) at (0,0) [my box] {00-01};

  \node (00-12) at (90:2) [my box] {00-12};
  \node (00-21) at (90+120:2) [my box] {00-21};
  \node (01-20) at (90+240:2) [my box] {01-20};

  \node (01-12) at (90+60:2) [my box] {01-12};
  \node (01-21) at (90+60+120:2) [my box] {01-21};
  \node (02-11) at (90+60+240:2) [my box] {02-11};

  \draw [->] (00-01) to (00-12);
  \draw [->] (00-21) to (01-21);
  \draw [->] (00-12) to (02-11);
  \draw [->] (01-20) to (02-11);
  \draw [->] (00-00) to (00-01);
  \draw [->] (00-12) to (01-12);
  \draw [->] (00-21) to (01-12);
  \draw [->] (00-01) to (01-20);
  \draw [->] (01-20) to (01-21);
  \draw [->] (00-01) to (00-21);

\end{tikzpicture}
\end{center}

\vspace{1\baselineskip}

\begin{center}
\begin{tikzpicture}
    [scale=1.5,
     my box/.style={draw,minimum size=2em,inner sep=.1em,outer sep=.3em},
     font=\small,
    ]

  \node (00-00) at (0,0) [my box] {00-00};

  \node (00-01) at (0,-1) [my box] {00-01};

  \node (00-12) at (1,-2) [my box] {00-12};
  \node (00-21) at (0,-2) [my box] {00-21};
  \node (01-20) at (-1,-2) [my box] {01-20};

  \node (01-12) at (1,-3) [my box] {01-12};
  \node (02-11) at (0,-3) [my box] {02-11};
  \node (01-21) at (-1,-3) [my box] {01-21};

  \draw [->] (00-01) to (00-12);
  \draw [->] (00-21) to (01-21);
  \draw [->] (00-12) to (02-11);
  \draw [->] (01-20) to (02-11);
  \draw [->] (00-00) to (00-01);
  \draw [->] (00-12) to (01-12);
  \draw [->] (00-21) to (01-12);
  \draw [->] (00-01) to (01-20);
  \draw [->] (01-20) to (01-21);
  \draw [->] (00-01) to (00-21);

\end{tikzpicture}
\end{center}



%------------------------------------------------------------------------------
\section{Noughts and Crosses 2x2, 1 Player, up to Reflection}

\url{https://hog.grinvin.org/ViewGraphInfo.action?id=856}

\begin{center}
\begin{tikzpicture}
    [scale=1.5,
     my box/.style={draw,minimum size=2em,inner sep=.1em,outer sep=.3em},
     font=\small,
    ]

  \node (0000) at (0,0) [my box] {0000};
  \node (0010) at (1,0) [my box] {0010};
  \node (1000) at (-1,0) [my box] {1000};

  \node (0011) at (2,0) [my box] {0011};
  \node (1010) at (0,-1) [my box] {1010};
  \node (1001) at (0,1) [my box] {1001};
  \node (1100) at (-2,0) [my box] {1100};

  \draw [->] (1000) to (1100);
  \draw [->] (1000) to (1001);
  \draw [->] (0000) to (0010);
  \draw [->] (0010) to (1010);
  \draw [->] (1000) to (1010);
  \draw [->] (0000) to (1000);
  \draw [->] (0010) to (0011);
  \draw [->] (0010) to (1001);

\end{tikzpicture}
\end{center}

%------------------------------------------------------------------------------
\section{Noughts and Crosses 2x2, 3 Players, up to Rotation}

\url{https://hog.grinvin.org/ViewGraphInfo.action?id=27025}

\begin{center}
\begin{tikzpicture}
    [scale=1.5,
     my box/.style={draw,minimum size=2em,inner sep=.1em,outer sep=.3em},
     font=\small,
    ]

  \node (0000) at (45:1) [my box] {0000};
  \node (1000) at (0,0) [my box] {1000};
  \node (1002) at (3*30:2) [my box] {1002};
  \node (1020) at (7*30:2) [my box] {1020};
  \node (1023) at (6*30:2) [my box] {1023};
  \node (1032) at (2*30:2) [my box] {1032};
  \node (1123) at (5*30:2) [my box] {1123};
  \node (1200) at (11*30:2) [my box] {1200};
  \node (1203) at (0*30:2) [my box] {1203};
  \node (1213) at (1*30:2) [my box] {1213};
  \node (1230) at (10*30:2) [my box] {1230};
  \node (1231) at (9*30:2) [my box] {1231};
  \node (1302) at (4*30:2) [my box] {1302};
  \node (1320) at (8*30:2) [my box] {1320};

  \draw [] (1000) to (1200);
  \draw [] (1123) to (1302);
  \draw [] (1230) to (1231);
  \draw [] (1200) to (1230);
  \draw [] (1000) to (1020);
  \draw [] (1002) to (1302);
  \draw [] (1002) to (1032);
  \draw [] (1231) to (1320);
  \draw [] (1020) to (1320);
  \draw [] (1020) to (1023);
  \draw [] (1200) to (1203);
  \draw [] (1000) to (1002);
  \draw [] (1032) to (1213);
  \draw [] (1123) to (1023);
  \draw [] (0000) to (1000);
  \draw [] (1203) to (1213);

\end{tikzpicture}
\end{center}

\begin{center}
\begin{tikzpicture}
    [scale=1.5,
     my box/.style={draw,minimum size=2em,inner sep=.1em,outer sep=.3em},
     font=\small,
    ]

  \node (0000) at (0,1) [my box] {0000};
  \node (1000) at (0,0) [my box] {1000};

  \node (1002) at (-2,-1) [my box] {1002};
  \node (1020) at (0,-1) [my box] {1020};
  \node (1200) at (2,-1) [my box] {1200};

  \node (1302) at (-2.5,-2) [my box] {1302};
  \node (1032) at (-1.5,-2) [my box] {1023};
  \node (1023) at (-.5,-2) [my box] {1320};
  \node (1320) at (.5,-2) [my box] {1032};
  \node (1203) at (1.5,-2) [my box] {1230};
  \node (1230) at (2.5,-2) [my box] {1203};

  \node (1123) at (-2.5,-3) [my box] {1123};
  \node (1213) at (0,-3) [my box] {1231};
  \node (1231) at (2.5,-3) [my box] {1213};

  \draw [] (1000) to (1200);
  \draw [] (1123) to (1302);
  \draw [] (1230) to (1231);
  \draw [] (1200) to (1230);
  \draw [] (1000) to (1020);
  \draw [] (1002) to (1302);
  \draw [] (1002) to (1032);
  \draw [] (1231) to (1320);
  \draw [] (1020) to (1320);
  \draw [] (1020) to (1023);
  \draw [] (1200) to (1203);
  \draw [] (1000) to (1002);
  \draw [] (1032) to (1213);
  \draw [] (1123) to (1023);
  \draw [] (0000) to (1000);
  \draw [] (1203) to (1213);

\end{tikzpicture}
\end{center}

\begin{center}
\begin{tikzpicture}
    [scale=1.5,
     my box/.style={draw,minimum size=2em,inner sep=.1em,outer sep=.3em},
     font=\small,
    ]


  \node (0000) at (0,1) [my box] {0000};
  \node (1000) at (0,0) [my box] {1000};

  \node (1002) at (-2,-1) [my box] {1002};
  \node (1020) at (0,-1) [my box] {1020};
  \node (1200) at (2,-1) [my box] {1200};

  \node (1302) at (-3,-2) [my box] {1302};
  \node (1023) at (-2,-2) [my box] {1023};
  \node (1203) at (-.5,-2) [my box] {1320};
  \node (1032) at (.5,-2) [my box] {1032};
  \node (1320) at (2,-2) [my box] {1230};
  \node (1230) at (3,-2) [my box] {1203};

  \node (1123) at (-3,-3) [my box] {1123};
  \node (1213) at (0,-3) [my box] {1231};
  \node (1231) at (3,-3) [my box] {1213};

  \draw [] (1000) to (1200);
  \draw [] (1123) to (1302);
  \draw [] (1230) to (1231);
  \draw [] (1200) to (1230);
  \draw [] (1000) to (1020);
  \draw [] (1002) to (1302);
  \draw [] (1002) to (1032);
  \draw [] (1231) to (1320);
  \draw [] (1020) to (1320);
  \draw [] (1020) to (1023);
  \draw [] (1200) to (1203);
  \draw [] (1000) to (1002);
  \draw [] (1032) to (1213);
  \draw [] (1123) to (1023);
  \draw [] (0000) to (1000);
  \draw [] (1203) to (1213);

\end{tikzpicture}
\end{center}

%------------------------------------------------------------------------------
\section{Subgraph Non-Relations Among 4-Vertex Graphs}

\url{https://hog.grinvin.org/ViewGraphInfo.action?id=26965}

\begin{center}
\begin{tikzpicture}
    [scale=1.2,xscale=1.7,
     my box/.style={draw,minimum size=2em,inner sep=.1em,outer sep=.3em},
     font=\small,
    ]

  \node (complete) at (0,-2) [my box] {complete};
  \node (cycle across) at (1,-2) [my box] {cycle across};
  \node (4) at (2,-2) [my box] {p-1};
  \node (disconnected) at (4,-2) [my box] {disconnected};

  \node (cycle-1 hanging) at (0,0) [my box] {cycle-1 hanging};
  \node (cycle) at (1,0) [my box] {cycle};
  \node (5) at (5,0) [my box] {p-3=2edge};
  \node (7) at (4,0) [my box] {2-sep};
  \node (cycle-1 disc) at (2,1) [my box] {cycle-1 disc};
  \node (path) at (2.5,0) [my box] {path};
  \node (star) at (2,-1) [my box] {claw};

  \draw [] (star) to (path);
  \draw [] (star) to (cycle);
  \draw [] (cycle) to (cycle-1 hanging);
  \draw [] (5) to (7);
  \draw [] (7) to (star);
  \draw [] (cycle-1 disc) to (cycle);
  \draw [] (cycle-1 disc) to (path);
  \draw [] (star) to (cycle-1 disc);
  \draw [] (7) to (cycle-1 disc);

\end{tikzpicture}
\end{center}

%------------------------------------------------------------------------------
\section{Subgraph Relations Among 4-Vertex Graphs, 1-Edge Delta}

\url{https://hog.grinvin.org/ViewGraphInfo.action?id=26963}

\begin{center}
\begin{tikzpicture}
    [scale=1.2,xscale=1.7,
     my box/.style={draw,minimum size=2em,inner sep=.1em,outer sep=.3em},
     font=\small,
    ]
  \node (complete) at (0, 0) [my box] {compl};
  \node (cycle across) at (1, 0) [my box,align=center] {cycle \\ across};
  \node (cycle) at (2, 1) [my box] {cycle};
  \node (cycle hanging) at (2,0) [my box,align=center] {cycle-1 \\ hanging};
  \node (path) at (3, 1) [my box] {path};
  \node (claw) at (3,0) [my box] {claw};
  \node (cycle-1 disc) at (3,-1) [my box,align=center] {cycle-1 \\ discon};
  \node (7) at (4,1) [my box] {2-sep};
  \node (5) at (4,0) [my box] {L};
  \node (4) at (5,0) [my box] {1-edge};
  \node (disconnected) at (6,0) [my box] {disc};

  \draw [->] (complete) to (cycle across);
  \draw [->] (cycle hanging) to (claw);
  \draw [->] (cycle hanging) to (path);
  \draw [->] (cycle hanging) to (cycle-1 disc);
  \draw [->] (path) to (5);
  \draw [->] (cycle) to (path);
  \draw [->] (4) to (disconnected);
  \draw [->] (5) to (4);
  \draw [->] (7) to (4);
  \draw [->] (cycle across) to (cycle hanging);
  \draw [->] (cycle-1 disc) to (5);
  \draw [->] (cycle across) to (cycle);
  \draw [->] (claw) to (5);
  \draw [->] (path) to (7);

\end{tikzpicture}
\end{center}

\begin{center}
\begin{tikzpicture}
    [scale=2,xscale=1.5,
     my box/.style={draw,minimum size=2em,inner sep=.1em,outer sep=.3em},
     font=\small,
    ]
  \node (complete) at (0, 6) [my box] {complete};
  \node (cycle across) at (0, 5) [my box] {cycle across};
  \node (cycle) at (-1, 4) [my box] {cycle};
  \node (cycle hanging) at (0,4) [my box] {cycle-1 hanging};
  \node (path) at (-1, 3) [my box] {path};
  \node (claw) at (0,3) [my box] {claw};
  \node (cycle-1 disc) at (1,3) [my box] {cycle-1 discon};
  \node (7) at (-1,2) [my box] {2-sep};
  \node (5) at (0,2) [my box] {L};
  \node (4) at (0,1) [my box] {1-edge};
  \node (disconnected) at (0,0) [my box] {disconnected};

  \draw [->] (complete) to (cycle across);
  \draw [->] (cycle hanging) to (claw);
  \draw [->] (cycle hanging) to (path);
  \draw [->] (cycle hanging) to (cycle-1 disc);
  \draw [->] (path) to (5);
  \draw [->] (cycle) to (path);
  \draw [->] (4) to (disconnected);
  \draw [->] (5) to (4);
  \draw [->] (7) to (4);
  \draw [->] (cycle across) to (cycle hanging);
  \draw [->] (cycle-1 disc) to (5);
  \draw [->] (cycle across) to (cycle);
  \draw [->] (claw) to (5);
  \draw [->] (path) to (7);

\end{tikzpicture}
\end{center}

%------------------------------------------------------------------------------
\section{Subgraph Relations Among 4-Vertex Connected Graphs}

\url{https://hog.grinvin.org/ViewGraphInfo.action?id=748}

\begin{center}
\begin{tikzpicture}
    [scale=2,xscale=1.6,
     my box/.style={draw,minimum size=2em,inner sep=.1em,outer sep=.3em},
     font=\small,
    ]
  \node (complete)            at (0,1) [my box] {complete};
  \node (cycle and across)    at (1,1) [my box] {K4 - 1};
  \node (cycle)               at (1.5,0) [my box] {cycle};
  \node (claw)                at (-.5,0) [my box] {claw};
  \node (cycle-1 and hanging) at (0,-1) [my box] {cycle-1 and hang};
  \node (path)                at (1,-1) [my box] {path};

  \draw [->,bend right=0] (complete) to (path);
  \draw [->] (complete) to (cycle and across);
  \draw [->] (complete) to (cycle);
  \draw [->] (complete) to (cycle-1 and hanging);
  \draw [->,bend left=0] (complete) to (claw);
  \draw [->] (cycle and across) to (cycle);
  \draw [->] (cycle and across) to (path);
  \draw [->] (cycle and across) to (cycle-1 and hanging);
  \draw [->] (cycle and across) to (claw);
  \draw [->] (cycle-1 and hanging) to (path);
  \draw [->] (cycle-1 and hanging) to (claw);
  \draw [->] (cycle) to (path);
\end{tikzpicture}
\end{center}

% \vspace{2ex}

\begin{center}
\begin{tikzpicture}
    [scale=2,xscale=1.8,
     my box/.style={draw,minimum size=2em,inner sep=.1em,outer sep=.3em},
     font=\small,
    ]
  \node (complete)            at (0, 1) [my box] {complete};
  \node (cycle and across)    at (2, 1) [my box] {K4 - 1};
  \node (cycle)               at (0, -1) [my box] {cycle};
  \node (claw)                at (2,-1) [my box] {claw};
  \node (cycle-1 and hanging) at (.5, 0) [my box] {cycle-1 and hang};
  \node (path)                at (1.5, 0) [my box] {path};

  \draw [->,bend right=0] (complete) to (path);
  \draw [->] (complete) to (cycle and across);
  \draw [->] (complete) to (cycle);
  \draw [->] (complete) to (cycle-1 and hanging);
  \draw [->,bend left=0] (complete) to (claw);
  \draw [->] (cycle and across) to (cycle);
  \draw [->] (cycle and across) to (path);
  \draw [->] (cycle and across) to (cycle-1 and hanging);
  \draw [->] (cycle and across) to (claw);
  \draw [->] (cycle-1 and hanging) to (path);
  \draw [->] (cycle-1 and hanging) to (claw);
  \draw [->] (cycle) to (path);
\end{tikzpicture}
\end{center}

\begin{center}
\begin{tikzpicture}
    [scale=3.5,rotate=90-72,
     my box/.style={draw,minimum size=2em,inner sep=.1em,outer sep=.3em},
     font=\small,
    ]
  \node (complete)            at (0,0) [my box] {complete};
  \node (claw)                at (0:1) [my box] {claw};
  \node (cycle)               at (2*72:1) [my box] {cycle};
  \node (cycle and across)    at (1*72:1) [my box] {cycle and across};
  \node (cycle-1 and hanging) at (4*72:1) [my box] {cycle-1 and hanging};
  \node (path)                at (3*72:1) [my box] {path};

  \draw [->,bend right=0] (complete) to (path);
  \draw [->] (complete) to (cycle and across);
  \draw [->] (complete) to (cycle);
  \draw [->] (complete) to (cycle-1 and hanging);
  \draw [->,bend left=0] (complete) to (claw);
  \draw [->] (cycle and across) to (cycle);
  \draw [->] (cycle and across) to (path);
  \draw [->] (cycle and across) to (cycle-1 and hanging);
  \draw [->] (cycle and across) to (claw);
  \draw [->] (cycle-1 and hanging) to (path);
  \draw [->] (cycle-1 and hanging) to (claw);
  \draw [->] (cycle) to (path);
\end{tikzpicture}
\end{center}

\begin{center}
\begin{tikzpicture}
    [scale=2,
     my box/.style={draw,minimum size=2em,inner sep=.1em,outer sep=.3em},
     font=\small,
    ]
  \node (complete) at (1,2) [my box] {complete};
  \node (claw) at (3,0) [my box] {claw};
  \node (cycle) at (0,1) [my box] {cycle};
  \node (cycle and across) at (1.5, 1) [my box] {cycle and across};
  \node (cycle-1 and hanging) at (1,0) [my box] {cycle-1 and hanging};
  \node (path) at (-1,0) [my box] {path};

  \draw [->,bend right=20] (complete) to (path);
  \draw [->] (complete) to (cycle and across);
  \draw [->] (complete) to (cycle);
  \draw [->] (complete) to (cycle-1 and hanging);
  \draw [->,bend left=20] (complete) to (claw);
  \draw [->] (cycle and across) to (cycle);
  \draw [->] (cycle and across) to (path);
  \draw [->] (cycle and across) to (cycle-1 and hanging);
  \draw [->] (cycle and across) to (claw);
  \draw [->] (cycle-1 and hanging) to (path);
  \draw [->] (cycle-1 and hanging) to (claw);
  \draw [->] (cycle) to (path);
\end{tikzpicture}
\end{center}

%------------------------------------------------------------------------------
\section{Transpositions 4-Star = Claw}

\begin{center}
\begin{tikzpicture}
    [scale=1.6,
     my box/.style={draw,minimum size=2em,inner sep=.1em,outer sep=.3em},
     font=\small,
    ]
  \newcommand\MyOuter{3.8}
  \newcommand\MyInner{2}
  \newcommand\MyOuterAngle{15}
  \newcommand\MyInnerAngle{15}
  \node (0-1-2-3) at (180+\MyOuterAngle:\MyOuter) [my box] {0,1,2,3};
  \node (0-1-3-2) at (300-\MyOuterAngle:\MyOuter) [my box] {0,1,3,2};
  \node (1-0-2-3) at (180+\MyInnerAngle:\MyInner) [my box] {1,0,2,3};
  \node (1-0-3-2) at (300+\MyOuterAngle:\MyOuter) [my box] {1,0,3,2};

  \node (0-2-1-3) at (60-\MyOuterAngle:\MyOuter) [my box] {0,2,1,3};
  \node (0-2-3-1) at (300-\MyInnerAngle:\MyInner) [my box] {0,2,3,1};
  \node (2-0-1-3) at (0+\MyOuterAngle:\MyOuter) [my box] {2,0,1,3};
  \node (2-0-3-1) at (300+\MyInnerAngle:\MyInner) [my box] {2,0,3,1};

  \node (0-3-1-2) at (120-\MyInnerAngle:\MyInner) [my box] {0,3,1,2};
  \node (0-3-2-1) at (120+\MyInnerAngle:\MyInner) [my box] {0,3,2,1};
  \node (3-0-1-2) at (0-\MyOuterAngle:\MyOuter) [my box] {3,0,1,2};
  \node (3-0-2-1) at (60-\MyInnerAngle:\MyInner) [my box] {3,0,2,1};

  \node (1-2-0-3) at (240-\MyInnerAngle:\MyInner) [my box] {1,2,0,3};
  \node (1-2-3-0) at (60+\MyInnerAngle:\MyInner) [my box] {1,2,3,0};
  \node (2-1-0-3) at (240-\MyOuterAngle:\MyOuter) [my box] {2,1,0,3};
  \node (2-1-3-0) at (180-\MyInnerAngle:\MyInner) [my box] {2,1,3,0};
                                                  
  \node (1-3-0-2) at (240+\MyInnerAngle:\MyInner) [my box] {1,3,0,2};
  \node (1-3-2-0) at (120+\MyOuterAngle:\MyOuter) [my box] {1,3,2,0};
  \node (3-1-0-2) at (240+\MyOuterAngle:\MyOuter) [my box] {3,1,0,2};
  \node (3-1-2-0) at (180-\MyOuterAngle:\MyOuter) [my box] {3,1,2,0};
                                                  
  \node (2-3-0-1) at (0-\MyInnerAngle:\MyInner) [my box] {2,3,0,1};
  \node (2-3-1-0) at (120-\MyOuterAngle:\MyOuter) [my box] {2,3,1,0};
  \node (3-2-0-1) at (0+\MyInnerAngle:\MyInner) [my box] {3,2,0,1};
  \node (3-2-1-0) at (60+\MyOuterAngle:\MyOuter) [my box] {3,2,1,0};

  \draw [] (2-0-3-1) to (3-0-2-1);
  \draw [] (2-0-3-1) to (1-0-3-2);
  \draw [] (0-2-1-3) to (3-2-1-0);
  \draw [] (2-1-3-0) to (0-1-3-2);
  \draw [] (3-2-0-1) to (0-2-3-1);
  \draw [] (3-2-1-0) to (2-3-1-0);
  \draw [] (1-3-0-2) to (0-3-1-2);
  \draw [] (0-1-2-3) to (2-1-0-3);
  \draw [] (0-3-2-1) to (1-3-2-0);
  \draw [] (0-1-2-3) to (1-0-2-3);
  \draw [] (2-1-3-0) to (1-2-3-0);
  \draw [] (2-0-1-3) to (3-0-1-2);
  \draw [] (0-1-2-3) to (3-1-2-0);
  \draw [] (1-2-0-3) to (0-2-1-3);
  \draw [] (0-2-3-1) to (2-0-3-1);
  \draw [] (1-2-0-3) to (2-1-0-3);
  \draw [] (0-2-1-3) to (2-0-1-3);
  \draw [] (3-2-1-0) to (1-2-3-0);
  \draw [] (3-0-1-2) to (1-0-3-2);
  \draw [] (2-3-1-0) to (1-3-2-0);
  \draw [] (1-2-0-3) to (3-2-0-1);
  \draw [] (2-3-1-0) to (0-3-1-2);
  \draw [] (1-3-0-2) to (3-1-0-2);
  \draw [] (2-3-0-1) to (0-3-2-1);
  \draw [] (1-0-2-3) to (3-0-2-1);
  \draw [] (1-0-2-3) to (2-0-1-3);
  \draw [] (3-2-0-1) to (2-3-0-1);
  \draw [] (2-1-0-3) to (3-1-0-2);
  \draw [] (1-3-0-2) to (2-3-0-1);
  \draw [] (3-1-2-0) to (1-3-2-0);
  \draw [] (3-0-2-1) to (0-3-2-1);
  \draw [] (0-1-3-2) to (3-1-0-2);
  \draw [] (0-1-3-2) to (1-0-3-2);
  \draw [] (3-0-1-2) to (0-3-1-2);
  \draw [] (0-2-3-1) to (1-2-3-0);
  \draw [] (3-1-2-0) to (2-1-3-0);

\end{tikzpicture}
\end{center}


%------------------------------------------------------------------------------
\section{Transpositions 4-Cycle}

\url{https://hog.grinvin.org/ViewGraphInfo.action?id=1292}

\begin{center}
\begin{tikzpicture}
    [scale=1.6,
     my box/.style={draw,minimum size=2em,inner sep=.1em,outer sep=.3em},
     font=\small,
    ]
  \newcommand\MyOuter{3.8}
  \newcommand\MyInner{2}
  \newcommand\MyOuterAngle{15}
  \newcommand\MyInnerAngle{15}
  \node (0-1-2-3) at (180+\MyOuterAngle:\MyOuter) [my box] {0,1,2,3};
  \node (0-1-3-2) at (180-\MyOuterAngle:\MyOuter) [my box] {0,1,3,2};
  \node (1-0-2-3) at (180+\MyInnerAngle:\MyInner) [my box] {1,0,2,3};
  \node (1-0-3-2) at (180-\MyInnerAngle:\MyInner) [my box] {1,0,3,2};

  \node (0-2-1-3) at (240-\MyOuterAngle:\MyOuter) [my box] {0,2,1,3};
  \node (0-2-3-1) at (240-\MyInnerAngle:\MyInner) [my box] {0,2,3,1};
  \node (2-0-1-3) at (240+\MyOuterAngle:\MyOuter) [my box] {2,0,1,3};
  \node (2-0-3-1) at (240+\MyInnerAngle:\MyInner) [my box] {2,0,3,1};

  \node (0-3-1-2) at (120+\MyOuterAngle:\MyOuter) [my box] {0,3,1,2};
  \node (0-3-2-1) at (120+\MyInnerAngle:\MyInner) [my box] {0,3,2,1};
  \node (3-0-1-2) at (120-\MyOuterAngle:\MyOuter) [my box] {3,0,1,2};
  \node (3-0-2-1) at (120-\MyInnerAngle:\MyInner) [my box] {3,0,2,1};

  \node (1-2-0-3) at (300-\MyInnerAngle:\MyInner) [my box] {1,2,0,3};
  \node (1-2-3-0) at (300+\MyInnerAngle:\MyInner) [my box] {1,2,3,0};
  \node (2-1-0-3) at (300-\MyOuterAngle:\MyOuter) [my box] {2,1,0,3};
  \node (2-1-3-0) at (300+\MyOuterAngle:\MyOuter) [my box] {2,1,3,0};
                                                  
  \node (1-3-0-2) at (60+\MyInnerAngle:\MyInner) [my box] {1,3,0,2};
  \node (1-3-2-0) at (60-\MyInnerAngle:\MyInner) [my box] {1,3,2,0};
  \node (3-1-0-2) at (60+\MyOuterAngle:\MyOuter) [my box] {3,1,0,2};
  \node (3-1-2-0) at (60-\MyOuterAngle:\MyOuter) [my box] {3,1,2,0};
                                                  
  \node (2-3-0-1) at (0-\MyInnerAngle:\MyInner) [my box] {2,3,0,1};
  \node (2-3-1-0) at (0-\MyOuterAngle:\MyOuter) [my box] {2,3,1,0};
  \node (3-2-0-1) at (0+\MyInnerAngle:\MyInner) [my box] {3,2,0,1};
  \node (3-2-1-0) at (0+\MyOuterAngle:\MyOuter) [my box] {3,2,1,0};

  \draw [] (1-0-3-2) to (0-1-3-2);
  \draw [] (2-1-0-3) to (1-2-0-3);
  \draw [] (2-0-1-3) to (3-0-1-2);
  \draw [] (2-0-1-3) to (2-0-3-1);
  \draw [] (2-1-0-3) to (2-1-3-0);
  \draw [] (2-3-1-0) to (2-3-0-1);
  \draw [] (2-1-3-0) to (2-3-1-0);
  \draw [] (2-3-1-0) to (0-3-1-2);
  \draw [] (3-1-0-2) to (1-3-0-2);
  \draw [] (2-0-1-3) to (0-2-1-3);
  \draw [] (0-2-1-3) to (3-2-1-0);
  \draw [] (3-1-2-0) to (3-1-0-2);
  \draw [] (3-2-0-1) to (3-2-1-0);
  \draw [] (2-1-3-0) to (0-1-3-2);
  \draw [] (2-1-0-3) to (2-0-1-3);
  \draw [] (2-3-1-0) to (3-2-1-0);
  \draw [] (1-2-3-0) to (1-3-2-0);
  \draw [] (1-0-2-3) to (1-2-0-3);
  \draw [] (0-3-1-2) to (0-3-2-1);
  \draw [] (1-0-3-2) to (2-0-3-1);
  \draw [] (1-0-3-2) to (1-3-0-2);
  \draw [] (3-1-0-2) to (3-0-1-2);
  \draw [] (3-2-0-1) to (3-0-2-1);
  \draw [] (1-2-0-3) to (1-2-3-0);
  \draw [] (0-2-1-3) to (0-2-3-1);
  \draw [] (0-3-2-1) to (3-0-2-1);
  \draw [] (0-1-2-3) to (3-1-2-0);
  \draw [] (3-0-2-1) to (3-0-1-2);
  \draw [] (0-3-2-1) to (0-2-3-1);
  \draw [] (1-2-3-0) to (0-2-3-1);
  \draw [] (0-3-1-2) to (3-0-1-2);
  \draw [] (0-3-2-1) to (1-3-2-0);
  \draw [] (3-1-2-0) to (1-3-2-0);
  \draw [] (1-2-0-3) to (3-2-0-1);
  \draw [] (1-0-2-3) to (3-0-2-1);
  \draw [] (2-3-0-1) to (1-3-0-2);
  \draw [] (0-1-2-3) to (1-0-2-3);
  \draw [] (2-3-0-1) to (3-2-0-1);
  \draw [] (1-0-2-3) to (1-0-3-2);
  \draw [] (2-1-3-0) to (1-2-3-0);
  \draw [] (2-1-0-3) to (3-1-0-2);
  \draw [] (1-3-2-0) to (1-3-0-2);
  \draw [] (0-3-1-2) to (0-1-3-2);
  \draw [] (3-1-2-0) to (3-2-1-0);
  \draw [] (0-1-2-3) to (0-2-1-3);
  \draw [] (2-3-0-1) to (2-0-3-1);
  \draw [] (0-1-2-3) to (0-1-3-2);
  \draw [] (2-0-3-1) to (0-2-3-1);

\end{tikzpicture}
\end{center}


%------------------------------------------------------------------------------
\section{Transpositions 4-Row}

\url{https://hog.grinvin.org/ViewGraphInfo.action?id=1391}

4 by 1, row swap adjacent
\begin{center}
\begin{tikzpicture}
    [scale=1.8,
     my box/.style={draw,minimum size=2em,inner sep=.1em,outer sep=.3em},
     font=\small,
    ]
  \newcommand\MyOuter{3}
  \newcommand\MyInner{2}
  \newcommand\MyOuterAngle{15}
  \newcommand\MyInnerAngle{15}
  \node (0-1-2-3) at (180+\MyOuterAngle:\MyOuter) [my box] {0,1,2,3};
  \node (0-1-3-2) at (180-\MyOuterAngle:\MyOuter) [my box] {0,1,3,2};
  \node (1-0-2-3) at (180+\MyInnerAngle:\MyInner) [my box] {1,0,2,3};
  \node (1-0-3-2) at (180-\MyInnerAngle:\MyInner) [my box] {1,0,3,2};

  \node (0-2-1-3) at (240-\MyOuterAngle:\MyOuter) [my box] {0,2,1,3};
  \node (0-2-3-1) at (240-\MyInnerAngle:\MyInner) [my box] {0,2,3,1};
  \node (2-0-1-3) at (240+\MyOuterAngle:\MyOuter) [my box] {2,0,1,3};
  \node (2-0-3-1) at (240+\MyInnerAngle:\MyInner) [my box] {2,0,3,1};

  \node (0-3-1-2) at (120+\MyOuterAngle:\MyOuter) [my box] {0,3,1,2};
  \node (0-3-2-1) at (120+\MyInnerAngle:\MyInner) [my box] {0,3,2,1};
  \node (3-0-1-2) at (120-\MyOuterAngle:\MyOuter) [my box] {3,0,1,2};
  \node (3-0-2-1) at (120-\MyInnerAngle:\MyInner) [my box] {3,0,2,1};

  \node (1-2-0-3) at (300-\MyInnerAngle:\MyInner) [my box] {1,2,0,3};
  \node (1-2-3-0) at (300+\MyInnerAngle:\MyInner) [my box] {1,2,3,0};
  \node (2-1-0-3) at (300-\MyOuterAngle:\MyOuter) [my box] {2,1,0,3};
  \node (2-1-3-0) at (300+\MyOuterAngle:\MyOuter) [my box] {2,1,3,0};
                                                  
  \node (1-3-0-2) at (60+\MyInnerAngle:\MyInner) [my box] {1,3,0,2};
  \node (1-3-2-0) at (60-\MyInnerAngle:\MyInner) [my box] {1,3,2,0};
  \node (3-1-0-2) at (60+\MyOuterAngle:\MyOuter) [my box] {3,1,0,2};
  \node (3-1-2-0) at (60-\MyOuterAngle:\MyOuter) [my box] {3,1,2,0};
                                                  
  \node (2-3-0-1) at (0-\MyInnerAngle:\MyInner) [my box] {2,3,0,1};
  \node (2-3-1-0) at (0-\MyOuterAngle:\MyOuter) [my box] {2,3,1,0};
  \node (3-2-0-1) at (0+\MyInnerAngle:\MyInner) [my box] {3,2,0,1};
  \node (3-2-1-0) at (0+\MyOuterAngle:\MyOuter) [my box] {3,2,1,0};

  \draw [] (2-3-1-0) to (2-1-3-0);
  \draw [] (2-0-3-1) to (2-0-1-3);
  \draw [] (2-3-1-0) to (3-2-1-0);
  \draw [] (0-1-3-2) to (0-3-1-2);
  \draw [] (3-1-0-2) to (3-0-1-2);
  \draw [] (3-2-1-0) to (3-1-2-0);
  \draw [] (3-2-0-1) to (3-2-1-0);
  \draw [] (0-3-2-1) to (3-0-2-1);
  \draw [] (3-0-1-2) to (3-0-2-1);
  \draw [] (2-1-3-0) to (1-2-3-0);
  \draw [] (1-3-2-0) to (3-1-2-0);
  \draw [] (1-0-2-3) to (1-2-0-3);
  \draw [] (0-1-2-3) to (0-1-3-2);
  \draw [] (2-1-0-3) to (1-2-0-3);
  \draw [] (3-2-0-1) to (3-0-2-1);
  \draw [] (1-3-0-2) to (1-0-3-2);
  \draw [] (2-0-1-3) to (0-2-1-3);
  \draw [] (0-1-2-3) to (1-0-2-3);
  \draw [] (1-3-0-2) to (3-1-0-2);
  \draw [] (2-0-3-1) to (0-2-3-1);
  \draw [] (1-3-2-0) to (1-3-0-2);
  \draw [] (1-0-2-3) to (1-0-3-2);
  \draw [] (0-3-1-2) to (3-0-1-2);
  \draw [] (2-0-1-3) to (2-1-0-3);
  \draw [] (3-1-2-0) to (3-1-0-2);
  \draw [] (2-1-0-3) to (2-1-3-0);
  \draw [] (1-2-3-0) to (1-2-0-3);
  \draw [] (0-1-3-2) to (1-0-3-2);
  \draw [] (1-2-3-0) to (1-3-2-0);
  \draw [] (0-3-2-1) to (0-2-3-1);
  \draw [] (0-1-2-3) to (0-2-1-3);
  \draw [] (2-3-0-1) to (2-3-1-0);
  \draw [] (2-3-0-1) to (3-2-0-1);
  \draw [] (0-3-1-2) to (0-3-2-1);
  \draw [] (2-0-3-1) to (2-3-0-1);
  \draw [] (0-2-1-3) to (0-2-3-1);

\end{tikzpicture}
\end{center}

\begin{center}
\begin{tikzpicture}
    [scale=1.3,xscale=1.2,
     my box/.style={draw,minimum size=2em,inner sep=.1em,outer sep=.3em},
     font=\small,
    ]
  \node (0-1-2-3) at (1,1) [my box] {0,1,2,3};
  \node (1-0-2-3) at (0,1) [my box] {1,0,2,3};
  \node (0-1-3-2) at (2,0) [my box] {0,1,3,2};
  \node (1-0-3-2) at (-1,1) [my box] {1,0,3,2};

  \node (0-2-1-3) at (6,3) [my box] {0,2,1,3};
  \node (0-2-3-1) at (7,3) [my box] {0,2,3,1};
  \node (2-0-1-3) at (6,4) [my box] {2,0,1,3};
  \node (2-0-3-1) at (7,4) [my box] {2,0,3,1};

  \node (0-3-1-2) at (4,0) [my box] {0,3,1,2};
  \node (0-3-2-1) at (6,0) [my box] {0,3,2,1};
  \node (3-0-1-2) at (4,1) [my box] {3,0,1,2};
  \node (3-0-2-1) at (5,1) [my box] {3,0,2,1};

  \node (1-2-0-3) at (2,10) [my box] {1,2,0,3};
  \node (1-2-3-0) at (1,11) [my box] {1,2,3,0};
  \node (2-1-0-3) at (3,10) [my box] {2,1,0,3};
  \node (2-1-3-0) at (3,11) [my box] {2,1,3,0};

  \node (1-3-0-2) at (-1,4) [my box] {1,3,0,2};
  \node (1-3-2-0) at (-1,7) [my box] {1,3,2,0};
  \node (3-1-0-2) at (0,5)  [my box] {3,1,0,2};
  \node (3-1-2-0) at (0,6)  [my box] {3,1,2,0};

  \node (2-3-0-1) at (6,8)  [my box] {2,3,0,1};
  \node (2-3-1-0) at (5,11) [my box] {2,3,1,0};
  \node (3-2-0-1) at (5,9)  [my box] {3,2,0,1};
  \node (3-2-1-0) at (5,10) [my box] {3,2,1,0};

  \draw [] (2-3-1-0) to (2-1-3-0);
  \draw [] (2-0-3-1) to (2-3-0-1);
  \draw [] (0-3-2-1) to (0-2-3-1);
  \draw [] (3-0-1-2) to (3-0-2-1);
  \draw [] (0-1-2-3) to (0-1-3-2);
  \draw [] (2-3-0-1) to (2-3-1-0);
  \draw [] (2-0-1-3) to (2-1-0-3);
  \draw [] (1-2-3-0) to (1-2-0-3);
  \draw [] (2-0-3-1) to (2-0-1-3);
  \draw [] (2-3-1-0) to (3-2-1-0);
  \draw [] (2-3-0-1) to (3-2-0-1);
  \draw [] (3-1-2-0) to (3-1-0-2);
  \draw [] (2-0-1-3) to (0-2-1-3);
  \draw [] (1-3-0-2) to (3-1-0-2);
  \draw [] (0-1-3-2) to (0-3-1-2);
  \draw [] (2-1-0-3) to (1-2-0-3);
  \draw [] (0-2-1-3) to (0-2-3-1);
  \draw [] (0-3-2-1) to (3-0-2-1);
  \draw [] (1-2-3-0) to (1-3-2-0);
  \draw [] (1-0-2-3) to (1-2-0-3);
  \draw [] (3-2-0-1) to (3-2-1-0);
  \draw [] (1-3-0-2) to (1-0-3-2);
  \draw [] (0-3-1-2) to (0-3-2-1);
  \draw [] (3-2-0-1) to (3-0-2-1);
  \draw [] (1-3-2-0) to (3-1-2-0);
  \draw [] (0-1-3-2) to (1-0-3-2);
  \draw [] (2-1-0-3) to (2-1-3-0);
  \draw [] (0-1-2-3) to (1-0-2-3);
  \draw [] (3-2-1-0) to (3-1-2-0);
  \draw [] (1-0-2-3) to (1-0-3-2);
  \draw [] (0-1-2-3) to (0-2-1-3);
  \draw [] (2-1-3-0) to (1-2-3-0);
  \draw [] (0-3-1-2) to (3-0-1-2);
  \draw [] (2-0-3-1) to (0-2-3-1);
  \draw [] (1-3-2-0) to (1-3-0-2);
  \draw [] (3-1-0-2) to (3-0-1-2);

\end{tikzpicture}
\end{center}

%------------------------------------------------------------------------------
\section{Petersen Triangle Replaced}

Vertex transitive non-Hamiltonian.

\url{http://mathworld.wolfram.com/NonhamiltonianVertex-TransitiveGraph.html}

\begin{center}
\begin{tikzpicture}
    [scale=1.3,
     my box/.style={circle,draw,minimum size=2em,inner sep=.1em,outer sep=.2em},
     font=\small,
     rotate=90
    ]
  \newcommand\MyOuter{4}
  \newcommand\MyInner{2.7}
  \newcommand\MyTriangle{.9}

  \node (1-1) at ($(0:\MyOuter)+(30:\MyTriangle)$) [my box] {1-1};
  \node (1-2) at (0:\MyOuter) [my box] {1-2};
  \node (1-3) at ($(0:\MyOuter)+(-30:\MyTriangle)$) [my box] {1-3};
  \node (2-1) at ($(72:\MyOuter)+(-30+72:\MyTriangle)$) [my box] {2-1};
  \node (2-2) at ($(72:\MyOuter)+(30+72:\MyTriangle)$) [my box] {2-2};
  \node (2-3) at (72:\MyOuter) [my box] {2-3};
  \node (3-1) at ($(2*72:\MyOuter)+(30+2*72:\MyTriangle)$) [my box] {3-1};
  \node (3-2) at ($(2*72:\MyOuter)+(-30+2*72:\MyTriangle)$) [my box] {3-2};
  \node (3-3) at (2*72:\MyOuter) [my box] {3-3};
  \node (4-1) at ($(3*72:\MyOuter)+(-30+3*72:\MyTriangle)$) [my box] {4-1};
  \node (4-2) at ($(3*72:\MyOuter)+(30+3*72:\MyTriangle)$) [my box] {4-2};
  \node (4-3) at (3*72:\MyOuter) [my box] {4-3};
  \node (5-1) at ($(4*72:\MyOuter)+(30+4*72:\MyTriangle)$) [my box] {5-1};
  \node (5-2) at (4*72:\MyOuter) [my box] {5-2};
  \node (5-3) at ($(4*72:\MyOuter)+(-30+4*72:\MyTriangle)$) [my box] {5-3};
  \node (6-1) at (0*72:\MyInner) [my box] {6-1};
  \node (6-2) at ($(0*72:\MyInner)+(180-30+0*72:\MyTriangle)$) [my box] {6-2};
  \node (6-3) at ($(0*72:\MyInner)+(180+30+0*72:\MyTriangle)$) [my box] {6-3};
  \node (7-1) at ($(1*72:\MyInner)+(180+30+1*72:\MyTriangle)$) [my box] {7-1};
  \node (7-2) at (1*72:\MyInner) [my box] {7-2};
  \node (7-3) at ($(1*72:\MyInner)+(180-30+1*72:\MyTriangle)$) [my box] {7-3};
  \node (8-1) at ($(2*72:\MyInner)+(180-30+2*72:\MyTriangle)$) [my box] {8-1};
  \node (8-2) at ($(2*72:\MyInner)+(180+30+2*72:\MyTriangle)$) [my box] {8-2};
  \node (8-3) at (2*72:\MyInner) [my box] {8-3};
  \node (9-1) at ($(3*72:\MyInner)+(180-30+3*72:\MyTriangle)$) [my box] {9-1};
  \node (9-2) at ($(3*72:\MyInner)+(180+30+3*72:\MyTriangle)$) [my box] {9-2};
  \node (9-3) at (3*72:\MyInner) [my box] {9-3};
  \node (10-1) at ($(4*72:\MyInner)+(180+30+4*72:\MyTriangle)$) [my box] {10-1};
  \node (10-2) at (4*72:\MyInner) [my box] {10-2};
  \node (10-3) at ($(4*72:\MyInner)+(180-30+4*72:\MyTriangle)$) [my box] {10-3};

  \draw [] (5-1) to (1-3);
  \draw [] (7-1) to (10-3);
  \draw [] (6-1) to (6-3);
  \draw [] (6-2) to (8-2);
  \draw [] (5-1) to (5-3);
  \draw [] (1-1) to (1-2);
  \draw [] (4-1) to (3-1);
  \draw [] (4-1) to (4-3);
  \draw [] (4-1) to (4-2);
  \draw [] (6-2) to (6-3);
  \draw [] (1-2) to (1-3);
  \draw [] (10-1) to (10-3);
  \draw [] (2-3) to (2-1);
  \draw [] (2-2) to (3-2);
  \draw [] (6-1) to (6-2);
  \draw [] (2-2) to (2-3);
  \draw [] (6-3) to (9-1);
  \draw [] (4-2) to (4-3);
  \draw [] (5-2) to (5-3);
  \draw [] (7-1) to (7-3);
  \draw [] (7-1) to (7-2);
  \draw [] (8-1) to (8-2);
  \draw [] (6-1) to (1-2);
  \draw [] (9-2) to (9-3);
  \draw [] (7-3) to (9-2);
  \draw [] (7-2) to (7-3);
  \draw [] (3-3) to (8-3);
  \draw [] (9-1) to (9-2);
  \draw [] (5-1) to (5-2);
  \draw [] (3-1) to (3-2);
  \draw [] (2-2) to (2-1);
  \draw [] (10-2) to (10-3);
  \draw [] (1-1) to (2-1);
  \draw [] (5-2) to (10-2);
  \draw [] (8-2) to (8-3);
  \draw [] (10-1) to (8-1);
  \draw [] (2-3) to (7-2);
  \draw [] (9-1) to (9-3);
  \draw [] (10-1) to (10-2);
  \draw [] (1-1) to (1-3);
  \draw [] (3-1) to (3-3);
  \draw [] (4-3) to (9-3);
  \draw [] (4-2) to (5-3);
  \draw [] (8-1) to (8-3);
  \draw [] (3-2) to (3-3);

\end{tikzpicture}
\end{center}

%------------------------------------------------------------------------------
\section{Petersen Line Graph}

Petersen 3-regular so line graph $3+3-2=4$ regular
\begin{center}
\begin{tikzpicture}
    [scale=1.3,
     my box/.style={circle,draw,minimum size=2em,inner sep=.15em,outer sep=.3em},
     font=\small,
    ]
  \newcommand\MyOuter{4}
  \node (1-5)  at (18+3*72:\MyOuter) [my box] {1:5};
  \node (1-2)  at (18+0*72:\MyOuter) [my box] {1:2};
  \node (2-3)  at (18+2*72:\MyOuter) [my box] {2:3};
  \node (3-4)  at (18+4*72:\MyOuter) [my box] {3:4};
  \node (4-5)  at (18+1*72:\MyOuter) [my box] {4:5};
  \node (1-6)  at (18+4*72:2.1) [my box] {1:6};
  \node (5-10) at (18+2*72:2.1) [my box] {5:10};
  \node (4-9)  at (18+0*72:2.1) [my box] {4:9};
  \node (2-7)  at (18+1*72:2.1) [my box] {2:7};
  \node (3-8)  at (18+3*72:2.1) [my box] {3:8};
  \node (10-7) at (-18+2*72:.8) [my box] {10:7};
  \node (6-8)  at (-18+4*72:.8) [my box] {6:8};
  \node (6-9)  at (-18+0*72:.8) [my box] {6:9};
  \node (7-9)  at (-18+1*72:.8) [my box] {7:9};
  \node (8-10) at (-18+3*72:.8) [my box] {8:10};

  \draw [] (2-3) to (2-7);
  \draw [] (7-9) to (10-7);
  \draw [] (5-10) to (4-5);
  \draw [] (6-9) to (1-6);
  \draw [] (2-3) to (3-8);
  \draw [] (6-8) to (6-9);
  \draw [] (8-10) to (6-8);
  \draw [] (2-3) to (1-2);
  \draw [] (2-3) to (3-4);
  \draw [] (2-7) to (1-2);
  \draw [] (4-9) to (6-9);
  \draw [] (1-2) to (1-5);
  \draw [] (7-9) to (2-7);
  \draw [] (7-9) to (6-9);
  \draw [] (5-10) to (1-5);
  \draw [] (1-2) to (1-6);
  \draw [] (4-5) to (1-5);
  \draw [] (8-10) to (3-8);
  \draw [] (10-7) to (5-10);
  \draw [] (3-8) to (6-8);
  \draw [] (5-10) to (8-10);
  \draw [] (4-9) to (4-5);
  \draw [] (10-7) to (2-7);
  \draw [] (4-5) to (3-4);
  \draw [] (6-8) to (1-6);
  \draw [] (3-8) to (3-4);
  \draw [] (4-9) to (3-4);
  \draw [] (1-5) to (1-6);
  \draw [] (7-9) to (4-9);
  \draw [] (10-7) to (8-10);

\end{tikzpicture}
\end{center}

\begin{center}
\begin{tikzpicture}
    [scale=1.3,
     my box/.style={circle,draw,minimum size=2em,inner sep=.15em,outer sep=.3em},
     font=\small,
    ]

  \node (1-5)  at (18+2*72:3.5) [my box] {1:5};
  \node (1-2)  at (18+1*72:3.5) [my box] {1:2};
  \node (2-3)  at (18+0*72:3.5) [my box] {2:3};
  \node (3-4)  at (18+4*72:3.5) [my box] {3:4};
  \node (4-5)  at (18+3*72:3.5) [my box] {4:5};
  \node (1-6)  at (-18+2*72:2) [my box] {1:6};
  \node (5-10) at (-18+3*72:2) [my box] {5:10};
  \node (4-9)  at (-18+4*72:2) [my box] {4:9};
  \node (2-7)  at (-18+1*72:2) [my box] {2:7};
  \node (3-8)  at (-18+0*72:2) [my box] {3:8};
  \node (10-7) at (-18+2*72:1) [my box] {10:7};
  \node (6-8)  at (-18+1*72:1) [my box] {6:8};
  \node (6-9)  at (-18+3*72:1) [my box] {6:9};
  \node (7-9)  at (-18+0*72:1) [my box] {7:9};
  \node (8-10) at (-18+4*72:1) [my box] {8:10};

  \draw [] (2-3) to (2-7);
  \draw [] (7-9) to (10-7);
  \draw [] (5-10) to (4-5);
  \draw [] (6-9) to (1-6);
  \draw [] (2-3) to (3-8);
  \draw [] (6-8) to (6-9);
  \draw [] (8-10) to (6-8);
  \draw [] (2-3) to (1-2);
  \draw [] (2-3) to (3-4);
  \draw [] (2-7) to (1-2);
  \draw [] (4-9) to (6-9);
  \draw [] (1-2) to (1-5);
  \draw [] (7-9) to (2-7);
  \draw [] (7-9) to (6-9);
  \draw [] (5-10) to (1-5);
  \draw [] (1-2) to (1-6);
  \draw [] (4-5) to (1-5);
  \draw [] (8-10) to (3-8);
  \draw [] (10-7) to (5-10);
  \draw [] (3-8) to (6-8);
  \draw [] (5-10) to (8-10);
  \draw [] (4-9) to (4-5);
  \draw [] (10-7) to (2-7);
  \draw [] (4-5) to (3-4);
  \draw [] (6-8) to (1-6);
  \draw [] (3-8) to (3-4);
  \draw [] (4-9) to (3-4);
  \draw [] (1-5) to (1-6);
  \draw [] (7-9) to (4-9);
  \draw [] (10-7) to (8-10);

\end{tikzpicture}
\end{center}


\begin{center}
\begin{tikzpicture}
    [scale=1.3,
     my box/.style={circle,draw,minimum size=2em,inner sep=.15em,outer sep=.3em},
     font=\small,
    ]

  \node (1-5)  at (18+2*72:3.5) [my box] {1:5};
  \node (1-2)  at (18+1*72:3.5) [my box] {1:2};
  \node (2-3)  at (18+0*72:3.5) [my box] {2:3};
  \node (3-4)  at (18+4*72:3.5) [my box] {3:4};
  \node (4-5)  at (18+3*72:3.5) [my box] {4:5};
  \node (1-6)  at (-18+2*72:2) [my box] {1:6};
  \node (5-10) at (-18+3*72:2) [my box] {5:10};
  \node (4-9)  at (-18+4*72:2) [my box] {4:9};
  \node (2-7)  at (-18+1*72:2) [my box] {2:7};
  \node (3-8)  at (-18+0*72:2) [my box] {3:8};
  \node (10-7) at (18+3*72:1) [my box] {10:7};
  \node (6-8)  at (18+2*72:1) [my box] {6:8};
  \node (6-9)  at (18+4*72:1) [my box] {6:9};
  \node (7-9)  at (18+1*72:1) [my box] {7:9};
  \node (8-10) at (18+0*72:1) [my box] {8:10};

  \draw [] (2-3) to (2-7);
  \draw [] (7-9) to (10-7);
  \draw [] (5-10) to (4-5);
  \draw [] (6-9) to (1-6);
  \draw [] (2-3) to (3-8);
  \draw [] (6-8) to (6-9);
  \draw [] (8-10) to (6-8);
  \draw [] (2-3) to (1-2);
  \draw [] (2-3) to (3-4);
  \draw [] (2-7) to (1-2);
  \draw [] (4-9) to (6-9);
  \draw [] (1-2) to (1-5);
  \draw [] (7-9) to (2-7);
  \draw [] (7-9) to (6-9);
  \draw [] (5-10) to (1-5);
  \draw [] (1-2) to (1-6);
  \draw [] (4-5) to (1-5);
  \draw [] (8-10) to (3-8);
  \draw [] (10-7) to (5-10);
  \draw [] (3-8) to (6-8);
  \draw [] (5-10) to (8-10);
  \draw [] (4-9) to (4-5);
  \draw [] (10-7) to (2-7);
  \draw [] (4-5) to (3-4);
  \draw [] (6-8) to (1-6);
  \draw [] (3-8) to (3-4);
  \draw [] (4-9) to (3-4);
  \draw [] (1-5) to (1-6);
  \draw [] (7-9) to (4-9);
  \draw [] (10-7) to (8-10);

\end{tikzpicture}
\end{center}

%------------------------------------------------------------------------------
\section{Beineke All as Subgraphs, 12 Edges}

\url{https://hog.grinvin.org/ViewGraphInfo.action?id=748}

cf.\@ Introduction to Line Graphs
Emphasizing their construction, clique decompositions, and regularity

\begin{center}
\begin{tikzpicture}
    [scale=1,
     my box/.style={circle,draw,inner sep=.15em,outer sep=.3em},
     font=\small,
    ]

  \node at (0,1) [my box,name=1] {1};
  \node at (0,2) [my box,name=3] {3};
  \node at (0,3) [my box,name=0] {0};
  \node at (0,4) [my box,name=2] {2};
  \node at (-2, .25) [my box,name=4] {4};
  \node at (2, .25) [my box,name=5] {5};

  \draw [] (0) to (4);
  \draw [] (0) to (2);
  \draw [] (2) to (5);
  \draw [] (1) to (4);
  \draw [] (0) to (5);
  \draw [] (3) to (5);
  \draw [] (4) to (5);
  \draw [] (0) to (3);
  \draw [] (3) to (1);
  \draw [] (3) to (4);
  \draw [] (2) to (4);
  \draw [] (1) to (5);
\end{tikzpicture}
\end{center}

\begin{center}
\begin{tikzpicture}
    [scale=2,
     my box/.style={circle,draw,inner sep=.15em,outer sep=.3em},
     font=\small,
    ]

  \node at (4*72+18:1) [my box,name=0] {0};
  \node at (2*72+18:1) [my box,name=1] {1};
  \node at (0*72+18:1) [my box,name=2] {2};
  \node at (3*72+18:1) [my box,name=3] {3};
  \node at (0,0) [my box,name=4] {4};
  \node at (1*72+18:1) [my box,name=5] {5};

  \draw [] (0) to (4);
  \draw [] (0) to (2);
  \draw [] (2) to (5);
  \draw [] (1) to (4);
  \draw [] (0) to (5);
  \draw [] (3) to (5);
  \draw [] (4) to (5);
  \draw [] (0) to (3);
  \draw [] (3) to (1);
  \draw [] (3) to (4);
  \draw [] (2) to (4);
  \draw [] (1) to (5);
\end{tikzpicture}
\end{center}

\begin{center}
\begin{tikzpicture}
    [scale=2,
     my box/.style={circle,draw,inner sep=.15em,outer sep=.3em},
     font=\small,
    ]

  \node at (0,0) [my box,name=0] {0};
  \node at (2,2) [my box,name=1] {1};
  \node at (2,0) [my box,name=2] {2};
  \node at (0,2) [my box,name=3] {3};
  \node at (1,1) [my box,name=4] {4};
  \node at (2,1) [my box,name=5] {5};

  \draw [] (0) to (4);
  \draw [] (0) to (2);
  \draw [] (2) to (5);
  \draw [] (1) to (4);
  \draw [] (0) to (5);
  \draw [] (3) to (5);
  \draw [] (4) to (5);
  \draw [] (0) to (3);
  \draw [] (3) to (1);
  \draw [] (3) to (4);
  \draw [] (2) to (4);
  \draw [] (1) to (5);
\end{tikzpicture}
\end{center}

\begin{center}
\begin{tikzpicture}
    [scale=2,
     my box/.style={circle,draw,inner sep=.15em,outer sep=.3em},
     font=\small,
    ]

  \node at (0,0) [my box,name=0] {0};
  \node at (2,2) [my box,name=1] {1};
  \node at (2,0) [my box,name=2] {2};
  \node at (0,2) [my box,name=3] {3};
  \node at (.5,1) [my box,name=4] {4};
  \node at (1.5,1) [my box,name=5] {5};

  \draw [] (0) to (4);
  \draw [] (0) to (2);
  \draw [] (2) to (5);
  \draw [] (1) to (4);
  \draw [] (0) to (5);
  \draw [] (3) to (5);
  \draw [] (4) to (5);
  \draw [] (0) to (3);
  \draw [] (3) to (1);
  \draw [] (3) to (4);
  \draw [] (2) to (4);
  \draw [] (1) to (5);
\end{tikzpicture}
\end{center}

\begin{center}
\begin{tikzpicture}
    [scale=2,
     my box/.style={circle,draw,inner sep=.15em,outer sep=.3em},
     font=\small,
    ]

  \node at (1,0) [my box,name=1] {1};
  \node at (2,0) [my box,name=3] {3};
  \node at (3,0) [my box,name=0] {0};
  \node at (4,0) [my box,name=2] {2};
  \node at (2.5,1) [my box,name=4] {4};
  \node at (2.5,-1) [my box,name=5] {5};

  \draw [] (0) to (4);
  \draw [] (0) to (2);
  \draw [] (2) to (5);
  \draw [] (1) to (4);
  \draw [] (0) to (5);
  \draw [] (3) to (5);
  \draw [] (4) to (5);
  \draw [] (0) to (3);
  \draw [] (3) to (1);
  \draw [] (3) to (4);
  \draw [] (2) to (4);
  \draw [] (1) to (5);
\end{tikzpicture}
\end{center}




%------------------------------------------------------------------------------
\section{Beineke Subgraph Relations}

\begin{center}
\begin{tikzpicture}
    [scale=1.5,
     my box/.style={circle,draw,inner sep=.15em,outer sep=.3em},
     font=\small,
    ]

  \node at (0,0) [my box,name=1] {1};
  \node at (-1,0) [my box,name=2] {2};
  \node at (-1,-1) [my box,name=3] {3};
  \node at (1,0) [my box,name=4] {4};
  \node at (-1,1) [my box,name=5] {5};
  \node at (0,1) [my box,name=6] {6};
  \node at (1, -1) [my box,name=7] {7};
  \node at (1,1) [my box,name=8] {8};
  \node at (0,-1) [my box,name=9] {9};

  \draw [->] (6) to (8);
  \draw [->] (6) to (1);
  \draw [->] (2) to (1);
  \draw [->] (8) to (4);
  \draw [->] (4) to (1);
  \draw [->,bend left=0] (5) to (4);
  \draw [->] (9) to (4);
  \draw [->] (6) to (4);
  \draw [->,bend right=0] (7) to (4);
  \draw [->] (5) to (2);
  \draw [->] (3) to (1);
  \draw [->] (5) to (1);
  \draw [->] (6) to (5);
  \draw [->] (9) to (1);
  \draw [->] (3) to (2);
  \draw [->] (8) to (1);
  \draw [->] (9) to (7);
  \draw [->] (6) to (7);
  \draw [->] (7) to (1);
  \draw [->,bend left=0] (6) to (2);

\end{tikzpicture}
\end{center}

\begin{center}
\begin{tikzpicture}
    [scale=2,
     my box/.style={circle,draw,inner sep=.15em,outer sep=.3em},
     font=\small,
    ]

  \node (1) at (0,0)           [my box] {1};
  \node (4) at (0*45+22.5*1:1) [my box] {4};
  \node (8) at (1*45+22.5*1:1) [my box] {8};
  \node (6) at (2*45+22.5*1:1) [my box] {6};
  \node (5) at (3*45+22.5*1:1) [my box] {5};
  \node (2) at (4*45+22.5*1:1) [my box] {2};
  \node (3) at (5*45+22.5*1:1) [my box] {3};
  \node (9) at (6*45+22.5*1:1) [my box] {9};
  \node (7) at (7*45+22.5*1:1) [my box] {7};

  \draw [->] (6) to (8);
  \draw [->] (6) to (1);
  \draw [->] (2) to (1);
  \draw [->] (8) to (4);
  \draw [->] (4) to (1);
  \draw [->,bend left=0] (5) to (4);
  \draw [->] (9) to (4);
  \draw [->] (6) to (4);
  \draw [->,bend right=0] (7) to (4);
  \draw [->] (5) to (2);
  \draw [->] (3) to (1);
  \draw [->] (5) to (1);
  \draw [->] (6) to (5);
  \draw [->] (9) to (1);
  \draw [->] (3) to (2);
  \draw [->] (8) to (1);
  \draw [->] (9) to (7);
  \draw [->] (6) to (7);
  \draw [->] (7) to (1);
  \draw [->,bend left=0] (6) to (2);

\end{tikzpicture}
\end{center}


%------------------------------------------------------------------------------
\section{Hanoi 2 Discs, 4 Spindles, Star}

\url{https://hog.grinvin.org/ViewGraphInfo.action?id=21152}

\begin{center}
\begin{tikzpicture}
    [scale=1.25,
     my box/.style={circle,draw,inner sep=.15em,outer sep=.3em},
     font=\scriptsize,
    ]
  % degree=2 count 9
  % degree=3 count 4
  % degree=4 count 3

  \node at (0,0)   [my box,name=00] {00};
  \node at (30:1)  [my box,name=01] {01};
  \node at (-90:1) [my box,name=02] {02};
  \node at (150:1) [my box,name=03] {03};

  \begin{scope}[shift={(210:2)}]
  \node at (0,0)   [my box,name=10] {10};
  \node at (210:1) [my box,name=11] {11};
  \node at (-30:1) [my box,name=12] {12};
  \node at (90:1) [my box,name=13] {13};
  \end{scope}

  \begin{scope}[shift={(90:2)}]
  \node at (0,0)   [my box,name=20] {20};
  \node at (-30:1) [my box,name=21] {21};
  \node at (90:1) [my box,name=22] {22};
  \node at (210:1) [my box,name=23] {23};
  \end{scope}

  \begin{scope}[shift={(-30:2)}]
  \node at (0,0)   [my box,name=30] {30};
  \node at (90:1) [my box,name=31] {31};
  \node at (210:1) [my box,name=32] {32};
  \node at (-30:1) [my box,name=33] {33};
  \end{scope}


  \draw [] (01) to (21);
  \draw [] (20) to (21);
  \draw [] (03) to (23);
  \draw [] (00) to (01);
  \draw [] (00) to (02);
  \draw [] (02) to (12);
  \draw [] (20) to (23);
  \draw [] (00) to (03);
  \draw [] (10) to (12);
  \draw [] (30) to (33);
  \draw [] (01) to (31);
  \draw [] (03) to (13);
  \draw [] (10) to (11);
  \draw [] (30) to (32);
  \draw [] (10) to (13);
  \draw [] (20) to (22);
  \draw [] (02) to (32);
  \draw [] (30) to (31);

\end{tikzpicture}
\end{center}

Each vertex is a configuration of discs on spindles for a variation by
Stockmeyer of the towers of Hanoi puzzle.  There are 4 spindles in a
star pattern and discs can only move between the centre spindle and
one of the 3 outer spindles (not among those outer spindles).

The present graph is for 2 discs.  The centre claw has big disc on the
centre spindle and the small disc moving to or from one of the outer
spindles.  When the small disc is not on the centre spindle the big
disc can move from there to one of the other outer spindles (where
further claws are the small disc moving).

The puzzle is to move both discs from one outer spindle to another outer 
spindle.  This is a path between two of the degree\hyp{}1 vertices.

Paul K. Stockmeyer, "Variations on the Four-Post Tower of Hanoi
Puzzle", Congressus Numerantium, volume 102, 1994, pages 3-12,
\url{http://www.cs.wm.edu/~pkstoc/boca.ps}


%------------------------------------------------------------------------------
\section{Hanoi Graph 2 Discs, 4 Spindles, Linear}

\url{https://hog.grinvin.org/ViewGraphInfo.action?id=25143}

% 3, 10, 19, 34, 57, 88
% A160002
% my(n=2); 3^n+n-1 == 10
% my(n=3); 3^n+n-1 == 29

\begin{center}
\begin{tikzpicture}
  [scale=1.5,
   my box/.style={circle,draw,inner sep=.15em,outer sep=.3em,minimum size=1.6em},
  ]

  \node at (4,-1) [my box,name=0] {0};
  \node at (4,0) [my box,name=1] {1};
  \node at (4,1) [my box,name=2] {2};
  \node at (4,2) [my box,name=3] {3};

  \node at (1,1) [my box,name=4] {4};
  \node at (2,1) [my box,name=5] {5};
  \node at (3,1) [my box,name=6] {6};
  \node at (3,2) [my box,name=7] {7};

  \node at (0,1) [my box,name=8] {8};
  \node at (0,2) [my box,name=9] {9};
  \node at (1,2) [my box,name=10] {10};
  \node at (2,2) [my box,name=11] {11};

  \node at (-1,1) [my box,name=12] {12};
  \node at (-1,2) [my box,name=13] {13};
  \node at (-1,3) [my box,name=14] {14};
  \node at (-1,4) [my box,name=15] {15};


  \draw [] (14) to (15);
  \draw [] (9) to (10);
  \draw [] (0) to (1);
  \draw [] (9) to (8);
  \draw [] (4) to (5);
  \draw [] (3) to (7);
  \draw [] (1) to (2);
  \draw [] (7) to (11);
  \draw [] (6) to (7);
  \draw [] (4) to (8);
  \draw [] (12) to (8);
  \draw [] (9) to (13);
  \draw [] (2) to (6);
  \draw [] (13) to (14);
  \draw [] (2) to (3);
  \draw [] (6) to (5);
  \draw [] (12) to (13);
  \draw [] (10) to (11);

\end{tikzpicture}
\end{center}

Each vertex is a configuration of discs on spindles for a variation on the towers of Hanoi puzzle by Stockmeyer where discs can only move forward or backward between adjacent spindles along a row of 4 spindles.

The two degree-1 vertices are the 2 discs on the first or last
spindle.  The problem is to move the discs from one end to the other.
This is the graph diameter of 10 moves (OEIS A160002).

Paul K. Stockmeyer, "Variations on the Four-Post Tower of Hanoi
Puzzle", Congressus Numerantium, volume 102, 1994, pages 3-12,
\url{http://www.cs.wm.edu/~pkstoc/boca.ps}


% my(n=2); 3^n+n-1 == 10

Stockmeyer gives a straightforward recursive solution for the general case n discs along 4 spindles.  This establishes an upper bound of $3^n+n-1$ moves.  The n=2 case here is the last where that solution is optimal.

"The Four-in-a-Row Puzzle"

%------------------------------------------------------------------------------
\section{Hanoi Graph 2 Discs, 4 Spindles, Cyclic}

\url{https://hog.grinvin.org/ViewGraphInfo.action?id=25141}

\begin{center}
\begin{tikzpicture}
    [scale=2,
     my box/.style={circle,draw,inner sep=.15em,outer sep=.3em},
     font=\scriptsize,
    ]

  \node at (0,0) [my box,name=0] {0};
  \node at (0,1) [my box,name=1] {1};
  \node at (1,1) [my box,name=2] {2};
  \node at (1,0) [my box,name=3] {3};

  \node at (3,1) [my box,name=4] {4};
  \node at (3,0) [my box,name=5] {5};
  \node at (2,0) [my box,name=6] {6};
  \node at (2,1) [my box,name=7] {7};

  \node at (2,2) [my box,name=8] {8};
  \node at (2,3) [my box,name=9] {9};
  \node at (3,3) [my box,name=10] {10};
  \node at (3,2) [my box,name=11] {11};

  \node at (1,3) [my box,name=12] {12};
  \node at (1,2) [my box,name=13] {13};
  \node at (0,2) [my box,name=14] {14};
  \node at (0,3) [my box,name=15] {15};

  \draw [] (2) to (6);
  \draw [] (1) to (13);
  \draw [] (13) to (14);
  \draw [] (7) to (4);
  \draw [] (1) to (2);
  \draw [] (15) to (14);
  \draw [] (11) to (10);
  \draw [] (4) to (5);
  \draw [] (8) to (4);
  \draw [] (6) to (7);
  \draw [] (12) to (13);
  \draw [] (8) to (12);
  \draw [] (0) to (3);
  \draw [] (9) to (10);
  \draw [] (8) to (9);
  \draw [] (0) to (1);
  \draw [] (8) to (11);
  \draw [] (3) to (7);
  \draw [] (6) to (5);
  \draw [] (3) to (2);
  \draw [] (12) to (15);
  \draw [] (2) to (14);
  \draw [] (9) to (13);
  \draw [] (11) to (7);

\end{tikzpicture}
\end{center}

Each vertex is a configuration of discs on spindles for a variation on the towers of Hanoi puzzle where discs can only move to an adjacent spindle around a cycle of 4 spindles.

The degree-2 vertices are where the 2 discs are on the same spindle so the only moves are the smaller disc back or forward.  The 4-cycle at each of those vertices is the small disc going successively around the 4 spindles.  The cross connections between those cycles are where the big disc moves (where permitted).

%------------------------------------------------------------------------------
\section{Hanoi 2 Discs 4 Spindles}

\url{https://hog.grinvin.org/ViewGraphInfo.action?id=22742}

\begin{center}
\begin{tikzpicture}
    [scale=.75,
     my box/.style={circle,draw,inner sep=.15em,outer sep=.3em},
     font=\scriptsize,
    ]
  \newcommand\MyOuter{3}

  \begin{scope}[shift={(0,\MyOuter)}]
    \node at (0,1) [my box,name=0] {0};
    \node at (-1,0) [my box,name=1] {1};
    \node at (1,0) [my box,name=2] {2};
    \node at (0,-1) [my box,name=3] {3};
  \end{scope}

  \begin{scope}[shift={(\MyOuter,0)}]
    \node at (0,-1) [my box,name=4] {4};
    \node at (1,0) [my box,name=5] {5};
    \node at (-1,0) [my box,name=6] {6};
    \node at (0,1) [my box,name=7] {7};
  \end{scope}

  \begin{scope}[shift={(-\MyOuter,0)}]
    \node at (0,-1) [my box,name=8] {8};
    \node at (1,0) [my box,name=9] {9};
    \node at (-1,0) [my box,name=10] {10};
    \node at (0,1) [my box,name=11] {11};
  \end{scope}

  \begin{scope}[shift={(0,-\MyOuter)}]
    \node at (0,1) [my box,name=12] {12};
    \node at (-1,0) [my box,name=13] {13};
    \node at (1,0) [my box,name=14] {14};
    \node at (0,-1) [my box,name=15] {15};
  \end{scope}

  \draw [] (9) to (11);
  \draw [] (4) to (6);
  \draw [] (3) to (11);
  \draw [] (8) to (9);
  \draw [] (15) to (14);
  \draw [] (12) to (15);
  \draw [] (5) to (6);
  \draw [] (12) to (14);
  \draw [] (15) to (13);
  \draw [] (8) to (12);
  \draw [] (8) to (11);
  \draw [] (1) to (13);
  \draw [] (9) to (13);
  \draw [] (6) to (14);
  \draw [] (4) to (7);
  \draw [] (5) to (7);
  \draw [] (7) to (11);
  \draw [] (0) to (2);
  \draw [] (10) to (9);
  \draw [] (1) to (9);
  \draw [] (3) to (7);
  \draw [] (4) to (12);
  \draw [] (6) to (7);
  \draw [] (0) to (1);
  \draw [] (1) to (2);
  \draw [] (4) to (5);
  \draw [] (12) to (13);
  \draw [] (10) to (11);
  \draw [] (13) to (14);
  \draw [] (4) to (8);
  \draw [] (8) to (10);
  \draw [] (2) to (6);
  \draw [] (2) to (3);
  \draw [] (0) to (3);
  \draw [] (2) to (14);
  \draw [] (1) to (3);

\end{tikzpicture}
\end{center}

Each vertex is a configuration of discs on spindles for a variation on
the towers of Hanoi puzzle with 2 discs on 4 spindles.

The degree-3 vertices are where the 2 discs are on the same spindle so
only the smaller disc can move.  The complete-4 clique at each of
those is the small disc moving among the 4 spindles.  The cross
connections between those subgraphs are where the big disc moves (to
one of the 2 spindles different from itself and the smaller disc).

\medskip

Cf.\ picture in Andreas M. Hinz, Sandi Klav\v{z}ar, Sara Sabrina
Zemlji\v{c}, ``Sierpinski Graphs as Spanning Subgraphs of Hanoi
Graphs'', Cent.\ Eur.\ J. Math., 11(6), 2013, 1153-1157.
DOI 10.2478/s11533-013-0227-7



%------------------------------------------------------------------------------
\section{Johnson 5,2}

\url{https://hog.grinvin.org/ViewGraphInfo.action?id=21154}

\begin{center}
\begin{tikzpicture}
    [scale=.74,
     my box/.style={circle,draw,inner sep=.15em,outer sep=.3em},
     font=\small,
    ]
  \newcommand\MyOuter{9}
  \newcommand\MyInner{6.2}
  % \newcommand\MyInnerAngle{54+180}
  \newcommand\MyInnerAngle{54}

  \node at (90-0*72: \MyOuter) [my box,name=1-2] {1,2};
  \node at (90-1*72: \MyOuter) [my box,name=2-3] {2,3};
  \node at (90-2*72: \MyOuter) [my box,name=3-4] {3,4};
  \node at (90-3*72: \MyOuter) [my box,name=4-5] {4,5};
  \node at (90-4*72: \MyOuter) [my box,name=1-5] {1,5};

  \node at (\MyInnerAngle-0*72: \MyInner) [my box,name=1-3] {1,3};
  \node at (\MyInnerAngle-1*72: \MyInner) [my box,name=2-4] {2,4};
  \node at (\MyInnerAngle-2*72: \MyInner) [my box,name=3-5] {3,5};
  \node at (\MyInnerAngle-3*72: \MyInner) [my box,name=1-4] {1,4};
  \node at (\MyInnerAngle-4*72: \MyInner) [my box,name=2-5] {2,5};

  \draw (1-2) to (1-3);
  \draw (1-2) to (1-4);
  \draw (1-2) to (1-5);
  \draw (1-2) to (2-3);
  \draw (1-2) to (2-4);
  \draw (1-2) to (2-5);
  
  \draw (1-4) to (4-5);
  \draw (2-4) to (3-4);
  \draw (1-3) to (1-5);
  \draw (1-5) to (2-5);
  \draw (1-3) to (2-3);
  \draw (2-5) to (4-5);
  \draw (3-5) to (4-5);
  \draw (1-3) to (3-4);
  \draw (2-3) to (3-5);
  \draw (3-4) to (4-5);
  \draw (1-4) to (1-5);
  \draw (1-5) to (3-5);
  \draw (1-4) to (3-4);
  \draw (1-4) to (2-4);
  \draw (1-3) to (1-4);
  \draw (2-3) to (3-4);
  \draw (3-4) to (3-5);
  \draw (2-3) to (2-4);
  \draw (1-3) to (3-5);
  \draw (2-5) to (3-5);
  \draw (2-4) to (4-5);
  \draw (1-5) to (4-5);
  \draw (2-4) to (2-5);
  \draw (2-3) to (2-5);

\end{tikzpicture}
\end{center}

\begin{center}
\begin{tikzpicture}
    [scale=1,
     my box/.style={circle,draw,inner sep=.15em,outer sep=.3em},
     font=\scriptsize,
    ]

  \node at (-0*36: 5) [my box,name=1-2] {1,2};    % 3
  \node at (-1*36: 5) [my box,name=2-5] {2,5};    % 7
  \node at (-2*36: 5) [my box,name=2-4] {2,4};    % 6
  \node at (-3*36: 5) [my box,name=2-3] {2,3};    % 5
  \node at (-4*36: 5) [my box,name=3-5] {3,5};    % 8
  \node at (-5*36: 5) [my box,name=3-4] {3,4};    % 7
  \node at (-6*36: 5) [my box,name=1-3] {1,3};    % 4

  \node at (-7*36: 5) [my box,name=1-5] {1,5};    % 6
  \node at (-8*36: 5) [my box,name=1-4] {1,4};    % 5
  \node at (-9*36: 5) [my box,name=4-5] {4,5};    % 9

  \draw (1-2) to (1-3);
  \draw (1-2) to (1-4);
  \draw (1-2) to (1-5);
  \draw (1-2) to (2-3);
  \draw (1-2) to (2-4);
  \draw (1-2) to (2-5);

  \draw (1-4) to (4-5);
  \draw (2-4) to (3-4);
  \draw (1-3) to (1-5);
  \draw (1-5) to (2-5);
  \draw (1-3) to (2-3);
  \draw (2-5) to (4-5);
  \draw (3-5) to (4-5);
  \draw (1-3) to (3-4);
  \draw (2-3) to (3-5);
  \draw (3-4) to (4-5);
  \draw (1-4) to (1-5);
  \draw (1-5) to (3-5);
  \draw (1-4) to (3-4);
  \draw (1-4) to (2-4);
  \draw (1-3) to (1-4);
  \draw (2-3) to (3-4);
  \draw (3-4) to (3-5);
  \draw (2-3) to (2-4);
  \draw (1-3) to (3-5);
  \draw (2-5) to (3-5);
  \draw (2-4) to (4-5);
  \draw (1-5) to (4-5);
  \draw (2-4) to (2-5);
  \draw (2-3) to (2-5);

\end{tikzpicture}
\end{center}

\begin{center}
\begin{tikzpicture}
    [scale=.74,
     my box/.style={circle,draw,minimum width=2.5em,inner sep=.15em,outer sep=.3em},
     font=\small,
    ]
  \newcommand\MyOuter{9}
  \newcommand\MyInner{5}
  \newcommand\MyInIn{5}
  % \newcommand\MyInnerAngle{54+180}
  \newcommand\MyInnerAngle{54}

  \node at (-0*36: \MyOuter) [my box,name=1-2] {1-2};
  \node at (-2*36: \MyOuter) [my box,name=2-3] {2-3};
  \node at (-4*36: \MyOuter) [my box,name=3-4] {3-4};
  \node at (-6*36: \MyOuter) [my box,name=4-5] {4-5};
  \node at (-8*36: \MyOuter) [my box,name=1-5] {1-5};

  \node at (-1*36: \MyInner) [my box,name=2-7] {2-7};
  \node at (-3*36: \MyInner) [my box,name=3-8] {3-8};
  \node at (-5*36: \MyInner) [my box,name=4-9] {4-9};
  \node at (-7*36: \MyInner) [my box,name=5-10] {5-10};
  \node at (-9*36: \MyInner) [my box,name=1-6] {1-6};

  \node at (-4*36: \MyInIn) [my box,name=10-7] {10-7};
  \node at (-6*36: \MyInIn) [my box,name=6-8] {6-8};
  \node at (-8*36: \MyInIn) [my box,name=7-9] {7-9};
  \node at (-10*36: \MyInIn) [my box,name=8-10] {8-10};
  \node at (-12*36: \MyInIn) [my box,name=6-9] {6-9};

  \draw [] (2-3) to (1-2);
  \draw [] (2-3) to (3-4);
  \draw [] (1-5) to (4-5);
  \draw [] (1-6) to (6-8);
  \draw [] (1-5) to (1-6);
  \draw [] (2-3) to (3-8);
  \draw [] (8-10) to (3-8);
  \draw [] (1-6) to (6-9);
  \draw [] (4-5) to (5-10);
  \draw [] (4-5) to (3-4);
  \draw [] (4-9) to (7-9);
  \draw [] (10-7) to (2-7);
  \draw [] (6-9) to (7-9);
  \draw [] (4-5) to (4-9);
  \draw [] (2-7) to (1-2);
  \draw [] (8-10) to (5-10);
  \draw [] (6-8) to (3-8);
  \draw [] (10-7) to (8-10);
  \draw [] (4-9) to (6-9);
  \draw [] (1-5) to (5-10);
  \draw [] (1-5) to (1-2);
  \draw [] (4-9) to (3-4);
  \draw [] (1-6) to (1-2);
  \draw [] (2-7) to (2-3);
  \draw [] (8-10) to (6-8);
  \draw [] (2-7) to (7-9);
  \draw [] (10-7) to (5-10);
  \draw [] (6-9) to (6-8);
  \draw [] (3-8) to (3-4);
  \draw [] (10-7) to (7-9);

\end{tikzpicture}
\end{center}

%------------------------------------------------------------------------------

\end{document}

% Local variables:
% compile-command: "latexmk -file-line-error -pdf pictures.tex"
% End:
