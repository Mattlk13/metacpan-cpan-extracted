% Copyright 2014, 2015, 2016, 2017, 2018, 2019 Kevin Ryde
%

%; whizzy section

\documentclass{article}
\usepackage[T1]{fontenc}  % T1 for accents, before babel
\usepackage{amsmath}
\allowdisplaybreaks

\usepackage{needspace}
\usepackage{gensymb}    % for \degree
\usepackage{hyphenat}   % for \hyp hyphenation of words with -

\usepackage[pdfusetitle,
            pdflang={en}, % RFC3066 style ISO639
           ]{hyperref}
\renewcommand\figureautorefname{figure}   % lower case
\usepackage[all]{hypcap} % figure links to top of figure

% breakurl must be after hyperref
% whizzytex circa its version 1.3.3 does something hairy which upsets breakurl,
% so skip when whizzytex previewing
\usepackage{ifthen}
\ifthenelse{\isundefined{\Whizzytex}}{
\usepackage{breakurl} % for breaking long \url in DVI
}

\usepackage{mathtools}  % for \mathclap and showonlyrefs
\mathtoolsset{showonlyrefs=true,showmanualtags=true}

\usepackage{amsthm}

\usepackage{tikz}
\usetikzlibrary{arrows.meta}  % for Latex arrows To[length etc
\usetikzlibrary{bending}      % for arrow [bend]
\usetikzlibrary{calc}         % for ($(...)$) coordinate calculations
\usetikzlibrary{decorations}  % for [decoration=]
\usetikzlibrary{decorations.pathreplacing}  % for decoration=brace
\usetikzlibrary{shapes}       % for shape aspect=1
\tikzset{font=\small,         % same as text
         >=Latex}             % arrowhead style
  % must be capital Latex for [harpoon] half arrows


%------------------------------------------------------------------------------
% personal preferences

\hypersetup{
  pdfborderstyle={/W 0},  % no border on hyperlinks
}

% these must be after \begin{document} to take effect, hence \AtBeginDocument
\AtBeginDocument{%
  \setlength\abovedisplayskip{.7\baselineskip}
  \setlength\belowdisplayskip{.7\baselineskip}
  \setlength\abovedisplayshortskip{.5\baselineskip}
  \setlength\belowdisplayshortskip{.5\baselineskip}
}

% less space after "plain" style \end{theorem} etc
\makeatletter
\g@addto@macro\th@plain{\thm@postskip=1\baselineskip}
\makeatother

%------------------------------------------------------------------------------
% Generic Macros

% GP-DEFINE  default(strictargs,1);

\newcommand\MySlash{\slash\hspace{0pt}}
\newcommand\MyTightDots{.\kern.1em.\kern.1em.}

\newbox\MyNegativePhantomBox
\newcommand\MyNegativePhantom[1]{%
  \setbox\MyNegativePhantomBox\hbox{#1}%
  \hbox to -\wd\MyNegativePhantomBox{}}
% -\wd\MyNegativePhantomBox

% % Uncomment this to see the bounding box around each picture.
% \tikzset{
%   every picture/.append style={
%     execute at end picture={
%       \draw (current bounding box.south west)
%         rectangle (current bounding box.north east);
% }}}


%------------------------------------------------------------------------------
\begin{document}



%------------------------------------------------------------------------------
\section{9,3 Symmetric}

9,3 symmetric.  9 points, 3 ``lines''.
Pair of points at most 1 line.
Each line 3 points, each point 3 lines.

\subsection{Tri-Hex}
\begin{center}
\begin{tikzpicture}
    [scale=1.7,
     my box/.style={draw,minimum size=1.3em,inner sep=.15em,outer sep=.1em},
    ]

  \node (1) at (90:2.5) [my box] {1};
  \node (5) at (-150:2.5) [my box] {5};
  \node (3) at (-30:2.5) [my box] {3};

  \node (4) at (270:1.2) [my box] {4};
  \node (2) at (30:1.2) [my box] {2};
  \node (6) at (150:1.2) [my box] {6};

  \node (7) at (90:.6) [my box] {7};
  \node (8) at (-30:.6) [my box] {8};
  \node (9) at (-150:.6) [my box] {9};

  \draw [] (8) to (7);
  \draw [] (4) to (5);
  \draw [] (8) to (4);
  \draw [] (1) to (2);
  \draw [] (2) to (7);
  \draw [] (6) to (5);
  \draw [] (8) to (9);
  \draw [] (7) to (5);
  \draw [] (1) to (6);
  \draw [] (6) to (9);
  \draw [] (3) to (4);
  \draw [] (1) to (8);
  \draw [] (3) to (9);
  \draw [] (7) to (9);
  \draw [] (4) to (9);
  \draw [] (2) to (3);
  \draw [] (6) to (7);
  \draw [] (2) to (8);

\end{tikzpicture}
\end{center}

\subsection{Pappus}

Geometric lines: \url{https://hog.grinvin.org/ViewGraphInfo.action?id=33731}

Lines as cliques: \url{https://hog.grinvin.org/ViewGraphInfo.action?id=370}
\begin{center}
\begin{tikzpicture}
    [scale=1.7,
     my box/.style={draw,minimum size=1.3em,inner sep=.15em,outer sep=.1em},
    ]

  \node (1) at (-2,-1) [my box] {1};
  \node (2) at (0,-1) [my box] {2};
  \node (3) at (2,-1) [my box] {3};
  \node (4) at (1,0) [my box] {4};
  \node (5) at (2,1) [my box] {5};
  \node (6) at (0,1) [my box] {6};
  \node (7) at (-2,1) [my box] {7};
  \node (8) at (-1,0) [my box] {8};
  \node (9) at (0,0) [my box] {9};

  \draw [] (1) to (2);
  \draw [] (2) to (4);
  \draw [] (2) to (8);
  \draw [] (9) to (7);
  \draw [] (2) to (3);
  \draw [] (4) to (6);
  \draw [] (8) to (6);
  \draw [] (5) to (6);
  \draw [] (1) to (8);
  \draw [] (3) to (4);
  \draw [] (4) to (5);
  \draw [] (9) to (3);
  \draw [] (9) to (4);
  \draw [] (1) to (9);
  \draw [] (8) to (9);
  \draw [] (6) to (7);
  \draw [] (9) to (5);
  \draw [] (8) to (7);

\end{tikzpicture}
\end{center}

\subsection{Other}
\begin{center}
\begin{tikzpicture}
    [scale=1.8,
     my box/.style={draw,circle,minimum size=1.1em,inner sep=.15em},
    ]

  \node (1) at (0,0) [my box] {1};
  \node (2) at (1.5,0) [my box] {2};
  \node (3) at (4,0) [my box] {3};
  \node (4) at (3.2, .8) [my box] {4};
  \node (5) at (2,2) [my box] {5};
  \node (6) at (1.3, 1.3) [my box] {6};
  \node (7) at (1.4, .6) [my box] {7};
  \node (8) at (2.3, .4) [my box] {8};
  \node (9) at (2.2, 1.1) [my box] {9};

  \draw [] (7) to (8);
  \draw [] (7) to (6);
  \draw [] (9) to (8);
  \draw [] (1) to (2);
  \draw [] (6) to (5);
  \draw [] (9) to (6);
  \draw [] (3) to (8);
  \draw [] (3) to (4);
  \draw [] (1) to (7);
  \draw [] (2) to (3);
  \draw [] (7) to (9);
  \draw [] (2) to (7);
  \draw [] (2) to (8);
  \draw [] (5) to (4);
  \draw [] (9) to (5);
  \draw [] (9) to (4);
  \draw [] (1) to (6);
  \draw [] (8) to (4);

\end{tikzpicture}
\end{center}


\pagebreak 

Circulant N=9 1,2
\begin{center}
\begin{tikzpicture}
    [scale=1.8,
     my box/.style={draw,circle,minimum size=1.1em,inner sep=.15em},
    ]

  \node (1) at (0*40:1) [my box] {1};
  \node (2) at (1*40:1) [my box] {2};
  \node (3) at (2*40:1) [my box] {3};
  \node (4) at (3*40:1) [my box] {4};
  \node (5) at (4*40:1) [my box] {5};
  \node (6) at (5*40:1) [my box] {6};
  \node (7) at (6*40:1) [my box] {7};
  \node (8) at (7*40:1) [my box] {8};
  \node (9) at (8*40:1) [my box] {9};

  \draw [] (2) to (3);
  \draw [] (2) to (9);
  \draw [] (3) to (4);
  \draw [] (8) to (9);
  \draw [] (5) to (7);
  \draw [] (1) to (2);
  \draw [] (7) to (8);
  \draw [] (6) to (8);
  \draw [] (7) to (9);
  \draw [] (3) to (5);
  \draw [] (4) to (6);
  \draw [] (1) to (9);
  \draw [] (6) to (7);
  \draw [] (5) to (6);
  \draw [] (1) to (8);
  \draw [] (1) to (3);
  \draw [] (2) to (4);
  \draw [] (4) to (5);

\end{tikzpicture}
\end{center}
\begin{center}
\begin{tikzpicture}
    [scale=3,
     my box/.style={draw,circle,minimum size=1.1em,inner sep=.15em},
    ]

  \node (1) at (0,0) [my box] {1};
  \node (2) at (.8, .4) [my box] {2};
  \node (3) at (1, 0) [my box] {3};
  \node (4) at (2,0) [my box] {4};
  \node (5) at (1.3, .6) [my box] {5};
  \node (6) at ($(2,0) + (120:1)$) [my box] {6};
  \node (7) at (60:2) [my box] {7};
  \node (8) at (.9, .87) [my box] {8};
  \node (9) at (60:1) [my box] {9};

  \draw [] (2) to (3);
  \draw [] (2) to (9);
  \draw [] (3) to (4);
  \draw [] (8) to (9);
  \draw [] (5) to (7);
  \draw [] (1) to (2);
  \draw [] (7) to (8);
  \draw [] (6) to (8);
  \draw [] (7) to (9);
  \draw [] (3) to (5);
  \draw [] (4) to (6);
  \draw [] (1) to (9);
  \draw [] (6) to (7);
  \draw [] (5) to (6);
  \draw [] (1) to (8);
  \draw [] (1) to (3);
  \draw [] (2) to (4);
  \draw [] (4) to (5);

\end{tikzpicture}
\end{center}


%------------------------------------------------------------------------------
\section{Tamari Lattice Intervals Lattice N=3}

\begin{center}
\begin{tikzpicture}
    [scale=1.7,yscale=.85,
     my box/.style={draw,minimum size=1.3em,inner sep=.15em,outer sep=.1em},
    ]

  \node (111 111) at (0,0) [my box] {111 111};

  \node (111 112) at (1,1) [my box] {111 112};
  \node (111 121) at (2,-2) [my box] {111 121};

  \node (112 112) at (2,2) [my box] {112 112};
  \node (111 113) at (2,0) [my box] {111 113};
  \node (121 121) at (3,-3) [my box] {121 121};

  \node (112 113) at (3,1) [my box] {112 113};
  \node (111 123) at (3,-1) [my box] {111 123};

  \node (112 123) at (4,0) [my box] {112 123};
  \node (113 113) at (4,2) [my box] {113 113};

  \node (113 123) at (5,1) [my box] {113 123};
  \node (121 123) at (4,-2) [my box] {121 123};
  \node (123 123) at (6,0) [my box] {123 123};

  \draw [->] (111 111) to (111 121);
  \draw [->] (111 111) to (111 112);

  \draw [->] (112 123) to (113 123);
  \draw [->] (111 113) to (111 123);
  \draw [->] (121 123) to (123 123);
  \draw [->] (121 121) to (121 123);
  \draw [->] (111 112) to (111 113);
  \draw [->] (112 112) to (112 113);
  \draw [->] (111 112) to (112 112);
  \draw [->] (111 113) to (112 113);
  \draw [->] (112 113) to (113 113);
  \draw [->] (113 123) to (123 123);
  \draw [->] (112 113) to (112 123);
  \draw [->] (111 121) to (121 121);
  \draw [->] (111 121) to (111 123);
  \draw [->] (111 123) to (112 123);
  \draw [->] (113 113) to (113 123);
  \draw [->] (111 123) to (121 123);

\end{tikzpicture}
\end{center}

\noindent
\hspace{-5em}
\begin{tikzpicture}
    [scale=1.1,yscale=1,
     my box/.style={draw,minimum size=1.3em,inner sep=.2em,outer sep=.1em,align=center,font=\footnotesize},
    ]

  \node (1111-1111) at (0,0) [my box] {1111 \\ 1111};

  \node (1111-1112) at (1,1) [my box] {1111 \\ 1112};
  \node (1111-1211) at (1,0) [my box] {1111 \\ 1211};
  \node (1111-1121) at (1,-1) [my box] {1111 \\ 1121};

  \node (1112-1112) at (2,2) [my box] {1112 \\ 1112};
  \node (1111-1113) at (2,1) [my box] {1111 \\ 1113};
  \node (1111-1212) at (2,0) [my box] {1111 \\ 1212};
  \node (1111-1131) at (2,-1) [my box] {1111 \\ 1131};
  \node (1121-1121) at (2,-2) [my box] {1121 \\ 1121};

  \node (1112-1212) at (3,3) [my box] {1112 \\ 1212};
  \node (1112-1113) at (3,2) [my box] {1112 \\ 1113};
  \node (1111-1114) at (3,1) [my box] {1111 \\ 1114};
  \node (1111-1123) at (3,0) [my box] {1111 \\ 1123};
  \node (1111-1231) at (3,-1) [my box] {1111 \\ 1231};

  \node (1112-1214) at (4,4) [my box] {1112 \\ 1214};
  \node (1111-1124) at (4,1) [my box] {1111 \\ 1124};
  \node (1121-1123) at (4,-1) [my box] {1121 \\ 1123};
  \node (1121-1231) at (4,-2) [my box] {1121 \\ 1231};
  \node (1211-1211) at (4,-3) [my box] {1211 \\ 1211};

  \node (1112-1123) at (5,4) [my box] {1112 \\ 1123};
  \node (1113-1113) at (5,3) [my box] {1113 \\ 1113};
  \node (1111-1134) at (5,0) [my box] {1111 \\ 1134};

  \node (1112-1124) at (6,4) [my box] {1112 \\ 1124};
  \node (1111-1234) at (6,1) [my box] {1111 \\ 1234};
  \node (1113-1123) at (6,2) [my box] {1113 \\ 1123};
  \node (1112-1134) at (6,0) [my box] {1112 \\ 1134};
  \node (1121-1134) at (6,-1) [my box] {1121 \\ 1134};

  \node (1212-1214) at (5,6) [my box] {1212 \\ 1214};

  \node (1111-1214) at (4,2) [my box] {1111 \\ 1214};
  \node (1121-1131) at (3,-3) [my box] {1121 \\ 1131};

  \node (1121-1124) at (5,-1) [my box] {1121 \\ 1124};

  \node (1212-1234) at (7,6) [my box] {1212 \\ 1234};
  \node (1113-1124) at (7,3) [my box] {1113 \\ 1124};
  \node (1131-1134) at (7,-2) [my box] {1131 \\ 1134};

  \node (1114-1114) at (8,7) [my box] {1114 \\ 1114};
  \node (1114-1234) at (8,3) [my box] {1114 \\ 1234};
  \node (1113-1134) at (8,2) [my box] {1113 \\ 1134};

  \node (1114-1124) at (9,7) [my box] {1114 \\ 1124};
  \node (1211-1234) at (9,-3) [my box] {1211 \\ 1234};


  \node (1124-1124) at (10,7) [my box] {1124 \\ 1124};

  \node (1212-1212) at (4,6) [my box] {1212 \\ 1212};

  \node (1214-1214) at (7,7) [my box] {1214 \\ 1214};

  \node (1112-1114) at (4+2,0) [my box] {1112 \\ 1114};
  \node (1112-1234) at (4+2,6) [my box] {1112 \\ 1234};
  \node (1113-1114) at (4+3,0) [my box] {1113 \\ 1114};
  \node (1113-1214) at (4+3,4) [my box] {1113 \\ 1214};
  \node (1113-1234) at (4+3,5) [my box] {1113 \\ 1234};
  \node (1114-1134) at (4+4,0) [my box] {1114 \\ 1134};
  \node (1114-1214) at (4+4,1) [my box] {1114 \\ 1214};
  \node (1121-1234) at (4+5,1) [my box] {1121 \\ 1234};
  \node (1123-1123) at (4+5,2) [my box] {1123 \\ 1123};
  \node (1123-1124) at (4+5,3) [my box] {1123 \\ 1124};
  \node (1123-1134) at (4+5,4) [my box] {1123 \\ 1134};
  \node (1123-1234) at (9,5) [my box] {1123 \\ 1234};
  \node (1124-1134) at (9,-1) [my box] {1124 \\ 1134};
  \node (1124-1234) at (10,0) [my box] {1124 \\ 1234};
  \node (1131-1131) at (4,-4) [my box] {1131 \\ 1131};
  \node (1131-1231) at (10,3) [my box] {1131 \\ 1231};
  \node (1131-1234) at (11,4) [my box] {1131 \\ 1234};
  \node (1134-1134) at (11,3) [my box] {1134 \\ 1134};
  \node (1134-1234) at (12,4) [my box] {1134 \\ 1234};
  \node (1211-1212) at (11,0) [my box] {1211 \\ 1212};
  \node (1211-1214) at (11,1) [my box] {1211 \\ 1214};
  \node (1211-1231) at (5,-2) [my box] {1211 \\ 1231};
  \node (1214-1234) at (11,-4) [my box] {1214 \\ 1234};

  \node (1231-1231) at (11,-1) [my box] {1231 \\ 1231};

  \node (1231-1234) at (12,-1) [my box] {1231 \\ 1234};

  \node (1234-1234) at (13,-3) [my box] {1234 \\ 1234};

  \draw [->] (1112-1114) to (1112-1124);
  \draw [->] (1212-1214) to (1214-1214);
  \draw [->] (1211-1214) to (1211-1234);
  \draw [->] (1211-1214) to (1212-1214);
  \draw [->] (1121-1121) to (1121-1123);
  \draw [->] (1111-1113) to (1112-1113);
  \draw [->] (1211-1234) to (1212-1234);
  \draw [->] (1111-1214) to (1211-1214);
  \draw [->] (1112-1234) to (1113-1234);
  \draw [->] (1113-1214) to (1113-1234);
  \draw [->] (1112-1212) to (1212-1212);
  \draw [->] (1112-1114) to (1112-1214);
  \draw [->] (1211-1212) to (1212-1212);
  \draw [->] (1111-1111) to (1111-1211);
  \draw [->] (1111-1111) to (1111-1112);
  \draw [->] (1112-1123) to (1112-1124);
  \draw [->] (1112-1114) to (1113-1114);
  \draw [->] (1211-1231) to (1231-1231);
  \draw [->] (1112-1212) to (1112-1214);
  \draw [->] (1113-1123) to (1113-1124);
  \draw [->] (1113-1124) to (1123-1124);
  \draw [->] (1111-1114) to (1112-1114);
  \draw [->] (1121-1124) to (1123-1124);
  \draw [->] (1111-1113) to (1111-1114);
  \draw [->] (1111-1212) to (1112-1212);
  \draw [->] (1114-1124) to (1114-1134);
  \draw [->] (1111-1112) to (1112-1112);
  \draw [->] (1111-1134) to (1112-1134);
  \draw [->] (1111-1121) to (1111-1131);
  \draw [->] (1114-1124) to (1124-1124);
  \draw [->] (1112-1134) to (1112-1234);
  \draw [->] (1134-1234) to (1234-1234);
  \draw [->] (1121-1231) to (1131-1231);
  \draw [->] (1113-1114) to (1113-1214);
  \draw [->] (1114-1114) to (1114-1124);
  \draw [->] (1124-1134) to (1134-1134);
  \draw [->] (1114-1214) to (1114-1234);
  \draw [->] (1111-1211) to (1111-1231);
  \draw [->] (1121-1131) to (1131-1131);
  \draw [->] (1114-1234) to (1214-1234);
  \draw [->] (1111-1112) to (1111-1113);
  \draw [->] (1111-1214) to (1111-1234);
  \draw [->] (1112-1112) to (1112-1113);
  \draw [->] (1112-1113) to (1112-1114);
  \draw [->] (1111-1234) to (1112-1234);
  \draw [->] (1124-1124) to (1124-1134);
  \draw [->] (1131-1231) to (1131-1234);
  \draw [->] (1121-1123) to (1121-1124);
  \draw [->] (1113-1114) to (1113-1124);
  \draw [->] (1211-1212) to (1211-1214);
  \draw [->] (1114-1234) to (1124-1234);
  \draw [->] (1112-1123) to (1113-1123);
  \draw [->] (1111-1121) to (1111-1123);
  \draw [->] (1112-1113) to (1113-1113);
  \draw [->] (1111-1231) to (1211-1231);
  \draw [->] (1131-1134) to (1131-1234);
  \draw [->] (1112-1234) to (1212-1234);
  \draw [->] (1113-1134) to (1114-1134);
  \draw [->] (1111-1131) to (1121-1131);
  \draw [->] (1123-1123) to (1123-1124);
  \draw [->] (1112-1214) to (1113-1214);
  \draw [->] (1123-1124) to (1124-1124);
  \draw [->] (1111-1114) to (1111-1214);
  \draw [->] (1113-1123) to (1123-1123);
  \draw [->] (1111-1211) to (1211-1211);
  \draw [->] (1112-1113) to (1112-1123);
  \draw [->] (1121-1124) to (1121-1134);
  \draw [->] (1134-1134) to (1134-1234);
  \draw [->] (1113-1134) to (1113-1234);
  \draw [->] (1212-1212) to (1212-1214);
  \draw [->] (1211-1234) to (1231-1234);
  \draw [->] (1123-1234) to (1124-1234);
  \draw [->] (1121-1234) to (1131-1234);
  \draw [->] (1111-1211) to (1111-1212);
  \draw [->] (1113-1134) to (1123-1134);
  \draw [->] (1231-1234) to (1234-1234);
  \draw [->] (1112-1214) to (1212-1214);
  \draw [->] (1113-1234) to (1114-1234);
  \draw [->] (1113-1124) to (1114-1124);
  \draw [->] (1111-1212) to (1111-1214);
  \draw [->] (1114-1114) to (1114-1214);
  \draw [->] (1111-1214) to (1112-1214);
  \draw [->] (1121-1134) to (1123-1134);
  \draw [->] (1214-1234) to (1234-1234);
  \draw [->] (1111-1134) to (1111-1234);
  \draw [->] (1121-1131) to (1121-1134);
  \draw [->] (1111-1111) to (1111-1121);
  \draw [->] (1113-1234) to (1123-1234);
  \draw [->] (1111-1124) to (1111-1134);
  \draw [->] (1114-1214) to (1214-1214);
  \draw [->] (1124-1234) to (1134-1234);
  \draw [->] (1212-1214) to (1212-1234);
  \draw [->] (1111-1234) to (1121-1234);
  \draw [->] (1111-1114) to (1111-1124);
  \draw [->] (1121-1234) to (1123-1234);
  \draw [->] (1113-1114) to (1114-1114);
  \draw [->] (1121-1231) to (1121-1234);
  \draw [->] (1112-1124) to (1113-1124);
  \draw [->] (1231-1231) to (1231-1234);
  \draw [->] (1211-1231) to (1211-1234);
  \draw [->] (1111-1121) to (1121-1121);
  \draw [->] (1111-1212) to (1211-1212);
  \draw [->] (1121-1123) to (1123-1123);
  \draw [->] (1113-1124) to (1113-1134);
  \draw [->] (1123-1134) to (1123-1234);
  \draw [->] (1121-1134) to (1131-1134);
  \draw [->] (1111-1231) to (1111-1234);
  \draw [->] (1111-1131) to (1111-1231);
  \draw [->] (1111-1134) to (1121-1134);
  \draw [->] (1112-1134) to (1113-1134);
  \draw [->] (1111-1123) to (1112-1123);
  \draw [->] (1112-1112) to (1112-1212);
  \draw [->] (1114-1134) to (1124-1134);
  \draw [->] (1131-1131) to (1131-1134);
  \draw [->] (1131-1234) to (1134-1234);
  \draw [->] (1121-1121) to (1121-1131);
  \draw [->] (1112-1214) to (1112-1234);
  \draw [->] (1113-1214) to (1114-1214);
  \draw [->] (1111-1123) to (1121-1123);
  \draw [->] (1131-1134) to (1134-1134);
  \draw [->] (1123-1124) to (1123-1134);
  \draw [->] (1111-1113) to (1111-1123);
  \draw [->] (1113-1113) to (1113-1123);
  \draw [->] (1124-1134) to (1124-1234);
  \draw [->] (1131-1131) to (1131-1231);
  \draw [->] (1211-1211) to (1211-1212);
  \draw [->] (1112-1124) to (1112-1134);
  \draw [->] (1123-1134) to (1124-1134);
  \draw [->] (1111-1124) to (1121-1124);
  \draw [->] (1131-1234) to (1231-1234);
  \draw [->] (1111-1124) to (1112-1124);
  \draw [->] (1214-1214) to (1214-1234);
  \draw [->] (1114-1134) to (1114-1234);
  \draw [->] (1211-1211) to (1211-1231);
  \draw [->] (1111-1112) to (1111-1212);
  \draw [->] (1111-1123) to (1111-1124);
  \draw [->] (1121-1134) to (1121-1234);
  \draw [->] (1113-1113) to (1113-1114);
  \draw [->] (1111-1131) to (1111-1134);
  \draw [->] (1121-1131) to (1121-1231);
  \draw [->] (1111-1234) to (1211-1234);
  \draw [->] (1212-1234) to (1214-1234);
  \draw [->] (1111-1231) to (1121-1231);
  \draw [->] (1131-1231) to (1231-1231);

\end{tikzpicture}


%------------------------------------------------------------------------------
\section{Binary Tree Dexter}

\subsection{N=4}
\url{https://hog.grinvin.org/ViewGraphInfo.action?id=33621}

Each graph vertex is a balanced binary string (Dyck word) of N pairs.
There are Catalan(4)=14 such.  Each graph edge is the "dexter"
transform by Chapoton which shifts a block of 1s to raise their
adjacent balanced substring.  This is some multiple binary tree
``rotates''.  Chapoton notes right-arm rotates are a subset of dexter.

F. Chapoton, ``Some Properties of a New Partial Order on Dyck Paths'',
September 2018.
\newline
\url{https://hal.archives-ouvertes.fr/hal-01878792}
\newline
\url{https://arxiv.org/abs/1809.10981}

\begin{center}
\begin{tikzpicture}
    [scale=2,yscale=.8,
     my box/.style={draw,minimum size=1.3em,inner sep=.15em,outer sep=.1em},
    ]

  \node (1111) at (0,0) [my box] {1111};

  \node (1112) at (-1,1) [my box] {1112};
  \node (1211) at (0,1) [my box] {1211};
  \node (1121) at (1,1) [my box] {1121};

  \node (1113) at (-2,2) [my box] {1113};
  \node (1212) at (-1,2) [my box] {1212};
  \node (1123) at (0,2) [my box] {1123};
  \node (1231) at (1,2) [my box] {1231};
  \node (1131) at (2,2) [my box] {1131};

  \node (1114) at (-2,3) [my box] {1114};
  \node (1214) at (-1,3) [my box] {1214};
  \node (1234) at (0,3) [my box] {1234};

  \node (1124) at (-1,4) [my box] {1124};
  \node (1134) at (1,4) [my box] {1134};

  \draw [->] (1111) to (1112);
  \draw [->] (1113) to (1114);
  \draw [->] (1123) to (1124);
  \draw [->] (1112) to (1123);
  \draw [->] (1113) to (1214);
  \draw [->] (1211) to (1212);
  \draw [->] (1114) to (1124);
  \draw [->] (1131) to (1134);
  \draw [->] (1212) to (1214);
  \draw [->] (1112) to (1113);
  \draw [->] (1121) to (1131);
  \draw [->] (1123) to (1234);
  \draw [->] (1123) to (1134);
  \draw [->] (1111) to (1211);
  \draw [->] (1211) to (1231);
  \draw [->] (1112) to (1212);
  \draw [->] (1212) to (1234);
  \draw [->] (1231) to (1234);
  \draw [->] (1121) to (1123);
  \draw [->] (1111) to (1121);
  \draw [->] (1121) to (1231);

\end{tikzpicture}
\end{center}

%------------------------------------------------------------------------------
\section{Binary Tree Rotate A-Empty}

\subsection{N=4}

\url{https://hog.grinvin.org/ViewGraphInfo.action?id=33607}

Each vertex represents a binary tree of N=4 vertices (there are
Catalan(4)=14 such).  Each edge is by a ``rotate'' where the first
subtree of the rotate is empty.  (Or equivalently where the end
subtree is empty, that being the same by mirror image and rotate other
way.)  The result is edge intersection of Tamari (rotate) and Stanley
(flip) lattices.

A. Bonnin and J.M. Pallo, ``A Shortest Path Metric on Unlabelled Binary
Trees'', Pattern Recognition Letters, volume 13, 1992, pages 411-415.

\begin{center}
\begin{tikzpicture}
    [scale=2,yscale=.5,
     my box/.style={draw,minimum size=1.3em,inner sep=.15em,outer sep=.1em},
    ]

  \node (1111) at (0,0) [my box] {1111};

  \node (1211) at (-1,1) [my box] {1211};
  \node (1112) at (0,1) [my box] {1112};
  \node (1121) at (1,1) [my box] {1121};

  \node (1212) at (-1,2) [my box] {1212};
  \node (1113) at (0,2) [my box] {1113};
  \node (1131) at (1,2) [my box] {1131};

  \node (1114) at (-1,3) [my box] {1114};
  \node (1123) at (0,3) [my box] {1123};
  \node (1231) at (1,3) [my box] {1231};

  \node (1214) at (-1,4) [my box] {1214};

  \node (1124) at (0,4) [my box] {1124};

  \node (1134) at (0,5) [my box] {1134};
  \node (1234) at (0,6) [my box] {1234};

  \draw [->] (1111) to (1211);
  \draw [->] (1131) to (1231);
  \draw [->] (1112) to (1212);
  \draw [->] (1134) to (1234);
  \draw [->] (1114) to (1124);
  \draw [->] (1111) to (1112);
  \draw [->] (1113) to (1114);
  \draw [->] (1114) to (1214);
  \draw [->] (1124) to (1134);
  \draw [->] (1121) to (1131);
  \draw [->] (1112) to (1113);
  \draw [->] (1211) to (1212);
  \draw [->] (1123) to (1124);
  \draw [->] (1113) to (1123);
  \draw [->] (1111) to (1121);

\end{tikzpicture}
\end{center}


%------------------------------------------------------------------------------
\section{Balanced Binary Filling}

\subsection{N=4}
\url{https://hog.grinvin.org/ViewGraphInfo.action?id=33595}

Each graph vertex is a balanced binary string (Dyck word) of N=4 pairs
(there are Catalan(4)=14 such).  Each edge is the "filling" defined by
Sapounakis, Tasoulas, and Tsikouras, which flips all 01 -> 10 to reach
a new string.  Each string has one destination so the result is a
tree.

A. Sapounakis, I. Tasoulas, P. Tsikouras, ``On the Dominance Partial
Ordering of Dyck Paths'', Journal of Integer Sequences, volume 9, 2006,
article 06.2.5.
\newline
\url{https://cs.uwaterloo.ca/journals/JIS/VOL9/Tsikouras/tsikouras67.html}

\begin{center}
\begin{tikzpicture}
    [scale=2,yscale=.5,
     my box/.style={draw,minimum size=1.3em,inner sep=.15em,outer sep=.1em},
    ]

  \node (11010010) at (0,4) [my box] {11010010};
  \node (11100010) at (0,3) [my box] {11100010};
  \node (11100100) at (1, 3.5) [my box] {11100100};

  \node (10111000) at (0, -3) [my box] {10111000};
  \node (10110100) at (0, -4) [my box] {10110100};
  \node (11011000) at (1, -3.5) [my box] {11011000};

  \node (10101010) at (0,2) [my box] {10101010};
  \node (10101100) at (0,1) [my box] {10101100};
  \node (10110010) at (0,0) [my box] {10110010};
  \node (11001100) at (0,-1) [my box] {11001100};
  \node (11001010) at (0,-2) [my box] {11001010};
  \node (11010100) at (1, 0) [my box] {11010100};

  \node (11101000) at (2, 0) [my box] {11101000};
  \node (11110000) at (3, 0) [my box] {11110000};

  \draw [->] (11100010) to (11100100);
  \draw [->] (10101100) to (11010100);
  \draw [->] (10110100) to (11011000);
  \draw [->] (11011000) to (11101000);
  \draw [->] (10110010) to (11010100);
  \draw [->] (11100100) to (11101000);
  \draw [->] (11010100) to (11101000);
  \draw [->] (11001100) to (11010100);
  \draw [->] (11001010) to (11010100);
  \draw [->] (10101010) to (11010100);
  \draw [->] (10111000) to (11011000);
  \draw [->] (11010010) to (11100100);
  \draw [->] (11101000) to (11110000);

\end{tikzpicture}
\end{center}


%------------------------------------------------------------------------------
\section{Excess Configurations}

\subsection{N=3}
\url{https://hog.grinvin.org/ViewGraphInfo.action?id=33585}

Janson, Knuth, Luczak, and Pittel, consider the "configuration" of a
multigraph.  Configuration is how many components have excess
r=1,2,3,etc, where excess r = number of edges - number of vertices.
Adding an edge to a multigraph can go to a new configuration.

The present graph is configurations of total excess 0 to 3 inclusive.
Configurations are partitions of the total excess, so there are 7
vertices (OEIS A000070).  0, 1 or 2 terms are taken from a
configuration to form a new one, so 8 edges (OEIS A029859).  The
start, being no components r>=1, is the degree-1 with degree-3
neighbour.

Svante Janson, Donald E. Knuth, Tomasz Luczak, Boris Pittel, ``Birth
of the Giant Component'', Random Structures and Algorithms, volume 4,
1993, pages 233-358.
\newline
\url{https://arxiv.org/abs/math/9310236}
\newline
\url{https://onlinelibrary.wiley.com/doi/abs/10.1002/rsa.3240040303}

\begin{center}
\begin{tikzpicture}
    [scale=2,
     my box/.style={draw,minimum size=1.3em,inner sep=.15em,outer sep=.1em},
    ]
  \node (0) at (0,0) [my box] {0};
  \node (1) at (1,0) [my box] {1};
  \node (0-1) at (2,0) [my box] {0,1};
  \node (2) at (2,-1) [my box] {2};

  \node (0-0-1) at (3,0) [my box] {0,0,1};
  \node (1-1) at (3,-1) [my box] {1,1};
  \node (3) at (3,-2) [my box] {3};

  \draw [->] (2) to (1-1);
  \draw [->] (1) to (2);
  \draw [->] (2) to (3);
  \draw [->] (2) to (0-0-1);
  \draw [->] (0-1) to (0-0-1);
  \draw [->] (0) to (1);
  \draw [->] (1) to (0-1);
  \draw [->] (0-1) to (1-1);

\end{tikzpicture}
\end{center}

\subsection{N=4}
\url{https://hog.grinvin.org/ViewGraphInfo.action?id=33587}

Janson, Knuth, Luczak, and Pittel, consider the "configuration" of a
multigraph.  Configuration is how many components have excess
r=1,2,3,etc, where excess r = number of edges - number of vertices.
Adding an edge to a multigraph can go to a new configuration.

The present graph is their figure 1 (and 3), being configurations of
total excess 0 to 4 inclusive.  Configurations are partitions of the
total excess, so there are 12 vertices (OEIS A000070).  0, 1 or 2
terms are taken from a configuration to form a new one, so 17 edges
(OEIS A029859).  The start, being no components r>=1, is the degree-1
with degree-3 neighbour.

Svante Janson, Donald E. Knuth, Tomasz Luczak, Boris Pittel, ``Birth
of the Giant Component'', Random Structures and Algorithms, volume 4,
1993, pages 233-358.
\newline
\url{https://arxiv.org/abs/math/9310236}
\newline
\url{https://onlinelibrary.wiley.com/doi/abs/10.1002/rsa.3240040303}

\begin{center}
\begin{tikzpicture}
    [scale=1.5,
     my box/.style={draw,minimum size=1.3em,inner sep=.15em,outer sep=.1em},
    ]
  \node at (.5, 1.2) {per Janson et al};

  \node (0) at (0,0) [my box] {0};
  \node (1) at (1,0) [my box] {1};

  \node (0-1) at (2, .5) [my box] {0,1};
  \node (2) at (2, -.5) [my box] {2};

  \node (0-0-1) at (3,1) [my box] {0,0,1};
  \node (1-1) at (3,0) [my box] {1,1};
  \node (3) at (3,-1) [my box] {3};

  \node (0-0-0-1) at (4, 1.5) [my box] {0,0,0,1};
  \node (1-0-1) at (4,.5) [my box] {1,0,1};
  \node (0-2) at (4,0) [my box] {0,2};
  \node (2-1) at (4,-.5) [my box] {2,1};
  \node (4) at (4, -1.5) [my box] {4};

  \draw [->] (1-1) to (0-2);
  \draw [->] (2) to (1-1);
  \draw [->] (3) to (2-1);
  \draw [->] (0-1) to (1-1);
  \draw [->] (3) to (1-0-1);
  \draw [->] (1-1) to (1-0-1);
  \draw [->] (1) to (0-1);
  \draw [->] (0-1) to (0-0-1);
  \draw [->] (1) to (2);
  \draw [->] (0-0-1) to (1-0-1);
  \draw [->] (1-1) to (0-0-0-1);
  \draw [->] (2) to (3);
  \draw [->] (0) to (1);
  \draw [->] (1-1) to (2-1);
  \draw [->] (3) to (4);
  \draw [->] (2) to (0-0-1);
  \draw [->] (0-0-1) to (0-0-0-1);

\end{tikzpicture}
\end{center}

\vspace{1\baselineskip}

\begin{center}
\begin{tikzpicture}
    [scale=1.5,
     my box/.style={draw,minimum size=1.3em,inner sep=.15em,outer sep=.1em},
    ]
  \node at (.5,-2) {sheared downwards};

  \node (0) at (0,0) [my box] {0};
  \node (1) at (1,0) [my box] {1};

  \node (0-1) at (2, 0) [my box] {0,1};
  \node (2) at (2, -1) [my box] {2};

  \node (0-0-1) at (3,0) [my box] {0,0,1};
  \node (1-1) at (3,-1) [my box] {1,1};
  \node (3) at (3,-2) [my box] {3};

  \node (0-0-0-1) at (4, 0) [my box] {0,0,0,1};
  \node (1-0-1) at (4, -1) [my box] {1,0,1};
  \node (0-2) at (4, -2) [my box] {0,2};
  \node (2-1) at (4, -3) [my box] {2,1};
  \node (4) at (4, -4) [my box] {4};

  \draw [->] (1-1) to (0-2);
  \draw [->] (2) to (1-1);
  \draw [->] (3) to (2-1);
  \draw [->] (0-1) to (1-1);
  \draw [->] (3) to (1-0-1);
  \draw [->] (1-1) to (1-0-1);
  \draw [->] (1) to (0-1);
  \draw [->] (0-1) to (0-0-1);
  \draw [->] (1) to (2);
  \draw [->] (0-0-1) to (1-0-1);
  \draw [->] (1-1) to (0-0-0-1);
  \draw [->] (2) to (3);
  \draw [->] (0) to (1);
  \draw [->] (1-1) to (2-1);
  \draw [->] (3) to (4);
  \draw [->] (2) to (0-0-1);
  \draw [->] (0-0-1) to (0-0-0-1);

\end{tikzpicture}
\end{center}


%------------------------------------------------------------------------------
\section{Bulgarian Solitaire}

\subsection{N=7}
\url{https://hog.grinvin.org/ViewGraphInfo.action?id=32380}

\begin{center}
\begin{tikzpicture}
    [scale=2,yscale=.8,
     my box/.style={draw,minimum size=1.3em,inner sep=.15em,outer sep=.1em},
    ]

  \node at (2.5, 3.5) {N=7};

  \node (1-1-1-1-1-1-1) at (1,0) [my box] {1,1,1,1,1,1,1};
  \node (1-1-1-1-1-2) at (2,1) [my box] {1,1,1,1,1,2};
  \node (1-1-1-1-3) at (2,2) [my box] {1,1,1,1,3};
  \node (1-1-1-2-2) at (-1.5,1) [my box] {1,1,1,2,2};
  \node (1-1-1-4) at (-.5, 2) [my box] {1,1,1,4};
  \node (1-1-2-3) at (0,3.5) [my box] {1,1,2,3};
  \node (1-1-5) at (-1.5, 2) [my box] {1,1,5};
  \node (1-2-2-2) at (-.5, 1) [my box] {1,2,2,2};
  \node (1-2-4) at (1,4) [my box] {1,2,4};
  \node (1-3-3) at (0,4.5) [my box] {1,3,3};
  \node (1-6) at (1,2) [my box] {1,6};
  \node (2-2-3) at (-1,4) [my box] {2,2,3};
  \node (2-5) at (1,3) [my box] {2,5};
  \node (3-4) at (-1,3) [my box] {3,4};
  \node (7) at (1,1) [my box] {7};

  \draw [->] (1-2-4) to (1-3-3);
  \draw [->] (1-1-1-1-1-2) to (1-6);
  \draw [->] (7) to (1-6);
  \draw [->] (1-1-5) to (3-4);
  \draw [->] (1-3-3) to (2-2-3);
  \draw [->] (1-6) to (2-5);
  \draw [->] (1-1-1-1-1-1-1) to (7);
  \draw [->] (2-5) to (1-2-4);
  \draw [->] (1-2-2-2) to (1-1-1-4);
  \draw [->] (1-1-1-1-3) to (2-5);
  \draw [->] (1-1-2-3) to (1-2-4);
  \draw [->] (3-4) to (2-2-3);
  \draw [->] (2-2-3) to (1-1-2-3);
  \draw [->] (1-1-1-2-2) to (1-1-5);
  \draw [->] (1-1-1-4) to (3-4);

\end{tikzpicture}
\end{center}


%------------------------------------------------------------------------------
\section{Tamari Lattice, Binary Tree Rotation Graph}

% GP-DEFINE  Cat(n) = binomial(n<<1,n) / (n+1);
% Cat(4) == 14
% Cat(5) == 42
% Cat(6) == 132
% Cat(7) == 429


\subsection{N=3}

\begin{center}
\begin{tikzpicture}
    [scale=2,
     my box/.style={draw,circle,minimum size=1.3em,inner sep=.1em,outer sep=.1em},
    ]

  \node at (5,1) {N=3};

  \coordinate (vec0) at (0:1);
  \coordinate (vec1) at (0:1);
  \coordinate (vec2) at (45:1);

  \node (111) at ($1*(vec0)+1*(vec1)+1*(vec2)$) [my box] {1,1,1};
  \node (112) at ($1*(vec0)+1*(vec1)+2*(vec2)$) [my box] {1,1,2};
  \node (113) at ($1*(vec0)+1*(vec1)+3*(vec2)$) [my box] {1,1,3};
  \node (121) at ($1*(vec0)+2*(vec1)+1*(vec2)$) [my box] {1,2,1};
  \node (123) at ($1*(vec0)+2*(vec1)+3*(vec2)$) [my box] {1,2,3};

  \draw [->] (112) to (113);
  \draw [->] (113) to (123);
  \draw [->] (121) to (123);
  \draw [->] (111) to (121);
  \draw [->] (111) to (112);

\end{tikzpicture}
\end{center}

\subsection{N=4}

\url{https://hog.grinvin.org/ViewGraphInfo.action?id=33547}

The Tamari lattice by Dov Tamari has graph vertices as
parenthesizations of N+1 objects into pairs, and graph edges between
those differing by one application of the associative law.  Hence also
called an associahedron.  Here N=4 and there are Catalan(4) = 14
vertices.

Equivalently, binary tree rotation graph.  Each vertex represents a
binary tree of N=4 vertices and graph edges are between trees
differing by one "rotation".  Each tree edge can rotate, so degree N-1
regular (21 edges, OEIS A002054).

\begin{center}
\begin{tikzpicture}
    [scale=1,
     my box/.style={fill,circle,inner sep=.15em,outer sep=.15em,node contents=},
    ]

  \node at (7.5, 2.5) [right] {N=4};

  \begin{scope}[z={(0,0)},
                rotate around x=-37,
                rotate around y=0,
                rotate around z=-35,
                xscale=2,
               ]

  \node (1111) at (1,1,1) [my box];

  \node (1121) at (1,2,1) [my box];
  \node (1112) at (1,1,2) [my box];
  \node (1211) at (2,1,1) [my box];

  \node (1113) at (1,1,3) [my box];

  \node (1131) at (1,3,1) [my box];
  \node (1123) at (1,2,3) [my box];
  \node (1114) at (1,1,4) [my box];
  \node (1212) at (2,1,2) [my box];

  \node (1124) at (1,2,4) [my box];

  \node (1134) at (1,3,4) [my box];
  \node (1231) at (2,3,1) [my box];
  \node (1214) at (2,1,4) [my box];

  \node (1234) at (2,3,4) [my box];

  \end{scope}


  \draw [] (1111) to (1211);
  \draw [] (1131) to (1231);
  \draw [] (1113) to (1114);
  \draw [] (1134) to (1234);
  \draw [] (1123) to (1124);
  \draw [] (1111) to (1112);
  \draw [] (1112) to (1212);
  \draw [] (1231) to (1234);
  \draw [] (1214) to (1234);
  \draw [] (1131) to (1134);
  \draw [] (1211) to (1212);
  \draw [] (1111) to (1121);
  \draw [] (1113) to (1123);
  \draw [] (1211) to (1231);
  \draw [] (1112) to (1113);
  \draw [] (1114) to (1214);
  \draw [] (1121) to (1131);
  \draw [] (1114) to (1124);
  \draw [] (1121) to (1123);
  \draw [] (1212) to (1214);
  \draw [] (1124) to (1134);

\end{tikzpicture}
\end{center}

cf Winfried Geyer, ''On Tamari Lattices'', Discrete Mathematics,
volume 133, 1994, pages 99-122, figure 6(b), who draws in a similar
fashion (by intervals \dots).


\begin{center}
\begin{tikzpicture}
    [scale=2,
     my box/.style={draw,circle,minimum size=1.3em,inner sep=.1em,outer sep=.1em},
    ]

  \coordinate (vec0) at (0:1);
  \coordinate (vec1) at (-25:2.5);
  \coordinate (vec2) at (25:1.8);
  \coordinate (vec3) at (90:1);

  % \coordinate (vec0) at (0:1);
  % \coordinate (vec1) at (10:1);
  % \coordinate (vec2) at (-20:1.5);
  % \coordinate (vec3) at (90:1);

  \node (1111) at ($1*(vec0)+1*(vec1)+1*(vec2)+1*(vec3)$) [my box] {1111};
  \node (1112) at ($1*(vec0)+1*(vec1)+1*(vec2)+2*(vec3)$) [my box] {1112};
  \node (1113) at ($1*(vec0)+1*(vec1)+1*(vec2)+3*(vec3)$) [my box] {1113};
  \node (1114) at ($1*(vec0)+1*(vec1)+1*(vec2)+4*(vec3)$) [my box] {1114};
  \node (1121) at ($1*(vec0)+1*(vec1)+2*(vec2)+1*(vec3)$) [my box] {1121};
  \node (1123) at ($1*(vec0)+1*(vec1)+2*(vec2)+3*(vec3)$) [my box] {1123};
  \node (1124) at ($1*(vec0)+1*(vec1)+2*(vec2)+4*(vec3)$) [my box] {1124};
  \node (1131) at ($1*(vec0)+1*(vec1)+3*(vec2)+1*(vec3)$) [my box] {1131};
  \node (1134) at ($1*(vec0)+1*(vec1)+3*(vec2)+4*(vec3)$) [my box] {1134};
  \node (1211) at ($1*(vec0)+2*(vec1)+1*(vec2)+1*(vec3)$) [my box] {1211};
  \node (1212) at ($1*(vec0)+2*(vec1)+1*(vec2)+2*(vec3)$) [my box] {1212};
  \node (1214) at ($1*(vec0)+2*(vec1)+1*(vec2)+4*(vec3)$) [my box] {1214};
  \node (1231) at ($1*(vec0)+2*(vec1)+3*(vec2)+1*(vec3)$) [my box] {1231};
  \node (1234) at ($1*(vec0)+2*(vec1)+3*(vec2)+4*(vec3)$) [my box] {1234};

  \draw [->] (1211) to (1231);
  \draw [->] (1131) to (1134);
  \draw [->] (1134) to (1234);
  \draw [->] (1114) to (1124);
  \draw [->] (1113) to (1123);
  \draw [->] (1112) to (1212);
  \draw [->] (1113) to (1114);
  \draw [->] (1114) to (1214);
  \draw [->] (1121) to (1131);
  \draw [->] (1111) to (1121);
  \draw [->] (1111) to (1211);
  \draw [->] (1231) to (1234);
  \draw [->] (1212) to (1214);
  \draw [->] (1123) to (1124);
  \draw [->] (1214) to (1234);
  \draw [->] (1131) to (1231);
  \draw [->] (1111) to (1112);
  \draw [->] (1112) to (1113);
  \draw [->] (1211) to (1212);
  \draw [->] (1124) to (1134);
  \draw [->] (1121) to (1123);

  \node at (1111.south) [below right=4ex] {N=4};

\end{tikzpicture}
\end{center}

\begin{center}
\begin{tikzpicture}
    [scale=2.5,
     my box/.style={draw,circle,minimum size=1.3em,inner sep=.1em,outer sep=.1em},
    ]

  \begin{scope}[z={(0,0)},
                rotate around x=-37,
                rotate around y=0,
                rotate around z=-35,
                xscale=2,
               ]

  \node (1111) at (1,1,1) [my box] {1111};

  \node (1121) at (1,2,1) [my box] {1121};
  \node (1112) at (1,1,2) [my box] {1112};
  \node (1211) at (2,1,1) [my box] {1211};

  \node (1113) at (1,1,3) [my box] {1113};

  \node (1131) at (1,3,1) [my box] {1131};
  \node (1123) at (1,2,3) [my box] {1123};
  \node (1114) at (1,1,4) [my box] {1114};
  \node (1212) at (2,1,2) [my box] {1212};

  \node (1124) at (1,2,4) [my box] {1124};

  \node (1134) at (1,3,4) [my box] {1134};
  \node (1231) at (2,3,1) [my box] {1231};
  \node (1214) at (2,1,4) [my box] {1214};

  \node (1234) at (2,3,4) [my box] {1234};

  \end{scope}


  \draw [->] (1111) to (1211);
  \draw [->] (1131) to (1231);
  \draw [->] (1113) to (1114);
  \draw [->] (1134) to (1234);
  \draw [->] (1123) to (1124);
  \draw [->] (1111) to (1112);
  \draw [->] (1112) to (1212);
  \draw [->] (1231) to (1234);
  \draw [->] (1214) to (1234);
  \draw [->] (1131) to (1134);
  \draw [->] (1211) to (1212);
  \draw [->] (1111) to (1121);
  \draw [->] (1113) to (1123);
  \draw [->] (1211) to (1231);
  \draw [->] (1112) to (1113);
  \draw [->] (1114) to (1214);
  \draw [->] (1121) to (1131);
  \draw [->] (1114) to (1124);
  \draw [->] (1121) to (1123);
  \draw [->] (1212) to (1214);
  \draw [->] (1124) to (1134);

\end{tikzpicture}
\end{center}

\begin{center}
\begin{tikzpicture}
    [scale=2.5,
     my box/.style={draw,circle,minimum size=1.3em,inner sep=.1em,outer sep=.2em},
    ]

  \node (1111) at (0,4) [my box] {1111};

  \node (1121) at (-1,3) [my box] {1121};
  \node (1112) at (0,3) [my box] {1112};
  \node (1211) at (2,3) [my box] {1211};

  \node (1113) at (0,2) [my box] {1113};

  \node (1131) at (-1.5, 1) [my box] {1131};
  \node (1123) at (-.5, 1) [my box] {1123};
  \node (1114) at (.5, 1) [my box] {1114};
  \node (1212) at (2,1) [my box] {1212};

  \node (1124) at (0,0) [my box] {1124};

  \node (1134) at (-1,-1) [my box] {1134};
  \node (1231) at (0, -1) [my box] {1231};
  \node (1214) at (1,-1) [my box] {1214};

  \node (1234) at (0,-2) [my box] {1234};

  \node at (1111) [right=5em] {N=4};

  \draw [->] (1111) to (1211);
  \draw [->] (1131) to (1231);
  \draw [->] (1113) to (1114);
  \draw [->] (1134) to (1234);
  \draw [->] (1123) to (1124);
  \draw [->] (1111) to (1112);
  \draw [->] (1112) to (1212);
  \draw [->] (1231) to (1234);
  \draw [->] (1214) to (1234);
  \draw [->] (1131) to (1134);
  \draw [->] (1211) to (1212);
  \draw [->] (1111) to (1121);
  \draw [->] (1113) to (1123);
  \draw [->] (1211) to (1231);
  \draw [->] (1112) to (1113);
  \draw [->] (1114) to (1214);
  \draw [->] (1121) to (1131);
  \draw [->] (1114) to (1124);
  \draw [->] (1121) to (1123);
  \draw [->] (1212) to (1214);
  \draw [->] (1124) to (1134);

\end{tikzpicture}
\end{center}

\begin{center}
\begin{tikzpicture}
    [scale=1.8,
     my box/.style={draw,circle,minimum size=1.3em,inner sep=.1em,outer sep=.2em},
    ]

  \node (1111) at (-1,4) [my box] {1111};
  \node (1211) at (0,4) [my box] {1211};
  \node (1212) at (1.5,3) [my box] {1212};
  \node (1112) at (.5,3) [my box] {1112};

  \node (1121) at (-2, 1.5) [my box] {1121};


  \node (1113) at (0,2) [my box] {1113};

  \node (1123) at (-1, 1) [my box] {1123};
  \node (1114) at (1, 1) [my box] {1114};

  \node (1124) at (0,0) [my box] {1124};

  \node (1134) at (0,-1) [my box] {1134};
  \node (1214) at (2,1) [my box] {1214};

  \node (1131) at (-1.5, -1.5) [my box] {1131};
  \node (1231) at (-.5, -2) [my box] {1231};
  \node (1234) at (1,-1.5) [my box] {1234};


  \node at (1212) [right=2em] {N=4 Associahedron};

  \draw [->] (1111) to (1211);
  \draw [->] (1131) to (1231);
  \draw [->] (1113) to (1114);
  \draw [->] (1134) to (1234);
  \draw [->] (1123) to (1124);
  \draw [->] (1111) to (1112);
  \draw [->] (1112) to (1212);
  \draw [->] (1231) to (1234);
  \draw [->] (1214) to (1234);
  \draw [->] (1131) to (1134);
  \draw [->] (1211) to (1212);
  \draw [->] (1111) to (1121);
  \draw [->] (1113) to (1123);
  \draw [->] (1211) to (1231);
  \draw [->] (1112) to (1113);
  \draw [->] (1114) to (1214);
  \draw [->] (1121) to (1131);
  \draw [->] (1114) to (1124);
  \draw [->] (1121) to (1123);
  \draw [->] (1212) to (1214);
  \draw [->] (1124) to (1134);

\end{tikzpicture}
\end{center}

\subsection{N=5}

N=5
\url{https://hog.grinvin.org/ViewGraphInfo.action?id=33549}

N=6
\url{https://hog.grinvin.org/ViewGraphInfo.action?id=33551}

cf Geyer (above) again, figure 10(b).

\noindent
\hspace{-0em}
\begin{tikzpicture}
    [scale=1,
     my box/.style={fill,circle,inner sep=.15em,outer sep=.1em,node contents =},
    ]

  \coordinate (vec0) at (0:1);
  \coordinate (vec1) at (-20:2);
  \coordinate (vec2) at (25:1);
  \coordinate (vec3) at (90:1.6);
  \coordinate (vec4) at (7:2);

  % N=4 angles
  \coordinate (vec0) at (0:1);
  \coordinate (vec1) at (-25:2.5);
  \coordinate (vec2) at (25:1.8);
  \coordinate (vec3) at (90:1.7);
  \coordinate (vec4) at (70:3);

  \node (11111) at ($1*(vec0)+1*(vec1)+1*(vec2)+1*(vec3)+1*(vec4)$) [my box];
  \node (11112) at ($1*(vec0)+1*(vec1)+1*(vec2)+1*(vec3)+2*(vec4)$) [my box];
  \node (11113) at ($1*(vec0)+1*(vec1)+1*(vec2)+1*(vec3)+3*(vec4)$) [my box];
  \node (11114) at ($1*(vec0)+1*(vec1)+1*(vec2)+1*(vec3)+4*(vec4)$) [my box];
  \node (11115) at ($1*(vec0)+1*(vec1)+1*(vec2)+1*(vec3)+5*(vec4)$) [my box];
  \node (11121) at ($1*(vec0)+1*(vec1)+1*(vec2)+2*(vec3)+1*(vec4)$) [my box];
  \node (11123) at ($1*(vec0)+1*(vec1)+1*(vec2)+2*(vec3)+3*(vec4)$) [my box];
  \node (11124) at ($1*(vec0)+1*(vec1)+1*(vec2)+2*(vec3)+4*(vec4)$) [my box];
  \node (11125) at ($1*(vec0)+1*(vec1)+1*(vec2)+2*(vec3)+5*(vec4)$) [my box];
  \node (11131) at ($1*(vec0)+1*(vec1)+1*(vec2)+3*(vec3)+1*(vec4)$) [my box];
  \node (11134) at ($1*(vec0)+1*(vec1)+1*(vec2)+3*(vec3)+4*(vec4)$) [my box];
  \node (11135) at ($1*(vec0)+1*(vec1)+1*(vec2)+3*(vec3)+5*(vec4)$) [my box];
  \node (11141) at ($1*(vec0)+1*(vec1)+1*(vec2)+4*(vec3)+1*(vec4)$) [my box];
  \node (11145) at ($1*(vec0)+1*(vec1)+1*(vec2)+4*(vec3)+5*(vec4)$) [my box];
  \node (11211) at ($1*(vec0)+1*(vec1)+2*(vec2)+1*(vec3)+1*(vec4)$) [my box];
  \node (11212) at ($1*(vec0)+1*(vec1)+2*(vec2)+1*(vec3)+2*(vec4)$) [my box];
  \node (11214) at ($1*(vec0)+1*(vec1)+2*(vec2)+1*(vec3)+4*(vec4)$) [my box];
  \node (11215) at ($1*(vec0)+1*(vec1)+2*(vec2)+1*(vec3)+5*(vec4)$) [my box];
  \node (11231) at ($1*(vec0)+1*(vec1)+2*(vec2)+3*(vec3)+1*(vec4)$) [my box];
  \node (11234) at ($1*(vec0)+1*(vec1)+2*(vec2)+3*(vec3)+4*(vec4)$) [my box];
  \node (11235) at ($1*(vec0)+1*(vec1)+2*(vec2)+3*(vec3)+5*(vec4)$) [my box];
  \node (11241) at ($1*(vec0)+1*(vec1)+2*(vec2)+4*(vec3)+1*(vec4)$) [my box];
  \node (11245) at ($1*(vec0)+1*(vec1)+2*(vec2)+4*(vec3)+5*(vec4)$) [my box];
  \node (11311) at ($1*(vec0)+1*(vec1)+3*(vec2)+1*(vec3)+1*(vec4)$) [my box];
  \node (11312) at ($1*(vec0)+1*(vec1)+3*(vec2)+1*(vec3)+2*(vec4)$) [my box];
  \node (11315) at ($1*(vec0)+1*(vec1)+3*(vec2)+1*(vec3)+5*(vec4)$) [my box];
  \node (11341) at ($1*(vec0)+1*(vec1)+3*(vec2)+4*(vec3)+1*(vec4)$) [my box];
  \node (11345) at ($1*(vec0)+1*(vec1)+3*(vec2)+4*(vec3)+5*(vec4)$) [my box];
  \node (12111) at ($1*(vec0)+2*(vec1)+1*(vec2)+1*(vec3)+1*(vec4)$) [my box];
  \node (12112) at ($1*(vec0)+2*(vec1)+1*(vec2)+1*(vec3)+2*(vec4)$) [my box];
  \node (12113) at ($1*(vec0)+2*(vec1)+1*(vec2)+1*(vec3)+3*(vec4)$) [my box];
  \node (12115) at ($1*(vec0)+2*(vec1)+1*(vec2)+1*(vec3)+5*(vec4)$) [my box];
  \node (12121) at ($1*(vec0)+2*(vec1)+1*(vec2)+2*(vec3)+1*(vec4)$) [my box];
  \node (12123) at ($1*(vec0)+2*(vec1)+1*(vec2)+2*(vec3)+3*(vec4)$) [my box];
  \node (12125) at ($1*(vec0)+2*(vec1)+1*(vec2)+2*(vec3)+5*(vec4)$) [my box];
  \node (12141) at ($1*(vec0)+2*(vec1)+1*(vec2)+4*(vec3)+1*(vec4)$) [my box];
  \node (12145) at ($1*(vec0)+2*(vec1)+1*(vec2)+4*(vec3)+5*(vec4)$) [my box];
  \node (12311) at ($1*(vec0)+2*(vec1)+3*(vec2)+1*(vec3)+1*(vec4)$) [my box];
  \node (12312) at ($1*(vec0)+2*(vec1)+3*(vec2)+1*(vec3)+2*(vec4)$) [my box];
  \node (12315) at ($1*(vec0)+2*(vec1)+3*(vec2)+1*(vec3)+5*(vec4)$) [my box];
  \node (12341) at ($1*(vec0)+2*(vec1)+3*(vec2)+4*(vec3)+1*(vec4)$) [my box];
  \node (12345) at ($1*(vec0)+2*(vec1)+3*(vec2)+4*(vec3)+5*(vec4)$) [my box];

  \draw [->] (11215) to (11315);
  \draw [->] (12121) to (12123);
  \draw [->] (11145) to (11245);
  \draw [->] (11111) to (12111);
  \draw [->] (11215) to (11235);
  \draw [->] (11245) to (11345);
  \draw [->] (11211) to (11311);
  \draw [->] (11121) to (11123);
  \draw [->] (12311) to (12312);
  \draw [->] (11214) to (11215);
  \draw [->] (11134) to (11135);
  \draw [->] (11125) to (12125);
  \draw [->] (11115) to (12115);
  \draw [->] (11235) to (11245);
  \draw [->] (11315) to (11345);
  \draw [->] (12311) to (12341);
  \draw [->] (12112) to (12312);
  \draw [->] (12312) to (12315);
  \draw [->] (11115) to (11215);
  \draw [->] (11312) to (11315);
  \draw [->] (12113) to (12115);
  \draw [->] (11234) to (11235);
  \draw [->] (11125) to (11135);
  \draw [->] (11341) to (12341);
  \draw [->] (12115) to (12315);
  \draw [->] (11111) to (11112);
  \draw [->] (11345) to (12345);
  \draw [->] (12113) to (12123);
  \draw [->] (11113) to (12113);
  \draw [->] (11121) to (12121);
  \draw [->] (11123) to (11124);
  \draw [->] (11241) to (11341);
  \draw [->] (12145) to (12345);
  \draw [->] (12111) to (12121);
  \draw [->] (11141) to (11241);
  \draw [->] (11131) to (11134);
  \draw [->] (12141) to (12145);
  \draw [->] (11212) to (11214);
  \draw [->] (11135) to (11145);
  \draw [->] (12112) to (12113);
  \draw [->] (11231) to (11241);
  \draw [->] (11311) to (11341);
  \draw [->] (11211) to (11231);
  \draw [->] (11131) to (11231);
  \draw [->] (11315) to (12315);
  \draw [->] (11214) to (11234);
  \draw [->] (11114) to (11214);
  \draw [->] (11131) to (11141);
  \draw [->] (11111) to (11211);
  \draw [->] (11124) to (11134);
  \draw [->] (11113) to (11123);
  \draw [->] (11112) to (12112);
  \draw [->] (12315) to (12345);
  \draw [->] (11241) to (11245);
  \draw [->] (11341) to (11345);
  \draw [->] (11115) to (11125);
  \draw [->] (12111) to (12112);
  \draw [->] (11141) to (12141);
  \draw [->] (11211) to (11212);
  \draw [->] (12121) to (12141);
  \draw [->] (11121) to (11131);
  \draw [->] (12341) to (12345);
  \draw [->] (11124) to (11125);
  \draw [->] (11112) to (11113);
  \draw [->] (12141) to (12341);
  \draw [->] (11112) to (11212);
  \draw [->] (11113) to (11114);
  \draw [->] (11145) to (12145);
  \draw [->] (11111) to (11121);
  \draw [->] (11231) to (11234);
  \draw [->] (11311) to (12311);
  \draw [->] (11114) to (11115);
  \draw [->] (12115) to (12125);
  \draw [->] (11212) to (11312);
  \draw [->] (11135) to (11235);
  \draw [->] (12125) to (12145);
  \draw [->] (11311) to (11312);
  \draw [->] (11123) to (12123);
  \draw [->] (11134) to (11234);
  \draw [->] (12123) to (12125);
  \draw [->] (11141) to (11145);
  \draw [->] (11114) to (11124);
  \draw [->] (12111) to (12311);
  \draw [->] (11312) to (12312);

\end{tikzpicture}



%------------------------------------------------------------------------------
\section{Kreweras Lattice N=4}

\url{https://hog.grinvin.org/ViewGraphInfo.action?id=33557}

Each graph vertex is one of the Catalan(4)=14 partitions of the
integers 1..4 into non-crossing sets.  Such a partition corresponds to
an ordered rooted forest (its sets of siblings), and hence to a binary
tree too.

Each graph edge is where one set in the partition splits into two to
reach another partition (and hence is "graded" by number of sets in
the partition).

G. Kreweras, ``Sur les Partitions Non-Crois\'ees d'Un Cycle'', Discrete
Mathematics, volume 1, number 4, 1971, pages 333-350.  The present
graph is figure 1.

\begin{center}
\begin{tikzpicture}
    [scale=2,yscale=1.5,
     my box/.style={draw,circle,inner sep=.1em,outer sep=.0em},
    ]
  \node (0000) at (-.5, 0) [my box] {0000};

  \node (0111) at (-2,1) [my box] {0111};
  \node (0020) at (-3,1) [my box] {0020};
  \node (0110) at (-1,1) [my box] {0110};
  \node (0100) at (0,1) [my box] {0100};
  \node (0003) at (1,1) [my box] {0003};
  \node (0022) at (2,1) [my box] {0022};

  \node (0113) at (-3,2) [my box] {0113};
  \node (0122) at (-2,2) [my box] {0122};
  \node (0120) at (-1,2) [my box] {0120};
  \node (0103) at (0,2) [my box] {0103};
  \node (0023) at (1,2) [my box] {0023};
  \node (0121) at (2,2) [my box] {0121};

  \node (0123) at (-.5, 3) [my box] {0123};

  \draw [] (0000) to (0100);
  \draw [] (0000) to (0020);
  \draw [] (0000) to (0022);
  \draw [] (0000) to (0003);
  \draw [] (0000) to (0110);
  \draw [] (0000) to (0111);

  \draw [] (0003) to (0103);
  \draw [] (0110) to (0122);
  \draw [] (0100) to (0121);
  \draw [] (0121) to (0022);
  \draw [] (0120) to (0110);
  \draw [] (0113) to (0123);
  \draw [] (0100) to (0103);
  \draw [] (0003) to (0023);
  \draw [] (0113) to (0111);
  \draw [] (0123) to (0122);
  \draw [] (0003) to (0122);
  \draw [] (0121) to (0111);
  \draw [] (0123) to (0103);
  \draw [] (0022) to (0023);
  \draw [] (0020) to (0023);
  \draw [] (0111) to (0122);
  \draw [] (0121) to (0123);
  \draw [] (0120) to (0123);
  \draw [] (0100) to (0120);
  \draw [] (0123) to (0023);
  \draw [] (0120) to (0020);
  \draw [] (0113) to (0020);

\end{tikzpicture}
\end{center}

cf. Knuth fasc 4a figure 39 draws with the degree-3s in the middle,
and ``dual'' elements symmetric around the centre.


N=5
\url{https://hog.grinvin.org/ViewGraphInfo.action?id=33559}

N=6
\url{https://hog.grinvin.org/ViewGraphInfo.action?id=33561}


%------------------------------------------------------------------------------
\section{Dudeney Towns Hamiltonian}

\url{https://hog.grinvin.org/ViewGraphInfo.action?id=32239}

Dudeney gives this graph in a puzzle.  Vertices are towns and edges
are roads between them.  The traveller is to visit each town once
only, starting and ending at the same place, so a Hamiltonian cycle.
Per Dudeney's solution, the degree-2 vertices force a unique
Hamiltonian cycle.

\begin{center}
\begin{tikzpicture}
    [scale=3,
     my box/.style={draw,circle,minimum size=1.3em,inner sep=.1em,outer sep=.0em},
    ]

  \begin{scope}[rotate=6.5*360/16]
  \node (1) at (0:1) [my box] {1};
  \node (9) at (1*360/16:1) [my box] {9};
  \node (5) at (2*360/16:1) [my box] {5};
  \node (14) at (3*360/16:1) [my box] {14};
  \node (8) at (4*360/16:1) [my box] {8};
  \node (4) at (5*360/16:1) [my box] {4};
  \node (15) at (6*360/16:1) [my box] {15};
  \node (6) at (7*360/16:1) [my box] {6};
  \node (10) at (8*360/16:1) [my box] {10};
  \node (2) at (9*360/16:1) [my box] {2};
  \node (13) at (10*360/16:1) [my box] {13};
  \node (7) at (11*360/16:1) [my box] {7};
  \node (3) at (12*360/16:1) [my box] {3};
  \node (11) at (13*360/16:1) [my box] {11};
  \node (16) at (14*360/16:1) [my box] {16};
  \node (12) at (15*360/16:1) [my box] {12};
  \end{scope}

  \draw [] (4) to (8);
  \draw [] (13) to (7);
  \draw [] (5) to (14);
  \draw [] (4) to (15);
  \draw [] (15) to (6);
  \draw [] (12) to (3);
  \draw [] (15) to (9);
  \draw [] (3) to (14);
  \draw [] (2) to (13);
  \draw [] (2) to (10);
  \draw [] (13) to (11);
  \draw [] (12) to (16);
  \draw [] (8) to (14);
  \draw [] (3) to (11);
  \draw [] (6) to (10);
  \draw [] (1) to (9);
  \draw [] (3) to (7);
  \draw [] (16) to (9);
  \draw [] (16) to (11);
  \draw [] (5) to (9);
  \draw [] (1) to (12);
  \draw [] (4) to (13);
  \draw [] (12) to (10);

\end{tikzpicture}
\end{center}

\begin{center}
\begin{tikzpicture}
    [scale=1.5,xscale=1,
     my box/.style={draw,circle,minimum size=1.3em,inner sep=.1em,outer sep=.3em},
    ]

  \node (1) at (-2,3) [my box] {1};
  \node (2) at (2,-1) [my box] {2};
  \node (3) at (0,1) [my box] {3};
  \node (4) at (-1,-1) [my box] {4};
  \node (5) at (-3,1) [my box] {5};
  \node (6) at (3, 1) [my box] {6};
  \node (7) at (0,-.5) [my box] {7};
  \node (8) at (-2,-1) [my box] {8};
  \node (9) at (-1,1) [my box] {9};
  \node (10) at (3,3) [my box] {10};
  \node (11) at (1,2) [my box] {11};
  \node (12) at (0,3) [my box] {12};
  \node (13) at (1,-1) [my box] {13};
  \node (14) at (-3,0) [my box] {14};
  \node (15) at (-1,0) [my box] {15};
  \node (16) at (-1,2) [my box] {16};

  \draw [] (4) to (8);
  \draw [] (13) to (7);
  \draw [] (5) to (14);
  \draw [] (4) to (15);
  \draw [] (15) to (6);
  \draw [] (12) to (3);
  \draw [] (15) to (9);
  \draw [] (3) to (14);
  \draw [] (2) to (13);
  \draw [] (2) to (10);
  \draw [] (13) to (11);
  \draw [] (12) to (16);
  \draw [] (8) to (14);
  \draw [] (3) to (11);
  \draw [] (6) to (10);
  \draw [] (1) to (9);
  \draw [] (3) to (7);
  \draw [] (16) to (9);
  \draw [] (16) to (11);
  \draw [] (5) to (9);
  \draw [] (1) to (12);
  \draw [] (4) to (13);
  \draw [] (12) to (10);

\end{tikzpicture}
\end{center}



%------------------------------------------------------------------------------
\section{Dudeney 248 Cyclists}

Cyclists starting at ``star'' ending at E, Hamiltonian path (visit
each vertex once).  Answer: N O W A Y I M S U R E.

\medskip

\url{https://hog.grinvin.org/ViewGraphInfo.action?id=32237}

Dudeney gives this graph in a puzzle.  Vertices are cities and edges
are roads between them.  The puzzle is to find a path for cyclists
starting top left and ending top right, in the drawing here, and
visiting other all cities exactly once, so a Hamiltonian path.  The
puzzle is drawn winding around to make it more difficult than what is
in fact a partial grid.  The solution is unique.

Henry Ernest Dudeney, ``Amusements in Mathematics'', 1917, puzzle 248
``The Cyclists' Tour'', page 71.\newline
\url{http://www.gutenberg.org/ebooks/16713}\newline
\url{https://www.gutenberg.org/files/16713/16713-h/images/q248.png}

Reproduced in Martin Gardner, ``The Travelling Salesman'', Discover
magazine, April 1985.  And that article reprinted in Martin Gardner,
``Gardner's Whys and Wherefores'', Prometheus Books, 1999, ISBN
1-57392-744-9, chapter 12, page 90.

\begin{center}
\begin{tikzpicture}
    [scale=1.5,xscale=1,
     my box/.style={draw,circle,minimum size=1.3em,inner sep=.1em,outer sep=.3em},
    ]

  \node (A) at (3,0) [my box] {A};
  \node (E) at (5,1) [my box] {E};
  \node (I) at (4,1) [my box] {I};
  \node (M) at (4,0) [my box] {M};
  \node (N) at (2,0) [my box] {N};
  \node (O) at (2,-1) [my box] {O};
  \node (R) at (5,0) [my box] {R};
  \node (S) at (4,-1) [my box] {S};
  \node (U) at (5,-1) [my box] {U};
  \node (W) at (3,-1) [my box] {W};
  \node (Y) at (3,1) [my box] {Y};
  \node (star) at (2,1) [my box] {star};

  \draw [] (O) to (W);
  \draw [] (W) to (A);
  \draw [] (R) to (E);
  \draw [] (N) to (star);
  \draw [] (A) to (Y);
  \draw [] (S) to (U);
  \draw [] (A) to (M);
  \draw [] (O) to (N);
  \draw [] (E) to (I);
  \draw [] (I) to (M);
  \draw [] (star) to (Y);
  \draw [] (R) to (M);
  \draw [] (Y) to (I);
  \draw [] (N) to (A);
  \draw [] (M) to (S);
  \draw [] (R) to (U);

\end{tikzpicture}
\end{center}


%------------------------------------------------------------------------------
\section{Pi Graph Digits}

\url{https://hog.grinvin.org/ViewGraphInfo.action?id=32206}

\smallskip

This is the simple graph of a multigraph form which Knuth gives as an
example of building a graph by adding edges.  Each vertex is a decimal
digit, 0 through 9.  Edges are between digits in pi = 3.14159265\dots.
First edge 3-1, next edge 4-1, next 5-9, etc.

The present graph is after 17 steps (which is 15 distinct edges) and
which is where the graph first becomes fully connected.  The degree-1
is digit 0.

Donald Knuth, MSRI lecture "The Birth of the Giant Component", 1 Oct 2004.
\url{http://archive.org/details/lecture12091/}

\smallskip

\begin{center}
\begin{tikzpicture}
    [scale=1,
     my box/.style={draw,circle,minimum size=1.3em,inner sep=.1em,outer sep=.3em},
    ]

  \node (0) at (1,2) [my box] {0};
  \node (1) at (0,0) [my box] {1};
  \node (2) at (1,1) [my box] {2};
  \node (3) at (1,0) [my box] {3};
  \node (4) at (-1,0) [my box] {4};
  \node (5) at (2,-1) [my box] {5};
  \node (6) at (-1,1) [my box] {6};
  \node (7) at (3,1) [my box] {7};
  \node (8) at (0,-1) [my box] {8};
  \node (9) at (3,0) [my box] {9};

  \draw [] (7) to (9);
  \draw [] (4) to (6);
  \draw [] (3) to (9);
  \draw [] (4) to (8);
  \draw [] (3) to (5);
  \draw [,loop below] (3) to ();
  \draw [] (0) to (2);
  \draw [] (5) to (8);
  \draw [] (2) to (3);
  \draw [] (2) to (6);
  \draw [] (5) to (9);
  \draw [] (1) to (3);
  \draw [] (3) to (8);
  \draw [] (2) to (7);
  \draw [] (1) to (4);

\end{tikzpicture}
\end{center}

\begin{center}
\begin{tikzpicture}
    [scale=1.5,xscale=1.1,
     my box/.style={draw,circle,minimum size=1.3em,inner sep=.1em,outer sep=.3em},
    ]

  \node (0) at (1,2) [my box] {0};
  \node (1) at (.2,0) [my box] {1};
  \node (2) at (1,1) [my box] {2};
  \node (3) at (1,0) [my box] {3};
  \node (4) at (0,-1) [my box] {4};
  \node (5) at (2,-1) [my box] {5};
  \node (6) at (0,1) [my box] {6};
  \node (7) at (2,1) [my box] {7};
  \node (8) at (1,-1) [my box] {8};
  \node (9) at (2,0) [my box] {9};

  \draw [] (7) to (9);
  \draw [] (4) to (6);
  \draw [] (3) to (9);
  \draw [] (4) to (8);
  \draw [] (3) to (5);
  \draw [,loop below] (3) to ();
  \draw [] (0) to (2);
  \draw [] (5) to (8);
  \draw [] (2) to (3);
  \draw [] (2) to (6);
  \draw [] (5) to (9);
  \draw [] (1) to (3);
  \draw [] (3) to (8);
  \draw [] (2) to (7);
  \draw [] (1) to (4);

\end{tikzpicture}
\end{center}



%------------------------------------------------------------------------------
\section{Unlabelled Vpar Differences}

%---------------------
\subsection{Lex-Max Vpar Forests Differences N=4}


Automorphisms: \newline
swap 2,4 hanging triangle \newline
swap pairs 6,0 and 7,8

\url{https://hog.grinvin.org/ViewGraphInfo.action?id=32274}

Each graph vertex is an unlabelled rooted forest of n=4 vertices (9 of
them) represented by a labelled rooted forest in vertex parent array
form (vpar).  Labelling is chosen to give each the lexicographically
greatest vpar array.  Graph edges are between arrays differing in one
position.

The 3 degree-1 vertices are vpar=[0,0,0,0] singletons, vpar=[4,3,1,0]
path-4, and vpar=[4,4,0,3] star-4.  The latter has the degree-6
neighbour.  The clique-4 is vpar=[4,X,0,0] with X=0,1,3,4.


\begin{center}
\begin{tikzpicture}
    [scale=1.5,
     my box/.style={draw,circle,minimum size=1.3em,inner sep=.1em,outer sep=.0em},
    ]

  \node (0) at (3,1) [my box] {0};
  \node (1) at (0,0) [my box] {1};
  \node (2) at (0.5, -1) [my box] {2};
  \node (3) at (1,1) [my box] {3};
  \node (4) at (1.5, -1) [my box] {4};
  \node (5) at (1,0) [my box] {5};
  \node (6) at (2,1) [my box] {6};
  \node (7) at (2,0) [my box] {7};
  \node (8) at (3,0) [my box] {8};

  \draw [] (2) to (4);
  \draw [] (5) to (6);
  \draw [] (7) to (8);
  \draw [] (4) to (5);
  \draw [] (3) to (6);
  \draw [] (2) to (5);
  \draw [] (6) to (7);
  \draw [] (1) to (5);
  \draw [] (3) to (7);
  \draw [] (5) to (7);
  \draw [] (3) to (5);
  \draw [] (0) to (6);

\end{tikzpicture}
\end{center}

%---------------------
\subsection{Lex-Max Vpar Forests Differences N=5}

Automorphism swap 9,13 hanging triangle only.

\url{https://hog.grinvin.org/ViewGraphInfo.action?id=32276}

Each graph vertex is an unlabelled rooted forest of n=5 vertices (20
of them) represented by a labelled rooted forest in vertex parent
array form (vpar).  Labelling is chosen to give each the
lexicographically greatest vpar array.  Graph edges are between arrays
differing in one position.

The 2 degree-1 vertices are vpar=[0,0,0,0,0] singletons and
vpar=[5,5,4,2,0] path-5.  The latter has the degree-6 neighbour.

\begin{center}
\begin{tikzpicture}
    [scale=1.5,
     my box/.style={draw,circle,minimum size=1.3em,inner sep=.1em,outer sep=.0em},
    ]

  \node (0) at (1,  2) [my box] {0};
  \node (1) at (-1.6, -2) [my box] {1};
  \node (2) at (1.6, -2) [my box] {2};
  \node (3) at (-1.6, -1) [my box] {3};
  \node (4) at (0, 2) [my box] {4};
  \node (5) at (1.6, -3) [my box] {5};
  \node (6) at (-.6, -3) [my box] {6};
  \node (7) at (-.6, -2) [my box] {7};
  \node (8) at (2, 1) [my box] {8};
  \node (9) at (2.6, -2) [my box] {9};
  \node (10) at (1, 0) [my box] {10};
  \node (11) at (1, 1) [my box] {11};
  \node (12) at (-1, 1) [my box] {12};
  \node (13) at (2.6, -1) [my box] {13};
  \node (14) at (1, -1) [my box] {14};
  \node (15) at (0, -1) [my box] {15};
  \node (16) at (0, 0) [my box] {16};
  \node (17) at (0,1) [my box] {17};
  \node (18) at (-1,0) [my box] {18};
  \node (19) at (-2,0) [my box] {19};

  \draw [] (12) to (18);
  \draw [] (0) to (8);
  \draw [] (14) to (15);
  \draw [] (10) to (16);
  \draw [] (5) to (6);
  \draw [] (17) to (18);
  \draw [] (13) to (14);
  \draw [] (16) to (18);
  \draw [] (15) to (16);
  \draw [] (2) to (6);
  \draw [] (2) to (7);
  \draw [] (14) to (5);
  \draw [] (16) to (7);
  \draw [] (10) to (2);
  \draw [] (12) to (16);
  \draw [] (16) to (17);
  \draw [] (11) to (17);
  \draw [] (2) to (5);
  \draw [] (15) to (3);
  \draw [] (0) to (4);
  \draw [] (15) to (6);
  \draw [] (14) to (9);
  \draw [] (10) to (15);
  \draw [] (10) to (8);
  \draw [] (17) to (4);
  \draw [] (12) to (17);
  \draw [] (10) to (11);
  \draw [] (5) to (7);
  \draw [] (10) to (14);
  \draw [] (6) to (7);
  \draw [] (14) to (16);
  \draw [] (13) to (9);
  \draw [] (1) to (15);
  \draw [] (0) to (11);
  \draw [] (18) to (19);
  \draw [] (1) to (6);

\end{tikzpicture}
\end{center}


%---------------------
\subsection{Lex-Min Vpar Forests Differences N=4}

\url{https://hog.grinvin.org/ViewGraphInfo.action?id=32190}

\smallskip

Each graph vertex is an unlabelled rooted forest of n=4 vertices (9 of
them) represented by a labelled rooted forest in vertex parent array
form (vpar).  Labelling is chosen to give each the lexicographically
smallest vpar array.  Graph edges are between arrays differing in one
position.

The two degree-1s are vpar=[0,0,0,0] singletons and vpar=[0,1,2,3] path-4.

\begin{center}
\begin{tikzpicture}
    [scale=1.2,
     my box/.style={draw,circle,minimum size=1.3em,inner sep=.1em,outer sep=.0em},
    ]

  \node (0) at (0,0) [my box] {0};
  \node (1) at (1,0) [my box] {1};
  \node (2) at (2,0) [my box] {2};
  \node (3) at (3.5,-1) [my box] {3};
  \node (4) at (3,1) [my box] {4};
  \node (5) at (4,0) [my box] {5};
  \node (6) at (3,0) [my box] {6};
  \node (7) at (5,0) [my box] {7};
  \node (8) at (6,0) [my box] {8};

  \draw [] (3) to (6);
  \draw [] (3) to (5);
  \draw [] (2) to (4);
  \draw [] (4) to (5);
  \draw [] (5) to (7);
  \draw [] (7) to (8);
  \draw [] (5) to (6);
  \draw [] (0) to (1);
  \draw [] (2) to (6);
  \draw [] (1) to (2);

\end{tikzpicture}
\end{center}


%---------------------
\subsection{Lex-Min Vpar Forests Differences N=5}

\url{https://hog.grinvin.org/ViewGraphInfo.action?id=32188}

\smallskip

Each graph vertex is an unlabelled rooted forest of n=5 vertices (20
of them) represented by a labelled rooted forest in vertex parent
array form (vpar).  Labelling is chosen to give each the
lexicographically smallest vpar array.  Graph edges are between arrays
differing in one position.

The 3 degree-1s are vpar=[0,0,0,0,0] singletons, vpar=[0,1,2,3,4]
path-5, and vpar=[0,0,1,3,4] path-4 and singleton.  The latter has the
degree-3 neighbour.


\begin{center}
\begin{tikzpicture}
    [scale=1.2,
     my box/.style={draw,circle,minimum size=1.3em,inner sep=.1em,outer sep=.0em},
    ]

  \node (0) at (-2, 0) [my box] {0};
  \node (1) at (-1, 0) [my box] {1};
  \node (2) at (0, 0) [my box] {2};
  \node (3) at (.5, .75) [my box] {3};
  \node (4) at (2.5, -2.25) [my box] {4};
  \node (5) at (0,  1.5) [my box] {5};
  \node (6) at (1,  1.5) [my box] {6};
  \node (7) at (2.5, -1.25) [my box] {7};
  \node (8) at (1, 0) [my box] {8};
  \node (9) at (2.5, 1.5) [my box] {9};
  \node (10) at (3, 0) [my box] {10};
  \node (11) at (2, 0) [my box] {11};
  \node (12) at (4.5, .75) [my box] {12};
  \node (13) at (3, 1) [my box] {13};
  \node (14) at (4, 0) [my box] {14};
  \node (15) at (4, 1.5) [my box] {15};
  \node (16) at (5, 0) [my box] {16};
  \node (17) at (5, 1.5) [my box] {17};
  \node (18) at (6, 0) [my box] {18};
  \node (19) at (7, 0) [my box] {19};

  \draw [] (5) to (6);
  \draw [] (16) to (17);
  \draw [] (3) to (8);
  \draw [] (10) to (15);
  \draw [] (10) to (7);
  \draw [] (10) to (11);
  \draw [] (13) to (9);
  \draw [] (2) to (8);
  \draw [] (16) to (18);
  \draw [] (0) to (1);
  \draw [] (2) to (5);
  \draw [] (11) to (8);
  \draw [] (14) to (15);
  \draw [] (4) to (7);
  \draw [] (1) to (2);
  \draw [] (14) to (16);
  \draw [] (13) to (14);
  \draw [] (10) to (14);
  \draw [] (12) to (17);
  \draw [] (3) to (6);
  \draw [] (11) to (7);
  \draw [] (12) to (16);
  \draw [] (18) to (19);
  \draw [] (15) to (17);
  \draw [] (15) to (9);
  \draw [] (6) to (8);
  \draw [] (6) to (9);

\end{tikzpicture}
\end{center}

\needspace{4\baselineskip}

%---------------------
\subsection{Pre-Order Lex-Max Vpar Forests Differences N=4}

\url{https://hog.grinvin.org/ViewGraphInfo.action?id=32194}

\smallskip

Each graph vertex is an unlabelled rooted forest of n=4 vertices (9 of
them) represented by a labelled rooted forest in vertex parent array
form (vpar).  Labelling is pre-order and lexicographically biggest, in
the manner of Beyer and Hedetniemi.  Each graph edge is between vpar
arrays differing in one position.

The degree-1 vertex is vpar=[0,0,0,0] all singletons.  Its sole one
different is vpar=[0,1,0,0].

The graph has a Hamiltonian path starting at the degree-1 and ending
anywhere except its neighbour or one other degree-4.  These
Hamiltonian paths are a kind of Gray code traversal of the vpar arrays
changing one entry each time.

The clique-4 is vpar=[0,1,2,X].  The last vertex parent X can be any 0
to 3 (so v=4 is a new root or a child of any preceding).  These differ
from each other in just that last entry.

\begin{center}
\begin{tikzpicture}
    [scale=1.4,xscale=1,
     my box/.style={draw,circle,minimum size=1.3em,inner sep=.1em,outer sep=.3em},
    ]

  \node at (3,1.5) [right=1em] {degree-4s shown thick};

  \node (0) at (0,1) [thick,my box] {0};
  \node (1) at (0,2) [my box] {1};
  \node (2) at (1,2) [thick,my box] {2};
  \node (3) at (1,1) [my box] {3};
  \node (4) at (2,2) [my box] {4};
  \node (5) at (2,1) [my box] {5};
  \node (6) at (0,0) [my box] {6};
  \node (7) at (1,0) [thick,my box] {7};
  \node (8) at (2,0) [my box] {8};

  \draw [] (2) to (3);
  \draw [] (6) to (7);
  \draw [] (0) to (3);
  \draw [] (7) to (8);
  \draw [] (3) to (7);
  \draw [] (3) to (5);
  \draw [] (0) to (1);
  \draw [] (1) to (3);
  \draw [] (1) to (2);
  \draw [] (2) to (4);
  \draw [] (0) to (6);
  \draw [] (5) to (7);
  \draw [] (4) to (5);
  \draw [] (0) to (2);

\end{tikzpicture}
\end{center}

%---------------------
\subsection{Pre-Order Lex-Max Vpar Forests Differences N=5}

\url{https://hog.grinvin.org/ViewGraphInfo.action?id=32192}

\smallskip

Each graph vertex is an unlabelled rooted forest of n=5 vertices (20
of them) represented by a labelled rooted forest in vertex parent
array form (vpar).  Labelling is pre-order and lexicographically
biggest, in the manner of Beyer and Hedetniemi.  Each graph edge is
between vpar arrays differing in one position.

The degree-1 vertex is vpar=[0,0,0,0,0] all singletons.  Its sole one
different is vpar=[0,1,0,0,0].

The graph has a Hamiltonian path starting at the degree-1 and ending
anywhere except its neighbour or the degree-8.  These Hamiltonian
paths are a kind of Gray code traversal of the vpar arrays changing
one entry each time.

The clique-5 is vpar=[0,1,2,3,X].  The last vertex parent X can be any
0 to 4 (so v=5 is a new root or a child of any preceding).  These
differ from each other in just that last entry.

\begin{center}
\begin{tikzpicture}
    [scale=1.5,xscale=1,
     my box/.style={draw,circle,minimum size=1.3em,inner sep=.1em,outer sep=.3em},
    ]

  \node (0) at (0, 5.5) [my box] {0};
  \node (1) at (-1,6.2) [my box] {1};
  \node (2) at (-2, 5.5) [my box] {2};
  \node (3) at (-2, 4) [my box] {3};
  \node (4) at (0,4) [my box] {4};
  \node (5) at (-3, 3.1) [my box] {5};
  \node (6) at (-2,2) [my box] {6};
  \node (7) at (-1, 3.1) [my box] {7};
  \node (8) at (1, 3.1) [my box] {8};
  \node (9) at (-1, 2) [my box] {9};
  \node (10) at (0, 2) [my box] {10};
  \node (11) at (2, 3.1) [my box] {11};
  \node (12) at (1, 2) [my box] {12};
  \node (13) at (-1,1) [my box] {13};
  \node (14) at (0,1) [my box] {14};
  \node (15) at (2, 2) [my box] {15};
  \node (16) at (2, 1) [my box] {16};
  \node (17) at (2.4, 4) [my box] {17};
  \node (18) at (3, 3.1) [my box] {18};
  \node (19) at (4, 3.1) [my box] {19};

  \draw [] (6) to (7);
  \draw [] (14) to (16);
  \draw [] (12) to (18);
  \draw [] (2) to (4);
  \draw [] (1) to (2);
  \draw [] (10) to (8);
  \draw [] (13) to (9);
  \draw [] (15) to (16);
  \draw [] (18) to (19);
  \draw [] (11) to (15);
  \draw [] (3) to (6);
  \draw [] (12) to (4);
  \draw [] (6) to (9);
  \draw [] (12) to (7);
  \draw [] (17) to (4);
  \draw [] (0) to (11);
  \draw [] (4) to (7);
  \draw [] (2) to (5);
  \draw [] (3) to (4);
  \draw [] (11) to (8);
  \draw [] (3) to (9);
  \draw [] (0) to (4);
  \draw [] (12) to (16);
  \draw [] (13) to (14);
  \draw [] (2) to (3);
  \draw [] (17) to (18);
  \draw [] (11) to (12);
  \draw [] (10) to (7);
  \draw [] (5) to (7);
  \draw [] (0) to (1);
  \draw [] (0) to (8);
  \draw [] (10) to (14);
  \draw [] (0) to (3);
  \draw [] (10) to (9);
  \draw [] (1) to (3);
  \draw [] (0) to (2);
  \draw [] (10) to (4);
  \draw [] (8) to (9);
  \draw [] (16) to (18);
  \draw [] (5) to (6);
  \draw [] (1) to (4);
  \draw [] (10) to (12);

\end{tikzpicture}
\end{center}


%------------------------------------------------------------------------------
\section{Vpar Differences}

\subsection{N=3 Trees}

Each vertex is a labelled rooted tree of 3 vertices, in the form of a
vertex parent array.  Each graph edge is between arrays differing in
one place.

\begin{center}
\begin{tikzpicture}
    [scale=1,
     my box/.style={draw,circle,minimum size=1.3em,inner sep=.1em,outer sep=.3em},
    ]

  \node (0) at (1,0) [my box] {0};
  \node (1) at (0,1) [my box] {1};
  \node (2) at (0,2) [my box] {2};
  \node (3) at (2,0) [my box] {3};
  \node (4) at (1,1) [my box] {4};
  \node (5) at (2,2) [my box] {5};
  \node (6) at (0,0) [my box] {6};
  \node (7) at (2,1) [my box] {7};
  \node (8) at (1,2) [my box] {8};

  \draw [] (5) to (8);
  \draw [] (2) to (8);
  \draw [] (4) to (7);
  \draw [] (0) to (3);
  \draw [] (1) to (4);
  \draw [] (0) to (6);

\end{tikzpicture}
\end{center}

\subsection{Vpar Forests Differences N=3}

\url{https://hog.grinvin.org/ViewGraphInfo.action?id=32175}

Each graph vertex represents a labelled rooted forest of 3 vertices
($4^2 {=} 16$ of them) in the form of a vertex parent array (vpar).
Each graph edge is between arrays differing in one entry.

The degree=6 vertex is vpar=[0,0,0] forest of all singletons.
Degree=5 vertices (six of) have two singletons.  Degree=4 (three of)
are tree root having 2 children.  Degree=3 (six of) are path-3 down
from root.

The graph is Hamiltonian.  Such a path or cycle through the vertices
is an iteration through the vpar arrays changing just one entry each
time.

\begin{center}
\begin{tikzpicture}
    [scale=4,
     my box/.style={draw,circle,minimum size=1.3em,inner sep=.1em,outer sep=.3em},
     rotate=90-12,
    ]

  \node (0) at (0,0) [my box] {0};  % degree=6

  \node (12) at (0*24:1) [my box] {12};
  \node (8) at (1*24:1) [my box] {8};

  \node (11) at (2*24:1) [my box] {11};
  \node (15) at (3*24:1) [my box] {15};
  \node (7) at (4*24:1) [my box] {7};

  \node (4) at (5*24:1) [my box] {4};
  \node (3) at (6*24:1) [my box] {3};

  \node (13) at (7*24:1) [my box] {13};
  \node (5) at (8*24:1) [my box] {5};
  \node (9) at (9*24:1) [my box] {9};

  \node (2) at (10*24:1) [my box] {2};
  \node (1) at (11*24:1) [my box] {1};

  \node (6) at (12*24:1) [my box] {6};
  \node (10) at (13*24:1) [my box] {10};
  \node (14) at (14*24:1) [my box] {14};

  \draw [] (15) to (7);
  \draw [] (15) to (3);
  \draw [] (1) to (5);
  \draw [] (0) to (2);
  \draw [] (1) to (13);
  \draw [] (10) to (8);
  \draw [] (12) to (15);
  \draw [] (0) to (8);
  \draw [] (2) to (9);
  \draw [] (5) to (9);
  \draw [] (3) to (4);
  \draw [] (0) to (1);
  \draw [] (10) to (14);
  \draw [] (4) to (7);
  \draw [] (11) to (3);
  \draw [] (0) to (4);
  \draw [] (0) to (12);
  \draw [] (10) to (6);
  \draw [] (1) to (6);
  \draw [] (4) to (5);
  \draw [] (12) to (7);
  \draw [] (1) to (2);
  \draw [] (13) to (3);
  \draw [] (12) to (8);
  \draw [] (0) to (3);
  \draw [] (4) to (9);
  \draw [] (12) to (14);
  \draw [] (11) to (8);
  \draw [] (6) to (8);
  \draw [] (10) to (2);
  \draw [] (13) to (5);
  \draw [] (14) to (2);
  \draw [] (11) to (15);

\end{tikzpicture}
\end{center}

\begin{center}
\begin{tikzpicture}
    [scale=2,
     my box/.style={draw,circle,minimum size=1.3em,inner sep=.1em,outer sep=.3em},
    ]

  \node (0) at (0,0) [my box] {0};  % degree=6
  \node (8) at (0:1) [my box] {8};
  \node (12) at (60:1) [my box] {12};
  \node (3) at (2*60:1) [my box] {3};
  \node (4) at (3*60:1) [my box] {4};
  \node (1) at (4*60:1) [my box] {1};
  \node (2) at (5*60:1) [my box] {2};

  \node (6) at (5*60:2) [my box] {6};
  \node (14) at (0*60:2) [my box] {14};

  \node (7) at (2*60:2) [my box] {7};
  \node (11) at (1*60:2) [my box] {11};

  \node (9) at (4*60:2) [my box] {9};
  \node (13) at (3*60:2) [my box] {13};

  \node (10) at ($(6.center)!.5!(14.center)$) [my box] {10};
  \node (5) at ($(9.center)!.5!(13.center)$) [my box] {5};
  \node (15) at ($(7.center)!.5!(11.center)$) [my box] {15};

  \draw [] (15) to (7);
  \draw [] (15) to (3);
  \draw [] (1) to (5);
  \draw [] (0) to (2);
  \draw [] (1) to (13);
  \draw [] (10) to (8);
  \draw [] (12) to (15);
  \draw [] (0) to (8);
  \draw [] (2) to (9);
  \draw [] (5) to (9);
  \draw [] (3) to (4);
  \draw [] (0) to (1);
  \draw [] (10) to (14);
  \draw [] (4) to (7);
  \draw [] (11) to (3);
  \draw [] (0) to (4);
  \draw [] (0) to (12);
  \draw [] (10) to (6);
  \draw [] (1) to (6);
  \draw [] (4) to (5);
  \draw [] (12) to (7);
  \draw [] (1) to (2);
  \draw [] (13) to (3);
  \draw [] (12) to (8);
  \draw [] (0) to (3);
  \draw [] (4) to (9);
  \draw [] (12) to (14);
  \draw [] (11) to (8);
  \draw [] (6) to (8);
  \draw [] (10) to (2);
  \draw [] (13) to (5);
  \draw [] (14) to (2);
  \draw [] (11) to (15);

\end{tikzpicture}
\end{center}

\begin{center}
\begin{tikzpicture}
    [scale=4,
     my box/.style={draw,minimum size=1.3em,inner ysep=.3em,outer sep=.3em},
     font=\normalsize,
     rotate=-2.25*24,
    ]

  \node (0) at (0,0) [my box] {000};  % degree=6



  \node (7) at (5*24:1) [my box] {310};
  \node (12) at (4*24:1) [my box] {300};
  \node (14) at (3*24:1) [my box] {302};
  \node (10) at (2*24:1) [my box] {202};
  \node (6) at (1*24:1) [my box] {201};
  \node (8) at (0*24:1) [my box] {200};
  \node (11) at (14*24:1) [my box] {230};
  \node (15) at (13*24:1) [my box] {330};
  \node (1) at (12*24:1) [my box] {001};

  \node (5) at (11*24:1) [my box] {011};
  \node (3) at (8*24:1) [my box] {030};
  \node (13) at (10*24:1) [my box] {031};
  \node (2) at (9*24:1) [my box] {002};

  \node (4) at (6*24:1) [my box] {010};

  \node (9) at (7*24:1) [my box] {012};



  \draw [] (15) to (7);
  \draw [] (15) to (3);
  \draw [] (1) to (5);
  \draw [] (0) to (2);
  \draw [] (1) to (13);
  \draw [] (10) to (8);
  \draw [] (12) to (15);
  \draw [] (0) to (8);
  \draw [] (2) to (9);
  \draw [] (5) to (9);
  \draw [] (3) to (4);
  \draw [] (0) to (1);
  \draw [] (10) to (14);
  \draw [] (4) to (7);
  \draw [] (11) to (3);
  \draw [] (0) to (4);
  \draw [] (0) to (12);
  \draw [] (10) to (6);
  \draw [] (1) to (6);
  \draw [] (4) to (5);
  \draw [] (12) to (7);
  \draw [] (1) to (2);
  \draw [] (13) to (3);
  \draw [] (12) to (8);
  \draw [] (0) to (3);
  \draw [] (4) to (9);
  \draw [] (12) to (14);
  \draw [] (11) to (8);
  \draw [] (6) to (8);
  \draw [] (10) to (2);
  \draw [] (13) to (5);
  \draw [] (14) to (2);
  \draw [] (11) to (15);

\end{tikzpicture}
\end{center}


%------------------------------------------------------------------------------
\section{Ordered Forests Depths Vector Differences N=4}

\url{https://hog.grinvin.org/ViewGraphInfo.action?id=32179}

Each graph vertex represents an ordered rooted forest of N=4 vertices
(Catalan number C(4)=14 of them) in the form of depths vector (or
``level sequence'') resulting from pre-order traversal.  Edges are
between depths vectors differing in one entry.

The graph is Hamiltonian.  Such path or cycle through the vertices is
a traversal of depths vectors changing one entry each time.

\medskip

Automorphism
(2 3) (5 6) (7 8) (10 11) (12 13)

\begin{center}
\begin{tikzpicture}
    [scale=1.5,
     my box/.style={draw,circle,minimum size=1.3em,inner sep=.1em,outer sep=.3em},
     rotate=0,
    ]

  \node (0) at (.1, .5) [my box] {0};
  \node (1) at (-.6, .5) [my box] {1};
  \node (4) at (-1.5, .5) [my box] {4};
  \node (9) at (-2.5, .5) [my box] {9};

  \node (2) at (.5, 1) [my box] {2};
  \node (3) at (.5, 0) [my box] {3};

  \node (5) at (-1,1) [my box] {5};
  \node (6) at (-1,0) [my box] {6};

  \node (7) at (-2, 2) [my box] {7};
  \node (8) at (-2, -1) [my box] {8};

  \node (10) at (-3, 1) [my box] {10};
  \node (11) at (-3, 0) [my box] {11};

  \node (12) at (-4.5, 1) [my box] {12};
  \node (13) at (-4.5, 0) [my box] {13};

  \draw [] (5) to (7);
  \draw [] (11) to (13);
  \draw [] (10) to (12);
  \draw [] (4) to (9);
  \draw [] (2) to (5);
  \draw [] (3) to (6);
  \draw [] (10) to (11);
  \draw [] (0) to (1);
  \draw [] (12) to (13);
  \draw [] (1) to (3);
  \draw [] (7) to (8);
  \draw [] (2) to (7);
  \draw [] (10) to (5);
  \draw [] (6) to (8);
  \draw [] (1) to (2);
  \draw [] (11) to (9);
  \draw [] (13) to (8);
  \draw [] (1) to (4);
  \draw [] (0) to (2);
  \draw [] (10) to (9);
  \draw [] (3) to (8);
  \draw [] (2) to (3);
  \draw [] (4) to (6);
  \draw [] (12) to (7);
  \draw [] (0) to (3);
  \draw [] (5) to (6);
  \draw [] (4) to (5);
  \draw [] (11) to (6);

\end{tikzpicture}
\end{center}

\bigskip

\begin{center}
\begin{tikzpicture}
    [scale=1.5,
     my box/.style={draw,circle,minimum size=1.3em,inner sep=.1em,outer sep=.3em},
     rotate=0,
    ]

  \node (0) at (.1, .5) [my box] {0};
  \node (1) at (-.6, .5) [my box] {1};
  \node (4) at (-2, .5) [my box] {4};
  \node (9) at (-3.25, .5) [my box] {9};

  \node (2) at (.5, 1) [my box] {2};
  \node (3) at (.5, 0) [my box] {3};

  \node (5) at (-1.5, 2) [my box] {5};
  \node (6) at (-1.5, -1) [my box] {6};

  \node (7) at (-1,1) [my box] {7};
  \node (8) at (-1,0) [my box] {8};

  \node (10) at (-4, 1) [my box] {10};
  \node (11) at (-4, 0) [my box] {11};

  \node (12) at (-2.5, 1) [my box] {12};
  \node (13) at (-2.5, 0) [my box] {13};

  \draw [] (5) to (7);
  \draw [] (11) to (13);
  \draw [] (10) to (12);
  \draw [] (4) to (9);
  \draw [] (2) to (5);
  \draw [] (3) to (6);
  \draw [] (10) to (11);
  \draw [] (0) to (1);
  \draw [] (12) to (13);
  \draw [] (1) to (3);
  \draw [] (7) to (8);
  \draw [] (2) to (7);
  \draw [] (10) to (5);
  \draw [] (6) to (8);
  \draw [] (1) to (2);
  \draw [] (11) to (9);
  \draw [] (13) to (8);
  \draw [] (1) to (4);
  \draw [] (0) to (2);
  \draw [] (10) to (9);
  \draw [] (3) to (8);
  \draw [] (2) to (3);
  \draw [] (4) to (6);
  \draw [] (12) to (7);
  \draw [] (0) to (3);
  \draw [] (5) to (6);
  \draw [] (4) to (5);
  \draw [] (11) to (6);

\end{tikzpicture}
\end{center}

\bigskip

\begin{center}
\begin{tikzpicture}
    [scale=1.7,
     my box/.style={draw,circle,minimum size=1.2em,inner sep=.1em,outer sep=.3em},
     rotate=0,
    ]

  \node (2) at (0,1) [my box] {2};
  \node (3) at (0,0) [my box] {3};
  \node (6) at (-.5, 2.5) [my box] {6};

  \node (0) at (1,0) [my box] {0};
  \node (1) at (1,1) [my box] {1};
  \node (5) at (-.5, 1.5) [my box] {5};

  \node (4) at (1,2) [my box] {4};
  \node (10) at (-2,2) [my box] {10};

  \node (7) at (-1,1) [my box] {7};
  \node (8) at (-1,0) [my box] {8};
  \node (9) at (-.5, 4.5) [my box] {9};
  \node (11) at (-.5, 3.5) [my box] {11};
  \node (12) at (-2,1) [my box] {12};
  \node (13) at (-2,0) [my box] {13};

  \draw [] (5) to (7);
  \draw [] (11) to (13);
  \draw [] (10) to (12);
  \draw [] (4) to (9);
  \draw [] (2) to (5);
  \draw [] (3) to (6);
  \draw [] (10) to (11);
  \draw [] (0) to (1);
  \draw [] (12) to (13);
  \draw [] (1) to (3);
  \draw [] (7) to (8);
  \draw [] (2) to (7);
  \draw [] (10) to (5);
  \draw [] (6) to (8);
  \draw [] (1) to (2);
  \draw [] (11) to (9);
  \draw [] (13) to (8);
  \draw [] (1) to (4);
  \draw [] (0) to (2);
  \draw [] (10) to (9);
  \draw [] (3) to (8);
  \draw [] (2) to (3);
  \draw [] (4) to (6);
  \draw [] (12) to (7);
  \draw [] (0) to (3);
  \draw [] (5) to (6);
  \draw [] (4) to (5);
  \draw [] (11) to (6);

\end{tikzpicture}
\end{center}

\subsection{Difference 1}

Hamiltonian ends 0001 0010 0100 0111 0012 0120

\begin{center}
\begin{tikzpicture}
    [scale=1.7,
     my box/.style={draw,circle,minimum size=1.2em,inner sep=.1em,outer sep=.3em},
     rotate=0,
    ]

  \node (0000) at (0,0) [my box] {0000};
  \node (0001) at (1,1) [my box] {0001};
  \node (0010) at (1,0) [my box] {0010};
  \node (0100) at (1,-1) [my box] {0100};
  \node (0011) at (2,1) [my box] {0011};
  \node (0101) at (2,0) [my box] {0101};
  \node (0110) at (2,-1) [my box] {0110};
  \node (0012) at (3,1) [my box] {0012};
  \node (0111) at (3,0) [my box] {0111};
  \node (0120) at (3,-1) [my box] {0120};
  \node (0112) at (4,1) [my box] {0112};
  \node (0121) at (4,-1) [my box] {0121};
  \node (0122) at (5,0) [my box] {0122};
  \node (0123) at (6,0) [my box] {0123};

  % Hamiltonian ends 0001 0010 0100 0111 0012 0120
  \node at (0001.north) [above,inner sep=0pt] {*};
  \node at (0010.north) [above,inner sep=0pt] {*};
  \node at (0100.north) [above,inner sep=0pt] {*};
  \node at (0111.north) [above,inner sep=0pt] {*};
  \node at (0012.north) [above,inner sep=0pt] {*};
  \node at (0120.north) [above,inner sep=0pt] {*};

  \draw [->] (0000) to (0010);
  \draw [->] (0121) to (0122);
  \draw [->] (0110) to (0111);
  \draw [->] (0111) to (0121);
  \draw [->] (0001) to (0101);
  \draw [->] (0111) to (0112);
  \draw [->] (0122) to (0123);
  \draw [->] (0010) to (0011);
  \draw [->] (0011) to (0012);
  \draw [->] (0100) to (0101);
  \draw [->] (0110) to (0120);
  \draw [->] (0000) to (0001);
  \draw [->] (0011) to (0111);
  \draw [->] (0001) to (0011);
  \draw [->] (0010) to (0110);
  \draw [->] (0120) to (0121);
  \draw [->] (0101) to (0111);
  \draw [->] (0112) to (0122);
  \draw [->] (0000) to (0100);
  \draw [->] (0100) to (0110);
  \draw [->] (0012) to (0112);

\end{tikzpicture}
\end{center}

\pagebreak

N=5

\begin{center}
\begin{tikzpicture}
    [scale=1.3,
     my box/.style={draw,minimum size=1.2em,inner sep=.1em,outer sep=.2em},
     rotate=90,
    ]

  \node (00000) at (-1,0) [my box] {00000};

  \node (00001) at (0,2.5) [my box] {00001};
  \node (00010) at (0,1) [my box] {00010};
  \node (00100) at (0,-1) [my box] {00100};
  \node (01000) at (0,-2.5) [my box] {01000};

  \node (00011) at (2-.3, 3) [my box] {00011};
  \node (01010) at (2-.3, -1.5) [my box] {01010};
  \node (00110) at (2-.1, 0) [my box] {00110};
  \node (01001) at (2-.7, 0) [my box] {01001};
  \node (00101) at (2-.3, 1.5) [my box] {00101};
  \node (01100) at (2-.3, -3) [my box] {01100};

  \node (00111) at (3,3) [my box] {00111};
  \node (00012) at (3,2) [my box] {00012};
  \node (01011) at (3,1) [my box] {01011};
  \node (00120) at (3,0) [my box] {00120};
  \node (01101) at (3,-1) [my box] {01101};
  \node (01200) at (3,-2) [my box] {01200};
  \node (01110) at (3,-3) [my box] {01110};

  \node (00112) at (4,3) [my box] {00112};
  \node (00121) at (4,2) [my box] {00121};
  \node (01012) at (4,1) [my box] {01012};
  \node (01111) at (4,0) [my box] {01111};
  \node (01201) at (4,-1) [my box] {01201};
  \node (01120) at (4,-2) [my box] {01120};
  \node (01210) at (4,-3) [my box] {01210};

  \node (00122) at (5,2) [my box] {00122};
  \node (01112) at (5,1) [my box] {01112};
  \node (01121) at (5,0) [my box] {01121};
  \node (01211) at (5,-1) [my box] {01211};
  \node (01220) at (5,-2) [my box] {01220};

  \node (00123) at (6,2) [my box] {00123};
  \node (01212) at (6,0) [my box] {01212};
  \node (01230) at (6,-2) [my box] {01230};
  \node (01122) at (6,1) [my box] {01122};
  \node (01221) at (6,-1) [my box] {01221};

  \node (01123) at (7,2) [my box] {01123};
  \node (01222) at (7,0) [my box] {01222};
  \node (01231) at (7,-2) [my box] {01231};

  \node (01223) at (8,1) [my box] {01223};
  \node (01232) at (8,-1) [my box] {01232};

  \node (01233) at (9,0) [my box] {01233};
  \node (01234) at (10,0) [my box] {01234};

  \draw [->] (01121) to (01122);
  \draw [->] (01011) to (01111);
  \draw [->] (00012) to (01012);
  \draw [->] (01001.north west) to (01011);
  \draw [->] (00010) to (01010);
  \draw [->] (00122) to (01122);
  \draw [->] (01222) to (01232);
  \draw [->] (00121) to (01121);
  \draw [->] (01000) to (01100);
  \draw [->] (01200) to (01210);
  \draw [->] (01212) to (01222);
  \draw [->] (01001.north east) to (01101);
  \draw [->] (01220) to (01230);
  \draw [->] (01110) to (01111);
  \draw [->] (00121) to (00122);
  \draw [->] (00122) to (00123);
  \draw [->] (00011) to (01011);
  \draw [->] (01100) to (01110);
  \draw [->] (01223) to (01233);
  \draw [->] (00101) to (00111);
  \draw [->] (01110) to (01120);
  \draw [->] (01101) to (01201);
  \draw [->] (01111) to (01211);
  \draw [->] (00111) to (00121);
  \draw [->] (01011) to (01012);
  \draw [->] (01112) to (01212);
  \draw [->] (00010) to (00011);
  \draw [->] (00000) to (00100);
  \draw [->] (01000) to (01010);
  \draw [->] (00001) to (00101);
  \draw [->] (00000) to (00010);
  \draw [->] (00100) to (00101);
  \draw [->] (01211) to (01212);
  \draw [->] (01232) to (01233);
  \draw [->] (00120) to (00121);
  \draw [->] (00011) to (00111);
  \draw [->] (01010) to (01011);
  \draw [->] (00120) to (01120);
  \draw [->] (01010) to (01110);
  \draw [->] (00001) to (01001);
  \draw [->] (01100) to (01101);
  \draw [->] (00110) to (01110);
  \draw [->] (01220) to (01221);
  \draw [->] (00112) to (01112);
  \draw [->] (00000) to (01000);
  \draw [->] (01000) to (01001);
  \draw [->] (00000) to (00001);
  \draw [->] (01100) to (01200);
  \draw [->] (00101) to (01101);
  \draw [->] (01121) to (01221);
  \draw [->] (01221) to (01231);
  \draw [->] (00100) to (00110.east);
  \draw [->] (00110) to (00120);
  \draw [->] (01201) to (01211);
  \draw [->] (01233) to (01234);
  \draw [->] (01211) to (01221);
  \draw [->] (00123) to (01123);
  \draw [->] (00110) to (00111);
  \draw [->] (00001) to (00011);
  \draw [->] (00100) to (01100);
  \draw [->] (00111) to (00112);
  \draw [->] (01123) to (01223);
  \draw [->] (01120) to (01121);
  \draw [->] (01122) to (01123);
  \draw [->] (01210) to (01211);
  \draw [->] (01221) to (01222);
  \draw [->] (01012) to (01112);
  \draw [->] (00011) to (00012);
  \draw [->] (01111) to (01121);
  \draw [->] (01200) to (01201);
  \draw [->] (01101) to (01111);
  \draw [->] (01122) to (01222);
  \draw [->] (01120) to (01220);
  \draw [->] (01230) to (01231);
  \draw [->] (01111) to (01112);
  \draw [->] (01222) to (01223);
  \draw [->] (00111) to (01111);
  \draw [->] (00112) to (00122);
  \draw [->] (00012) to (00112);
  \draw [->] (01210) to (01220);
  \draw [->] (01110) to (01210);
  \draw [->] (01112) to (01122);
  \draw [->] (01231) to (01232);
  \draw [->] (00010) to (00110.west);

\end{tikzpicture}
\end{center}


%------------------------------------------------------------------------------
\section{Complement of Smallest Asymmetric Tree}

\url{https://hog.grinvin.org/ViewGraphInfo.action?id=31120}

\smallskip

This is one of the 18 minimal asymmetric graphs (asymmetric and has no
induced subgraph asymmetric).

Pascal Schweitzer and Patrick Schweitzer, ``Minimal Asymmetric Graphs'',
Journal of Combinatorial Theory, Series B, volume 127, November 2017,
pages 215-227.  Graph X14 in their numbering.
\newline
\url{https://arxiv.org/abs/1605.01320}
\newline
\url{https://www.sciencedirect.com/science/article/pii/S0095895617300539}


\begin{center}
\begin{tikzpicture}
    [scale=1.5,
     my box/.style={draw,circle,minimum size=1.3em,inner sep=.1em,outer sep=.3em},
    ]

  \node at (2.5, 1) [right,align=center]
    {X14 \\[1ex]
     n=7 \\[1ex]
     {\ttfamily }};

  \node (0) at (0, 0) [my box] {0};
  \node (1) at (0, .9) [my box] {1};
  \node (2) at (90:2) [my box] {2};
  \node (3) at (90+2*72:2) [my box] {3};
  \node (4) at (90-2*72:2) [my box] {4};
  \node (5) at (90+72:2) [my box] {5};
  \node (6) at (90-72:2) [my box] {6};

  \draw [] (3) to (5);
  \draw [] (1) to (5);
  \draw [] (4) to (3);
  \draw [] (1) to (2);
  \draw [] (3) to (1);
  \draw [] (1) to (6);
  \draw [] (0) to (3);
  \draw [] (0) to (5);
  \draw [] (3) to (6);
  \draw [] (0) to (6);
  \draw [] (4) to (1);
  \draw [] (4) to (6);
  \draw [] (6) to (2);
  \draw [] (4) to (0);
  \draw [] (5) to (2);

\end{tikzpicture}
\end{center}

\begin{center}
\begin{tikzpicture}
    [scale=2,
     my box/.style={draw,circle,minimum size=1.3em,inner sep=.1em,outer sep=.3em},
    ]

  \node at (1.5, 1) [right,align=center]
    {X14 \\[1ex]
     n=7 \\[1ex]
     {\ttfamily }};

  \node (0) at (0, -.8) [my box] {0};
  \node (1) at (0,1) [my box] {1};
  \node (2) at (0, 1.8) [my box] {2};
  \node (3) at (-1,0) [my box] {3};
  \node (4) at (1,0) [my box] {4};
  \node (5) at (-1,1) [my box] {5};
  \node (6) at (1,1) [my box] {6};

  \draw [] (3) to (5);
  \draw [] (1) to (5);
  \draw [] (4) to (3);
  \draw [] (1) to (2);
  \draw [] (3) to (1);
  \draw [] (1) to (6);
  \draw [] (0) to (3);
  \draw [] (0) to (5);
  \draw [] (3) to (6);
  \draw [] (0) to (6);
  \draw [] (4) to (1);
  \draw [] (4) to (6);
  \draw [] (6) to (2);
  \draw [] (4) to (0);
  \draw [] (5) to (2);

\end{tikzpicture}
\end{center}


%------------------------------------------------------------------------------
\section{Erdos and Renyi Asymmetric Del or Add}

Erd\H{o}s and R\'enyi, ``Asymmetric Graphs'', Acta Mathematica
Academiae Scientiarum Hungaricae, volume 14, 1963, pages 295-315.

\url{https://users.renyi.hu/~p_erdos/1963-04.pdf}

%-----------------
\subsection{Erdos and Renyi Figure 5}

\url{https://hog.grinvin.org/ViewGraphInfo.action?id=31137}

\smallskip

Erd\H{o}s and R\'enyi give this graph (their figure 5) as an example
which is asymmetric but can be made symmetric by deleting an edge.
They intended that adding one edge does not become symmetric; but in
fact can add one edge to make symmetric.  (A different delete than
theirs gives an involution automorphism and, as they note concerning
involutions, an addition can too.)

\begin{center}
\begin{tikzpicture}
    [scale=1.5,
     my box/.style={draw,circle,minimum size=1.3em,inner sep=.1em,outer sep=.3em},
    ]

      % .  0
      % .
      % .     1     2
      % .
      % .  3     4     5    6
      % .
      % .     7     8
      % .
      % .  9

  \node at (2, .8) [right] {figure 5 n=10};

  \node (0) at (120:2) [my box] {0};
  \node (1) at (120:1) [my box] {1};
  \node (2) at (60:1) [my box] {2};
  \node (3) at (180:1) [my box] {3};
  \node (4) at (0,0) [my box] {4};
  \node (5) at (0:1) [my box] {5};
  \node (6) at (0:2) [my box] {6};
  \node (7) at (-120:1) [my box] {7};
  \node (8) at (-60:1) [my box] {8};
  \node (9) at (-120:2) [my box] {9};

  \draw [] (2) to (3);
  \draw [] (2) to (8);
  \draw [] (1) to (4);
  \draw [] (4) to (8);
  \draw [] (6) to (8);
  \draw [] (0) to (1);
  \draw [] (1) to (3);
  \draw [] (3) to (9);
  \draw [] (0) to (2);
  \draw [] (4) to (7);
  \draw [] (8) to (9);
  \draw [] (5) to (6);
  \draw [] (2) to (4);
  \draw [] (7) to (8);
  \draw [] (4) to (5);
  \draw [] (3) to (4);
  \draw [] (2) to (5);
  \draw [] (7) to (9);
  \draw [] (3) to (8);

\end{tikzpicture}
\end{center}

\begin{center}
\begin{tikzpicture}
    [scale=2,
     my box/.style={draw,circle,minimum size=1.3em,inner sep=.1em,outer sep=.2em},
    ]
  \newcommand\MyHoriz{.9}

  \node at (1.8, .7) [right,align=center]
    {figure 5 n=10 \\ another drawing \\[1ex]
     other mods del 4,5 or add 3,5 \\
     which balance 3,4 edges upwards};

  \node (0) at (0, 1.6) [my box] {0};
  \node (1) at (-\MyHoriz, 1) [my box] {1};
  \node (2) at (\MyHoriz, 1) [my box] {2};
  \node (3) at (-\MyHoriz, 0) [my box] {3};
  \node (4) at (\MyHoriz, 0) [my box] {4};
  \node (5) at (0, 1) [my box] {5};
  \node (6) at (0, .2) [my box] {6};
  \node (7) at (\MyHoriz, -1) [my box] {7};
  \node (8) at (0, -.5) [my box] {8};
  \node (9) at (-\MyHoriz, -1) [my box] {9};

  \draw [] (2) to (3);
  \draw [] (2) to (8);
  \draw [] (1) to (4);
  \draw [] (4) to (8);
  \draw [] (6) to (8);
  \draw [] (0) to (1);
  \draw [] (1) to (3);
  \draw [] (3) to (9);
  \draw [] (0) to (2);
  \draw [] (4) to (7);
  \draw [] (8) to (9);
  \draw [] (5) to (6);
  \draw [] (2) to (4);
  \draw [] (7) to (8);
  \draw [] (4) to (5);
  \draw [] (3) to (4);
  \draw [] (2) to (5);
  \draw [] (7) to (9);
  \draw [] (3) to (8);

\end{tikzpicture}
\end{center}

%-----------------
\needspace{4\baselineskip}
\subsection{Erdos and Renyi Figure 5, edge delete}

\url{https://hog.grinvin.org/ViewGraphInfo.action?id=31139}

\smallskip

Erd\H{o}s and R\'enyi give a sample graph (their figure 5) which is
asymmetric but can be made symmetric by deleting an edge.  The present
graph is after their delete.  It has automorphism group C3.

\begin{center}
\begin{tikzpicture}
    [scale=2,
     my box/.style={draw,circle,minimum size=1.3em,inner sep=.1em,outer sep=.3em},
    ]

  \node at (1.5, .25) [right,align=center]
    {figure 5 n=10 \\
     after delete edge 8,9 \\[1ex]
     n=10 \\[1ex]
     {\ttfamily HPVBG\}z}};

  \begin{scope}[rotate=40]

  \node (0) at (90:1) [my box] {0};
  \node (1) at (130:1) [my box] {1};
  \node (2) at (50:1) [my box] {2};
  \node (3) at (170:1) [my box] {3};
  \node (4) at (0,0) [my box] {4};
  \node (5) at (10:1) [my box] {5};
  \node (6) at (330:1) [my box] {6};
  \node (7) at (250:1) [my box] {7};
  \node (8) at (290:1) [my box] {8};
  \node (9) at (210:1) [my box] {9};

  \draw [] (2) to (3);
  \draw [] (2) to (8);
  \draw [] (1) to (4);
  \draw [] (4) to (8);
  \draw [] (6) to (8);
  \draw [] (0) to (1);
  \draw [] (1) to (3);
  \draw [] (3) to (9);
  \draw [] (0) to (2);
  \draw [] (4) to (7);
%  \draw [] (8) to (9);
  \draw [] (5) to (6);
  \draw [] (2) to (4);
  \draw [] (7) to (8);
  \draw [] (4) to (5);
  \draw [] (3) to (4);
  \draw [] (2) to (5);
  \draw [] (7) to (9);
  \draw [] (3) to (8);

  \end{scope}

\end{tikzpicture}
\end{center}

%-----------------
\subsection{Erdos and Renyi Figure 5, other edge delete}

\url{https://hog.grinvin.org/ViewGraphInfo.action?id=31141}

\smallskip

Erd\H{o}s and R\'enyi give a sample graph (their figure 5) which is
asymmetric but can be made symmetric by deleting an edge.  The present
graph is after a different edge delete than the one they give, and
which also makes symmetric.  It has an automorphism swapping two
pairs.

\begin{center}
\begin{tikzpicture}
    [scale=2,
     my box/.style={draw,circle,minimum size=1.3em,inner sep=.1em,outer sep=.3em},
    ]
  \newcommand\MyHoriz{.7}

  \node at (2, .1) [right,align=center]
    {figure 5 n=10\\
     after delete edge 4,5 \\ (other) \\
     autom swap 3,4 and 7,9};

  \node (0) at (0, 1.5) [my box] {0};
  \node (1) at (-\MyHoriz, 1) [my box] {1};
  \node (2) at (\MyHoriz, 1) [my box] {2};
  \node (3) at (-\MyHoriz, 0) [my box] {3};
  \node (4) at (\MyHoriz, 0) [my box] {4};
  \node (7) at (\MyHoriz, -1) [my box] {7};
  \node (8) at (0, -.5) [my box] {8};
  \node (9) at (-\MyHoriz, -1) [my box] {9};

  \coordinate (delta) at ($(2.center) - (8.center)$);
  \node (6) at (1.3, -.5) [my box] {6};
  \node (5) at ($(6.center) + .45*(delta)$) [my box] {5};

  \draw [] (5) to (6);
  \draw [] (4) to (8);
  \draw [] (2) to (5);
  \draw [] (3) to (9);
  \draw [] (7) to (9);
  \draw [] (6) to (8);
  \draw [] (0) to (1);
  \draw [] (7) to (8);
  \draw [] (2) to (3);
  \draw [] (2) to (8);
  \draw [] (1) to (3);
  \draw [] (3) to (4);
  \draw [] (3) to (8);
  \draw [] (1) to (4);
  \draw [] (4) to (7);
  \draw [] (8) to (9);
  \draw [] (2) to (4);
  \draw [] (0) to (2);

\end{tikzpicture}
\end{center}

%-----------------
\subsection{Erdos and Renyi Figure 5, other edge add}

\url{https://hog.grinvin.org/ViewGraphInfo.action?id=31143}

\smallskip

Erd\H{o}s and R\'enyi give a sample graph (their figure 5) which is
asymmetric but on deleting an edge becomes symmetric.  They intended
that adding one edge does not become symmetric; but can in fact add
one to make symmetric.  The graph here is with that addition.  It has
an automorphism swapping two pairs.

\begin{center}
\begin{tikzpicture}
    [scale=2.9, xscale=1.5,
     my box/.style={draw,circle,minimum size=1.3em,inner sep=.1em,outer sep=.3em},
    ]

  \node at (.3, .5) [right,align=center]
    {figure 5 n=10 \\
     after add edge 3,5 \\
     autom swap 3,4 and 7,9};

  \node (0) at (-.5, 1.4) [my box] {0};
  \node (1) at (-1-.0, 1) [my box] {1};
  \node (2) at (0+.0, 1) [my box] {2};
  \node (3) at (-1,0) [my box] {3};
  \node (4) at (0,0) [my box] {4};
  \node (5) at (-.5, 1) [my box] {5};
  \node (6) at (-.5, .2) [my box] {6};
  \node (7) at (0,-1) [my box] {7};
  \node (8) at (-.5, -.5) [my box] {8};
  \node (9) at (-1,-1) [my box] {9};

  \draw [] (3) to (4);
  \draw [] (1) to (3);
  \draw [] (8) to (9);
  \draw [] (3) to (8);
  \draw [] (4) to (8);
  \draw [] (3) to (9);
  \draw [] (7) to (9);
  \draw [] (4) to (5);
  \draw [] (6) to (8);
  \draw [] (2) to (3);
  \draw [] (1) to (4);
  \draw [] (2) to (4);
  \draw [] (2) to (5);
  \draw [] (5) to (6);
  \draw [] (7) to (8);
  \draw [] (4) to (7);
  \draw [] (3) to (5);
  \draw [] (0) to (1);
  \draw [] (2) to (8);
  \draw [] (0) to (2);

\end{tikzpicture}
\end{center}


%------------------------------------------------------------------------------
\section{Nine Vertices, Asymmetric, Made Symmetric by Delete Edge}

\subsection{Degrees le 5}

\url{https://hog.grinvin.org/ViewGraphInfo.action?id=32230}

\begin{center}
\begin{tikzpicture}
    [scale=3,
     my box/.style={draw,circle,minimum size=1.3em,inner sep=.1em,outer sep=.3em},
    ]

  \node at (1.1, .5) [right,align=center]
    {n=9 can del to symmetric, \\
     cannot add to symmetric, \\
     0,6 is the delete};

  % autos 0,6,7 1,5,8, 2,3,4

  \newcommand\MyInner{.235}
  \newcommand\MyOuter{.7}

  \begin{scope}[rotate=120]
  \begin{scope}[shift={(-30:\MyOuter)},rotate=0+30]
  \node (4) at (240:\MyInner) [my box] {4};
  \node (6) at (0:\MyInner) [my box] {6};
  \end{scope}

  \begin{scope}[shift={(90:\MyOuter)},rotate=120+30]
  \node (2) at (240:\MyInner) [my box] {2};
  \node (7) at (0:\MyInner) [my box] {7};
  \end{scope}

  \begin{scope}[shift={(-150:\MyOuter)},rotate=240+30]
  \node (0) at (0:\MyInner) [my box] {0};
  \node (3) at (240:\MyInner) [my box] {3};
  \end{scope}

  \begin{scope}[rotate=30]
  \node (5) at (120:\MyInner) [my box] {5};
  \node (1) at (0:\MyInner)   [my box] {1};
  \node (8) at (240:\MyInner) [my box] {8};
  \end{scope}

  \end{scope}

  \draw [very thick,dashed] (0) to (6);


  \draw [] (4) to (0);
  \draw [] (6) to (1);
  \draw [] (3) to (8);
  \draw [] (3) to (7);
  \draw [] (8) to (6);
  \draw [] (1) to (5);
  \draw [] (3) to (0);
  \draw [] (8) to (5);
  \draw [] (1) to (4);
  \draw [] (5) to (7);
  \draw [] (6) to (4);
  \draw [] (8) to (1);
  \draw [] (2) to (7);
  \draw [] (1) to (7);
  \draw [] (6) to (2);
  \draw [] (0) to (5);
  \draw [] (8) to (0);
  \draw [] (2) to (5);

\end{tikzpicture}
\end{center}

\subsection{With Degree 6}

\url{https://hog.grinvin.org/ViewGraphInfo.action?id=32228}

\begin{center}
\begin{tikzpicture}
    [scale=3,
     my box/.style={draw,circle,minimum size=1.3em,inner sep=.1em,outer sep=.3em},
    ]
  \node at (1, .5) [right,align=center]
    {n=9 can del to symmetric, \\
     cannot add to symmetric, \\
     3,8 is the delete};

  % autos 0,6,7 1,5,8, 2,3,4

  \newcommand\MyInner{.3}
  \newcommand\MyOuter{.5}

  \begin{scope}[rotate=120]
  \begin{scope}[shift={(-30:\MyOuter)},rotate=0+30]
  \node (0) at (0:\MyInner) [my box] {0};
  \node (3) at (240:\MyInner) [my box] {3};
  \end{scope}

  \begin{scope}[shift={(90:\MyOuter)},rotate=120+30]
  \node (4) at (0:\MyInner) [my box] {4};
  \node (5) at (240:\MyInner) [my box] {5};
  \end{scope}

  \begin{scope}[shift={(-150:\MyOuter)},rotate=240+30]
  \node (2) at (240:\MyInner) [my box] {2};
  \node (6) at (0:\MyInner) [my box] {6};
  \end{scope}

  \begin{scope}[rotate=30]
  \node (8) at (240:\MyInner) [my box] {8};
  \node (1) at (120:\MyInner)   [my box] {1};
  \node (7) at (0:\MyInner) [my box] {7};
  \end{scope}

  \end{scope}

  \draw [dashed,very thick] (3) to (8);

  \draw [] (0) to (5);
  \draw [] (7) to (1);
  \draw [] (0) to (7);
  \draw [] (8) to (2);
  \draw [] (8) to (0);
  \draw [] (1) to (5);
  \draw [] (5) to (4);
  \draw [] (4) to (2);
  \draw [] (6) to (2);
  \draw [] (8) to (1);
  \draw [] (6) to (8);
  \draw [] (1) to (4);
  \draw [] (7) to (3);
  \draw [] (6) to (1);
  \draw [] (8) to (7);
  \draw [] (0) to (3);
  \draw [] (7) to (4);
  \draw [] (6) to (3);

\end{tikzpicture}
\end{center}

\subsection{Same After Edge Delete of Either}

\url{https://hog.grinvin.org/ViewGraphInfo.action?id=32232}

>>graph6<<HCrbRjU

\begin{center}
\begin{tikzpicture}
    [scale=1.3,
     my box/.style={draw,circle,minimum size=1.3em,inner sep=.1em,outer sep=.3em},
    ]
  \node at (1.25, .5) [right] {n=9 deleted 0,6};

  % autos 0,6,7 1,5,8, 2,3,4

  \begin{scope}[rotate=-70]
  \node (0) at (0:1) [my box] {0};
  \node (6) at (120:1) [my box] {6};
  \node (7) at (240:1) [my box] {7};

  \node (1) at (120+80:1)   [my box] {1};
  \node (5) at (240+80:1) [my box] {5};
  \node (8) at (0+80:1) [my box] {8};

  \node (2) at (240+40:1.9) [my box] {2};
  \node (3) at (0+40:1.9)   [my box] {3};
  \node (4) at (120+40:1.9) [my box] {4};
  \end{scope}

  \draw [] (4) to (0);
  \draw [] (6) to (1);
  \draw [] (3) to (8);
  \draw [] (3) to (7);
  \draw [] (8) to (6);
  \draw [] (1) to (5);
  \draw [] (3) to (0);
  \draw [] (8) to (5);
  \draw [] (1) to (4);
  \draw [] (5) to (7);
  \draw [] (6) to (4);
  \draw [] (8) to (1);
  \draw [] (2) to (7);
  \draw [] (1) to (7);
  \draw [] (6) to (2);
  \draw [] (0) to (5);
  \draw [] (8) to (0);
  \draw [] (2) to (5);

\end{tikzpicture}
\end{center}

%----------------
\subsection{Complement}

\begin{center}
\begin{tikzpicture}
    [scale=2.5,
     my box/.style={draw,circle,minimum size=1.3em,inner sep=.1em,outer sep=.3em},
    ]
  \node at (1.25, .5) [right,align=center]
    {n=9 can del to symmetric, \\
     cannot add to symmetric};

  % autos 0,6,7 1,5,8, 2,3,4

  \begin{scope}[rotate=10]
  \node (7) at (0+40:1) [my box] {7};
  \node (6) at (120+40:1) [my box] {6};
  \node (0) at (240+40:1) [my box] {0};

  \node (5) at (0+80:1) [my box] {5};
  \node (1) at (120+80:1)   [my box] {1};
  \node (8) at (240+80:1) [my box] {8};

  \node (4) at (0:1) [my box] {4};
  \node (3) at (120:1) [my box] {3};
  \node (2) at (240:1)   [my box] {2};
  \end{scope}

  \draw [] (4) to (0);
  \draw [] (6) to (1);
  \draw [] (3) to (8);
  \draw [] (3) to (7);
  \draw [] (8) to (6);
  \draw [] (1) to (5);
  \draw [] (3) to (0);
  \draw [] (8) to (5);
  \draw [] (1) to (4);
  \draw [] (5) to (7);
  \draw [] (6) to (4);
  \draw [] (8) to (1);
  \draw [] (2) to (7);
  \draw [] (1) to (7);
  \draw [] (6) to (2);
  \draw [] (0) to (5);
  \draw [] (8) to (0);
  \draw [] (2) to (5);

\end{tikzpicture}
\end{center}



%---------------------
% Nine

With 2,5,7

2 -- 0
 1 4
 3 8

5 -- 3 4
 6

7 -- 0
 4
 6
 8


\begin{center}
\begin{tikzpicture}
    [scale=2.5,
     my box/.style={draw,circle,minimum size=1.3em,inner sep=.1em,outer sep=.3em},
    ]
  \node at (4, 2) [right] {Nine (AA)};

  \node (0) at (2, 1.5) [my box] {0};
  \node (1) at (3.5, 3) [my box] {1};
  \node (2) at (1.5, 1) [my box] {2};
  \node (3) at (2.5, 2.5) [my box] {3};
  \node (4) at (3, 2.5) [my box] {4};
  \node (5) at (2, 2.5) [my box] {5};
  \node (6) at (3.5, 1) [my box] {6};
  \node (7) at (1.5, 3) [my box] {7};
  \node (8) at (3, 1.5) [my box] {8};

  \draw [] (1) to (8);
  \draw [] (3) to (8);
  \draw [] (0) to (6);
  \draw [] (0) to (5);
  \draw [] (6) to (8);
  \draw [] (0) to (4);
  \draw [] (2) to (6);
  \draw [] (1) to (4);
  \draw [] (0) to (8);
  \draw [] (1) to (5);
  \draw [] (3) to (7);
  \draw [] (2) to (7);
  \draw [] (5) to (7);
  \draw [] (0) to (3);
  \draw [] (1) to (6);
  \draw [] (1) to (7);
  \draw [] (5) to (8);
  \draw [] (2) to (5);
  \draw [] (4) to (6);

\end{tikzpicture}
\end{center}

\begin{center}
\begin{tikzpicture}
    [scale=1.5,
     my box/.style={draw,circle,minimum size=1.3em,inner sep=.1em,outer sep=.3em},
    ]
  \node at (4, 1) [right] {Nine (B)};

  \node (0) at (2, -1) [my box] {0};
  \node (1) at (2, 1) [my box] {1};
  \node (2) at (2, 2) [my box] {2};
  \node (3) at (4, -1) [my box] {3};
  \node (4) at (1, -1) [my box] {4};
  \node (5) at (3, 0) [my box] {5};
  \node (6) at (1,1) [my box] {6};
  \node (7) at (3, 1) [my box] {7};
  \node (8) at (2, 0) [my box] {8};

  \draw [] (1) to (8);
  \draw [] (3) to (8);
  \draw [] (0) to (6);
  \draw [] (0) to (5);
  \draw [] (6) to (8);
  \draw [] (0) to (4);
  \draw [] (2) to (6);
  \draw [] (1) to (4);
  \draw [] (0) to (8);
  \draw [] (1) to (5);
  \draw [] (3) to (7);
  \draw [] (2) to (7);
  \draw [] (5) to (7);
  \draw [] (0) to (3);
  \draw [] (1) to (6);
  \draw [] (1) to (7);
  \draw [] (5) to (8);
  \draw [] (2) to (5);
  \draw [] (4) to (6);

\end{tikzpicture}
\end{center}

\subsection{Nine Vertices, Asymmetric, Made Symmetric by Add Edge}

\begin{center}
\begin{tikzpicture}
    [scale=3,
     my box/.style={draw,circle,minimum size=1.3em,inner sep=.1em,outer sep=.3em},
    ]
  \node at (1, .8) [right] {nine complement};

  \node (4) at (90:1.4) [my box] {4};
  \node (2) at (210:1.4) [my box] {2};
  \node (3) at (330:1.4) [my box] {3};

  \node (7) at (90:.4) [my box] {7};
  \node (0) at (210:.4) [my box] {0};
  \node (6) at (330:.4) [my box] {6};

  \begin{scope}[rotate=0]
  \node (5) at (30:.4) [my box] {5};
  \node (8) at (150:.4) [my box] {8};
  \node (1) at (270:.4) [my box] {1};
  \end{scope}

  \draw [] (1) to (0);
  \draw [] (2) to (8);
  \draw [] (4) to (7);
  \draw [] (6) to (7);
  \draw [] (3) to (4);
  \draw [] (8) to (4);
  \draw [] (8) to (7);
  \draw [] (2) to (4);
  \draw [] (2) to (3);
  \draw [] (3) to (5);
  \draw [] (4) to (5);
  \draw [] (0) to (7);
  \draw [] (3) to (6);
  \draw [] (2) to (0);
  \draw [] (5) to (6);
  \draw [] (2) to (1);
  \draw [] (3) to (1);

\end{tikzpicture}
\end{center}

\begin{center}
\begin{tikzpicture}
    [scale=2,
     my box/.style={draw,circle,minimum size=1.3em,inner sep=.1em,outer sep=.3em},
    ]
  \node at (1, .8) [right] {nine Complement};

  \begin{scope}[rotate=196]

  \newcommand\MyOuter{1.6}
  \begin{scope}[rotate=225]
  \node (0) at (30:\MyOuter) [my box] {0};
  \node (6) at (150:\MyOuter) [my box] {6};
  \node (7) at (270:\MyOuter) [my box] {7};
  \end{scope}

  \node (5) at (30:.8) [my box] {5};
  \node (8) at (150:.8) [my box] {8};
  \node (1) at (270:.8) [my box] {1};

  \node (4) at (90:.4) [my box] {4};
  \node (2) at (210:.4) [my box] {2};
  \node (3) at (330:.4) [my box] {3};

  \end{scope}

  \draw [] (1) to (0);
  \draw [] (2) to (8);
  \draw [] (4) to (7);
  \draw [] (6) to (7);
  \draw [] (3) to (4);
  \draw [] (8) to (4);
  \draw [] (8) to (7);
  \draw [] (2) to (4);
  \draw [] (2) to (3);
  \draw [] (3) to (5);
  \draw [] (4) to (5);
  \draw [] (0) to (7);
  \draw [] (3) to (6);
  \draw [] (2) to (0);
  \draw [] (5) to (6);
  \draw [] (2) to (1);
  \draw [] (3) to (1);

\end{tikzpicture}
\end{center}

\begin{center}
\begin{tikzpicture}
    [scale=2.5,
     my box/.style={draw,circle,minimum size=1.3em,inner sep=.1em,outer sep=.3em},
    ]
  \node at (1.8, 0) [right] {Nine Circ};

  % 1,3,5
  % 0,4,7
  % 2,8,6
  \node (0) at (0:1) [my box] {0};
  \node (1) at (0+40:1) [my box] {1};
  \node (2) at (0+80:1) [my box] {2};
  \node (3) at (120+40:1) [my box] {3};
  \node (4) at (120:1) [my box] {4};
  \node (5) at (240+40:1) [my box] {5};
  \node (6) at (240+80:1) [my box] {6};
  \node (7) at (240:1) [my box] {7};
  \node (8) at (120+80:1) [my box] {8};

  \draw [] (1) to (8);
  \draw [] (3) to (8);
  \draw [] (0) to (6);
  \draw [] (0) to (5);
  \draw [] (6) to (8);
  \draw [] (0) to (4);
  \draw [] (2) to (6);
  \draw [] (1) to (4);
  \draw [] (0) to (8);
  \draw [] (1) to (5);
  \draw [] (3) to (7);
  \draw [] (2) to (7);
  \draw [] (5) to (7);
  \draw [] (0) to (3);
  \draw [] (1) to (6);
  \draw [] (1) to (7);
  \draw [] (5) to (8);
  \draw [] (2) to (5);
  \draw [] (4) to (6);

\end{tikzpicture}
\end{center}

\begin{center}
\begin{tikzpicture}
    [scale=2,
     my box/.style={draw,circle,minimum size=1.3em,inner sep=.1em,outer sep=.3em},
    ]

  \node (0) at (0,1) [my box] {0};
  \node (1) at (2, -1) [my box] {1};
  \node (2) at (1.5, 0) [my box] {2};
  \node (3) at (2.5, 0) [my box] {3};
  \node (4) at (2, 1) [my box] {4};
  \node (5) at (3,1) [my box] {5};
  \node (6) at (2.5,2) [my box] {6};
  \node (7) at (1.5, 2) [my box] {7};
  \node (8) at (1, 1) [my box] {8};

  \draw [] (1) to (0);
  \draw [] (2) to (8);
  \draw [] (4) to (7);
  \draw [] (6) to (7);
  \draw [] (3) to (4);
  \draw [] (8) to (4);
  \draw [] (8) to (7);
  \draw [] (2) to (4);
  \draw [] (2) to (3);
  \draw [] (3) to (5);
  \draw [] (4) to (5);
  \draw [] (0) to (7);
  \draw [] (3) to (6);
  \draw [] (2) to (0);
  \draw [] (5) to (6);
  \draw [] (2) to (1);
  \draw [] (3) to (1);

\end{tikzpicture}
\end{center}

Nine Second

\begin{center}
\begin{tikzpicture}
    [scale=2,
     my box/.style={draw,circle,minimum size=1.3em,inner sep=.1em,outer sep=.3em},
    ]

  \node (0) at (-1, 1.5) [my box] {0};
  \node (1) at (1, 1.5) [my box] {1};
  \node (2) at (2, 1) [my box] {2};
  \node (3) at (-1, .3) [my box] {3};
  \node (4) at (1,3) [my box] {4};
  \node (5) at (0, 3) [my box] {5};
  \node (6) at (1, .3) [my box] {6};
  \node (7) at (0, 2) [my box] {7};
  \node (8) at (0,1) [my box] {8};

  \draw [] (2) to (8);
  \draw [] (6) to (8);
  \draw [] (1) to (7);
  \draw [] (0) to (3);
  \draw [] (1) to (5);
  \draw [] (4) to (5);
  \draw [] (3) to (6);
  \draw [] (1) to (4);
  \draw [] (1) to (6);
  \draw [] (2) to (4);
  \draw [] (4) to (7);
  \draw [] (7) to (8);
  \draw [] (0) to (7);
  \draw [] (0) to (8);
  \draw [] (2) to (6);
  \draw [] (3) to (7);
  \draw [] (0) to (5);
  \draw [] (1) to (8);
  \draw [] (3) to (8);

\end{tikzpicture}
\end{center}


%------------------------------------------------------------------------------
\section{Automorphisms C3, N=9}

\begin{center}
\begin{tikzpicture}
    [scale=1,
     my box/.style={draw,circle,minimum size=1.3em,inner sep=.1em,outer sep=.3em},
    ]

  \node at (2, 0) [right,align=center] {N=9, 15 edges \\[1ex]
                                        {\ttfamily HCOedLj} \\[1ex]
                                        degrees 2,3,5};

  \node (0) at (90+240-30:1) [my box] {0};
  \node (1) at (90-30:1) [my box] {1};
  \node (2) at (90+120-30:1) [my box] {2};

  \node (3) at (90+240:1.8) [my box] {3};
  \node (4) at (90:1.8) [my box] {4};
  \node (5) at (90+120:1.8) [my box] {5};

  \node (6) at (90+240+30:1) [my box] {6};
  \node (7) at (90+120+30:1) [my box] {7};
  \node (8) at (90+30:1) [my box] {8};

  \draw [] (2) to (7);
  \draw [] (7) to (8);
  \draw [] (6) to (8);
  \draw [] (4) to (8);
  \draw [] (2) to (8);
  \draw [] (5) to (7);
  \draw [] (3) to (6);
  \draw [] (1) to (4);
  \draw [] (0) to (6);
  \draw [] (0) to (7);
  \draw [] (1) to (6);
  \draw [] (1) to (8);
  \draw [] (0) to (3);
  \draw [] (2) to (5);
  \draw [] (6) to (7);

\end{tikzpicture}
\end{center}
\begin{center}
\begin{tikzpicture}
    [scale=1.8,
     my box/.style={draw,circle,minimum size=1.3em,inner sep=.1em,outer sep=.2em},
    ]

  \node at (1.5, 0) [right,align=center] {N=9, 15 edges \\[1ex]
                                        {\ttfamily HCOedLj} \\[1ex]
                                        per Frucht (is it?)};


  \node (6) at (90:1) [my box] {6};
  \node (1) at ($(6) + (150:.5)$) [my box] {1};
  \node (3) at ($(6) + (30:.5)$) [my box] {3};

  \node (8) at (90+120:1) [my box] {8};
  \node (2) at ($(8) + (-90:.5)$) [my box] {2};
  \node (4) at ($(8) + (150:.5)$) [my box] {4};

  \node (7) at (90+240:1) [my box] {7};
  \node (5) at ($(7) + (-90:.5)$) [my box] {5};
  \node (0) at ($(7) + (30:.5)$) [my box] {0};


  \draw [] (2) to (7);
  \draw [] (7) to (8);
  \draw [] (6) to (8);
  \draw [] (4) to (8);
  \draw [] (2) to (8);
  \draw [] (5) to (7);
  \draw [] (3) to (6);
  \draw [] (1) to (4);
  \draw [] (0) to (6);
  \draw [] (0) to (7);
  \draw [] (1) to (6);
  \draw [] (1) to (8);
  \draw [] (0) to (3);
  \draw [] (2) to (5);
  \draw [] (6) to (7);

\end{tikzpicture}
\end{center}

\begin{center}
\begin{tikzpicture}
    [scale=1.3,
     my box/.style={draw,circle,minimum size=1.3em,inner sep=.1em,outer sep=.3em},
    ]

  \node at (2, 0) [right,align=center] {N=9, 18 edges \\[1ex]
                                        {\ttfamily }};

  \node (0) at (90+240+30:1) [my box] {0};
  \node (2) at (90+120-30:1) [my box] {2};
  \node (4) at (90-30:1) [my box] {4};

  \node (7) at (90+240:2.8) [my box] {7};
  \node (1) at (90:2.8) [my box] {1};
  \node (5) at (90+120:2.8) [my box] {5};

  \node (3) at (90+240-30:1) [my box] {3};
  \node (6) at (90+120+30:1) [my box] {6};
  \node (8) at (90+30:1) [my box] {8};

  \draw [] (0) to (6);
  \draw [] (5) to (6);
  \draw [] (1) to (8);
  \draw [] (1) to (5);
  \draw [] (3) to (7);
  \draw [] (3) to (6);
  \draw [] (1) to (4);
  \draw [] (2) to (5);
  \draw [] (0) to (8);
  \draw [] (4) to (8);
  \draw [] (6) to (8);
  \draw [] (0) to (4);
  \draw [] (5) to (7);
  \draw [] (0) to (3);
  \draw [] (0) to (7);
  \draw [] (2) to (6);
  \draw [] (1) to (7);
  \draw [] (2) to (8);

\end{tikzpicture}
\end{center}
\begin{center}
\begin{tikzpicture}
    [scale=3,
     my box/.style={draw,circle,minimum size=1.3em,inner sep=.1em,outer sep=.3em},
    ]

  \node at (1, .5) [right,align=center] {N=9, 18 edges \\[1ex]
                                        {\ttfamily }};

  \node (0) at (90+240+30:1) [my box] {0};
  \node (2) at (90+120-30:1) [my box] {2};
  \node (4) at (90-30:1) [my box] {4};

  \node (1) at (90    +0:.3) [my box] {1};
  \node (7) at (90+240+0:.3) [my box] {7};
  \node (5) at (90+120+0:.3) [my box] {5};

  \node (3) at (90+240-30:1) [my box] {3};
  \node (6) at (90+120+30:1) [my box] {6};
  \node (8) at (90+30:1) [my box] {8};

  \draw [] (0) to (6);
  \draw [] (5) to (6);
  \draw [] (1) to (8);
  \draw [] (1) to (5);
  \draw [] (3) to (7);
  \draw [] (3) to (6);
  \draw [] (1) to (4);
  \draw [] (2) to (5);
  \draw [] (0) to (8);
  \draw [] (4) to (8);
  \draw [] (6) to (8);
  \draw [] (0) to (4);
  \draw [] (5) to (7);
  \draw [] (0) to (3);
  \draw [] (0) to (7);
  \draw [] (2) to (6);
  \draw [] (1) to (7);
  \draw [] (2) to (8);

\end{tikzpicture}
\end{center}
\begin{center}
\begin{tikzpicture}
    [scale=3,
     my box/.style={draw,circle,minimum size=1.3em,inner sep=.1em,outer sep=.3em},
    ]

  \node (3) at (60+240-30:.7) [my box] {3};
  \node (2) at (60+120-30:.7) [my box] {2};
  \node (4) at (60    -30:.7) [my box] {4};

  \node (1) at (60    -30:.3) [my box] {1};
  \node (5) at (60+120-30:.3) [my box] {5};
  \node (7) at (60+240-30:.3) [my box] {7};

  \node (0) at (60+240+30:2) [my box] {0};
  \node (6) at (60+120+30:2) [my box] {6};
  \node (8) at (60+30:2) [my box] {8};

  \draw [] (0) to (6);
  \draw [] (5) to (6);
  \draw [] (1) to (8);
  \draw [] (1) to (5);
  \draw [] (3) to (7);
  \draw [] (3) to (6);
  \draw [] (1) to (4);
  \draw [] (2) to (5);
  \draw [] (0) to (8);
  \draw [] (4) to (8);
  \draw [] (6) to (8);
  \draw [] (0) to (4);
  \draw [] (5) to (7);
  \draw [] (0) to (3);
  \draw [] (0) to (7);
  \draw [] (2) to (6);
  \draw [] (1) to (7);
  \draw [] (2) to (8);

\end{tikzpicture}
\end{center}


%------------------------------------------------------------------------------
\section{Automorphisms C3, N=10}

%----------------
\subsection{Automorphism C3, N=10, E=15}

\url{https://hog.grinvin.org/ViewGraphInfo.action?id=31116}

\smallskip

This graph has automorphism group C3.  It is one of two such having
n=10 vertices and e=15 edges.  This one has the degree 4s connected to
the middle fixed vertex.

\begin{center}
\begin{tikzpicture}
    [scale=1.6,
     my box/.style={draw,circle,minimum size=1.3em,inner sep=.1em,outer sep=.3em},
    ]

  \node at (2, 0) [right,align=center] {N=10, 15 edges \\[1ex]
                                        {\ttfamily I?\_aCwuU\_} \\[1ex]
                                        degrees 2,3,4};

  \begin{scope}[rotate=0,xscale=-1]
  \node (8) at (50-320:1) [my box] {8};
  \node (0) at (50:1) [my box] {0};
  \node (4) at (50-40:1) [my box] {4};
  \node (7) at (50-80:1) [my box] {7};
  \node (1) at (50-120:1) [my box] {1};
  \node (5) at (50-160:1) [my box] {5};
  \node (9) at (50-200:1) [my box] {9};
  \node (2) at (50-240:1) [my box] {2};
  \node (6) at (50-280:1) [my box] {6};
  \node (3) at (0,0) [my box] {3};
  \end{scope}

  \draw [] (3) to (7);
  \draw [] (4) to (7);
  \draw [] (2) to (8);
  \draw [] (1) to (9);
  \draw [] (2) to (6);
  \draw [] (0) to (4);
  \draw [] (5) to (9);
  \draw [] (6) to (8);
  \draw [] (2) to (9);
  \draw [] (3) to (8);
  \draw [] (0) to (8);
  \draw [] (1) to (5);
  \draw [] (3) to (9);
  \draw [] (1) to (7);
  \draw [] (0) to (7);

\end{tikzpicture}
\end{center}

\begin{center}
\begin{tikzpicture}
    [scale=1.3,
     my box/.style={draw,circle,minimum size=1.3em,inner sep=.1em,outer sep=.3em},
    ]

  \node at (2, 0) [right,align=center] {N=10, 15 edges \\[1ex]
                                        {\ttfamily I?\_aCwuU\_} \\[1ex]
                                        degrees 2,3,4};

  \node (0) at (90-30:1) [my box] {0};
  \node (1) at (90+120-30:1) [my box] {1};
  \node (2) at (90+240-30:1) [my box] {2};

  \node (3) at (0,0) [my box] {3};
  \node (4) at (90:1.8) [my box] {4};
  \node (5) at (90+120:1.8) [my box] {5};

  \node (6) at (90+240:1.8) [my box] {6};
  \node (7) at (90+30:1) [my box] {7};
  \node (8) at (90+240+30:1) [my box] {8};

  \node (9) at (90+120+30:1) [my box] {9};

  \draw [] (3) to (7);
  \draw [] (4) to (7);
  \draw [] (2) to (8);
  \draw [] (1) to (9);
  \draw [] (2) to (6);
  \draw [] (0) to (4);
  \draw [] (5) to (9);
  \draw [] (6) to (8);
  \draw [] (2) to (9);
  \draw [] (3) to (8);
  \draw [] (0) to (8);
  \draw [] (1) to (5);
  \draw [] (3) to (9);
  \draw [] (1) to (7);
  \draw [] (0) to (7);

\end{tikzpicture}
\end{center}

%----------------
\subsection{Automorphism C3, N=10, E=15}

\url{https://hog.grinvin.org/ViewGraphInfo.action?id=31118}

\smallskip

This graph has automorphism group C3.  It is one of two such having
n=10 vertices and e=15 edges.  This one has the degree 3s connected to
the middle fixed vertex.

\begin{center}
\begin{tikzpicture}
    [scale=1.6,
     my box/.style={draw,circle,minimum size=1.3em,inner sep=.1em,outer sep=.3em},
    ]

  \node at (2, 0) [right,align=center] {N=10, 15 edges \\[1ex]
                                        {\ttfamily I?cqcGbQW} \\[1ex]
                                        degrees 2,3,4};

  \begin{scope}[rotate=0,xscale=-1]
  \node (0) at (90:1) [my box] {0};
  \node (4) at (90+40:1) [my box] {4};
  \node (9) at (90+80:1) [my box] {9};
  \node (3) at (90+120:1) [my box] {3};
  \node (6) at (90+160:1) [my box] {6};
  \node (1) at (90+200:1) [my box] {1};
  \node (5) at (90+240:1) [my box] {5};
  \node (2) at (90+280:1) [my box] {2};
  \node (7) at (90+320:1) [my box] {7};
  \node (8) at (0,0) [my box] {8};
  \end{scope}

  \draw [] (3) to (8);
  \draw [] (0) to (8);
  \draw [] (1) to (9);
  \draw [] (5) to (8);
  \draw [] (1) to (5);
  \draw [] (3) to (9);
  \draw [] (3) to (6);
  \draw [] (1) to (7);
  \draw [] (1) to (6);
  \draw [] (0) to (7);
  \draw [] (7) to (9);
  \draw [] (2) to (7);
  \draw [] (2) to (5);
  \draw [] (0) to (4);
  \draw [] (4) to (9);

\end{tikzpicture}
\end{center}

\begin{center}
\begin{tikzpicture}
    [scale=1.3,
     my box/.style={draw,circle,minimum size=1.3em,inner sep=.1em,outer sep=.3em},
    ]

  \node at (2, 0) [right,align=center] {N=10, 15 edges \\[1ex]
                                        {\ttfamily I?cqcGbQW} \\[1ex]
                                        degrees 2,3,4};

  \node (0) at (90+30:1) [my box] {0};
  \node (1) at (90+240-30:1) [my box] {1};
  \node (2) at (90+120:1.8) [my box] {2};

  \node (3) at (90+240+30:1) [my box] {3};
  \node (4) at (90:1.8) [my box] {4};
  \node (5) at (90+120+30:1) [my box] {5};

  \node (6) at (90+240:1.8) [my box] {6};
  \node (7) at (90+120-30:1) [my box] {7};
  \node (8) at (0,0) [my box] {8};

  \node (9) at (90-30:1) [my box] {9};

  \draw [] (3) to (8);
  \draw [] (0) to (8);
  \draw [] (1) to (9);
  \draw [] (5) to (8);
  \draw [] (1) to (5);
  \draw [] (3) to (9);
  \draw [] (3) to (6);
  \draw [] (1) to (7);
  \draw [] (1) to (6);
  \draw [] (0) to (7);
  \draw [] (7) to (9);
  \draw [] (2) to (7);
  \draw [] (2) to (5);
  \draw [] (0) to (4);
  \draw [] (4) to (9);

\end{tikzpicture}
\end{center}



%------------------------------------------------------------------------------
\section{N=6 Asymmetrics}

\begin{center}
\begin{tikzpicture}
    [scale=1,
     my box/.style={draw,circle,minimum size=1.3em,inner sep=.1em,outer sep=.3em},
    ]

  \node at (2, 0) [right,align=center] {N=6 vertices, E=6 edges \\[1ex]
                                        {\ttfamily EGcw} \\[1ex]
                                        };

  \node (0) at (90+240:1.8) [my box] {0};
  \node (1) at (90    :.8) [my box] {1};
  \node (2) at (90+120:1.8) [my box] {2};
  \node (3) at (90+240:2.8) [my box] {3};
  \node (4) at (90+120:.8) [my box] {4};
  \node (5) at (90+240:.8) [my box] {5};

  \draw [] (0) to (5);
  \draw [] (2) to (4);
  \draw [] (4) to (5);
  \draw [] (0) to (3);
  \draw [] (1) to (4);
  \draw [] (1) to (5);

\end{tikzpicture}
\end{center}

\begin{center}
\begin{tikzpicture}
    [scale=1,
     my box/.style={draw,circle,minimum size=1.3em,inner sep=.1em,outer sep=.3em},
    ]

  \node at (3.5, 1) [right,align=center] {N=6 vertices, E=9 edges, \\[1ex]
                                          complement \\[1ex]
                                          {\ttfamily EHuw} \\[1ex]
                                          };

  \node (0) at (1,1) [my box] {0};
  \node (1) at (3,1) [my box] {1};
  \node (2) at (0,2) [my box] {2};
  \node (3) at (2,0) [my box] {3};
  \node (4) at (0,0) [my box] {4};
  \node (5) at (2,2) [my box] {5};

  \draw [] (3) to (4);
  \draw [] (2) to (4);
  \draw [] (3) to (5);
  \draw [] (0) to (5);
  \draw [] (2) to (5);
  \draw [] (0) to (4);
  \draw [] (1) to (5);
  \draw [] (0) to (3);
  \draw [] (1) to (3);

\end{tikzpicture}
\end{center}


%------------------------------------------------------------------------------
\section{N=20 Five Domnum Different}

\url{https://hog.grinvin.org/ViewGraphInfo.action?id=31101}

\smallskip

This tree has the following 5 domination quantities all different: \newline
domination number = 6, \newline
independent domination number = 7, \newline
perfect domination number = 8, \newline
semi-total domination number = 9, \newline
total domination number = 10.

Its n=20 vertices is the fewest where such a 5-different tree occurs, and it is the only such n=20.  The quantities are consecutive 6,7,8,9,10.  The independence number = 11 is different and consecutive too.


\begin{center}
\begin{tikzpicture}
    [scale=1.3,
     my box/.style={draw,circle,minimum size=1.3em,inner sep=.1em,outer sep=.3em},
    ]

  \node (00) at (0,0) [my box] {0};
  \node (01) at (-1,0) [my box] {1};
  \node (02) at (-1,-1) [my box] {2};
  \node (03) at (0,-1) [my box] {3};
  \node (04) at (0,-2) [my box] {4};
  \node (05) at (0+.25,-3) [my box] {5};
  \node (06) at (0-.25,-3) [my box] {6};
  \node (07) at (-1,-2) [my box] {7};
  \node (08) at (-1+.25,-3) [my box] {8};
  \node (09) at (-1-.25,-3) [my box] {9};

  \node (10) at (-2,0) [my box] {10};
  \node (11) at (-3,0) [my box] {11};
  \node (12) at (-4,0) [my box] {12};
  \node (13) at (-5,0) [my box] {13};
  \node (14) at (-1,1) [my box] {14};
  \node (15) at (1,0) [my box] {15};
  \node (16) at (2,0) [my box] {16};
  \node (17) at (3,0) [my box] {17};
  \node (18) at (4,0) [my box] {18};
  \node (19) at (0,1) [my box] {19};

  \draw [] (04) to (06);
  \draw [] (04) to (05);
  \draw [] (07) to (09);
  \draw [] (00) to (15);
  \draw [] (11) to (12);
  \draw [] (03) to (04);
  \draw [] (12) to (13);
  \draw [] (01) to (02);
  \draw [] (00) to (01);
  \draw [] (01) to (10);
  \draw [] (07) to (08);
  \draw [] (02) to (03);
  \draw [] (00) to (19);
  \draw [] (17) to (18);
  \draw [] (02) to (07);
  \draw [] (15) to (16);
  \draw [] (01) to (14);
  \draw [] (16) to (17);
  \draw [] (10) to (11);

\end{tikzpicture}
\end{center}

\begin{center}
\begin{tikzpicture}
    [scale=1.3,
     my box/.style={draw,circle,minimum size=1.3em,inner sep=.1em,outer sep=.3em},
    ]

  \node (00) at (0,0) [my box] {0};
  \node (01) at (-1,0) [my box] {1};
  \node (02) at (-2,0) [my box] {2};
  \node (03) at (-3,0) [my box] {3};
  \node (04) at (-4,0) [my box] {4};
  \node (05) at (-5,0) [my box] {5};
  \node (06) at (-4,-1) [my box] {6};
  \node (07) at (-2,-1) [my box] {7};
  \node (08) at (-2.5,-2) [my box] {8};
  \node (09) at (-2,-2) [my box] {9};

  \node (10) at (-1,-1) [my box] {10};
  \node (11) at (-1,-2) [my box] {11};
  \node (12) at (-1,-3) [my box] {12};
  \node (13) at (-1,-4) [my box] {13};
  \node (14) at (-1,1) [my box] {14};
  \node (15) at (1,0) [my box] {15};
  \node (16) at (2,0) [my box] {16};
  \node (17) at (3,0) [my box] {17};
  \node (18) at (4,0) [my box] {18};
  \node (19) at (0,1) [my box] {19};

  \draw [] (04) to (06);
  \draw [] (04) to (05);
  \draw [] (07) to (09);
  \draw [] (00) to (15);
  \draw [] (11) to (12);
  \draw [] (03) to (04);
  \draw [] (12) to (13);
  \draw [] (01) to (02);
  \draw [] (00) to (01);
  \draw [] (01) to (10);
  \draw [] (07) to (08);
  \draw [] (02) to (03);
  \draw [] (00) to (19);
  \draw [] (17) to (18);
  \draw [] (02) to (07);
  \draw [] (15) to (16);
  \draw [] (01) to (14);
  \draw [] (16) to (17);
  \draw [] (10) to (11);

\end{tikzpicture}
\end{center}

%------------------------------------------------------------------------------
\section{N=20 Seven Domnum Etc Different}

Different domnum, indomnum, perfect domnum, semitotdomnum, totdomnum,
indnum, matchnum.

Diameter 10 Min
\begin{center}
\begin{tikzpicture}
    [scale=1.3,
     my box/.style={draw,circle,minimum size=1.3em,inner sep=.1em,outer sep=.3em},
    ]

  \node (00) at (1,0) [my box] {00};
  \node (01) at (0,0) [my box] {01};
  \node (02) at (-1,-1) [my box] {02};
  \node (03) at (-1,-2) [my box] {03};
  \node (04) at (-1,-3) [my box] {04};
  \node (05) at (-1,-4) [my box] {05};
  \node (06) at (0,-1) [my box] {06};
  \node (07) at (0,-2) [my box] {07};
  \node (08) at (0,-3) [my box] {08};
  \node (09) at (0,-4) [my box] {09};
  \node (10) at (1,-1) [my box] {10};
  \node (11) at (1,-2) [my box] {11};
  \node (12) at (1,-3) [my box] {12};
  \node (13) at (1,-4) [my box] {13};
  \node (14) at (0,1) [my box] {14};
  \node (15) at (-1,2) [my box] {15};
  \node (16) at (0,2) [my box] {16};
  \node (17) at (1,2) [my box] {17};
  \node (18) at (2,-0) [my box] {18};
  \node (19) at (2,-1) [my box] {19};
  \node (20) at (2,-2) [my box] {20};
  \node (21) at (2,-3) [my box] {21};
  \node (22) at (2,-4) [my box] {22};
  \node (23) at (3,0) [my box] {23};

  \draw [] (10) to (11);
  \draw [] (00) to (18);
  \draw [] (01) to (10);
  \draw [] (01) to (02);
  \draw [] (00) to (01);
  \draw [] (02) to (03);
  \draw [] (01) to (14);
  \draw [] (12) to (13);
  \draw [] (18) to (23);
  \draw [] (03) to (04);
  \draw [] (14) to (16);
  \draw [] (04) to (05);
  \draw [] (14) to (15);
  \draw [] (21) to (22);
  \draw [] (14) to (17);
  \draw [] (08) to (09);
  \draw [] (01) to (06);
  \draw [] (11) to (12);
  \draw [] (07) to (08);
  \draw [] (18) to (19);
  \draw [] (20) to (21);
  \draw [] (19) to (20);
  \draw [] (06) to (07);

\end{tikzpicture}
\end{center}

Diameter 15 Max
\begin{center}
\begin{tikzpicture}
    [scale=.8,yscale=1.5,
     my box/.style={draw,circle,minimum size=1.3em,inner sep=.1em,outer sep=.3em},
    ]

  \node (00) at (0,0) [my box] {00};
  \node (01) at (1,0) [my box] {01};
  \node (02) at (2,0) [my box] {02};
  \node (03) at (3,0) [my box] {03};
  \node (04) at (4,0) [my box] {04};
  \node (05) at (5,0) [my box] {05};
  \node (06) at (6,0) [my box] {06};
  \node (07) at (7,0) [my box] {07};
  \node (08) at (8,0) [my box] {08};
  \node (09) at (4,-1) [my box] {09};
  \node (10) at (3,1) [my box] {10};
  \node (11) at (3,-1) [my box] {11};
  \node (12) at (2,-1) [my box] {12};
  \node (13) at (2+.3,-2) [my box] {13};
  \node (14) at (2-.3,-2) [my box] {14};
  \node (15) at (-1,0) [my box] {15};
  \node (16) at (-2,0) [my box] {16};
  \node (17) at (-3,0) [my box] {17};
  \node (18) at (-4,0) [my box] {18};
  \node (19) at (-5,0) [my box] {19};
  \node (20) at (-6,0) [my box] {20};
  \node (21) at (-7,0) [my box] {21};
  \node (22) at (-3,-1) [my box] {22};
  \node (23) at (0,-1) [my box] {23};

  \draw [] (00) to (23);
  \draw [] (05) to (06);
  \draw [] (10) to (03);
  \draw [] (12) to (14);
  \draw [] (20) to (21);
  \draw [] (02) to (03);
  \draw [] (03) to (04);
  \draw [] (06) to (07);
  \draw [] (12) to (02);
  \draw [] (04) to (05);
  \draw [] (07) to (08);
  \draw [] (16) to (17);
  \draw [] (00) to (15);
  \draw [] (00) to (01);
  \draw [] (11) to (03);
  \draw [] (04) to (09);
  \draw [] (19) to (20);
  \draw [] (17) to (22);
  \draw [] (18) to (19);
  \draw [] (12) to (13);
  \draw [] (17) to (18);
  \draw [] (15) to (16);
  \draw [] (01) to (02);

\end{tikzpicture}
\end{center}

%------------------------------------------------------------------------------
\section{Regular Cospectral}

First

\begin{center}
\begin{tikzpicture}
    [scale=2,
     my box/.style={draw,circle,minimum size=1.3em,inner sep=.1em,outer sep=.3em},
    ]

  \node (1) at (-1,0) [my box] {1};

  \node (6) at (.5, .5) [my box] {6};
  \node (8) at (.5, -.5) [my box] {8};
  \node (10) at (2,0) [my box] {10};

  \node (9) at (0,1) [my box] {9};
  \node (7) at (0,0) [my box] {7};
  \node (4) at (1,0) [my box] {4};
  \node (5) at (1,1) [my box] {5};

  \node (2) at (0,-1) [my box] {2};
  \node (3) at (1,-1) [my box] {3};

  \draw [] (4) to (7);
  \draw [] (3) to (8);
  \draw [] (3) to (10);
  \draw [] (6) to (9);
  \draw [] (2) to (3);
  \draw [] (2) to (7);
  \draw [] (5) to (6);
  \draw [] (1) to (9);
  \draw [] (5) to (10);
  \draw [] (5) to (9);
  \draw [] (7) to (9);
  \draw [] (2) to (8);
  \draw [] (3) to (4);
  \draw [] (1) to (7);
  \draw [] (6) to (8);
  \draw [] (4) to (10);
  \draw [] (10) to (8);
  \draw [] (4) to (5);
  \draw [] (1) to (2);
  \draw [] (1) to (6);

\end{tikzpicture}
\end{center}

\begin{center}
\begin{tikzpicture}
    [scale=2,
     my box/.style={draw,circle,minimum size=1.3em,inner sep=.1em,outer sep=.3em},
    ]

  \node (1) at (-1,0) [my box] {1};
  \node (2) at (-.5,-1) [my box] {2};

  \node (6) at (-1.5, .8) [my box] {6};
  \node (7) at (0,0) [my box] {7};
  \node (8) at (-1.5, -.8) [my box] {8};
  \node (9) at (-.5,1) [my box] {9};

  \node (3) at (1.5,-1) [my box] {3};
  \node (4) at (1,0) [my box] {4};
  \node (5) at (1.5,1) [my box] {5};
  \node (10) at (2,0) [my box] {10};

  \draw [] (4) to (7);
  \draw [] (3) to (8);
  \draw [] (3) to (10);
  \draw [] (6) to (9);
  \draw [] (2) to (3);
  \draw [] (2) to (7);
  \draw [] (5) to (6);
  \draw [] (1) to (9);
  \draw [] (5) to (10);
  \draw [] (5) to (9);
  \draw [] (7) to (9);
  \draw [] (2) to (8);
  \draw [] (3) to (4);
  \draw [] (1) to (7);
  \draw [] (6) to (8);
  \draw [] (4) to (10);
  \draw [] (10) to (8);
  \draw [] (4) to (5);
  \draw [] (1) to (2);
  \draw [] (1) to (6);

\end{tikzpicture}
\end{center}

Second

\begin{center}
\begin{tikzpicture}
    [scale=2,
     my box/.style={draw,circle,minimum size=1.3em,inner sep=.1em,outer sep=.3em},
    ]

  \node (1) at (-1, .5) [my box] {1};

  \node (6) at (0,1) [my box] {6};
  \node (8) at (0,0) [my box] {8};

  \node (10) at (-2,2) [my box] {10};

  \node (9) at (1.5, 0) [my box] {9};
  \node (7) at (.5, 0) [my box] {7};
  \node (4) at (3,1) [my box] {4};
  \node (5) at (1,1) [my box] {5};

  \node (2) at (0,-1) [my box] {2};
  \node (3) at (1,-1) [my box] {3};


  \draw [] (1) to (7);
  \draw [] (1) to (10);
  \draw [] (3) to (9);
  \draw [] (6) to (8);
  \draw [] (2) to (3);
  \draw [] (5) to (9);
  \draw [] (7) to (9);
  \draw [] (4) to (7);
  \draw [] (10) to (8);
  \draw [] (6) to (10);
  \draw [] (4) to (5);
  \draw [] (2) to (8);
  \draw [] (4) to (9);
  \draw [] (5) to (6);
  \draw [] (1) to (6);
  \draw [] (3) to (8);
  \draw [] (3) to (4);
  \draw [] (5) to (10);
  \draw [] (1) to (2);
  \draw [] (2) to (7);

\end{tikzpicture}
\end{center}


Third

\begin{center}
\begin{tikzpicture}
    [scale=2,
     my box/.style={draw,circle,minimum size=1.3em,inner sep=.1em,outer sep=.3em},
    ]

  \node (0) at (-.5,1) [my box] {0};
  \node (1) at (3,0) [my box] {1};
  \node (2) at (0,2.5) [my box] {2};
  \node (3) at (.5,1) [my box] {3};
  \node (4) at (0,0) [my box] {4};
  \node (5) at (1.5,1) [my box] {5};
  \node (6) at (2,2.5) [my box] {6};
  \node (7) at (1,3) [my box] {7};
  \node (8) at (1,2) [my box] {8};
  \node (9) at (2,.5) [my box] {9};

  \draw [] (5) to (7);
  \draw [] (1) to (4);
  \draw [] (2) to (7);
  \draw [] (1) to (9);
  \draw [] (6) to (8);
  \draw [] (3) to (9);
  \draw [] (2) to (3);
  \draw [] (0) to (1);
  \draw [] (6) to (9);
  \draw [] (4) to (5);
  \draw [] (7) to (8);
  \draw [] (2) to (8);
  \draw [] (0) to (2);
  \draw [] (3) to (4);
  \draw [] (5) to (9);
  \draw [] (0) to (4);
  \draw [] (1) to (6);
  \draw [] (5) to (8);
  \draw [] (6) to (7);
  \draw [] (0) to (3);

\end{tikzpicture}
\end{center}

Fourth

\begin{center}
\begin{tikzpicture}
    [scale=2,
     my box/.style={draw,circle,minimum size=1.3em,inner sep=.1em,outer sep=.3em},
    ]

  \node (2) at (0,-2) [my box] {2};
  \node (7) at (1,-2) [my box] {7};
  \node (4) at (0,-1) [my box] {4};
  \node (5) at (1,-1) [my box] {5};
  \node (3) at (0,0) [my box] {3};
  \node (9) at (1,0) [my box] {9};

  \node (0) at (-1, .5) [my box] {0};
  \node (1) at (1,2) [my box] {1};
  \node (6) at (2,.5) [my box] {6};
  \node (8) at (1.5, 1.5) [my box] {8};

  \draw [] (1) to (9);
  \draw [] (5) to (7);
  \draw [] (6) to (7);
  \draw [] (6) to (8);
  \draw [] (7) to (8);
  \draw [] (3) to (4);
  \draw [] (2) to (6);
  \draw [] (4) to (5);
  \draw [] (2) to (4);
  \draw [] (5) to (9);
  \draw [] (5) to (8);
  \draw [] (6) to (9);
  \draw [] (0) to (2);
  \draw [] (1) to (8);
  \draw [] (1) to (3);
  \draw [] (0) to (1);
  \draw [] (3) to (9);
  \draw [] (2) to (7);
  \draw [] (0) to (3);
  \draw [] (0) to (4);


\end{tikzpicture}
\end{center}


%------------------------------------------------------------------------------
\section{n=32 Tree Most Minimal Dominating Sets}

\url{https://hog.grinvin.org/ViewGraphInfo.action?id=30700}

\smallskip

This tree has 65960 minimal dominating sets which is the most of any
n=32 tree.  This is the smallest even n with more than $2^(n/2)$ sets,
and this tree is the only n=32 with more than $2^(n/2) = 65536$.

This tree can be taken as a bi-star 4,5 and bi-star 4,4 which are
subdivided and then connected by an edge from middle vertex of the
4-star in the first to middle vertex of 4-star in the other.


\begin{center}
\begin{tikzpicture}
    [scale=.8,
     my box/.style={draw,circle,minimum size=1.2em,inner sep=.1em,outer sep=.3em},
     font=\scriptsize,
    ]
  \newcommand\MyRight{5}

  \node (1) at (-4.5,0) [my box] {1};
  \node (27) at (-5,-1) [my box] {27};
  \node (28) at (-5,-2) [my box] {28};
  \node (29) at (-4,-1) [my box] {29};
  \node (30) at (-4,-2) [my box] {30};
  \node (31) at (-3,-1) [my box] {31};
  \node (32) at (-3,-2) [my box] {32};

  \node (19) at (-6,-1) [my box] {19};
  \node (20) at (-6,-2) [my box] {20};

  \node (21) at (-7,-3) [my box] {21};
  \node (22) at (-7,-4) [my box] {22};
  \node (23) at (-6,-3) [my box] {23};
  \node (24) at (-6,-4) [my box] {24};
  \node (25) at (-5,-3) [my box] {25};
  \node (26) at (-5,-4) [my box] {26};

  \begin{scope}[shift={(-.5,0)}]
  \node (2) at (.5,0) [my box] {2};
  \node (13) at (-1,-1) [my box] {13};
  \node (14) at (-1,-2) [my box] {14};
  \node (15) at (0,-1) [my box] {15};
  \node (16) at (0,-2) [my box] {16};
  \node (17) at (1,-1) [my box] {17};
  \node (18) at (1,-2) [my box] {18};
  \node (3) at (2,-1) [my box] {3};
  \node (4) at (2,-2) [my box] {4};

  \node (5)  at (.5,-3) [my box] {5};
  \node (6)  at (.5,-4) [my box] {6};
  \node (7)  at (1.5,-3) [my box] {7};
  \node (8)  at (1.5,-4) [my box] {8};
  \node (9)  at (2.5,-3) [my box] {9};
  \node (10) at (2.5,-4) [my box] {10};
  \node (11) at (3.5,-3) [my box] {11};
  \node (12) at (3.5,-4) [my box] {12};
  \end{scope}

  \draw [] (1) to (29);
  \draw [] (5) to (6);
  \draw [] (21) to (22);
  \draw [] (7) to (8);
  \draw [] (17) to (18);
  \draw [] (2) to (17);
  \draw [] (20) to (23);
  \draw [] (1) to (2);
  \draw [] (1) to (27);
  \draw [] (9) to (10);
  \draw [] (19) to (20);
  \draw [] (20) to (21);
  \draw [] (3) to (4);
  \draw [] (13) to (14);
  \draw [] (25) to (26);
  \draw [] (1) to (31);
  \draw [] (2) to (3);
  \draw [] (4) to (9);
  \draw [] (11) to (12);
  \draw [] (4) to (7);
  \draw [] (27) to (28);
  \draw [] (23) to (24);
  \draw [] (2) to (13);
  \draw [] (1) to (19);
  \draw [] (4) to (5);
  \draw [] (15) to (16);
  \draw [] (2) to (15);
  \draw [] (29) to (30);
  \draw [] (20) to (25);
  \draw [] (11) to (4);
  \draw [] (31) to (32);

\end{tikzpicture}
\end{center}

\bigskip

\begin{center}
\begin{tikzpicture}
    [scale=.8,yscale=-1,
     rotate=90,
     my box/.style={draw,circle,minimum size=1.2em,inner sep=.1em,outer sep=.3em},
     font=\scriptsize,
    ]
  \newcommand\MyRight{5}

  \begin{scope}[shift={(\MyRight,0)}]
  \node (1) at (0,0) [my box] {1};
  \node (27) at (45+135/2:1) [my box] {27};
  \node (28) at (45+135/2:2) [my box] {28};
  \node (29) at (45:1) [my box] {29};
  \node (30) at (45:2) [my box] {30};
  \node (31) at (45-135/2:1) [my box] {31};
  \node (32) at (45-135/2:2) [my box] {32};
  \end{scope}

  \begin{scope}[shift={(\MyRight,-4)}]
  \node (20) at (0,0) [my box] {20};
  \node (21) at (90+1*90:1) [my box] {21};
  \node (22) at (90+1*90:2) [my box] {22};
  \node (23) at (90+2*90:1) [my box] {23};
  \node (24) at (90+2*90:2) [my box] {24};
  \node (25) at (90+3*90:1) [my box] {25};
  \node (26) at (90+3*90:2) [my box] {26};

  \node (19) at (0,2) [my box] {19};
  \end{scope}

  \begin{scope}[shift={(0,0)}]
  \node (2) at (0,0) [my box] {2};
  \node (13) at (-10+1*70:1) [my box] {13};
  \node (14) at (-10+1*70:2) [my box] {14};
  \node (15) at (-10+2*70:1) [my box] {15};
  \node (16) at (-10+2*70:2) [my box] {16};
  \node (17) at (-10+3*70:1) [my box] {17};
  \node (18) at (-10+3*70:2) [my box] {18};
  \end{scope}

  \node (3) at (0,-2) [my box] {3};

  \begin{scope}[shift={(0,-4)}]
  \node (4) at (0,0) [my box] {4};
  \node (5)  at (90+70:1) [my box] {5};
  \node (6)  at (90+70:2) [my box] {6};
  \node (7)  at (-90+30:1) [my box] {7};
  \node (8)  at (-90+30:2) [my box] {8};
  \node (9)  at (-90-30:1) [my box] {9};
  \node (10) at (-90-30:2) [my box] {10};
  \node (11) at (90-70:1) [my box] {11};
  \node (12) at (90-70:2) [my box] {12};
  \end{scope}

  \draw [] (1) to (29);
  \draw [] (5) to (6);
  \draw [] (21) to (22);
  \draw [] (7) to (8);
  \draw [] (17) to (18);
  \draw [] (2) to (17);
  \draw [] (20) to (23);
  \draw [] (1) to (2);
  \draw [] (1) to (27);
  \draw [] (9) to (10);
  \draw [] (19) to (20);
  \draw [] (20) to (21);
  \draw [] (3) to (4);
  \draw [] (13) to (14);
  \draw [] (25) to (26);
  \draw [] (1) to (31);
  \draw [] (2) to (3);
  \draw [] (4) to (9);
  \draw [] (11) to (12);
  \draw [] (4) to (7);
  \draw [] (27) to (28);
  \draw [] (23) to (24);
  \draw [] (2) to (13);
  \draw [] (1) to (19);
  \draw [] (4) to (5);
  \draw [] (15) to (16);
  \draw [] (2) to (15);
  \draw [] (29) to (30);
  \draw [] (20) to (25);
  \draw [] (11) to (4);
  \draw [] (31) to (32);

\end{tikzpicture}
\end{center}

\bigskip

\begin{center}
\begin{tikzpicture}
    [scale=1,
     my box/.style={draw,circle,minimum size=1.5em,inner sep=.1em,outer sep=.3em},
    ]

  \begin{scope}[shift={(6,0)}]
  \node (1) at (0,0) [my box] {1};
  \node (27) at (45+135/2:1) [my box] {27};
  \node (28) at (45+135/2:2) [my box] {28};
  \node (29) at (45:1) [my box] {29};
  \node (30) at (45:2) [my box] {30};
  \node (31) at (45-135/2:1) [my box] {31};
  \node (32) at (45-135/2:2) [my box] {32};
  \end{scope}

  \begin{scope}[shift={(6,-4)}]
  \node (20) at (0,0) [my box] {20};
  \node (21) at (90+1*90:1) [my box] {21};
  \node (22) at (90+1*90:2) [my box] {22};
  \node (23) at (90+2*90:1) [my box] {23};
  \node (24) at (90+2*90:2) [my box] {24};
  \node (25) at (90+3*90:1) [my box] {25};
  \node (26) at (90+3*90:2) [my box] {26};

  \node (19) at (0,2) [my box] {19};
  \end{scope}

  \begin{scope}[shift={(0,0)}]
  \node (2) at (0,0) [my box] {2};
  \node (13) at (-10+1*70:1) [my box] {13};
  \node (14) at (-10+1*70:2) [my box] {14};
  \node (15) at (-10+2*70:1) [my box] {15};
  \node (16) at (-10+2*70:2) [my box] {16};
  \node (17) at (-10+3*70:1) [my box] {17};
  \node (18) at (-10+3*70:2) [my box] {18};
  \end{scope}

  \node (3) at (0,-2) [my box] {3};

  \begin{scope}[shift={(0,-4)}]
  \node (4) at (0,0) [my box] {4};
  \node (5)  at (90+70:1) [my box] {5};
  \node (6)  at (90+70:2) [my box] {6};
  \node (7)  at (-90+30:1) [my box] {7};
  \node (8)  at (-90+30:2) [my box] {8};
  \node (9)  at (-90-30:1) [my box] {9};
  \node (10) at (-90-30:2) [my box] {10};
  \node (11) at (90-70:1) [my box] {11};
  \node (12) at (90-70:2) [my box] {12};
  \end{scope}

  \draw [] (1) to (29);
  \draw [] (5) to (6);
  \draw [] (21) to (22);
  \draw [] (7) to (8);
  \draw [] (17) to (18);
  \draw [] (2) to (17);
  \draw [] (20) to (23);
  \draw [] (1) to (2);
  \draw [] (1) to (27);
  \draw [] (9) to (10);
  \draw [] (19) to (20);
  \draw [] (20) to (21);
  \draw [] (3) to (4);
  \draw [] (13) to (14);
  \draw [] (25) to (26);
  \draw [] (1) to (31);
  \draw [] (2) to (3);
  \draw [] (4) to (9);
  \draw [] (11) to (12);
  \draw [] (4) to (7);
  \draw [] (27) to (28);
  \draw [] (23) to (24);
  \draw [] (2) to (13);
  \draw [] (1) to (19);
  \draw [] (4) to (5);
  \draw [] (15) to (16);
  \draw [] (2) to (15);
  \draw [] (29) to (30);
  \draw [] (20) to (25);
  \draw [] (11) to (4);
  \draw [] (31) to (32);

\end{tikzpicture}
\end{center}


%------------------------------------------------------------------------------
\section{N=34 Cross-Connected Halves, Most Minimal Dominating Sets}

\subsection{Two Subdivided Bi-Star 4,5 Connected at 4-Star Middles}

\url{https://hog.grinvin.org/ViewGraphInfo.action?id=30703}

\smallskip

This tree has 134432 minimal dominating sets which is the most of any
n=34 tree, and this is the only n=34 tree with this many.

It comprises two identical halves cross-connected at the same vertex.
Each half is a bi-star 4,5 subdivided.  The cross connection is
between the middle of the 4-star of each.

n=34 vertices is the smallest where a tree of same-connected halves
like this gives more than $2^(n/2)$ minimal dominating sets.  This tree
is one of two such n=34.  The other has the connection between the
middle of the 5-stars.

\begin{center}
\begin{tikzpicture}
    [scale=.8,
     my box/.style={draw,circle,minimum size=1.2em,inner sep=.1em,outer sep=.3em},
     font=\scriptsize,
    ]
  \newcommand\MyRight{5}

  \node (1) at (-4.5,0) [my box] {1};
  \node (27) at (-5,-1) [my box] {27};
  \node (28) at (-5,-2) [my box] {28};
  \node (29) at (-4,-1) [my box] {29};
  \node (30) at (-4,-2) [my box] {30};
  \node (31) at (-3,-1) [my box] {31};
  \node (32) at (-3,-2) [my box] {32};

  \node (19) at (-6,-1) [my box] {19};
  \node (20) at (-6,-2) [my box] {20};

  \node (21) at (-7.5,-3) [my box] {21};
  \node (22) at (-7.5,-4) [my box] {22};
  \node (23) at (-6.5,-3) [my box] {23};
  \node (24) at (-6.5,-4) [my box] {24};
  \node (25) at (-5.5,-3) [my box] {25};
  \node (26) at (-5.5,-4) [my box] {26};
  \node (33) at (-4.5,-3) [my box] {33};
  \node (34) at (-4.5,-4) [my box] {34};

  \begin{scope}[shift={(-1+.25,0)}]
  \node (2) at (.5,0) [my box] {2};
  \node (13) at (-1,-1) [my box] {13};
  \node (14) at (-1,-2) [my box] {14};
  \node (15) at (0,-1) [my box] {15};
  \node (16) at (0,-2) [my box] {16};
  \node (17) at (1,-1) [my box] {17};
  \node (18) at (1,-2) [my box] {18};
  \node (3) at (2,-1) [my box] {3};
  \node (4) at (2,-2) [my box] {4};

  \node (5)  at (.5,-3) [my box] {5};
  \node (6)  at (.5,-4) [my box] {6};
  \node (7)  at (1.5,-3) [my box] {7};
  \node (8)  at (1.5,-4) [my box] {8};
  \node (9)  at (2.5,-3) [my box] {9};
  \node (10) at (2.5,-4) [my box] {10};
  \node (11) at (3.5,-3) [my box] {11};
  \node (12) at (3.5,-4) [my box] {12};
  \end{scope}

  \draw [] (1) to (29);
  \draw [] (5) to (6);
  \draw [] (21) to (22);
  \draw [] (7) to (8);
  \draw [] (17) to (18);
  \draw [] (2) to (17);
  \draw [] (20) to (23);
  \draw [] (1) to (2);
  \draw [] (1) to (27);
  \draw [] (9) to (10);
  \draw [] (19) to (20);
  \draw [] (20) to (21);
  \draw [] (3) to (4);
  \draw [] (13) to (14);
  \draw [] (25) to (26);
  \draw [] (1) to (31);
  \draw [] (2) to (3);
  \draw [] (4) to (9);
  \draw [] (11) to (12);
  \draw [] (4) to (7);
  \draw [] (27) to (28);
  \draw [] (23) to (24);
  \draw [] (2) to (13);
  \draw [] (1) to (19);
  \draw [] (4) to (5);
  \draw [] (15) to (16);
  \draw [] (2) to (15);
  \draw [] (29) to (30);
  \draw [] (20) to (25);
  \draw [] (11) to (4);
  \draw [] (31) to (32);

  \draw [] (20) to (33);
  \draw [] (33) to (34);

\end{tikzpicture}
\end{center}

%----------------
\subsection{Two Subdivided Bi-Star 4,5 Connected at 4-Star Middles}

\url{https://hog.grinvin.org/ViewGraphInfo.action?id=30705}

\smallskip

This tree has 132400 minimal dominating sets which is the second most
for an n=34 tree comprising two identical halves cross-connected at
the same vertex.  Each half is a bi-star 4,5 subdivided.  The cross
connection is between the middle of the 5-star of each.

n=34 vertices is the smallest where a tree of same-connected halves
like this gives more than $2^(n/2)$ minimal dominating sets.  This
tree is one of two such n=34.  The other has the connection between
the middle of the 4-stars.

\begin{center}
\begin{tikzpicture}
    [scale=.8,
     my box/.style={draw,circle,minimum size=1.2em,inner sep=.1em,outer sep=.3em},
     font=\scriptsize,
    ]
  \newcommand\MyRight{5}

  \node (19) at (-6,0) [my box] {19};

  \node (18) at (-8,-1) [my box] {18};
  \node (28) at (-8,-2) [my box] {28};
  \node (20) at (-7,-1) [my box] {20};
  \node (21) at (-7,-2) [my box] {21};
  \node (22) at (-6,-1) [my box] {22};
  \node (23) at (-6,-2) [my box] {23};
  \node (24) at (-5,-1) [my box] {24};
  \node (25) at (-5,-2) [my box] {25};
  \node (26) at (-4,-1) [my box] {26};
  \node (27) at (-4,-2) [my box] {27};


  \node (29) at (-9,-3) [my box] {29};
  \node (30) at (-9,-4) [my box] {30};
  \node (31) at (-8,-3) [my box] {31};
  \node (32) at (-8,-4) [my box] {32};
  \node (33) at (-7,-3) [my box] {33};
  \node (34) at (-7,-4) [my box] {34};


  \begin{scope}[shift={(-2+.25,0)}]
  \node (2) at (1,0) [my box] {2};

  \node (3)  at (-1,-1) [my box] {3};
  \node (4)  at (-1,-2) [my box] {4};
  \node (5)  at (0,-1) [my box] {5};
  \node (6)  at (0,-2) [my box] {6};
  \node (7)  at (1,-1) [my box] {7};
  \node (8)  at (1,-2) [my box] {8};
  \node (9)  at (2,-1) [my box] {9};
  \node (10) at (2,-2) [my box] {10};
  \node (1)  at (3,-1) [my box] {1};
  \node (11) at (3,-2) [my box] {11};

  \node (12) at (2,-3) [my box] {12};
  \node (13) at (2,-4) [my box] {13};
  \node (14) at (3,-3) [my box] {14};
  \node (15) at (3,-4) [my box] {15};
  \node (16) at (4,-3) [my box] {16};
  \node (17) at (4,-4) [my box] {17};

  \end{scope}


  \draw [] (2) to (5);
  \draw [] (2) to (3);
  \draw [] (14) to (15);
  \draw [] (5) to (6);
  \draw [] (16) to (17);
  \draw [] (19) to (22);
  \draw [] (1) to (11);
  \draw [] (2) to (9);
  \draw [] (2) to (7);
  \draw [] (18) to (28);
  \draw [] (28) to (31);
  \draw [] (2) to (19);
  \draw [] (1) to (2);
  \draw [] (19) to (26);
  \draw [] (33) to (34);
  \draw [] (9) to (10);
  \draw [] (31) to (32);
  \draw [] (11) to (14);
  \draw [] (11) to (12);
  \draw [] (28) to (29);
  \draw [] (20) to (21);
  \draw [] (3) to (4);
  \draw [] (24) to (25);
  \draw [] (28) to (33);
  \draw [] (19) to (20);
  \draw [] (22) to (23);
  \draw [] (19) to (24);
  \draw [] (7) to (8);
  \draw [] (11) to (16);
  \draw [] (12) to (13);
  \draw [] (29) to (30);
  \draw [] (26) to (27);
  \draw [] (18) to (19);

\end{tikzpicture}
\end{center}


%------------------------------------------------------------------------------
\section{Graphs Most Minimal Dominating Sets}

\subsection{N=6}

Octahedral.

\url{https://hog.grinvin.org/ViewGraphInfo.action?id=226}
\begin{center}
\begin{tikzpicture}
    [scale=2,
     my box/.style={draw,circle,minimum size=1.2em,inner sep=.1em,outer sep=.1em},
     font=\normalsize,
    ]

  \node (0) at (5*60:1) [my box] {0};
  \node (1) at (2*60:1) [my box] {1};
  \node (2) at (1*60:1) [my box] {2};
  \node (3) at (4*60:1) [my box] {3};
  \node (4) at (3*60:1) [my box] {4};
  \node (5) at (0*60:1) [my box] {5};

  \draw [] (0) to (4);
  \draw [] (1) to (4);
  \draw [] (2) to (4);
  \draw [] (1) to (2);
  \draw [] (3) to (4);
  \draw [] (3) to (5);
  \draw [] (2) to (5);
  \draw [] (0) to (2);
  \draw [] (0) to (5);
  \draw [] (1) to (3);
  \draw [] (1) to (5);
  \draw [] (0) to (3);

\end{tikzpicture}
\end{center}

\begin{center}
\begin{tikzpicture}
    [scale=2,
     my box/.style={draw,circle,minimum size=1.2em,inner sep=.1em,outer sep=.1em},
     font=\normalsize,
    ]

  \node (0) at (0*60:1) [my box] {0};
  \node (1) at (1*60:1) [my box] {1};
  \node (2) at (2*60:1) [my box] {2};
  \node (3) at (3*60:1) [my box] {3};
  \node (4) at (4*60:1) [my box] {4};
  \node (5) at (5*60:1) [my box] {5};

  \draw [] (3) to (4);
  \draw [] (0) to (5);
  \draw [] (2) to (4);
  \draw [] (4) to (5);
  \draw [] (1) to (5);
  \draw [] (0) to (4);
  \draw [] (0) to (1);
  \draw [] (1) to (3);
  \draw [] (0) to (2);
  \draw [] (3) to (5);
  \draw [] (2) to (3);
  \draw [] (1) to (2);

\end{tikzpicture}
\end{center}


\subsection{N=7}

\url{https://hog.grinvin.org/ViewGraphInfo.action?id=868}
\begin{center}
\begin{tikzpicture}
    [scale=1.5,
     my box/.style={draw,circle,minimum size=1.2em,inner sep=.1em,outer sep=.1em},
     font=\footnotesize,
    ]

  \node (0) at (0,0) [my box] {0};
  \node (1) at (2,0) [my box] {1};
  \node (2) at (2,1) [my box] {2};
  \node (3) at (0,1) [my box] {3};
  \node (4) at (2.8,.5) [my box] {4};
  \node (5) at (1,0) [my box] {5};
  \node (6) at (1,1) [my box] {6};

  \draw [] (2) to (4);
  \draw [] (2) to (6);
  \draw [] (0) to (5);
  \draw [] (1) to (5);
  \draw [] (1) to (4);
  \draw [] (3) to (6);
  \draw [] (5) to (6);
  \draw [] (2) to (5);
  \draw [] (1) to (6);
  \draw [] (0) to (3);

\end{tikzpicture}
\end{center}
\begin{center}
\begin{tikzpicture}
    [scale=1.5,
     my box/.style={draw,circle,minimum size=1.2em,inner sep=.1em,outer sep=.1em},
     font=\footnotesize,
     rotate=-90,
    ]

  \node (0) at (-1,0) [my box] {0};
  \node (1) at (0,3.5) [my box] {1};
  \node (2) at (0,1.7) [my box] {2};
  \node (3) at (1,0) [my box] {3};
  \node (4) at (0,2.5) [my box] {4};
  \node (5) at (-1,1) [my box] {5};
  \node (6) at (1,1) [my box] {6};

  \draw [] (2) to (4);
  \draw [] (2) to (6);
  \draw [] (0) to (5);
  \draw [] (1) to (5);
  \draw [] (1) to (4);
  \draw [] (3) to (6);
  \draw [] (5) to (6);
  \draw [] (2) to (5);
  \draw [] (1) to (6);
  \draw [] (0) to (3);

\end{tikzpicture}
\end{center}

\needspace{4\baselineskip}

\subsection{N=8 Connected}

Same most minimum domsets.

\url{https://hog.grinvin.org/ViewGraphInfo.action?id=30664}

\smallskip

This graph has 36 minimal dominating sets which is the most of any
n=8.  It is one of two n=8 graphs with this many, the other being two
disconnected 4-cycles.  The 4-cycle has most minimal dominating sets
of n=4.  The present graph can be taken as two 4-cycles
cross-connected by 6 edges.

\smallskip

3-7-1-5 cycle and others 0-4-2-6 cycle

\begin{center}
\begin{tikzpicture}
    [scale=.9,
     my box/.style={draw,circle,minimum size=1.2em,inner sep=.1em,outer sep=.1em},
     font=\footnotesize,
    ]

  \node (3) at (-1.5,0) [my box] {3};
  \node (7) at (-1.5,1) [my box] {7};
  \node (1) at (-2,2) [my box] {1};
  \node (5) at (-2,-1) [my box] {5};

  \node (4) at (1.5,1) [my box] {4};
  \node (0) at (1.5,0) [my box] {0};
  \node (6) at (2,-1) [my box] {6};
  \node (2) at (2,2) [my box] {2};

  \draw [] (0) to (6);
  \draw [] (2) to (5);
  \draw [] (1) to (7);
  \draw [] (2) to (6);
  \draw [] (0) to (3);
  \draw [] (1) to (5);
  \draw [] (1) to (6);
  \draw [] (0) to (4);
  \draw [] (2) to (4);
  \draw [] (2) to (7);
  \draw [] (1) to (4);
  \draw [] (3) to (7);
  \draw [] (3) to (5);
  \draw [] (4) to (7);

\end{tikzpicture}
\hspace{1em}
\begin{tikzpicture}
    [scale=.9,
     my box/.style={draw,circle,minimum size=1.2em,inner sep=.1em,outer sep=.1em},
     font=\footnotesize,
    ]

  \node (3) at (-1.5,0) [my box] {3};
  \node (7) at (-1.5,1) [my box] {7};
  \node (1) at (-2,2) [my box] {1};
  \node (5) at (-2,-1) [my box] {5};

  \node (2) at (1.5,1) [my box] {4};
  \node (6) at (1.5,0) [my box] {0};
  \node (0) at (2,-1) [my box] {6};
  \node (4) at (2,2) [my box] {2};

  \draw [] (0) to (6);
  \draw [] (2) to (5);
  \draw [] (1) to (7);
  \draw [] (2) to (6);
  \draw [] (0) to (3);
  \draw [] (1) to (5);
  \draw [] (1) to (6);
  \draw [] (0) to (4);
  \draw [] (2) to (4);
  \draw [] (2) to (7);
  \draw [] (1) to (4);
  \draw [] (3) to (7);
  \draw [] (3) to (5);
  \draw [] (4) to (7);

\end{tikzpicture}
\end{center}

3-7-4-0 cycle and others 1-5-2-6 cycle
\begin{center}
\begin{tikzpicture}
    [scale=1,
     my box/.style={draw,circle,minimum size=1.2em,inner sep=.1em,outer sep=.3em},
    ]

  \node (0) at (-2,-1) [my box] {0};
  \node (3) at (-1,0) [my box] {3};
  \node (7) at (-1,1) [my box] {7};
  \node (4) at (-2,2) [my box] {4};

  \node (6) at (1,1) [my box] {1};
  \node (2) at (1,0) [my box] {5};
  \node (5) at (2,-1) [my box] {2};
  \node (1) at (2,2) [my box] {6};

  \draw [] (0) to (6);
  \draw [] (2) to (5);
  \draw [] (1) to (7);
  \draw [] (2) to (6);
  \draw [] (0) to (3);
  \draw [] (1) to (5);
  \draw [] (1) to (6);
  \draw [] (0) to (4);
  \draw [] (2) to (4);
  \draw [] (2) to (7);
  \draw [] (1) to (4);
  \draw [] (3) to (7);
  \draw [] (3) to (5);
  \draw [] (4) to (7);

\end{tikzpicture}
\end{center}

F degree 4s
\begin{center}
\begin{tikzpicture}
    [scale=2,
     my box/.style={draw,circle,minimum size=1.2em,inner sep=.1em,outer sep=.3em},
     rotate=3*360/16,
    ]

  \node (0) at (5*360/8:1) [my box] {0};
  \node (1) at (2*360/8:1) [my box] {1F};
  \node (2) at (7*360/8:1) [my box] {2F};
  \node (3) at (4*360/8:1) [my box] {3};
  \node (4) at (1*360/8:1) [my box] {4F};
  \node (5) at (3*360/8:1) [my box] {5};
  \node (6) at (6*360/8:1) [my box] {6};
  \node (7) at (0*360/8:1) [my box] {7F};

  \draw [] (0) to (6);
  \draw [] (2) to (5);
  \draw [] (1) to (7);
  \draw [] (2) to (6);
  \draw [] (0) to (3);
  \draw [] (1) to (5);
  \draw [] (1) to (6);
  \draw [] (0) to (4);
  \draw [] (2) to (4);
  \draw [] (2) to (7);
  \draw [] (1) to (4);
  \draw [] (3) to (7);
  \draw [] (3) to (5);
  \draw [] (4) to (7);

\end{tikzpicture}
\end{center}

\begin{center}
\begin{tikzpicture}
    [scale=2,
     my box/.style={draw,circle,minimum size=1.2em,inner sep=.1em,outer sep=.3em},
     rotate=7*360/16,
    ]

  \node (0) at (5*360/8:1) [my box] {0};
  \node (1) at (6*360/8:1) [my box] {1F};
  \node (2) at (7*360/8:1) [my box] {2F};
  \node (3) at (0*360/8:1) [my box] {3};
  \node (4) at (1*360/8:1) [my box] {4F};
  \node (5) at (3*360/8:1) [my box] {5};
  \node (6) at (2*360/8:1) [my box] {6};
  \node (7) at (4*360/8:1) [my box] {7F};

  \draw [] (0) to (6);
  \draw [] (2) to (5);
  \draw [] (1) to (7);
  \draw [] (2) to (6);
  \draw [] (0) to (3);
  \draw [] (1) to (5);
  \draw [] (1) to (6);
  \draw [] (0) to (4);
  \draw [] (2) to (4);
  \draw [] (2) to (7);
  \draw [] (1) to (4);
  \draw [] (3) to (7);
  \draw [] (3) to (5);
  \draw [] (4) to (7);

\end{tikzpicture}
\end{center}

\begin{center}
\begin{tikzpicture}
    [scale=1,
     my box/.style={draw,circle,minimum size=1.2em,inner sep=.1em,outer sep=.3em},
    ]

  \node (0) at (0,3) [my box] {0};
  \node (1) at (-1,1) [my box] {1};
  \node (2) at (1,-2) [my box] {2};
  \node (3) at (0,0) [my box] {3};
  \node (4) at (1,3) [my box] {4};
  \node (5) at (-1,0) [my box] {5};
  \node (6) at (-3,1) [my box] {6};
  \node (7) at (4,1) [my box] {7};

  \draw [] (0) to (6);
  \draw [] (2) to (5);
  \draw [] (1) to (7);
  \draw [] (2) to (6);
  \draw [] (0) to (3);
  \draw [] (1) to (5);
  \draw [] (1) to (6);
  \draw [] (0) to (4);
  \draw [] (2) to (4);
  \draw [] (2) to (7);
  \draw [] (1) to (4);
  \draw [] (3) to (7);
  \draw [] (3) to (5);
  \draw [] (4) to (7);

\end{tikzpicture}
\end{center}

\subsubsection{N=8 Connected, Complement}

\begin{center}
\begin{tikzpicture}
    [scale=2,
     my box/.style={draw,circle,minimum size=1.2em,inner sep=.1em,outer sep=.3em},
     rotate=3*360/16,
    ]

  \node (0) at (6*360/8:1) [my box] {0};
  \node (1) at (4*360/8:1) [my box] {1F};
  \node (2) at (5*360/8:1) [my box] {2F};
  \node (3) at (3*360/8:1) [my box] {3};
  \node (4) at (2*360/8:1) [my box] {4F};
  \node (5) at (1*360/8:1) [my box] {5};
  \node (6) at (0*360/8:1) [my box] {6};
  \node (7) at (7*360/8:1) [my box] {7F};

  \draw [] (7) to (0);
  \draw [] (5) to (4);
  \draw [] (2) to (1);
  \draw [] (5) to (0);
  \draw [] (6) to (3);
  \draw [] (0) to (2);
  \draw [] (3) to (2);
  \draw [] (6) to (4);
  \draw [] (3) to (1);
  \draw [] (3) to (4);
  \draw [] (6) to (7);
  \draw [] (6) to (5);
  \draw [] (0) to (1);
  \draw [] (7) to (5);

\end{tikzpicture}
\end{center}

\begin{center}
\begin{tikzpicture}
    [scale=2,
     my box/.style={draw,circle,minimum size=1.2em,inner sep=.1em,outer sep=.3em},
     rotate=3*360/16,
    ]

  \node (0) at (5*360/8:1) [my box] {0};
  \node (1) at (2*360/8:1) [my box] {1F};
  \node (2) at (7*360/8:1) [my box] {2F};
  \node (3) at (4*360/8:1) [my box] {3};
  \node (4) at (1*360/8:1) [my box] {4F};
  \node (5) at (3*360/8:1) [my box] {5};
  \node (6) at (6*360/8:1) [my box] {6};
  \node (7) at (0*360/8:1) [my box] {7F};

  \draw [] (7) to (0);
  \draw [] (5) to (4);
  \draw [] (2) to (1);
  \draw [] (5) to (0);
  \draw [] (6) to (3);
  \draw [] (0) to (2);
  \draw [] (3) to (2);
  \draw [] (6) to (4);
  \draw [] (3) to (1);
  \draw [] (3) to (4);
  \draw [] (6) to (7);
  \draw [] (6) to (5);
  \draw [] (0) to (1);
  \draw [] (7) to (5);

\end{tikzpicture}
\end{center}


%-----------------
\needspace{6\baselineskip}
\subsection{N=9 Connected First, 18 edges}

\url{https://hog.grinvin.org/ViewGraphInfo.action?id=30667}

\smallskip

This graph has 51 minimal dominating sets which is the most of any
connected n=9 vertices.  It is one of three connected n=9 with this
many.  It can be taken as n=4 (cycle) and n=5 most minimal dominating
sets graphs cross-connected by 7 edges.

\smallskip

num edges 18, degrees 3,4,4,4,4,4,4,4,5
\begin{center}
\begin{tikzpicture}
    [scale=2,
     my box/.style={draw,circle,minimum size=1.2em,inner sep=.1em,outer sep=.1em},
     font=\normalsize,
     rotate=-7*360/36,
    ]

  \node (0) at (3*360/9:1) [my box] {0};
  \node (1) at (1*360/9:1) [my box] {1};
  \node (2) at (7*360/9:1) [my box] {2};
  \node (3) at (5*360/9:1) [my box] {3};
  \node (4) at (8*360/9:1) [my box] {4};
  \node (5) at (0*360/9:1) [my box] {5};
  \node (6) at (6*360/9:1) [my box] {6};
  \node (7) at (4*360/9:1) [my box] {7};
  \node (8) at (2*360/9:1) [my box] {8};

  \draw [] (1) to (8);
  \draw [] (2) to (8);
  \draw [] (0) to (3);
  \draw [] (0) to (7);
  \draw [] (2) to (5);
  \draw [] (3) to (8);
  \draw [] (2) to (6);
  \draw [] (0) to (6);
  \draw [] (0) to (8);
  \draw [] (3) to (6);
  \draw [] (4) to (7);
  \draw [] (1) to (6);
  \draw [] (3) to (7);
  \draw [] (2) to (4);
  \draw [] (4) to (8);
  \draw [] (5) to (7);
  \draw [] (1) to (5);
  \draw [] (1) to (4);

\end{tikzpicture}
\hspace{1em} \begin{tikzpicture}
    [scale=2,
     my box/.style={draw,circle,minimum size=1.2em,inner sep=.1em,outer sep=.1em},
     font=\normalsize,
     rotate=-7*360/36,
    ]

  \node (0) at (0*360/9:1) [my box] {0};
  \node (1) at (3*360/9:1) [my box] {1};
  \node (2) at (5*360/9:1) [my box] {2};
  \node (3) at (7*360/9:1) [my box] {3};
  \node (4) at (2*360/9:1) [my box] {4};
  \node (5) at (4*360/9:1) [my box] {5};
  \node (6) at (8*360/9:1) [my box] {6};
  \node (7) at (1*360/9:1) [my box] {7};
  \node (8) at (6*360/9:1) [my box] {8};

  \draw [] (1) to (8);
  \draw [] (2) to (8);
  \draw [] (0) to (3);
  \draw [] (0) to (7);
  \draw [] (2) to (5);
  \draw [] (3) to (8);
  \draw [] (2) to (6);
  \draw [] (0) to (6);
  \draw [] (0) to (8);
  \draw [] (3) to (6);
  \draw [] (4) to (7);
  \draw [] (1) to (6);
  \draw [] (3) to (7);
  \draw [] (2) to (4);
  \draw [] (4) to (8);
  \draw [] (5) to (7);
  \draw [] (1) to (5);
  \draw [] (1) to (4);

\end{tikzpicture}
\end{center}

cross connected 4-cycle and n=5 most

\begin{center}
\begin{tikzpicture}
    [scale=1.5,yscale=1.5,
     my box/.style={draw,circle,minimum size=1.2em,inner sep=.1em,outer sep=.1em},
     font=\normalsize,
    ]

  \node (0) at (2,-1) [my box] {0};
  \node (6) at (3,0) [my box] {6};
  \node (3) at (2,1) [my box] {3};
  \node (8) at (5,0) [my box] {8};
  \node (1) at (4,0) [my box] {1};

  \node (2) at (1,1) [my box] {2};
  \node (4) at (1,-1) [my box] {4};
  \node (5) at (0,.8) [my box] {5};
  \node (7) at (0,-.8) [my box] {7};

  \draw [] (4) to (7);
  \draw [] (0) to (7);
  \draw [] (0) to (6);
  \draw [] (5) to (7);
  \draw [] (1) to (6);
  \draw [] (3) to (8);
  \draw [] (3) to (7);
  \draw [] (3) to (6);
  \draw [] (2) to (4);
  \draw [] (2) to (5);
  \draw [] (1) to (5);
  \draw [] (2) to (6);
  \draw [] (0) to (3);
  \draw [] (2) to (8);
  \draw [] (1) to (4);
  \draw [] (4) to (8);
  \draw [] (1) to (8);
  \draw [] (0) to (8);

\end{tikzpicture}
\end{center}

\begin{center}
\begin{tikzpicture}
    [scale=1.5,yscale=1.5,
     my box/.style={draw,circle,minimum size=1.2em,inner sep=.1em,outer sep=.1em},
     font=\normalsize,
    ]

  \node (0) at (2,1) [my box] {0};
  \node (7) at (3,0) [my box] {7};
  \node (3) at (2,-1) [my box] {3};
  \node (8) at (5,0) [my box] {8};
  \node (4) at (4,0) [my box] {4};

  \node (2) at (0, -1) [my box] {2};
  \node (1) at (0, 1) [my box] {1};
  \node (6) at (.25, .25) [my box] {6};
  \node (5) at (-.25, -.25) [my box] {5};

  \draw [] (4) to (7);
  \draw [] (0) to (7);
  \draw [] (0) to (6);
  \draw [] (5) to (7);
  \draw [] (1) to (6);
  \draw [] (3) to (8);
  \draw [] (3) to (7);
  \draw [] (3) to (6);
  \draw [] (2) to (4);
  \draw [] (2) to (5);
  \draw [] (1) to (5);
  \draw [] (2) to (6);
  \draw [] (0) to (3);
  \draw [] (2) to (8);
  \draw [] (1) to (4);
  \draw [] (4) to (8);
  \draw [] (1) to (8);
  \draw [] (0) to (8);

\end{tikzpicture}
\end{center}

\begin{center}
\begin{tikzpicture}
    [scale=1.5,yscale=1.5,
     my box/.style={draw,circle,minimum size=1.2em,inner sep=.1em,outer sep=.1em},
     font=\normalsize,
    ]

  \node (0) at (5, .75) [my box] {0};
  \node (7) at (4, 0) [my box] {7};
  \node (3) at (5, -.75) [my box] {3};
  \node (8) at (2, 0) [my box] {8};
  \node (4) at (3, 0) [my box] {4};

  \begin{scope}[shift={(0, -.25)}]
  \node (2) at (0, -1) [my box] {2};
  \node (1) at (0, 1) [my box] {1};
  \node (6) at (.25, .25) [my box] {6};
  \node (5) at (-.25, -.25) [my box] {5};
  \end{scope}

  \draw [] (4) to (7);
  \draw [] (0) to (7);
  \draw [] (0) to (6);
  \draw [] (5) to (7);
  \draw [] (1) to (6);
  \draw [] (3) to (8);
  \draw [] (3) to (7);
  \draw [] (3) to (6);
  \draw [] (2) to (4);
  \draw [] (2) to (5);
  \draw [] (1) to (5);
  \draw [] (2) to (6);
  \draw [] (0) to (3);
  \draw [] (2) to (8);
  \draw [] (1) to (4);
  \draw [] (4) to (8);
  \draw [] (1) to (8);
  \draw [] (0) to (8);

\end{tikzpicture}
\end{center}

\begin{center}
\begin{tikzpicture}
    [scale=1.5,yscale=2,
     my box/.style={draw,circle,minimum size=1.2em,inner sep=.1em,outer sep=.1em},
     font=\normalsize,
    ]

  \node (0) at (7-2,1) [my box] {0};
  \node (7) at (7-3,0) [my box] {7};
  \node (3) at (7-2,-1) [my box] {3};
  \node (8) at (7-5,0) [my box] {8};
  \node (4) at (7-4,0) [my box] {4};

  \node (2) at (.5, -1) [my box] {2};
  \node (6) at (-1, 0) [my box] {6};
  \node (1) at (.5, 1) [my box] {1};
  \node (5) at (.5, .25) [my box] {5};

  \draw [] (4) to (7);
  \draw [] (0) to (7);
  \draw [] (0) to (6);
  \draw [] (5) to (7);
  \draw [] (1) to (6);
  \draw [] (3) to (8);
  \draw [] (3) to (7);
  \draw [] (3) to (6);
  \draw [] (2) to (4);
  \draw [] (2) to (5);
  \draw [] (1) to (5);
  \draw [] (2) to (6);
  \draw [] (0) to (3);
  \draw [] (2) to (8);
  \draw [] (1) to (4);
  \draw [] (4) to (8);
  \draw [] (1) to (8);
  \draw [] (0) to (8);

\end{tikzpicture}
\end{center}


\subsection{N=9 Connected Second, 19 edges}

Most minimal domsets (51 of), of them only 1 is minimum size domnum=2.

\url{https://hog.grinvin.org/ViewGraphInfo.action?id=30669}

\smallskip

This graph has 51 minimal dominating sets which is the most of any
connected n=9 vertices.  It is one of three connected n=9 with this
many.  It can almost be taken as as n=4 and n=5 most minimal
dominating sets graphs cross-connected by 7 edges, but the n=4 cycle
here has a further edge across the cycle.

\smallskip

degrees 4,4,4,4,4,4,4,5,5

7,8=five
\begin{center}
\begin{tikzpicture}
    [scale=2,
     my box/.style={draw,circle,minimum size=1.2em,inner sep=.1em,outer sep=.1em},
     font=\normalsize,
    ]

  \node (0) at (7,1) [my box] {0};
  \node (1) at (6,0) [my box] {1};
  \node (4) at (7,-1) [my box] {4};
  \node (3) at (5,0) [my box] {3};
  \node (7) at (4,0) [my box] {7};

  \begin{scope}[shift={(3, 0)}]
  \node (2) at (0,1) [my box] {2};
  \node (5) at (.3, .6) [my box] {5};
  \node (8) at (0,-1) [my box] {8};
  \node (6) at (.3, -.6) [my box] {6};
  \end{scope}

  \draw [] (3) to (6);
  \draw [] (0) to (2);
  \draw [] (5) to (8);
  \draw [] (1) to (3);
  \draw [] (2) to (8);
  \draw [] (2) to (6);
  \draw [] (4) to (7);
  \draw [] (3) to (5);
  \draw [] (5) to (7);
  \draw [] (6) to (7);
  \draw [] (2) to (5);
  \draw [] (0) to (7);
  \draw [] (0) to (4);
  \draw [] (3) to (7);
  \draw [] (0) to (1);
  \draw [] (4) to (8);
  \draw [] (1) to (4);
  \draw [] (6) to (8);
  \draw [] (1) to (8);

\end{tikzpicture}
\end{center}

%--------------
\needspace{6\baselineskip}
\subsection{N=9 Connected Third, 20 edges}

Most minimal domsets (51 of), of them only 1 is minimum size domnum=2.

\url{https://hog.grinvin.org/ViewGraphInfo.action?id=30671}

\smallskip

This graph has 51 minimal dominating sets which is the most of any
connected n=9.  It is one of three connected n=9 with this many.  It
can be taken as n=4 (cycle) and n=5 most minimal dominating sets
graphs cross-connected by 9 edges.

\smallskip

\begin{center}
\begin{tikzpicture}
    [scale=2,yscale=1.3,
     my box/.style={draw,circle,minimum size=1.2em,inner sep=.2em,outer sep=.1em},
     font=\normalsize,
    ]

  % n=5 part 2,3,5,6,8

  \node (2) at (7,1) [my box] {2};
  \node (8) at (7,-1) [my box] {8};
  \node (5) at (6.5, 0) [my box] {5};
  \node (3) at (5.8, 0) [my box] {3};
  \node (6) at (5, 0) [my box] {6};

  \begin{scope}[shift={(4, 0)}]
  \node (0) at (0,1) [my box] {0};
  \node (4) at (.3, .6) [my box] {4};
  \node (7) at (.3, -.6) [my box] {7};
  \node (1) at (0,-1) [my box] {1};
  \end{scope}

  \draw [] (3) to (5);
  \draw [] (2) to (6);
  \draw [] (4) to (5);
  \draw [] (5) to (7);
  \draw [] (1) to (8);
  \draw [] (7) to (8);
  \draw [] (6) to (8);
  \draw [] (0) to (1);
  \draw [] (1) to (3);
  \draw [] (5) to (8);
  \draw [] (2) to (5);
  \draw [] (2) to (8);
  \draw [] (4) to (7);
  \draw [] (0) to (4);
  \draw [] (1) to (7);
  \draw [] (3) to (6);
  \draw [] (4) to (6);
  \draw [] (6) to (7);
  \draw [] (3) to (4);
  \draw [] (0) to (2);

\end{tikzpicture}
\end{center}

%--------------
\needspace{6\baselineskip}
\subsection{N=9 Disconnected}

\url{https://hog.grinvin.org/ViewGraphInfo.action?id=30673}

\smallskip

This graph has 54 minimal dominating sets which is the most of any n=9
and it is the only n=9 with this many.  It is disjoint copies of the
n=4 (cycle) and n=5 graphs of most minimal dominating sets, for
product 6*9=54 sets.

\smallskip

\begin{center}
\begin{tikzpicture}
    [scale=1.5,
     my box/.style={draw,circle,minimum size=1.2em,inner sep=.1em,outer sep=.1em},
     font=\footnotesize,
    ]

  \node (0) at (3, 1.5) [my box] {0};
  \node (1) at (1, 1.5) [my box] {1};
  \node (4) at (0.5, .5) [my box] {4};
  \node (5) at (2, 1.5) [my box] {5};
  \node (8) at (0.5, 2.5) [my box] {8};

  \node (2) at (-1, 2) [my box] {2};
  \node (3) at (-2, 1) [my box] {3};
  \node (6) at (-2, 2) [my box] {6};
  \node (7) at (-1, 1) [my box] {7};

  \draw [] (0) to (4);
  \draw [] (1) to (8);
  \draw [] (0) to (8);
  \draw [] (0) to (5);
  \draw [] (2) to (7);
  \draw [] (1) to (5);
  \draw [] (1) to (4);
  \draw [] (3) to (7);
  \draw [] (3) to (6);
  \draw [] (2) to (6);
  \draw [] (4) to (8);

\end{tikzpicture}
\end{center}

\begin{center}
\begin{tikzpicture}
    [scale=1.5,
     my box/.style={draw,circle,minimum size=1.2em,inner sep=.1em,outer sep=.1em},
     font=\footnotesize,
    ]

  \node (0) at (0,2) [my box] {0};
  \node (1) at (2,0) [my box] {1};
  \node (4) at (0,0) [my box] {4};
  \node (5) at (2,2) [my box] {5};
  \node (8) at (1,1) [my box] {8};

  \node (2) at (-1,2) [my box] {2};
  \node (3) at (-2,1) [my box] {3};
  \node (6) at (-2,2) [my box] {6};
  \node (7) at (-1,1) [my box] {7};

  \draw [] (0) to (4);
  \draw [] (1) to (8);
  \draw [] (0) to (8);
  \draw [] (0) to (5);
  \draw [] (2) to (7);
  \draw [] (1) to (5);
  \draw [] (1) to (4);
  \draw [] (3) to (7);
  \draw [] (3) to (6);
  \draw [] (2) to (6);
  \draw [] (4) to (8);

\end{tikzpicture}
\end{center}


%--------------
\needspace{6\baselineskip}
\subsection{N=10 Connected}

n=10 connected, 87 minimal dominating sets

\url{https://hog.grinvin.org/ViewGraphInfo.action?id=30675}

\smallskip

This graph has 87 minimal dominating sets which is the most of any
connected n=10 vertices and it is the only connected n=10 with this
many.  It can be taken as two copies of the n=5 graph of most minimal
dominating sets cross-connected by 2 edges (and has more sets than
their product 9*9=81).

\smallskip

\begin{center}
\begin{tikzpicture}
    [scale=2,
     my box/.style={draw,circle,minimum size=1.2em,inner sep=.1em,outer sep=.1em},
     font=\normalsize,
    ]

  \node (3) at (-.3,1) [my box] {3};
  \node (7) at (.5,1) [my box] {7};
  \node (6) at (-1,1) [my box] {6};
  \node (2) at (1,2) [my box] {2};
  \node (9) at (1,0) [my box] {9};

  \node (0) at (2.5,1) [my box] {0};
  \node (1) at (4,1) [my box] {1};
  \node (4) at (2,0) [my box] {4};
  \node (5) at (3.2,1) [my box] {5};
  \node (8) at (2,2) [my box] {8};

  \draw [] (2) to (7);
  \draw [] (2) to (9);
  \draw [] (2) to (6);
  \draw [] (4) to (8);
  \draw [] (1) to (8);
  \draw [] (1) to (4);
  \draw [] (0) to (4);
  \draw [] (2) to (8);
  \draw [] (7) to (9);
  \draw [] (0) to (5);
  \draw [] (3) to (7);
  \draw [] (6) to (9);
  \draw [] (4) to (9);
  \draw [] (0) to (8);
  \draw [] (1) to (5);
  \draw [] (3) to (6);

\end{tikzpicture}
\end{center}

\begin{center}
\begin{tikzpicture}
    [scale=2,
     my box/.style={draw,circle,minimum size=1.2em,inner sep=.1em,outer sep=.1em},
     font=\normalsize,
    ]

  \node (3) at (0,1) [my box] {3};
  \node (7) at (0,0) [my box] {7};
  \node (6) at (0,2) [my box] {6};
  \node (2) at (1,2) [my box] {2};
  \node (9) at (1,0) [my box] {9};

  \node (0) at (3,0) [my box] {0};
  \node (1) at (3,2) [my box] {1};
  \node (4) at (2,0) [my box] {4};
  \node (5) at (3,1) [my box] {5};
  \node (8) at (2,2) [my box] {8};

  \draw [] (2) to (7);
  \draw [] (2) to (9);
  \draw [] (2) to (6);
  \draw [] (4) to (8);
  \draw [] (1) to (8);
  \draw [] (1) to (4);
  \draw [] (0) to (4);
  \draw [] (2) to (8);
  \draw [] (7) to (9);
  \draw [] (0) to (5);
  \draw [] (3) to (7);
  \draw [] (6) to (9);
  \draw [] (4) to (9);
  \draw [] (0) to (8);
  \draw [] (1) to (5);
  \draw [] (3) to (6);

\end{tikzpicture}
\end{center}

\subsection{N=11 Connected, 21 edges}

n=11 connected 127 minimal dominating sets

octahedral less edges cross connected to n=5

\url{https://hog.grinvin.org/ViewGraphInfo.action?id=30691}

\smallskip

This graph has 127 minimal dominating sets which is the most of any
connected n=11 vertices.  It is almost the n=5 and n=6 (octahedral)
most minimal dominating sets graphs cross-connected, but 2 edges
deleted from the octahedral.  It is one of two connected n=11 with
this many minimal dominating sets.  The other has 2 fewer
cross-connections.

\smallskip

\begin{center}
\begin{tikzpicture}
    [scale=1.7,
     my box/.style={draw,circle,minimum size=1.2em,inner sep=.1em,outer sep=.1em},
     font=\normalsize,
    ]

  \node (00) at (2, 1) [my box] {00};
  \node (01) at (0, 1) [my box] {01};
  \node (04) at (2, 3) [my box] {04};
  \node (05) at (0, 3) [my box] {05};
  \node (07) at (1, .5) [my box] {07};
  \node (08) at (1, 3.5) [my box] {08};

  \begin{scope}[shift={(0, 1)}]
  \node (02) at (5,1) [my box] {02};
  \node (09) at (3, 1-.4) [my box] {09};
  \node (06) at (4, 1) [my box] {06};
  \node (03) at (3.5, 1) [my box] {03};
  \node (10) at (3, 1+.4) [my box] {10};
  \end{scope}

  \draw [] (00) to (07);
  \draw [] (10) to (02);
  \draw [] (10) to (03);
  \draw [] (00) to (08);
  \draw [] (10) to (05);
  \draw [] (05) to (07);
  \draw [] (04) to (07);
  \draw [] (01) to (09);
  \draw [] (04) to (08);
  \draw [] (00) to (04);
  \draw [] (10) to (04);
  \draw [] (10) to (09);
  \draw [] (01) to (07);
  \draw [] (02) to (06);
  \draw [] (03) to (06);
  \draw [] (01) to (05);
  \draw [] (03) to (09);
  \draw [] (00) to (09);
  \draw [] (05) to (08);
  \draw [] (02) to (09);
  \draw [] (01) to (08);

\end{tikzpicture}
\end{center}

\begin{center}
\begin{tikzpicture}
    [scale=1,
     my box/.style={draw,circle,minimum size=1.2em,inner sep=.1em,outer sep=.1em},
    ]

  \node (00) at (2, -.5) [my box] {00};
  \node (01) at (-1,-.5) [my box] {01};
  \node (02) at (5,1) [my box] {02};
  \node (03) at (3.5, 1) [my box] {03};
  \node (04) at (2, 2.5) [my box] {04};
  \node (05) at (-1, 2.5) [my box] {05};
  \node (06) at (4, 1) [my box] {06};
  \node (07) at (.5,.5) [my box] {07};
  \node (08) at (.5, 1.5) [my box] {08};
  \node (09) at (3, .5) [my box] {09};
  \node (10) at (3, 1.5) [my box] {10};

  \draw [] (00) to (07);
  \draw [] (10) to (02);
  \draw [] (10) to (03);
  \draw [] (00) to (08);
  \draw [] (10) to (05);
  \draw [] (05) to (07);
  \draw [] (04) to (07);
  \draw [] (01) to (09);
  \draw [] (04) to (08);
  \draw [] (00) to (04);
  \draw [] (10) to (04);
  \draw [] (10) to (09);
  \draw [] (01) to (07);
  \draw [] (02) to (06);
  \draw [] (03) to (06);
  \draw [] (01) to (05);
  \draw [] (03) to (09);
  \draw [] (00) to (09);
  \draw [] (05) to (08);
  \draw [] (02) to (09);
  \draw [] (01) to (08);

\end{tikzpicture}
\end{center}

\begin{center}
\begin{tikzpicture}
    [scale=1,
     my box/.style={draw,circle,minimum size=1.2em,inner sep=.1em,outer sep=.1em},
     font=\footnotesize,
    ]

  \node (02) at (5,1) [my box] {02};
  \node (03) at (3.5, 1) [my box] {03};
  \node (09) at (3, .5) [my box] {09};
  \node (10) at (3, 1.5) [my box] {10};
  \node (06) at (4, 1) [my box] {06};

  \node (00) at (2, -.8) [my box] {00};
  \node (01) at (.7, 1-.5) [my box] {01};
  \node (04) at (2, 2.8) [my box] {04};
  \node (05) at (.7, 1+.5) [my box] {05};
  \node (07) at (-.9, 1) [my box] {07};
  \node (08) at (1.4, 1) [my box] {08};

  \draw [] (00) to (07);
  \draw [] (10) to (02);
  \draw [] (10) to (03);
  \draw [] (00) to (08);
  \draw [] (10) to (05);
  \draw [] (05) to (07);
  \draw [] (04) to (07);
  \draw [] (01) to (09);
  \draw [] (04) to (08);
  \draw [] (00) to (04);
  \draw [] (10) to (04);
  \draw [] (10) to (09);
  \draw [] (01) to (07);
  \draw [] (02) to (06);
  \draw [] (03) to (06);
  \draw [] (01) to (05);
  \draw [] (03) to (09);
  \draw [] (00) to (09);
  \draw [] (05) to (08);
  \draw [] (02) to (09);
  \draw [] (01) to (08);

\end{tikzpicture}
\end{center}

%--------------
\needspace{6\baselineskip}
\subsubsection{N=11 Connected Second, 19 edges}

\url{https://hog.grinvin.org/ViewGraphInfo.action?id=30689}

\smallskip

This graph has 127 minimal dominating sets which is the most of any
connected n=11 vertices.  It is almost the n=5 and n=6 (octahedral)
most minimal dominating sets graphs cross-connected, but 2 edges
deleted from the octahedral.  It is one of two connected n=11 with
this many minimal dominating sets.  The other has 2 additional
cross-connections.

\smallskip

\begin{center}
\begin{tikzpicture}
    [scale=1.7,
     my box/.style={draw,circle,minimum size=1.2em,inner sep=.1em,outer sep=.1em},
    ]

  \node (03) at (1, 1+1.2) [my box] {03};
  \node (10) at (1, 1-1.2) [my box] {10};
  \node (04) at (.2, 1+.4) [my box] {04};
  \node (00) at (.2, 1-.4) [my box] {00};
  \node (07) at (-1, 1) [my box] {07};
  \node (08) at (.6, 1) [my box] {08};

  \begin{scope}[shift={(-1.3, 0)}]
  \node (01) at (3.5, 1) [my box] {01};
  \node (02) at (5,1) [my box] {02};
  \node (09) at (3, 1+.4) [my box] {09};
  \node (06) at (4, 1) [my box] {06};
  \node (05) at (3, 1-.4) [my box] {05};
  \end{scope}


  \draw [] (01) to (05);
  \draw [] (04) to (08);
  \draw [] (05) to (09);
  \draw [] (10) to (07);
  \draw [] (04) to (07);
  \draw [] (02) to (09);
  \draw [] (00) to (08);
  \draw [] (10) to (03);
  \draw [] (00) to (07);
  \draw [] (10) to (05);
  \draw [] (03) to (09);
  \draw [] (02) to (06);
  \draw [] (00) to (04);
  \draw [] (01) to (09);
  \draw [] (01) to (06);
  \draw [] (03) to (08);
  \draw [] (03) to (07);
  \draw [] (10) to (08);
  \draw [] (02) to (05);


\end{tikzpicture}
\end{center}


%------------------------------------------------------------------------------
\section{Asymmetric Subtree Relations}

Ilhee Kim, Ringi Kim, Paul Seymour, ``The Minimal Automorphism-Free
Tree'', \url{https://arxiv.org/abs/1303.1551}

\subsection{n le 10}

\url{https://hog.grinvin.org/ViewGraphInfo.action?id=30651}

\smallskip

Each vertex is an asymmetric tree of $n \le 10$ vertices (tree with no
automorphisms except the identity).  Edges are between those where one
is a subgraph of the other and differing in size by 1 vertex.  The
number of trees is 1+1+3+6 = 11 (sum OEIS A000220).

Per Kim, Kim, and Seymour, any asymmetric tree of n vertices can be
obtained from an n-1 asymmetric tree by the addition of 1 extra leaf
vertex.

\smallskip

\begin{center}
\begin{tikzpicture}
    [scale=1,
     my box/.style={draw,circle,minimum size=1.5em,inner sep=.1em,outer sep=.3em},
    ]

  \node (0) at (0,4) [my box] {0};
  \node (1) at (0,3) [my box] {1};
  \node (2) at (-1.5,2) [my box] {2};
  \node (3) at (0,2) [my box] {3};
  \node (4) at (1.5,2) [my box] {4};
  \node (6) at (-2.5,0) [my box] {6};
  \node (9) at (-1.6,0) [my box] {9};
  \node (7) at (-1,0) [my box] {7};
  \node (8) at (0,0) [my box] {8};
  \node (5) at (1,0) [my box] {5};
  \node (10) at (2.2,0) [my box] {10};

  \draw [->] (9) to (3);
  \draw [->] (4) to (1);
  \draw [->] (5) to (2);
  \draw [->] (9) to (2);
  \draw [->] (5) to (4);
  \draw [->] (7) to (2);
  \draw [->] (1) to (0);
  \draw [->] (10) to (4);
  \draw [->] (8) to (3);
  \draw [->] (6) to (2);
  \draw [->] (7) to (3);
  \draw [->] (2) to (1);
  \draw [->] (3) to (1);
  \draw [->] (5) to (3);

\end{tikzpicture}
\end{center}

\subsection{n le 11}

\url{https://hog.grinvin.org/ViewGraphInfo.action?id=30653}

\smallskip

Each vertex is an asymmetric tree of $n \le 11$ vertices (tree with no
automorphisms except the identity).  Edges are between those where one
is a subgraph of the other and differing in size by 1 vertex.  The
number of trees is 1+1+3+6+15 = 26 (sum OEIS A000220).

Per Kim, Kim, and Seymour, any asymmetric tree of n vertices can be
obtained from an n-1 asymmetric tree by the addition of 1 extra leaf
vertex.

\smallskip

\begin{center}
\begin{tikzpicture}
    [scale=1,
     my box/.style={draw,circle,minimum size=1.5em,inner sep=.1em,outer sep=.3em},
    ]

  \node (0) at (0,4) [my box] {0};
  \node (1) at (0,3) [my box] {1};
  \node (2) at (-1,2) [my box] {2};
  \node (3) at (0,2) [my box] {3};
  \node (4) at (1,2) [my box] {4};
  \node (5) at (1,0) [my box] {5};
  \node (6) at (-5,0) [my box] {6};
  \node (7) at (-1,0) [my box] {7};
  \node (8) at (3,0) [my box] {8};
  \node (9) at (-3,0) [my box] {9};
  \node (10) at (5,0) [my box] {10};

  \node (11) at (-6,-2) [my box] {11};
  \node (12) at (-5,-2) [my box] {12};
  \node (18) at (-4.5,-2) [my box] {18};
  \node (19) at (-3,-2) [my box] {19};
  \node (13) at (-3.75,-2) [my box] {13};
  \node (15) at (-2,-2) [my box] {15};
  \node (24) at (-1,-2) [my box] {24};
  \node (17) at (0,-2) [my box] {17};
  \node (16) at (.5,-2) [my box] {16};
  \node (20) at (1.5,-2) [my box] {20};
  \node (22) at (2.5,-2) [my box] {22};
  \node (14) at (3.5,-2) [my box] {14};
  \node (21) at (4,-2) [my box] {21};
  \node (23) at (5,-2) [my box] {23};
  \node (25) at (6,-2) [my box] {25};

  \draw [->] (1) to (0);
  \draw [->] (12) to (6);
  \draw [->] (17) to (7);
  \draw [->] (15) to (5);
  \draw [->] (17) to (5);
  \draw [->] (15) to (6);
  \draw [->] (9) to (3);
  \draw [->] (14) to (10);
  \draw [->] (16) to (7);
  \draw [->] (10) to (4);
  \draw [->] (7) to (3);
  \draw [->] (12) to (9);
  \draw [->] (11) to (6);
  \draw [->] (16) to (5);
  \draw [->] (13) to (9);
  \draw [->] (5) to (2);
  \draw [->] (22) to (9);
  \draw [->] (3) to (1);
  \draw [->] (18) to (9);
  \draw [->] (14) to (8);
  \draw [->] (23) to (10);
  \draw [->] (6) to (2);
  \draw [->] (23) to (8);
  \draw [->] (20) to (8);
  \draw [->] (18) to (6);
  \draw [->] (25) to (10);
  \draw [->] (24) to (5);
  \draw [->] (5) to (3);
  \draw [->] (9) to (2);
  \draw [->] (24) to (9);
  \draw [->] (7) to (2);
  \draw [->] (15) to (9);
  \draw [->] (21) to (8);
  \draw [->] (20) to (7);
  \draw [->] (19) to (6);
  \draw [->] (4) to (1);
  \draw [->] (19) to (7);
  \draw [->] (14) to (5);
  \draw [->] (8) to (3);
  \draw [->] (22) to (8);
  \draw [->] (2) to (1);
  \draw [->] (5) to (4);

\end{tikzpicture}
\end{center}


%------------------------------------------------------------------------------
\section{Seven Segment LED Subsets, 7 without left, 9 with bottom}

\url{https://hog.grinvin.org/ViewGraphInfo.action?id=30627}

\medskip

Each vertex is a digit 0 to 9.  Edges are between those whose
7-segment LED segments are an immediate subset.  Immediate subset here
means no other subset between, so for example edge 8 to 6 and 6 to 5,
but no 8 to 5 because 6 between.

Seven segment displays vary in how they show 7 and 9.  The form here
is 7 without top left side (so 3 segments), and 9 with bottom segment
(so 6 segments).

Digit 8 is all segments.  It is the degree-4 with leaf neighbour.
Digit 2 is the leaf.

This graph is by Raphael M. Robinson from 1979, published and drawn by
Martin Gardner, ``The Dance of the Jolly Green Digits'', Isaac Asimov's
Science Fiction Magazine, volume 5, number 2, 16 February 1981, pages
50-51, 58, 88-89, 109. \newline
\url{https://archive.org/details/Asimovs_v05n02_1981-02-16} \newline
\url{https://archive.org/download/Asimovs_v05n02_1981-02-16/Asimovs_v05n02_1981-02-16.pdf}

Reprinted in Martin Gardner, ``Science Fiction Puzzle Tales'',
Clarkson
N. Potter, 1981, ISBN 0-517-54380X hardback and 0-517-543818
paperback.

\smallskip

\begin{center}
\begin{tikzpicture}
    [scale=1.5,yscale=.8,
     my box/.style={draw,circle,minimum size=1.2em,inner sep=.1em,outer sep=.3em},
    ]

  \node (0) at (-2.4, 6) [my box] {0};
  \node (1) at (0, 2) [my box] {1};
  \node (2) at (1.3, 5) [my box] {2};
  \node (3) at (0, 5) [my box] {3};
  \node (4) at (.8, 4) [my box] {4};
  \node (5) at (-1, 5) [my box] {5};
  \node (6) at (-1, 6) [my box] {6};
  \node (7) at (0, 3) [my box] {7};
  \node (8) at (0, 7) [my box] {8};
  \node (9) at (0, 6) [my box] {9};

  \draw [->,bend right=0] (0) to (7);
  \draw [->] (3) to (7);
  \draw [->] (8) to (2);
  \draw [->] (9) to (3);
  \draw [->] (7) to (1);
  \draw [->] (6) to (5);
  \draw [->] (8) to (9);
  \draw [->] (8) to (6);
  \draw [->] (4) to (1);
  \draw [->] (8) to (0);
  \draw [->] (9) to (5);
  \draw [->] (9) to (4);

  \node at (1.5, 4) [right] {After Gardner};

\end{tikzpicture}
\end{center}

\begin{center}
\begin{tikzpicture}
    [scale=1.5,
     my box/.style={draw,circle,minimum size=1.2em,inner sep=.1em,outer sep=.3em},
    ]

  \node (0) at (-1,3) [my box] {0};
  \node (1) at (0,0) [my box] {1};
  \node (2) at (2,3) [my box] {2};
  \node (3) at (-.5,2) [my box] {3};
  \node (4) at (0,1.5) [my box] {4};
  \node (5) at (1,2) [my box] {5};
  \node (6) at (1,3) [my box] {6};
  \node (7) at (-1,1) [my box] {7};
  \node (8) at (0,4) [my box] {8};
  \node (9) at (0,3) [my box] {9};

  \draw [->] (0) to (7);
  \draw [->] (3) to (7);
  \draw [->] (8) to (2);
  \draw [->] (9) to (3);
  \draw [->] (7) to (1);
  \draw [->] (6) to (5);
  \draw [->] (8) to (9);
  \draw [->] (8) to (6);
  \draw [->] (4) to (1);
  \draw [->] (8) to (0);
  \draw [->] (9) to (5);
  \draw [->] (9) to (4);

\end{tikzpicture}
\end{center}

\begin{center}
\begin{tikzpicture}
    [scale=2,
     my box/.style={draw,circle,minimum size=1.2em,inner sep=.2em,outer sep=.3em},
     font=\normalsize,
    ]

  \node (0) at (.5,1) [my box] {0};
  \node (1) at (.5,-1) [my box] {1};
  \node (2) at (2,2) [my box] {2};
  \node (3) at (1,0) [my box] {3};
  \node (4) at (1.5,-1) [my box] {4};
  \node (5) at ($(2,0) + (30:1)$) [my box] {5};
  \node (6) at ($(1.5,1) + (30:1)$) [my box] {6};
  \node (7) at (0,0) [my box] {7};
  \node (8) at (1.5,1) [my box] {8};
  \node (9) at (2,0) [my box] {9};

  \draw [->] (0) to (7);
  \draw [->] (3) to (7);
  \draw [->] (8) to (2);
  \draw [->] (9) to (3);
  \draw [->] (7) to (1);
  \draw [->] (6) to (5);
  \draw [->] (8) to (9);
  \draw [->] (8) to (6);
  \draw [->] (4) to (1);
  \draw [->] (8) to (0);
  \draw [->] (9) to (5);
  \draw [->] (9) to (4);

\end{tikzpicture}
\end{center}

\begin{center}
\begin{tikzpicture}
    [scale=2,
     my box/.style={draw,circle,minimum size=1.2em,inner sep=.1em,outer sep=.3em},
    ]

  \node (0) at (.5,1) [my box] {0};
  \node (1) at (.5,-1) [my box] {1};
  \node (2) at (2,2) [my box] {2};
  \node (3) at (1,0) [my box] {3};
  \node (4) at (1.5,-1) [my box] {4};
  \node (5) at (3,0) [my box] {5};
  \node (6) at (2.5,1) [my box] {6};
  \node (7) at (0,0) [my box] {7};
  \node (8) at (1.5,1) [my box] {8};
  \node (9) at (2,0) [my box] {9};

  \draw [->] (0) to (7);
  \draw [->] (3) to (7);
  \draw [->] (8) to (2);
  \draw [->] (9) to (3);
  \draw [->] (7) to (1);
  \draw [->] (6) to (5);
  \draw [->] (8) to (9);
  \draw [->] (8) to (6);
  \draw [->] (4) to (1);
  \draw [->] (8) to (0);
  \draw [->] (9) to (5);
  \draw [->] (9) to (4);

\end{tikzpicture}
\end{center}




%------------------------------------------------------------------------------
\section{Ramsey 3,4}

\subsection{One}

\url{https://hog.grinvin.org/ViewGraphInfo.action?id=588}

\smallskip

Generalized Petersen 4,2. \newline
Mobius Ladder 8 with 2 consecutive rungs removed.

Ramsey 3,4 graph.  This is one of three n=8 Ramsey 3,4 graphs.  The
others have the Mobius ladder rungs added back.

\begin{center}
\begin{tikzpicture}
    [scale=.8,
     my box/.style={draw,circle,minimum size=1.2em,inner sep=.1em,outer sep=.3em},
    ]
  \newcommand\MyOuter{{.95*(1+sqrt(2))}}
  \newcommand\MyInner{1}

  \node (0) at (5*45:\MyInner) [my box] {0};
  \node (1) at (3*45:\MyInner) [my box] {1};
  \node (2) at (1*45:\MyOuter) [my box] {2};
  \node (3) at (5*45:\MyOuter) [my box] {3};
  \node (4) at (7*45:\MyInner) [my box] {4};
  \node (5) at (1*45:\MyInner) [my box] {5};
  \node (6) at (3*45:\MyOuter) [my box] {6};
  \node (7) at (7*45:\MyOuter) [my box] {7};

  \draw [] (3) to (7);
  \draw [] (1) to (4);
  \draw [] (2) to (7);
  \draw [] (2) to (5);
  \draw [] (0) to (5);
  \draw [] (3) to (6);
  \draw [] (2) to (6);
  \draw [] (0) to (3);
  \draw [] (4) to (7);
  \draw [] (1) to (6);

\end{tikzpicture}
\hspace{1em}
\begin{tikzpicture}
    [scale=2,
     my box/.style={draw,circle,minimum size=1.5em,inner sep=.1em,outer sep=.3em},
    ]

  \node (0) at (.5,1) [my box] {0};
  \node (1) at (2.5,0) [my box] {1};
  \node (2) at (1,0) [my box] {2};
  \node (3) at (1,1) [my box] {3};
  \node (4) at (2.5,1) [my box] {4};
  \node (5) at (.5,0) [my box] {5};
  \node (6) at (2,0) [my box] {6};
  \node (7) at (2,1) [my box] {7};

  \draw [] (3) to (7);
  \draw [] (1) to (4);
  \draw [] (2) to (7);
  \draw [] (2) to (5);
  \draw [] (0) to (5);
  \draw [] (3) to (6);
  \draw [] (2) to (6);
  \draw [] (0) to (3);
  \draw [] (4) to (7);
  \draw [] (1) to (6);

\end{tikzpicture}
\end{center}
\begin{center}
\begin{tikzpicture}
    [scale=2,
     my box/.style={draw,circle,minimum size=1.5em,inner sep=.1em,outer sep=.3em},
    ]

  \node (0) at (1,0) [my box] {0};
  \node (1) at (1,1) [my box] {1};
  \node (2) at (3,0) [my box] {2};
  \node (3) at (0,0) [my box] {3};
  \node (4) at (2,1) [my box] {4};
  \node (5) at (2,0) [my box] {5};
  \node (6) at (0,1) [my box] {6};
  \node (7) at (3,1) [my box] {7};

  \draw [] (3) to (7);
  \draw [] (1) to (4);
  \draw [] (2) to (7);
  \draw [] (2) to (5);
  \draw [] (0) to (5);
  \draw [] (3) to (6);
  \draw [] (2) to (6);
  \draw [] (0) to (3);
  \draw [] (4) to (7);
  \draw [] (1) to (6);

\end{tikzpicture}
\end{center}

\subsection{Two}

\url{https://hog.grinvin.org/ViewGraphInfo.action?id=26996}

\smallskip

Mobius Ladder 8 with 1 rung removed.

Ramsey 3,4 graph.  This is one of three n=8 Ramsey 3,4 graphs.  The
others are with none or 2 consecutive Mobius ladder rungs removed.

\smallskip

\begin{center}
\begin{tikzpicture}
    [scale=1.5,rotate=45/2,
     my box/.style={draw,circle,minimum size=1.3em,inner sep=.1em,outer sep=.3em},
    ]

  \node (0) at (0*45:1) [my box] {0};
  \node (1) at (4*45:1) [my box] {1};
  \node (2) at (6*45:1) [my box] {2};
  \node (3) at (1*45:1) [my box] {3};
  \node (4) at (3*45:1) [my box] {4};
  \node (5) at (7*45:1) [my box] {5};
  \node (6) at (2*45:1) [my box] {6};
  \node (7) at (5*45:1) [my box] {7};

  \draw [] (4) to (6);
  \draw [] (1) to (5);
  \draw [] (0) to (5);
  \draw [] (1) to (4);
  \draw [] (1) to (7);
  \draw [] (2) to (5);
  \draw [] (2) to (7);
  \draw [] (3) to (7);
  \draw [] (2) to (6);
  \draw [] (0) to (3);
  \draw [] (3) to (6);


\end{tikzpicture}
\hspace{1em}
\begin{tikzpicture}
    [scale=1.5,
     my box/.style={draw,circle,minimum size=1.3em,inner sep=.1em,outer sep=.3em},
    ]

  \node (0) at (0*45:1) [my box] {0};
  \node (1) at (3*45:1) [my box] {1};
  \node (2) at (6*45:1) [my box] {2};
  \node (3) at (1*45:1) [my box] {3};
  \node (4) at (4*45:1) [my box] {4};
  \node (5) at (7*45:1) [my box] {5};
  \node (6) at (5*45:1) [my box] {6};
  \node (7) at (2*45:1) [my box] {7};

  \draw [] (4) to (6);
  \draw [] (1) to (5);
  \draw [] (0) to (5);
  \draw [] (1) to (4);
  \draw [] (1) to (7);
  \draw [] (2) to (5);
  \draw [] (2) to (7);
  \draw [] (3) to (7);
  \draw [] (2) to (6);
  \draw [] (0) to (3);
  \draw [] (3) to (6);


\end{tikzpicture}
\end{center}
\begin{center}
\begin{tikzpicture}
    [scale=1.5,
     my box/.style={draw,circle,minimum size=1.3em,inner sep=.1em,outer sep=.3em},
    ]

  \node (0) at (0*45:1) [my box] {0};
  \node (1) at (0,1) [my box] {1};
  \node (2) at (3,1) [my box] {2};
  \node (3) at (2,0) [my box] {3};
  \node (4) at (1,1) [my box] {4};
  \node (5) at (0,0) [my box] {5};
  \node (6) at (2,1) [my box] {6};
  \node (7) at (3,0) [my box] {7};

  \draw [] (4) to (6);
  \draw [] (1) to (5);
  \draw [] (0) to (5);
  \draw [] (1) to (4);
  \draw [] (1) to (7);
  \draw [] (2) to (5);
  \draw [] (2) to (7);
  \draw [] (3) to (7);
  \draw [] (2) to (6);
  \draw [] (0) to (3);
  \draw [] (3) to (6);


\end{tikzpicture}
\end{center}

\subsection{Three}

\url{https://hog.grinvin.org/ViewGraphInfo.action?id=640}

\smallskip

Ramsey 3,4 graph.  Shown in Bondy and Murty, ``Graph Theory With
Applications'', 1976, ISBN 0-444-19451-7, page 105, figure 7.2 (b) as
an example Ramsey 3,4.  This is one of three n=8 Ramsey 3,4 graphs.
The others have 1 or 2 consecutive Mobius ladder rungs removed.

Circulant-8 steps 1,4 (which is how drawn in Bondy and Murty), or
equivalently steps 3,4.

\smallskip

cf
\url{http://independencenumber.wordpress.com/2012/10/12/solutions-and-problems/}

\begin{center}
\begin{tikzpicture}
    [scale=1.5,
     my box/.style={draw,circle,minimum size=1.3em,inner sep=.1em,outer sep=.3em},
    ]

  \node (0) at (0:1) [my box] {0};
  \node (1) at (45:1) [my box] {1};
  \node (2) at (2*45:1) [my box] {2};
  \node (3) at (3*45:1) [my box] {3};
  \node (4) at (4*45:1) [my box] {4};
  \node (5) at (5*45:1) [my box] {5};
  \node (6) at (6*45:1) [my box] {6};
  \node (7) at (7*45:1) [my box] {7};

  \draw [] (1) to (5);
  \draw [] (4) to (7);
  \draw [] (2) to (7);
  \draw [] (3) to (7);
  \draw [] (0) to (3);
  \draw [] (2) to (5);
  \draw [] (1) to (6);
  \draw [] (2) to (6);
  \draw [] (1) to (4);
  \draw [] (0) to (5);
  \draw [] (0) to (4);
  \draw [] (3) to (6);

\end{tikzpicture} \hspace{1em}
\begin{tikzpicture}
    [scale=1.5,
     my box/.style={draw,circle,minimum size=1.3em,inner sep=.1em,outer sep=.3em},
    ]

  \node (0) at (5*45:1) [my box] {0};
  \node (1) at (1*45:1) [my box] {1};
  \node (2) at (7*45:1) [my box] {2};
  \node (3) at (4*45:1) [my box] {3};
  \node (4) at (2*45:1) [my box] {4};
  \node (5) at (6*45:1) [my box] {5};
  \node (6) at (0*45:1) [my box] {6};
  \node (7) at (3*45:1) [my box] {7};

  \draw [] (1) to (5);
  \draw [] (4) to (7);
  \draw [] (2) to (7);
  \draw [] (3) to (7);
  \draw [] (0) to (3);
  \draw [] (2) to (5);
  \draw [] (1) to (6);
  \draw [] (2) to (6);
  \draw [] (1) to (4);
  \draw [] (0) to (5);
  \draw [] (0) to (4);
  \draw [] (3) to (6);

\end{tikzpicture}
\end{center}
\begin{center}
\begin{tikzpicture}
    [scale=1.5,
     my box/.style={draw,circle,minimum size=1.3em,inner sep=.1em,outer sep=.3em},
    ]

  \node (0) at (5*45:1) [my box] {0};
  \node (1) at (2*45:1) [my box] {1};
  \node (2) at (7*45:1) [my box] {2};
  \node (3) at (4*45:1) [my box] {3};
  \node (4) at (1*45:1) [my box] {4};
  \node (5) at (6*45:1) [my box] {5};
  \node (6) at (3*45:1) [my box] {6};
  \node (7) at (0*45:1) [my box] {7};

  \draw [] (1) to (5);
  \draw [] (4) to (7);
  \draw [] (2) to (7);
  \draw [] (3) to (7);
  \draw [] (0) to (3);
  \draw [] (2) to (5);
  \draw [] (1) to (6);
  \draw [] (2) to (6);
  \draw [] (1) to (4);
  \draw [] (0) to (5);
  \draw [] (0) to (4);
  \draw [] (3) to (6);

\end{tikzpicture}
\hspace{1em}
\begin{tikzpicture}
    [scale=1.5,
     my box/.style={draw,circle,minimum size=1.3em,inner sep=.1em,outer sep=.3em},
    ]

  \node (0) at (1,0) [my box] {0};
  \node (1) at (0,1) [my box] {1};
  \node (2) at (-1,0) [my box] {2};
  \node (3) at (2,0) [my box] {3};
  \node (4) at (1,1) [my box] {4};
  \node (5) at (0,0) [my box] {5};
  \node (6) at (-1,1) [my box] {6};
  \node (7) at (2,1) [my box] {7};

  \draw [] (1) to (5);
  \draw [] (4) to (7);
  \draw [] (2) to (7);
  \draw [] (3) to (7);
  \draw [] (0) to (3);
  \draw [] (2) to (5);
  \draw [] (1) to (6);
  \draw [] (2) to (6);
  \draw [] (1) to (4);
  \draw [] (0) to (5);
  \draw [] (0) to (4);
  \draw [] (3) to (6);

\end{tikzpicture}
\end{center}


%------------------------------------------------------------------------------
\subsection{Ramsey 4,3}

H
\begin{center}
\begin{tikzpicture}
    [scale=3,
     my box/.style={draw,circle,minimum size=1.5em,inner sep=.1em,outer sep=.3em},
    ]

  \node (0) at (3*45:1) [my box] {0};
  \node (1) at (0*45:1) [my box] {1};
  \node (2) at (2*45:1) [my box] {2};
  \node (3) at (5*45:1) [my box] {3};
  \node (4) at (1*45:1) [my box] {4};
  \node (5) at (6*45:1) [my box] {5};
  \node (6) at (4*45:1) [my box] {6};
  \node (7) at (7*45:1) [my box] {7};

  \draw [] (3) to (5);
  \draw [] (2) to (7);
  \draw [] (2) to (5);
  \draw [] (3) to (7);
  \draw [] (0) to (4);
  \draw [] (1) to (7);
  \draw [] (2) to (4);
  \draw [] (1) to (3);
  \draw [] (5) to (7);
  \draw [] (4) to (6);
  \draw [] (0) to (2);
  \draw [] (0) to (6);
  \draw [] (1) to (6);
  \draw [] (3) to (6);
  \draw [] (1) to (4);
  \draw [] (0) to (5);

\end{tikzpicture}
\end{center}

Two
\begin{center}
\begin{tikzpicture}
    [scale=2,rotate=45,
     my box/.style={draw,circle,minimum size=1.5em,inner sep=.1em,outer sep=.3em},
    ]

  \node (0) at (3*45:1) [my box] {0};
  \node (1) at (0*45:1) [my box] {1};
  \node (2) at (2*45:1) [my box] {2};
  \node (3) at (6*45:1) [my box] {3};
  \node (4) at (1*45:1) [my box] {4};
  \node (5) at (4*45:1) [my box] {5};
  \node (6) at (7*45:1) [my box] {6};
  \node (7) at (5*45:1) [my box] {7};

  \draw [] (3) to (7);
  \draw [] (0) to (4);
  \draw [] (1) to (7);
  \draw [] (5) to (7);
  \draw [] (3) to (5);
  \draw [] (2) to (7);
  \draw [] (2) to (5);
  \draw [] (3) to (6);
  \draw [] (1) to (4);
  \draw [] (0) to (5);
  \draw [] (2) to (4);
  \draw [] (1) to (3);
  \draw [] (4) to (7);
  \draw [] (4) to (6);
  \draw [] (0) to (2);
  \draw [] (0) to (6);
  \draw [] (1) to (6);

\end{tikzpicture}
\end{center}


Three
\begin{center}
\begin{tikzpicture}
    [scale=2,
     my box/.style={draw,circle,minimum size=1.5em,inner sep=.1em,outer sep=.3em},
    ]

  \node (0) at (3*45:1) [my box] {0};
  \node (1) at (0*45:1) [my box] {1};
  \node (2) at (2*45:1) [my box] {2};
  \node (3) at (6*45:1) [my box] {3};
  \node (4) at (1*45:1) [my box] {4};
  \node (5) at (4*45:1) [my box] {5};
  \node (6) at (7*45:1) [my box] {6};
  \node (7) at (5*45:1) [my box] {7};

  \draw [] (5) to (7);
  \draw [] (2) to (5);
  \draw [] (3) to (5);
  \draw [] (1) to (7);
  \draw [] (5) to (6);
  \draw [] (0) to (4);
  \draw [] (4) to (7);
  \draw [] (2) to (7);
  \draw [] (4) to (6);
  \draw [] (0) to (2);
  \draw [] (0) to (6);
  \draw [] (1) to (6);
  \draw [] (2) to (4);
  \draw [] (1) to (3);
  \draw [] (3) to (7);
  \draw [] (3) to (6);
  \draw [] (0) to (5);
  \draw [] (1) to (4);

\end{tikzpicture}
\end{center}



%------------------------------------------------------------------------------
\section{Mobius Ladder 8 with 2 Non-Consecutive Rungs Deleted}

\url{https://hog.grinvin.org/ViewGraphInfo.action?id=30360}

\smallskip

Mobius Ladder 8 vertices with 2 non-consecutive rungs deleted.  The 4
degree-2 vertices are where the rungs are deleted.  They are an
independent set and are the only set attaining the independence
number 4.

\smallskip

\begin{center}
\begin{tikzpicture}
    [scale=3,
     my box/.style={draw,circle,minimum size=1.5em,inner sep=.1em,outer sep=.3em},
    ]

  \node (1) at (0,1) [my box] {1};
  \node (2) at (1,1) [my box] {2};
  \node (3) at (2,1) [my box] {3};
  \node (4) at (3,1) [my box] {4};
  \node (5) at (0,0) [my box] {5};
  \node (6) at (1,0) [my box] {6};
  \node (7) at (2,0) [my box] {7};
  \node (8) at (3,0) [my box] {8};

  \draw [] (2) to (3);
  \draw [] (2) to (6);
  \draw [] (1) to (8);
  \draw [] (6) to (7);
  \draw [] (4) to (5);
  \draw [] (7) to (8);
  \draw [] (3) to (4);
  \draw [] (1) to (2);
  \draw [] (4) to (8);
  \draw [] (5) to (6);

\end{tikzpicture}
\end{center}

\begin{center}
\begin{tikzpicture}
    [scale=3,rotate=90,
     my box/.style={draw,circle,minimum size=1.5em,inner sep=.1em,outer sep=.3em},
    ]

  \node (1) at (2,4) [my box] {1};
  \node (2) at (2,1) [my box] {2};
  \node (3) at (1.5,2) [my box] {3};
  \node (4) at (1,3) [my box] {4};
  \node (5) at (.5,2) [my box] {5};
  \node (6) at (0,1) [my box] {6};
  \node (7) at (0,4) [my box] {7};
  \node (8) at (1,4) [my box] {8};

  \draw [] (2) to (3);
  \draw [] (2) to (6);
  \draw [] (1) to (8);
  \draw [] (6) to (7);
  \draw [] (4) to (5);
  \draw [] (7) to (8);
  \draw [] (3) to (4);
  \draw [] (1) to (2);
  \draw [] (4) to (8);
  \draw [] (5) to (6);

\end{tikzpicture}
\end{center}


%------------------------------------------------------------------------------
\section{Circulant 17 steps 1,2,4,8}

\begin{center}
\begin{tikzpicture}
    [scale=4,
     my box/.style={draw,circle,minimum size=1.5em,inner sep=.1em,outer sep=.3em},
    ]

  \node (1) at (1*360/17:1) [my box] {1};
  \node (10) at (10*360/17:1) [my box] {10};
  \node (11) at (11*360/17:1) [my box] {11};
  \node (12) at (12*360/17:1) [my box] {12};
  \node (13) at (13*360/17:1) [my box] {13};
  \node (14) at (14*360/17:1) [my box] {14};
  \node (15) at (15*360/17:1) [my box] {15};
  \node (16) at (16*360/17:1) [my box] {16};
  \node (17) at (17*360/17:1) [my box] {17};
  \node (2) at (2*360/17:1) [my box] {2};
  \node (3) at (3*360/17:1) [my box] {3};
  \node (4) at (4*360/17:1) [my box] {4};
  \node (5) at (5*360/17:1) [my box] {5};
  \node (6) at (6*360/17:1) [my box] {6};
  \node (7) at (7*360/17:1) [my box] {7};
  \node (8) at (8*360/17:1) [my box] {8};
  \node (9) at (9*360/17:1) [my box] {9};

  \draw [] (15) to (7);
  \draw [] (14) to (16);
  \draw [] (5) to (6);
  \draw [] (8) to (10);
  \draw [] (17) to (9);
  \draw [] (4) to (8);
  \draw [] (2) to (6);
  \draw [] (2) to (10);
  \draw [] (7) to (9);
  \draw [] (16) to (3);
  \draw [] (14) to (5);
  \draw [] (3) to (7);
  \draw [] (1) to (9);
  \draw [] (15) to (17);
  \draw [] (1) to (2);
  \draw [] (17) to (8);
  \draw [] (11) to (7);
  \draw [] (3) to (5);
  \draw [] (1) to (17);
  \draw [] (4) to (5);
  \draw [] (11) to (9);
  \draw [] (6) to (8);
  \draw [] (16) to (17);
  \draw [] (2) to (11);
  \draw [] (17) to (4);
  \draw [] (13) to (14);
  \draw [] (2) to (4);
  \draw [] (16) to (7);
  \draw [] (8) to (9);
  \draw [] (3) to (4);
  \draw [] (2) to (15);
  \draw [] (6) to (10);
  \draw [] (7) to (8);
  \draw [] (13) to (9);
  \draw [] (11) to (15);
  \draw [] (6) to (7);
  \draw [] (1) to (14);
  \draw [] (1) to (5);
  \draw [] (11) to (13);
  \draw [] (12) to (13);
  \draw [] (5) to (7);
  \draw [] (14) to (10);
  \draw [] (11) to (12);
  \draw [] (2) to (17);
  \draw [] (1) to (10);
  \draw [] (4) to (6);
  \draw [] (12) to (8);
  \draw [] (13) to (5);
  \draw [] (13) to (17);
  \draw [] (12) to (14);
  \draw [] (13) to (4);
  \draw [] (1) to (16);
  \draw [] (2) to (3);
  \draw [] (9) to (10);
  \draw [] (11) to (3);
  \draw [] (12) to (3);
  \draw [] (15) to (16);
  \draw [] (11) to (10);
  \draw [] (12) to (10);
  \draw [] (14) to (15);
  \draw [] (15) to (6);
  \draw [] (16) to (8);
  \draw [] (13) to (15);
  \draw [] (14) to (6);
  \draw [] (5) to (9);
  \draw [] (12) to (16);
  \draw [] (1) to (3);
  \draw [] (12) to (4);

\end{tikzpicture}
\end{center}

\begin{center}
\begin{tikzpicture}
    [scale=4,
     my box/.style={draw,circle,minimum size=1.5em,inner sep=.1em,outer sep=.3em},
    ]

  \node (1) at (1*360/17*3:1) [my box] {1};
  \node (10) at (10*360/17*3:1) [my box] {10};
  \node (11) at (11*360/17*3:1) [my box] {11};
  \node (12) at (12*360/17*3:1) [my box] {12};
  \node (13) at (13*360/17*3:1) [my box] {13};
  \node (14) at (14*360/17*3:1) [my box] {14};
  \node (15) at (15*360/17*3:1) [my box] {15};
  \node (16) at (16*360/17*3:1) [my box] {16};
  \node (17) at (17*360/17*3:1) [my box] {17};
  \node (2) at (2*360/17*3:1) [my box] {2};
  \node (3) at (3*360/17*3:1) [my box] {3};
  \node (4) at (4*360/17*3:1) [my box] {4};
  \node (5) at (5*360/17*3:1) [my box] {5};
  \node (6) at (6*360/17*3:1) [my box] {6};
  \node (7) at (7*360/17*3:1) [my box] {7};
  \node (8) at (8*360/17*3:1) [my box] {8};
  \node (9) at (9*360/17*3:1) [my box] {9};

  \draw [] (15) to (7);
  \draw [] (14) to (16);
  \draw [] (5) to (6);
  \draw [] (8) to (10);
  \draw [] (17) to (9);
  \draw [] (4) to (8);
  \draw [] (2) to (6);
  \draw [] (2) to (10);
  \draw [] (7) to (9);
  \draw [] (16) to (3);
  \draw [] (14) to (5);
  \draw [] (3) to (7);
  \draw [] (1) to (9);
  \draw [] (15) to (17);
  \draw [] (1) to (2);
  \draw [] (17) to (8);
  \draw [] (11) to (7);
  \draw [] (3) to (5);
  \draw [] (1) to (17);
  \draw [] (4) to (5);
  \draw [] (11) to (9);
  \draw [] (6) to (8);
  \draw [] (16) to (17);
  \draw [] (2) to (11);
  \draw [] (17) to (4);
  \draw [] (13) to (14);
  \draw [] (2) to (4);
  \draw [] (16) to (7);
  \draw [] (8) to (9);
  \draw [] (3) to (4);
  \draw [] (2) to (15);
  \draw [] (6) to (10);
  \draw [] (7) to (8);
  \draw [] (13) to (9);
  \draw [] (11) to (15);
  \draw [] (6) to (7);
  \draw [] (1) to (14);
  \draw [] (1) to (5);
  \draw [] (11) to (13);
  \draw [] (12) to (13);
  \draw [] (5) to (7);
  \draw [] (14) to (10);
  \draw [] (11) to (12);
  \draw [] (2) to (17);
  \draw [] (1) to (10);
  \draw [] (4) to (6);
  \draw [] (12) to (8);
  \draw [] (13) to (5);
  \draw [] (13) to (17);
  \draw [] (12) to (14);
  \draw [] (13) to (4);
  \draw [] (1) to (16);
  \draw [] (2) to (3);
  \draw [] (9) to (10);
  \draw [] (11) to (3);
  \draw [] (12) to (3);
  \draw [] (15) to (16);
  \draw [] (11) to (10);
  \draw [] (12) to (10);
  \draw [] (14) to (15);
  \draw [] (15) to (6);
  \draw [] (16) to (8);
  \draw [] (13) to (15);
  \draw [] (14) to (6);
  \draw [] (5) to (9);
  \draw [] (12) to (16);
  \draw [] (1) to (3);
  \draw [] (12) to (4);

\end{tikzpicture}
\end{center}

%------------------------------------------------------------------------------
\section{Rook Grid 4,4}

\url{https://hog.grinvin.org/ViewGraphInfo.action?id=30317}

\smallskip

Each vertex is a square of a 4x4 chess board.  Edges are moves a rook
can make, being anywhere in same row or same column.

This graph is co-spectral with the Shrikhande graph.  The
characteristic polynomial of both their adjacency matrices is
 $(x-6) . (x-2)^6 . (x+2)^9$.

\smallskip

cf \url{http://www.win.tue.nl/~aeb/drg/graphs/Shrikhande.html}

\begin{center}
\begin{tikzpicture}
    [scale=3,
     my box/.style={draw,circle,minimum size=1.5em,inner sep=.1em,outer sep=.3em},
    ]
  \newcommand\MyW{-.2}

  \node (1) at (0,0) [my box] {1};
  \node (2) at (1,0-\MyW) [my box] {2};
  \node (3) at (2,0-\MyW) [my box] {3};
  \node (4) at (3,0) [my box] {4};
  \node (5) at (0-\MyW,1) [my box] {5};
  \node (6) at (1-\MyW, 1-\MyW) [my box] {6};
  \node (7) at (2+\MyW, 1-\MyW) [my box] {7};
  \node (8) at (3+\MyW,1) [my box] {8};
  \node (9) at (0-\MyW,2) [my box] {9};
  \node (10) at (1-\MyW, 2+\MyW) [my box] {10};
  \node (11) at (2+\MyW, 2+\MyW) [my box] {11};
  \node (12) at (3+\MyW,2) [my box] {12};
  \node (13) at (0,3) [my box] {13};
  \node (14) at (1,3+\MyW) [my box] {14};
  \node (15) at (2,3+\MyW) [my box] {15};
  \node (16) at (3,3) [my box] {16};

  \draw [] (5) to (8);
  \draw [] (10) to (11);
  \draw [] (6) to (7);
  \draw [] (10) to (2);
  \draw [] (5) to (6);
  \draw [] (3) to (7);
  \draw [] (7) to (8);
  \draw [] (4) to (8);
  \draw [] (14) to (6);
  \draw [] (6) to (8);
  \draw [] (12) to (8);
  \draw [] (5) to (13);
  \draw [] (10) to (12);
  \draw [] (2) to (4);
  \draw [] (1) to (4);
  \draw [] (10) to (6);
  \draw [] (12) to (4);
  \draw [] (12) to (9);
  \draw [] (11) to (3);
  \draw [] (9) to (13);
  \draw [] (1) to (9);
  \draw [] (11) to (7);
  \draw [] (2) to (3);
  \draw [] (10) to (9);
  \draw [] (16) to (13);
  \draw [] (15) to (16);
  \draw [] (10) to (14);
  \draw [] (11) to (9);
  \draw [] (3) to (4);
  \draw [] (5) to (7);
  \draw [] (15) to (7);
  \draw [] (1) to (13);
  \draw [] (15) to (3);
  \draw [] (15) to (13);
  \draw [] (11) to (15);
  \draw [] (14) to (13);
  \draw [] (5) to (9);
  \draw [] (16) to (8);
  \draw [] (14) to (15);
  \draw [] (1) to (5);
  \draw [] (1) to (2);
  \draw [] (12) to (16);
  \draw [] (1) to (3);
  \draw [] (16) to (4);
  \draw [] (14) to (2);
  \draw [] (14) to (16);
  \draw [] (11) to (12);
  \draw [] (2) to (6);

\end{tikzpicture}
\end{center}


%------------------------------------------------------------------------------
\section{Regular But Median Size Less than N}

polyhedral of vertices
A000944 0, 0, 0, 1, 2, 7, 34, 257, 2606, \dots
n =     1  2  3  4  5  6   7    8     9

n=8 regular 1 but median size 4 is not all vertices
\begin{center}
\begin{tikzpicture}
    [scale=2,
     my box/.style={draw,minimum size=2em,inner sep=.1em,outer sep=.3em},
    ]

  \node (0) at (1.5, .4) [my box] {0};
  \node (1) at (2, -1) [my box] {1};
  \node (2) at (.5,0) [my box] {2};
  \node (3) at (2,1) [my box] {3};
  \node (4) at (1,-1) [my box] {4};
  \node (5) at (1.5, -.4) [my box] {5};
  \node (6) at (2.5,0) [my box] {6};
  \node (7) at (1,1) [my box] {7};

  \draw [] (0) to (7);
  \draw [] (4) to (5);
  \draw [] (0) to (5);
  \draw [] (1) to (5);
  \draw [] (2) to (4);
  \draw [] (1) to (4);
  \draw [] (0) to (3);
  \draw [] (2) to (7);
  \draw [] (1) to (6);
  \draw [] (2) to (6);
  \draw [] (3) to (7);
  \draw [] (3) to (6);

\end{tikzpicture}
\end{center}

\begin{center}
\begin{tikzpicture}
    [scale=2,
     my box/.style={draw,minimum size=2em,inner sep=.1em,outer sep=.3em},
    ]

  \node (0) at (0,0) [my box] {0};
  \node (1) at (-1,1) [my box] {1};
  \node (2) at (-2, 2) [my box] {2};
  \node (3) at (1, 1) [my box] {3};
  \node (4) at (-2, 1) [my box] {4};
  \node (5) at (-1, 0) [my box] {5};
  \node (6) at (1, 2) [my box] {6};
  \node (7) at (0,1) [my box] {7};

  \draw [] (0) to (7);
  \draw [] (4) to (5);
  \draw [] (0) to (5);
  \draw [] (1) to (5);
  \draw [] (2) to (4);
  \draw [] (1) to (4);
  \draw [] (0) to (3);
  \draw [] (2) to (7);
  \draw [] (1) to (6);
  \draw [] (2) to (6);
  \draw [] (3) to (7);
  \draw [] (3) to (6);

\end{tikzpicture}
\end{center}

\begin{center}
\begin{tikzpicture}
    [scale=2,
     my box/.style={draw,minimum size=2em,inner sep=.1em,outer sep=.3em},
    ]

  \node (0) at (0/8*360:1) [my box] {0};
  \node (1) at (3/8*360:1) [my box] {1};
  \node (2) at (6/8*360:1) [my box] {2};
  \node (3) at (1/8*360:1) [my box] {3};
  \node (4) at (5/8*360:1) [my box] {4};
  \node (5) at (4/8*360:1) [my box] {5};
  \node (6) at (2/8*360:1) [my box] {6};
  \node (7) at (7/8*360:1) [my box] {7};

  \draw [] (0) to (7);
  \draw [] (4) to (5);
  \draw [] (0) to (5);
  \draw [] (1) to (5);
  \draw [] (2) to (4);
  \draw [] (1) to (4);
  \draw [] (0) to (3);
  \draw [] (2) to (7);
  \draw [] (1) to (6);
  \draw [] (2) to (6);
  \draw [] (3) to (7);
  \draw [] (3) to (6);

\end{tikzpicture}
\end{center}

\begin{center}
\begin{tikzpicture}
    [scale=2,
     my box/.style={draw,minimum size=2em,inner sep=.1em,outer sep=.3em},
    ]

  \node (0) at (6/8*360:1) [my box] {0};
  \node (1) at (3/8*360:1) [my box] {1};
  \node (2) at (1/8*360:1) [my box] {2};
  \node (3) at (7/8*360:1) [my box] {3};
  \node (4) at (4/8*360:1) [my box] {4};
  \node (5) at (5/8*360:1) [my box] {5};
  \node (6) at (2/8*360:1) [my box] {6};
  \node (7) at (0/8*360:1) [my box] {7};

  \draw [] (0) to (7);
  \draw [] (4) to (5);
  \draw [] (0) to (5);
  \draw [] (1) to (5);
  \draw [] (2) to (4);
  \draw [] (1) to (4);
  \draw [] (0) to (3);
  \draw [] (2) to (7);
  \draw [] (1) to (6);
  \draw [] (2) to (6);
  \draw [] (3) to (7);
  \draw [] (3) to (6);

\end{tikzpicture}
\end{center}

XXXXXXXX

n=8 regular 1 but median size 4 is not all vertices
\begin{center}
\begin{tikzpicture}
    [scale=1,
     my box/.style={draw,minimum size=2em,inner sep=.1em,outer sep=.3em},
    ]

  \node (0) at (0,-2) [my box] {0};
  \node (1) at (2,1) [my box] {1};
  \node (2) at (-2,1) [my box] {2};
  \node (3) at (0,-.7) [my box] {3};
  \node (4) at (0,2) [my box] {4};
  \node (5) at (0,.7) [my box] {5};
  \node (6) at (2,-1) [my box] {6};
  \node (7) at (-2,-1) [my box] {7};

  \draw [] (2) to (7);
  \draw [] (0) to (6);
  \draw [] (0) to (3);
  \draw [] (2) to (5);
  \draw [] (4) to (5);
  \draw [] (1) to (5);
  \draw [] (1) to (4);
  \draw [] (2) to (4);
  \draw [] (0) to (7);
  \draw [] (3) to (6);
  \draw [] (3) to (7);
  \draw [] (1) to (6);

\end{tikzpicture}
\end{center}

XXXXXXXX

n=9 not regular but median size = 9

\begin{center}
\begin{tikzpicture}
    [scale=1.2,
     my box/.style={draw,minimum size=2em,inner sep=.1em,outer sep=.3em},
    ]

  \node (0) at (2.8, 1.9) [my box] {0};
  \node (1) at (-.8, 1.9) [my box] {1};

  \node (3) at (2, 1.3) [my box] {3};
  \node (7) at (3,0) [my box] {7};
  \node (4) at (.2, 1.4) [my box] {4};
  \node (8) at (-.9, 0) [my box] {8};
  \node (6) at (1, 2.8) [my box] {6};
  \node (5) at (1,.5) [my box] {5};
  \node (2) at (1,-.5) [my box] {2};

  \draw [] (0) to (6);
  \draw [] (1) to (4);
  \draw [] (2) to (7);
  \draw [] (1) to (6);
  \draw [] (3) to (7);
  \draw [] (0) to (3);
  \draw [] (2) to (8);
  \draw [] (5) to (8);
  \draw [] (1) to (8);
  \draw [] (2) to (5);
  \draw [] (4) to (6);
  \draw [] (5) to (7);
  \draw [] (3) to (6);
  \draw [] (0) to (7);
  \draw [] (4) to (8);

\end{tikzpicture}
\end{center}

%------------------------------------------------------------------------------
\section{n=8 Pseudosimilar which are Line Graph}

\subsection{Smallest Pseudosimilar}

edges=9, is smallest of all n=8 pseudosimilars

\url{https://hog.grinvin.org/ViewGraphInfo.action?id=30306}

\smallskip

Harary and Palmer give this graph as the fewest vertices (n=8), and
for those vertices fewest edges (e=9), which has a pair of
pseudosimilar vertices.  These vertices are the degree-3s distance 2
apart (no other pair is pseudosimilar).

Frank Harary and Ed Palmer, ``On Similar Points of a Graph'', Journal of
Mathematics and Mechanics, volume 15, number 4, 1966, pages 623-630.

And shown in Harary, ``Graph Theory'', Addison-Wesley, 1969, chapter 14
``Groups'', figure 14.7, page 171.


\smallskip

\begin{center}
\begin{tikzpicture}
    [scale=1.2,
     my box/.style={draw,minimum size=2em,inner sep=.1em,outer sep=.3em},
    ]

  \node (0) at (2,1) [my box] {0};
  \node (1) at (0,1) [my box] {1};
  \node (2) at (-2,1) [my box] {2};
  \node (3) at (3,2) [my box] {3};
  \node (4) at (1,2) [my box] {4};
  \node (5) at (3,1) [my box] {5};
  \node (6) at (-1,1) [my box] {6};
  \node (7) at (1,1) [my box] {7};

  % 9 edges
  \draw [] (4) to (7);
  \draw [] (0) to (5);
  \draw [] (1) to (6);
  \draw [] (1) to (7);
  \draw [] (2) to (6);

  \draw [] (0) to (7);
  \draw [] (3) to (5);
  \draw [] (1) to (4);
  \draw [] (0) to (3);

\end{tikzpicture}
\end{center}

Graph remaining after either pseudosimilar vertex removed is \newline
\url{https://hog.grinvin.org/ViewGraphInfo.action?id=30308}

%------------------
\subsection{Line Graph n=8 Pseudosimilar Most Edges}

\url{https://hog.grinvin.org/ViewGraphInfo.action?id=30339}

\smallskip

This is a line graph with a pair of pseudosimilar vertices (the two
degree-3s with single degree-2 neighbour each).  n=8 vertices here is
the fewest where any graph has pseudosimilar vertices.  There are 4
n=8 which are line graphs.  This one has the second fewest edges (and
is a subgraph of the most and second most).

\smallskip

edges=10, vertices 1,5 pseudosimilar
\begin{center}
\begin{tikzpicture}
    [scale=1.2,
     my box/.style={draw,minimum size=2em,inner sep=.1em,outer sep=.3em},
     my circle box/.style={draw,circle,minimum size=2em,inner sep=.1em,outer sep=.3em},
    ]

  \node (0) at (1,1) [my box] {0};
  \node (1) at (-1,1) [my circle box] {1};
  \node (2) at (-.5,2) [my box] {2};
  \node (3) at (2,1) [my box] {3};
  \node (4) at (-2,1) [my box] {4};
  \node (5) at (1,2) [my circle box] {5};
  \node (6) at (-2,2) [my box] {6};
  \node (7) at (0,1) [my box] {7};

  % 10 edges
  \draw [] (1) to (6);
  \draw [] (5) to (7);
  \draw [] (0) to (5);
  \draw [] (1) to (7);
  \draw [] (0) to (3);

  \draw [] (1) to (4);
  \draw [] (2) to (5);
  \draw [] (4) to (6);
  \draw [] (2) to (6);
  \draw [] (0) to (7);

\end{tikzpicture}
\end{center}

%------------------
\subsection{Line Graph n=8 Pseudosimilar Second Most Edges}

\url{https://hog.grinvin.org/ViewGraphInfo.action?id=30337}

\smallskip

This is a line graph with a pair of pseudosimilar vertices (the two
degree-3s with degree-3 in between).  n=8 vertices here is the fewest
where any graph has pseudosimilar vertices.  There are 4 n=8 which are
line graphs.  This one has the second most edges (and is a subgraph of
the most).

\smallskip

edges=12, vertices 5,6 pseudosimilar

degree 3s
\begin{center}
\begin{tikzpicture}
    [scale=1.25,
     my box/.style={draw,minimum size=2em,inner sep=.1em,outer sep=.3em},
     my circle box/.style={draw,circle,minimum size=2em,inner sep=.1em,outer sep=.3em},
    ]

  \node (0) at (1,1) [my box] {0};
  \node (1) at (3,2) [my box] {1};
  \node (2) at (2,0) [my box] {2};
  \node (3) at (0,0) [my box] {3};
  \node (4) at (2,2) [my box] {4};
  \node (5) at (3,1) [my circle box] {5};
  \node (6) at (1,0) [my circle box] {6};
  \node (7) at (2,1) [my box] {7};

  % 12 edges
  \draw [] (0) to (7);
  \draw [] (1) to (4);
  \draw [] (2) to (5);
  \draw [] (5) to (7);
  \draw [] (2) to (7);

  \draw [] (4) to (7);
  \draw [] (3) to (6);
  \draw [] (0) to (6);
  \draw [] (0) to (3);
  \draw [] (2) to (6);

  \draw [] (1) to (5);
  \draw [] (0) to (4);

\end{tikzpicture}
\end{center}

%------------------
\subsection{Line Graph n=8 Pseudosimilar Most Edges}

edges=17

\url{https://hog.grinvin.org/ViewGraphInfo.action?id=30335}

\smallskip

This is a line graph with a pair of pseudosimilar vertices (the
non-adjacent degree-4s).  n=8 vertices here is the fewest where any
graph has pseudosimilar vertices.  There are 4 n=8 which are line
graphs.  This one has the most edges.

\smallskip

drawn to show contains second biggest as subgraph
\begin{center}
\begin{tikzpicture}
    [scale=1.2,yscale=1.3,xscale=1.5,
     my box/.style={draw,minimum size=2em,inner sep=.1em,outer sep=.3em},
     my circle box/.style={draw,circle,minimum size=2em,inner sep=.1em,outer sep=.3em},
    ]

  \node (0) at (3,2) [my box] {0};
  \node (1) at (0,0) [my box] {1};
  \node (3) at (1,1) [my box] {3};

  \node (2) at (2,0) [my box] {2};

  \node (4) at (3,1) [my circle box] {4};
  \node (5) at (1,0) [my circle box] {5};
  \node (6) at (2,2) [my box] {6};

  \node (7) at (2,1) [my box] {7};

  % 17 edges
  \draw [] (0) to (3);
  \draw [] (2) to (4);
  \draw [] (3) to (6);
  \draw [] (0) to (4);
  \draw [bend right=25] (2) to (6);

  \draw [] (6) to (7);
  \draw [] (1) to (3);
  \draw [] (3) to (5);
  \draw [] (2) to (5);
  \draw [] (0) to (6);

  \draw [] (4) to (6);
  \draw [] (1) to (7);
  \draw [] (3) to (7);
  \draw [] (1) to (5);
  \draw [] (5) to (7);

  \draw [] (4) to (7);
  \draw [] (2) to (7);

\end{tikzpicture}
\end{center}

vertices 4,5 pseudosimilar
\begin{center}
\begin{tikzpicture}
    [scale=1.2,yscale=1.3,xscale=1.5,
     my box/.style={draw,minimum size=2em,inner sep=.1em,outer sep=.3em},
     my circle box/.style={draw,circle,minimum size=2em,inner sep=.1em,outer sep=.3em},
    ]

  \node (0) at (-2,0) [my box] {0};
  \node (1) at (300:1) [my box] {1};
  \node (2) at (60:1) [my box] {2};
  \node (3) at (240:1) [my box] {3};
  \node (4) at (120:1) [my circle box] {4};
  \node (5) at (1,0) [my circle box] {5};
  \node (6) at (-1,0) [my box] {6};
  \node (7) at (0, 0) [my box] {7};

  % 17 edges
  \draw [] (0) to (3);
  \draw [] (2) to (4);
  \draw [] (3) to (6);
  \draw [] (0) to (4);
  \draw [] (2) to (6);

  \draw [] (6) to (7);
  \draw [] (1) to (3);
  \draw [] (3) to (5);
  \draw [] (2) to (5);
  \draw [] (0) to (6);

  \draw [] (4) to (6);
  \draw [] (1) to (7);
  \draw [] (3) to (7);
  \draw [] (1) to (5);
  \draw [] (5) to (7);

  \draw [] (4) to (7);
  \draw [] (2) to (7);

\end{tikzpicture}
\end{center}

root graph of the line graph
\begin{center}
\begin{tikzpicture}
    [scale=1.2,yscale=1.3,xscale=1.5,
     my box/.style={draw,minimum size=2em,inner sep=.1em,outer sep=.3em},
    ]
  \node (0) at (1,0) [my box] {0};
  \node (1) at (3,1) [my box] {1};
  \node (2) at (3,0) [my box] {2};
  \node (3) at (1,1) [my box] {3};
  \node (4) at (2,0) [my box] {4};
  \node (5) at (2,1) [my box] {5};

  \draw [] (4) to (1);
  \draw [] (1) to (5);
  \draw [] (3) to (5);
  \draw [] (0) to (5);
  \draw [] (4) to (5);
  \draw [] (0) to (4);
  \draw [] (4) to (2);
  \draw [] (0) to (3);
\end{tikzpicture}
\end{center}




%------------------------------------------------------------------------------
\section{Pseudosimilar n=8 Subgraph Relations}

\begin{center}
\begin{tikzpicture}
    [scale=1.2,yscale=.8,
     my box/.style={draw,inner sep=.1em,outer sep=0em,font=\scriptsize},
     rotate=90,
    ]
  \node (291-19) at (303,19) [my box] {291,19};

  \node (294-18) at (294,19) [my box] {294,18};
  \node (297-18) at (298,19) [my box] {297,18};
  \node (291-18) at (303,18) [my box] {291,18};

  \node (294-17) at (292,17) [my box] {294,17};
  \node (297-17) at (295,17) [my box] {297,17};
  \node (291-17) at (298,18) [my box] {291,17};

  \node (291-16) at (290,16) [my box] {291,16};
  \node (294-16) at (294,16) [my box] {294,16};
  \node (297-16) at (297,16) [my box] {297,16};
  \node (300-16) at (300,16) [my box] {300,16};
  \node (306-16) at (307,19) [my box] {306,16};
  \node (303-16) at (302,17) [my box] {303,16};

  \node (291-15) at (291,15) [my box] {291,15};
  \node (294-15) at (294,15) [my box] {294,15};
  \node (297-15) at (297,15) [my box] {297,15};
  \node (300-15) at (300,15) [my box] {300,15};
  \node (303-15) at (302,16) [my box] {303,15};
  \node (306-15) at (305,15) [my box] {306,15};
  \node (309-15) at (309,15) [my box] {309,15};




  \node (291-9) at (296,9) [my box] {291,9};

  \node (291-10) at (291, 9) [my box] {291,10};
  \node (294-10) at (294, 9) [my box] {294,10};
  \node (297-10) at (298, 9) [my box] {297,10};

  \node (291-11) at (291,11) [my box] {291,11};
  \node (294-11) at (294,11) [my box] {294,11};
  \node (297-11) at (301,10) [my box] {297,11};

  \node (291-12) at (290,12) [my box] {291,12};
  \node (294-12) at (294,12) [my box] {294,12};
  \node (297-12) at (297,12) [my box] {297,12};
  \node (300-12) at (300,12) [my box] {300,12};
  \node (303-12) at (303,12) [my box] {303,12};
  \node (306-12) at (306,12) [my box] {306,12};

  \node (291-13) at (290,13) [my box] {291,13};
  \node (294-13) at (294,13) [my box] {294,13};
  \node (297-13) at (297,13) [my box] {297,13};
  \node (300-13) at (300,13) [my box] {300,13};
  \node (303-13) at (303,13) [my box] {303,13};
  \node (306-13) at (306,13) [my box] {306,13};
  \node (309-13) at (309,13) [my box] {309,13};

  \node (291-14) at (291,14) [my box] {291,14};
  \node (294-14) at (294,14) [my box] {294,14};
  \node (297-14) at (297,14) [my box] {297,14};
  \node (300-14) at (300,14) [my box] {300,14};


  \draw [<-] (291-10) to (291-11);
  \draw [<-] (294-14) to (303-16);
  \draw [<-] (294-11) to (291-12);
  \draw [<-] (294-10) to (300-14);
  \draw [<-] (291-9) to (297-12);
  \draw [<-] (297-12) to (291-15);
  \draw [<-] (291-14) to (291-17);
  \draw [<-] (297-12) to (294-16);
  \draw [<-] (291-15) to (297-17);
  \draw [<-] (303-15) to (291-18);
  \draw [<-] (300-13) to (306-15);
  \draw [<-] (309-13) to (294-16);
  \draw [<-] (294-12) to (294-14);
  \draw [<-] (300-14) to (291-17);
  \draw [<-] (297-13) to (294-17);
  \draw [<-] (294-13) to (294-15);
  \draw [<-] (297-14) to (309-15);
  \draw [<-] (294-16) to (291-17);
  \draw [<-] (300-15) to (291-18);
  \draw [<-] (291-9) to (294-10);
  \draw [<-] (291-13) to (303-15);
  \draw [<-] (303-12) to (297-14);
  \draw [<-] (303-12) to (306-13);
  \draw [<-] (294-11) to (291-14);
  \draw [<-] (300-12) to (297-13);
  \draw [<-] (297-10) to (297-12);
  \draw [<-] (297-16) to (291-19);
  \draw [<-] (294-14) to (297-17);
  \draw [<-] (303-13) to (294-17);
  \draw [<-] (291-12) to (300-16);
  \draw [<-] (306-15) to (306-16);
  \draw [<-] (291-11) to (297-14);
  \draw [<-] (294-12) to (300-15);
  \draw [<-] (291-11) to (297-13);
  \draw [<-] (294-16) to (294-17);
  \draw [<-] (291-9) to (300-12);
  \draw [<-] (291-14) to (294-17);
  \draw [<-] (291-14) to (303-16);
  \draw [<-] (294-14) to (294-17);
  \draw [<-] (297-11) to (291-12);
  \draw [<-] (297-10) to (294-11);
  \draw [<-] (309-15) to (297-16);
  \draw [<-] (300-14) to (306-16);
  \draw [<-] (291-18) to (291-19);
  \draw [<-] (291-11) to (291-14);
  \draw [<-] (300-15) to (294-18);
  \draw [<-] (291-15) to (291-16);
  \draw [<-] (294-11) to (297-14);
  \draw [<-] (303-16) to (291-19);
  \draw [<-] (306-12) to (300-13);
  \draw [<-] (297-15) to (294-18);
  \draw [<-] (291-12) to (294-17);
  \draw [<-] (294-10) to (297-14);
  \draw [<-] (297-14) to (291-18);
  \draw [<-] (291-16) to (291-18);
  \draw [<-] (297-10) to (291-13);
  \draw [<-] (297-12) to (303-13);
  \draw [<-] (300-13) to (300-14);
  \draw [<-] (291-12) to (297-15);
  \draw [<-] (297-10) to (297-11);
  \draw [<-] (297-13) to (303-16);
  \draw [<-] (294-15) to (297-18);
  \draw [<-] (291-10) to (294-12);
  \draw [<-] (300-16) to (291-19);
  \draw [<-] (291-11) to (303-15);
  \draw [<-] (294-12) to (291-13);
  \draw [<-] (297-13) to (297-17);
  \draw [<-] (303-12) to (306-15);
  \draw [<-] (300-12) to (294-14);
  \draw [<-] (297-12) to (306-15);
  \draw [<-] (306-13) to (300-16);
  \draw [<-] (291-12) to (294-15);
  \draw [<-] (294-10) to (297-13);
  \draw [<-] (303-15) to (297-17);
  \draw [<-] (294-10) to (303-13);
  \draw [<-] (303-12) to (291-16);
  \draw [<-] (297-13) to (306-15);
  \draw [<-] (297-15) to (291-17);
  \draw [<-] (297-11) to (309-15);
  \draw [<-] (306-12) to (309-13);
  \draw [<-] (294-14) to (294-15);
  \draw [<-] (294-14) to (291-17);
  \draw [<-] (306-13) to (306-16);
  \draw [<-] (309-13) to (300-14);
  \draw [<-] (294-17) to (297-18);
  \draw [<-] (291-12) to (303-16);
  \draw [<-] (291-16) to (294-18);
  \draw [<-] (303-13) to (291-16);
  \draw [<-] (303-13) to (303-15);
  \draw [<-] (297-11) to (303-15);
  \draw [<-] (303-15) to (303-16);
  \draw [<-] (294-11) to (294-14);
  \draw [<-] (291-13) to (291-14);
  \draw [<-] (294-13) to (297-17);
  \draw [<-] (294-13) to (294-17);
  \draw [<-] (297-17) to (294-18);
  \draw [<-] (297-11) to (291-15);
  \draw [<-] (297-13) to (300-15);
  \draw [<-] (306-13) to (303-15);
  \draw [<-] (297-14) to (303-16);
  \draw [<-] (300-12) to (291-14);
  \draw [<-] (291-10) to (309-13);
  \draw [<-] (309-15) to (294-17);
  \draw [<-] (291-12) to (300-14);
  \draw [<-] (297-16) to (297-18);
  \draw [<-] (294-15) to (291-16);
  \draw [<-] (294-10) to (294-12);
  \draw [<-] (300-16) to (297-18);
  \draw [<-] (297-11) to (294-16);
  \draw [<-] (297-13) to (294-15);
  \draw [<-] (291-14) to (297-17);
  \draw [<-] (303-16) to (294-18);
  \draw [<-] (303-13) to (300-16);
  \draw [<-] (303-16) to (297-18);
  \draw [<-] (300-12) to (303-15);
  \draw [<-] (297-11) to (300-13);
  \draw [<-] (291-11) to (294-14);
  \draw [<-] (306-13) to (309-15);
  \draw [<-] (297-14) to (294-16);
  \draw [<-] (294-11) to (297-13);
  \draw [<-] (294-12) to (294-13);
  \draw [<-] (297-14) to (297-17);
  \draw [<-] (291-10) to (300-12);
  \draw [<-] (294-13) to (300-16);
  \draw [<-] (300-14) to (291-18);
  \draw [<-] (300-12) to (300-14);
  \draw [<-] (291-14) to (291-16);
  \draw [<-] (291-9) to (306-12);
  \draw [<-] (291-11) to (300-14);
  \draw [<-] (300-15) to (300-16);
  \draw [<-] (297-12) to (300-15);
  \draw [<-] (300-14) to (297-17);
  \draw [<-] (294-12) to (306-16);
  \draw [<-] (291-10) to (303-13);
  \draw [<-] (303-13) to (306-16);
  \draw [<-] (291-11) to (294-13);
  \draw [<-] (303-15) to (291-17);
  \draw [<-] (297-11) to (294-14);
  \draw [<-] (297-10) to (306-12);
  \draw [<-] (291-13) to (291-15);
  \draw [<-] (297-11) to (300-15);
  \draw [<-] (291-13) to (294-16);
  \draw [<-] (303-12) to (300-15);
  \draw [<-] (300-13) to (294-17);
  \draw [<-] (297-15) to (297-16);
  \draw [<-] (291-17) to (297-18);
  \draw [<-] (291-11) to (291-15);
  \draw [<-] (297-11) to (291-14);
  \draw [<-] (297-14) to (297-15);
  \draw [<-] (294-11) to (309-13);
  \draw [<-] (297-10) to (300-12);
  \draw [<-] (300-12) to (294-16);

\end{tikzpicture}
\end{center}




%------------------------------------------------------------------------------
\section{Crown 4}

\begin{center}
\begin{tikzpicture}
    [scale=1.2,
     my box/.style={draw,minimum size=2em,inner sep=.1em,outer sep=.3em},
    ]

  \node (1) at (0,0) [my box] {1};
  \node (2) at (0,1) [my box] {2};
  \node (3) at (0,2) [my box] {3};
  \node (4) at (0,3) [my box] {4};

  \node (5) at (3,0) [my box] {5};
  \node (6) at (3,1) [my box] {6};
  \node (7) at (3,2) [my box] {7};
  \node (8) at (3,3) [my box] {8};

  \draw [] (2) to (5);
  \draw [] (6) to (3);
  \draw [] (7) to (4);
  \draw [] (1) to (6);
  \draw [] (8) to (3);
  \draw [] (1) to (7);
  \draw [] (8) to (2);
  \draw [] (7) to (2);
  \draw [] (6) to (4);
  \draw [] (5) to (3);
  \draw [] (5) to (4);
  \draw [] (1) to (8);

\end{tikzpicture}
\end{center}



%------------------------------------------------------------------------------
\section{Most Minimum Dominating Sets}

Some same as most minimal domsets.  (RECHECK)

n=5
\kern1em
\url{https://hog.grinvin.org/ViewGraphInfo.action?id=438}

\begin{center}
\begin{tikzpicture}
    [scale=1.2,
     my box/.style={draw,minimum size=2em,inner sep=.1em,outer sep=.3em},
    ]

  \node (0) at (0,0) [my box] {0};
  \node (1) at (2,0) [my box] {1};
  \node (2) at (1,0) [my box] {2};
  \node (3) at (1,1) [my box] {3};
  \node (4) at (1,-1) [my box] {4};

  \draw [] (0) to (4);
  \draw [] (0) to (2);
  \draw [] (1) to (3);
  \draw [] (0) to (3);
  \draw [] (1) to (2);
  \draw [] (2) to (4);
  \draw [] (1) to (4);

\end{tikzpicture}
\end{center}

%------

n=6 domnum 3 in 15 ways

complete-5 less 3 non-touching edges

\url{https://hog.grinvin.org/ViewGraphInfo.action?id=226}

\begin{center}
\begin{tikzpicture}
    [scale=1,
     my box/.style={draw,minimum size=2em,inner sep=.1em,outer sep=.3em},
    ]

  \node (0) at (3,2) [my box] {0};
  \node (1) at (1,0) [my box] {1};
  \node (2) at (-.5,1) [my box] {2};
  \node (3) at (4.5,1) [my box] {3};
  \node (4) at (3,0) [my box] {4};
  \node (5) at (1,2) [my box] {5};

  \draw [] (0) to (4);
  \draw [] (3) to (4);
  \draw [] (1) to (5);
  \draw [] (0) to (2);
  \draw [] (0) to (5);
  \draw [] (0) to (3);
  \draw [] (2) to (5);
  \draw [] (2) to (4);
  \draw [] (3) to (5);
  \draw [] (1) to (3);
  \draw [] (1) to (2);
  \draw [] (1) to (4);

\end{tikzpicture}
\end{center}

%------

n=7 domnum 3 in 22 ways

\url{https://hog.grinvin.org/ViewGraphInfo.action?id=868}

\begin{center}
\begin{tikzpicture}
    [scale=1,
     my box/.style={draw,minimum size=2em,inner sep=.1em,outer sep=.3em},
    ]

  \node (0) at (5,-1) [my box] {0};
  \node (1) at (1,1) [my box] {1};
  \node (2) at (1,-1) [my box] {2};
  \node (3) at (5,1) [my box] {3};
  \node (4) at (-.5,0) [my box] {4};
  \node (5) at (3,-1) [my box] {5};
  \node (6) at (3,1) [my box] {6};

  \draw [] (2) to (6);
  \draw [] (0) to (3);
  \draw [] (5) to (6);
  \draw [] (1) to (5);
  \draw [] (1) to (6);
  \draw [] (3) to (6);
  \draw [] (2) to (4);
  \draw [] (0) to (5);
  \draw [] (1) to (4);
  \draw [] (2) to (5);

\end{tikzpicture}
\end{center}

%----------------------

n=8 domnum 3 in 36 ways

\url{https://hog.grinvin.org/ViewGraphInfo.action?id=30664}

\begin{center}
\begin{tikzpicture}
    [scale=1.3,
     my box/.style={draw,minimum size=2em,inner sep=.1em,outer sep=.2em},
    ]

  \node (0) at (0,-2.5) [my box] {0};
  \node (1) at (2,-1.5) [my box] {1};
  \node (2) at (2, 1.5) [my box] {2};
  \node (3) at (4, -2.5) [my box] {3};
  \node (4) at (1, 0) [my box] {4};
  \node (5) at (5, 0) [my box] {5};
  \node (6) at (-1, 0) [my box] {6};
  \node (7) at (3, 0) [my box] {7};

  \draw [] (2) to (5);
  \draw [] (0) to (4);
  \draw [] (2) to (4);
  \draw [] (4) to (7);
  \draw [] (1) to (4);
  \draw [] (3) to (7);
  \draw [] (1) to (6);
  \draw [] (2) to (7);
  \draw [] (1) to (7);
  \draw [] (3) to (5);
  \draw [] (1) to (5);
  \draw [] (0) to (3);
  \draw [] (2) to (6);
  \draw [] (0) to (6);

\end{tikzpicture}
\end{center}

\begin{center}
\begin{tikzpicture}
    [scale=1.3,
     my box/.style={draw,minimum size=2em,inner sep=.1em,outer sep=.2em},
    ]

  \node (0) at (-3,0) [my box] {0};
  \node (1) at (-4,2) [my box] {1};
  \node (2) at (0,2) [my box] {2};
  \node (3) at (-.5, 0) [my box] {3};
  \node (4) at (-2,1) [my box] {4};
  \node (5) at (-4,-2) [my box] {5};
  \node (6) at (0,-2) [my box] {6};
  \node (7) at (-2,3) [my box] {7};

  \draw [] (2) to (5);
  \draw [] (0) to (4);
  \draw [] (2) to (4);
  \draw [] (4) to (7);
  \draw [] (1) to (4);
  \draw [] (3) to (7);
  \draw [] (1) to (6);
  \draw [] (2) to (7);
  \draw [] (1) to (7);
  \draw [] (3) to (5);
  \draw [] (1) to (5);
  \draw [] (0) to (3);
  \draw [] (2) to (6);
  \draw [] (0) to (6);

\end{tikzpicture}
\end{center}


%------------------------------------------------------------------------------
\section{6 Vertices Mean Distance 2/3 of Diameter}

12 total

with one hanging
\begin{center}
\begin{tikzpicture}
    [scale=1.2,
     my box/.style={draw,minimum size=2em,inner sep=.1em,outer sep=.3em},
    ]

  \node (2) at (3,0) [my box] {2};
  \node (3) at (-2,0) [my box] {3};
  \node (0) at (0,1) [my box] {0};
  \node (1) at (0,3) [my box] {1};
  \node (4) at (0,2) [my box] {4};
  \node (5) at (2,0) [my box] {5};

  \draw [] (3) to (4);
  \draw [] (0) to (4);
  \draw [] (1) to (5);
  \draw [] (4) to (5);
  \draw [] (3) to (5);
  \draw [] (0) to (5);
  \draw [] (3) to (1);
  \draw [] (1) to (4);
  \draw [] (0) to (3);
  \draw [] (5) to (2);

\end{tikzpicture}
\end{center}




%------------------------------------------------------------------------------
\section{Noughts and Crosses 2x2, 1 Player}

Tesseract induced subgraph.

\url{https://hog.grinvin.org/ViewGraphInfo.action?id=27032}

\begin{center}
\begin{tikzpicture}
    [scale=1.2,
     my box/.style={draw,minimum size=2em,inner sep=.1em,outer sep=.3em},
    ]

  \node (0000) at (0,0) [my box] {0000};

  \node (0001) at (2,1.5) [my box] {0001};
  \node (0010) at (2,.5) [my box] {0010};
  \node (0100) at (2,-.5) [my box] {0100};
  \node (1000) at (2,-1.5) [my box] {1000};

  \node (0011) at (4,2.5) [my box] {0011};
  \node (0101) at (4,1.5) [my box] {0101};
  \node (1001) at (5.25,0) [my box] {1001};
  \node (0110) at (4,0) [my box] {0110};
  \node (1010) at (4,-1.5) [my box] {1010};
  \node (1100) at (4,-2.5) [my box] {1100};

  \draw [->] (0010) to (0011);
  \draw [->] (1000) to (1100);
  \draw [->] (1000) to (1010);
  \draw [->] (0001) to (0011);
  \draw [->] (0010) to (0110);
  \draw [->] (0100) to (1100);
  \draw [->] (1000) to (1001);
  \draw [->] (0100) to (0110);
  \draw [->] (0001) to (1001);
  \draw [->] (0000) to (0001);
  \draw [->] (0000) to (1000);
  \draw [->] (0000) to (0100);
  \draw [->] (0001) to (0101);
  \draw [->] (0010) to (1010);
  \draw [->] (0000) to (0010);
  \draw [->] (0100) to (0101);

\end{tikzpicture}
\end{center}

\begin{center}
\begin{tikzpicture}
    [scale=1.2,
     my box/.style={draw,minimum size=2em,inner sep=.1em,outer sep=.3em},
    ]

  \node (0000) at (0,0) [my box] {0000};

  \node (0001) at (2,1.5) [my box] {0001};
  \node (0010) at (2,.5) [my box] {0010};
  \node (0100) at (2,-.5) [my box] {0100};
  \node (1000) at (2,-1.5) [my box] {1000};

  \node (0011) at (4,2.5) [my box] {0011};
  \node (0110) at (4,1.5) [my box] {0110};
  \node (1100) at (4, .5) [my box] {1100};
  \node (1001) at (4,-.5) [my box] {1001};
  \node (0101) at (4,-1.5) [my box] {0101};
  \node (1010) at (4,-2.5) [my box] {1010};

  \draw [->] (0010) to (0011);
  \draw [->] (1000) to (1100);
  \draw [->] (1000) to (1010);
  \draw [->] (0001) to (0011);
  \draw [->] (0010) to (0110);
  \draw [->] (0100) to (1100);
  \draw [->] (1000) to (1001);
  \draw [->] (0100) to (0110);
  \draw [->] (0001) to (1001);
  \draw [->] (0000) to (0001);
  \draw [->] (0000) to (1000);
  \draw [->] (0000) to (0100);
  \draw [->] (0001) to (0101);
  \draw [->] (0010) to (1010);
  \draw [->] (0000) to (0010);
  \draw [->] (0100) to (0101);

\end{tikzpicture}
\end{center}

\begin{center}
\begin{tikzpicture}
    [scale=1.5,
     my box/.style={draw,minimum size=2em,inner sep=.1em,outer sep=.3em},
    ]

  \node (0000) at (0,0) [my box] {0000};
  \node (0001) at (90*0+90:2) [my box] {0001};
  \node (0010) at (90*1+90:2) [my box] {0010};
  \node (0100) at (90*2+90:2) [my box] {0100};
  \node (1000) at (90*3+90:2) [my box] {1000};

  \node (0011) at (-1,1) [my box] {0011};
  \node (0110) at (-1,-1) [my box] {0110};
  \node (1100) at (1,-1) [my box] {1100};
  \node (1001) at (1,1) [my box] {1001};
  \node (1010) at (0,-4) [my box] {1010};
  \node (0101) at (4,0) [my box] {0101};

  \draw [->] (0010) to (0011);
  \draw [->] (1000) to (1100);
  \draw [->] (1000) to (1010);
  \draw [->] (0001) to (0011);
  \draw [->] (0010) to (0110);
  \draw [->] (0100) to (1100);
  \draw [->] (1000) to (1001);
  \draw [->] (0100) to (0110);
  \draw [->] (0001) to (1001);
  \draw [->] (0000) to (0001);
  \draw [->] (0000) to (1000);
  \draw [->] (0000) to (0100);
  \draw [->] (0001) to (0101);
  \draw [->] (0010) to (1010);
  \draw [->] (0000) to (0010);
  \draw [->] (0100) to (0101);

\end{tikzpicture}
\end{center}

\begin{center}
\begin{tikzpicture}
    [scale=3,
     my box/.style={draw,minimum size=2em,inner sep=.1em,outer sep=.3em},
    ]

  \node (0000) at (0,0) [my box] {0000};
  \node (0001) at (90*0+90:2) [my box] {0001};
  \node (0010) at (90*1+90:2) [my box] {0010};
  \node (0100) at (90*2+90:2) [my box] {0100};
  \node (1000) at (90*3+90:2) [my box] {1000};

  \node (0011) at (-1,1) [my box] {0011};
  \node (0110) at (-1,-1) [my box] {0110};
  \node (1100) at (1,-1) [my box] {1100};
  \node (1001) at (1,1) [my box] {1001};
  \node (1010) at (.5,-.5) [my box] {1010};
  \node (0101) at (-.5,.5) [my box] {0101};

  \draw [->] (0010) to (0011);
  \draw [->] (1000) to (1100);
  \draw [->] (1000) to (1010);
  \draw [->] (0001) to (0011);
  \draw [->] (0010) to (0110);
  \draw [->] (0100) to (1100);
  \draw [->] (1000) to (1001);
  \draw [->] (0100) to (0110);
  \draw [->] (0001) to (1001);
  \draw [->] (0000) to (0001);
  \draw [->] (0000) to (1000);
  \draw [->] (0000) to (0100);
  \draw [->] (0001) to (0101);
  \draw [->] (0010) to (1010);
  \draw [->] (0000) to (0010);
  \draw [->] (0100) to (0101);

\end{tikzpicture}
\end{center}

\begin{center}
\begin{tikzpicture}
    [scale=4,
     my box/.style={draw,minimum size=2em,inner sep=.1em,outer sep=.3em},
    ]

  \node (0000) at (0,0) [my box] {0000};
  \node (0001) at (90*0+90:1) [my box] {0001};
  \node (0010) at (90*1+90:1) [my box] {0010};
  \node (0100) at (90*2+90:1) [my box] {0100};
  \node (1000) at (90*3+90:1) [my box] {1000};

  \node (0011) at (-1,1) [my box] {0011};
  \node (0110) at (-1,-1) [my box] {0110};
  \node (1100) at (1,-1) [my box] {1100};
  \node (1001) at (1,1) [my box] {1001};
  \node (1010) at (.5,-.5) [my box] {1010};
  \node (0101) at (-.5,.5) [my box] {0101};

  \draw [->] (0010) to (0011);
  \draw [->] (1000) to (1100);
  \draw [->] (1000) to (1010);
  \draw [->] (0001) to (0011);
  \draw [->] (0010) to (0110);
  \draw [->] (0100) to (1100);
  \draw [->] (1000) to (1001);
  \draw [->] (0100) to (0110);
  \draw [->] (0001) to (1001);
  \draw [->] (0000) to (0001);
  \draw [->] (0000) to (1000);
  \draw [->] (0000) to (0100);
  \draw [->] (0001) to (0101);
  \draw [->] (0010) to (1010);
  \draw [->] (0000) to (0010);
  \draw [->] (0100) to (0101);

\end{tikzpicture}
\end{center}


\begin{center}
\begin{tikzpicture}
    [scale=2,
     my box/.style={draw,minimum size=2em,inner sep=.1em,outer sep=.3em},
    ]

  \node (0000) at (0,0) [my box] {0000};
  \node (0001) at (90*0+90:1) [my box] {0001};
  \node (0010) at (90*1+90:1) [my box] {0010};
  \node (0100) at (90*2+90:1) [my box] {0100};
  \node (1000) at (90*3+90:1) [my box] {1000};

  \node (0011) at (-1,1) [my box] {0011};
  \node (0110) at (-1,-1) [my box] {0110};
  \node (1100) at (1,-1) [my box] {1100};
  \node (1001) at (1,1) [my box] {1001};
  \node (1010) at (0,-2) [my box] {1010};
  \node (0101) at (2,0) [my box] {0101};

  \draw [->] (0010) to (0011);
  \draw [->] (1000) to (1100);
  \draw [->] (1000) to (1010);
  \draw [->] (0001) to (0011);
  \draw [->] (0010) to (0110);
  \draw [->] (0100) to (1100);
  \draw [->] (1000) to (1001);
  \draw [->] (0100) to (0110);
  \draw [->] (0001) to (1001);
  \draw [->] (0000) to (0001);
  \draw [->] (0000) to (1000);
  \draw [->] (0000) to (0100);
  \draw [->] (0001) to (0101);
  \draw [->] (0010) to (1010);
  \draw [->] (0000) to (0010);
  \draw [->] (0100) to (0101);

\end{tikzpicture}
\end{center}

%------------------------------------------------------------------------------
\section{Noughts and Crosses 2x2}

\url{https://hog.grinvin.org/ViewGraphInfo.action?id=27017}

\begin{center}
\begin{tikzpicture}
    [scale=1.5,
     my box/.style={draw,minimum size=2em,inner sep=.1em,outer sep=.3em},
    ]

  \node (00-00) at (0,0) [my box] {00-00};

  % 1
  \node (00-01) at (1,0) [my box] {00-01};
  \node (00-10) at (0,1) [my box] {00-10};
  \node (01-00) at (-1,0) [my box] {01-00};
  \node (10-00) at (0,-1) [my box] {10-00};

  % 2
  \node (02-10) at (2,1) [my box] {02-10};
  \node (20-10) at (0,2) [my box] {20-10};
  \node (00-12) at (-2,1) [my box] {00-12};

  \node (21-00) at (-1,2) [my box] {21-00};
  \node (01-02) at (-2,0) [my box] {01-02};
  \node (01-20) at (-1,-2) [my box] {01-20};

  \node (10-02) at (-2,-1) [my box] {10-02};
  \node (10-20) at (0,-2) [my box] {10-20};
  \node (12-00) at (2,-1) [my box] {12-00};

  \node (00-21) at (1,-2) [my box] {00-21};
  \node (02-01) at (2,0) [my box] {02-01};
  \node (20-01) at (1,2) [my box] {20-01};

  % 3
  \node (12-10) at (3.5,0) [my box] {12-10};
  \node (02-11) at (3.5,1) [my box] {02-11};
  \node (12-01) at (3.5,-1) [my box] {12-01};

  \node (20-11) at (1,3.5) [my box] {20-11};
  \node (21-01) at (0,3.5) [my box] {21-01};
  \node (21-10) at (-1,3.5) [my box] {21-10};

  \node (01-12) at (-3.5,1) [my box] {01-12};

  \node (10-12) at (-3.5,0) [my box] {10-12};
  \node (11-02) at (-3.5,-1) [my box] {11-02};

  \node (11-20) at (-1,-3.5) [my box] {11-20};
  \node (01-21) at (0,-3.5) [my box] {01-21};
  \node (10-21) at (1,-3.5) [my box] {10-21};


  \draw [->] (12-00) to (12-10);
  \draw [->] (00-12) to (01-12);
  \draw [->] (00-10) to (02-10);
  \draw [->] (10-20) to (10-21);
  \draw [->] (01-00) to (01-20);
  \draw [->] (00-21) to (10-21);
  \draw [->] (10-00) to (10-20);
  \draw [->] (01-02) to (11-02);
  \draw [->] (21-00) to (21-01);
  \draw [->] (10-02) to (11-02);
  \draw [->] (20-01) to (20-11);
  \draw [->] (00-01) to (20-01);
  \draw [->] (00-00) to (10-00);
  \draw [->] (00-00) to (01-00);
  \draw [->] (00-12) to (10-12);
  \draw [->] (00-01) to (00-21);
  \draw [->] (02-01) to (02-11);
  \draw [->] (12-00) to (12-01);
  \draw [->] (02-10) to (02-11);
  \draw [->] (10-00) to (12-00);
  \draw [->] (20-01) to (21-01);
  \draw [->] (01-20) to (01-21);
  \draw [->] (01-20) to (11-20);
  \draw [->] (20-10) to (21-10);
  \draw [->] (10-02) to (10-12);
  \draw [->] (01-00) to (21-00);
  \draw [->] (00-10) to (20-10);
  \draw [->] (00-10) to (00-12);
  \draw [->] (01-02) to (01-12);
  \draw [->] (10-20) to (11-20);
  \draw [->] (00-00) to (00-10);
  \draw [->] (20-10) to (20-11);
  \draw [->] (02-01) to (12-01);
  \draw [->] (00-00) to (00-01);
  \draw [->] (00-21) to (01-21);
  \draw [->] (01-00) to (01-02);
  \draw [->] (02-10) to (12-10);
  \draw [->] (10-00) to (10-02);
  \draw [->] (21-00) to (21-10);
  \draw [->] (00-01) to (02-01);

\end{tikzpicture}
\end{center}



\begin{center}
\begin{tikzpicture}
    [scale=1.5,
     my box/.style={draw,minimum size=2em,inner sep=.1em,outer sep=.3em},
    ]

  \node (00-00) at (0,0) [my box] {00-00};

  % 1
  \node (00-01) at (1,0) [my box] {00-01};
  \node (00-10) at (0,1) [my box] {00-10};
  \node (01-00) at (-1,0) [my box] {01-00};
  \node (10-00) at (0,-1) [my box] {10-00};

  % 2
  \node (20-10) at (0,2) [my box] {20-10};
  \node (02-10) at ($(0,1) + (45:1)$) [my box] {02-10};
  \node (00-12) at ($(0,1) + (135:1)$) [my box] {00-12};

  \node (21-00) at ($(-1,0) + (135:1)$) [my box] {21-00};
  \node (01-02) at (-2,0) [my box] {01-02};
  \node (01-20) at ($(-1,0) + (-135:1)$) [my box] {01-20};

  \node (10-02) at ($(0,-1) + (-135:1)$) [my box] {10-02};
  \node (10-20) at (0,-2) [my box] {10-20};
  \node (12-00) at ($(0,-1) + (-45:1)$) [my box] {12-00};

  \node (00-21) at ($(1,0) + (-45:1)$) [my box] {00-21};
  \node (02-01) at (2,0) [my box] {02-01};
  \node (20-01) at ($(1,0) + (45:1)$) [my box] {20-01};

  % 3
  \node (12-10) at (0*30:4) [my box] {12-10};
  \node (02-11) at (1*30:4) [my box] {02-11};

  \node (20-11) at (2*30:4) [my box] {20-11};
  \node (21-01) at (3*30:4) [my box] {21-01};

  \node (21-10) at (4*30:4) [my box] {21-10};
  \node (01-12) at (5*30:4) [my box] {01-12};

  \node (10-12) at (6*30:4) [my box] {10-12};
  \node (11-02) at (7*30:4) [my box] {11-02};

  \node (11-20) at (8*30:4) [my box] {11-20};
  \node (01-21) at (9*30:4) [my box] {01-21};

  \node (10-21) at (10*30:4) [my box] {10-21};
  \node (12-01) at (11*30:4) [my box] {12-01};


  \draw [->] (12-00) to (12-10);
  \draw [->] (00-12) to (01-12);
  \draw [->] (00-10) to (02-10);
  \draw [->] (10-20) to (10-21);
  \draw [->] (01-00) to (01-20);
  \draw [->] (00-21) to (10-21);
  \draw [->] (10-00) to (10-20);
  \draw [->] (01-02) to (11-02);
  \draw [->] (21-00) to (21-01);
  \draw [->] (10-02) to (11-02);
  \draw [->] (20-01) to (20-11);
  \draw [->] (00-01) to (20-01);
  \draw [->] (00-00) to (10-00);
  \draw [->] (00-00) to (01-00);
  \draw [->] (00-12) to (10-12);
  \draw [->] (00-01) to (00-21);
  \draw [->] (02-01) to (02-11);
  \draw [->] (12-00) to (12-01);
  \draw [->] (02-10) to (02-11);
  \draw [->] (10-00) to (12-00);
  \draw [->] (20-01) to (21-01);
  \draw [->] (01-20) to (01-21);
  \draw [->] (01-20) to (11-20);
  \draw [->] (20-10) to (21-10);
  \draw [->] (10-02) to (10-12);
  \draw [->] (01-00) to (21-00);
  \draw [->] (00-10) to (20-10);
  \draw [->] (00-10) to (00-12);
  \draw [->] (01-02) to (01-12);
  \draw [->] (10-20) to (11-20);
  \draw [->] (00-00) to (00-10);
  \draw [->] (20-10) to (20-11);
  \draw [->] (02-01) to (12-01);
  \draw [->] (00-00) to (00-01);
  \draw [->] (00-21) to (01-21);
  \draw [->] (01-00) to (01-02);
  \draw [->] (02-10) to (12-10);
  \draw [->] (10-00) to (10-02);
  \draw [->] (21-00) to (21-10);
  \draw [->] (00-01) to (02-01);

\end{tikzpicture}
\end{center}


%------------------------------------------------------------------------------
\section{Noughts and Crosses 2x2 up to Rotation}

\url{https://hog.grinvin.org/ViewGraphInfo.action?id=27020}

\begin{center}
\begin{tikzpicture}
    [scale=1.5,
     my box/.style={draw,minimum size=2em,inner sep=.1em,outer sep=.3em},
    ]

  \node (00-00) at (45:1) [my box] {00-00};

  \node (00-01) at (0,0) [my box] {00-01};

  \node (00-12) at (90:2) [my box] {00-12};
  \node (00-21) at (90+120:2) [my box] {00-21};
  \node (01-20) at (90+240:2) [my box] {01-20};

  \node (01-12) at (90+60:2) [my box] {01-12};
  \node (01-21) at (90+60+120:2) [my box] {01-21};
  \node (02-11) at (90+60+240:2) [my box] {02-11};

  \draw [->] (00-01) to (00-12);
  \draw [->] (00-21) to (01-21);
  \draw [->] (00-12) to (02-11);
  \draw [->] (01-20) to (02-11);
  \draw [->] (00-00) to (00-01);
  \draw [->] (00-12) to (01-12);
  \draw [->] (00-21) to (01-12);
  \draw [->] (00-01) to (01-20);
  \draw [->] (01-20) to (01-21);
  \draw [->] (00-01) to (00-21);

\end{tikzpicture}
\end{center}

\vspace{1\baselineskip}

\begin{center}
\begin{tikzpicture}
    [scale=1.5,
     my box/.style={draw,minimum size=2em,inner sep=.1em,outer sep=.3em},
    ]

  \node (00-00) at (0,0) [my box] {00-00};

  \node (00-01) at (0,-1) [my box] {00-01};

  \node (00-12) at (1,-2) [my box] {00-12};
  \node (00-21) at (0,-2) [my box] {00-21};
  \node (01-20) at (-1,-2) [my box] {01-20};

  \node (01-12) at (1,-3) [my box] {01-12};
  \node (02-11) at (0,-3) [my box] {02-11};
  \node (01-21) at (-1,-3) [my box] {01-21};

  \draw [->] (00-01) to (00-12);
  \draw [->] (00-21) to (01-21);
  \draw [->] (00-12) to (02-11);
  \draw [->] (01-20) to (02-11);
  \draw [->] (00-00) to (00-01);
  \draw [->] (00-12) to (01-12);
  \draw [->] (00-21) to (01-12);
  \draw [->] (00-01) to (01-20);
  \draw [->] (01-20) to (01-21);
  \draw [->] (00-01) to (00-21);

\end{tikzpicture}
\end{center}



%------------------------------------------------------------------------------
\section{Noughts and Crosses 2x2, 1 Player, up to Reflection}

\url{https://hog.grinvin.org/ViewGraphInfo.action?id=856}

\begin{center}
\begin{tikzpicture}
    [scale=1.5,
     my box/.style={draw,minimum size=2em,inner sep=.1em,outer sep=.3em},
    ]

  \node (0000) at (0,0) [my box] {0000};
  \node (0010) at (1,0) [my box] {0010};
  \node (1000) at (-1,0) [my box] {1000};

  \node (0011) at (2,0) [my box] {0011};
  \node (1010) at (0,-1) [my box] {1010};
  \node (1001) at (0,1) [my box] {1001};
  \node (1100) at (-2,0) [my box] {1100};

  \draw [->] (1000) to (1100);
  \draw [->] (1000) to (1001);
  \draw [->] (0000) to (0010);
  \draw [->] (0010) to (1010);
  \draw [->] (1000) to (1010);
  \draw [->] (0000) to (1000);
  \draw [->] (0010) to (0011);
  \draw [->] (0010) to (1001);

\end{tikzpicture}
\end{center}

%------------------------------------------------------------------------------
\section{Noughts and Crosses 2x2, 3 Players, up to Rotation}

\url{https://hog.grinvin.org/ViewGraphInfo.action?id=27025}

\begin{center}
\begin{tikzpicture}
    [scale=1.5,
     my box/.style={draw,minimum size=2em,inner sep=.1em,outer sep=.3em},
    ]

  \node (0000) at (45:1) [my box] {0000};
  \node (1000) at (0,0) [my box] {1000};
  \node (1002) at (3*30:2) [my box] {1002};
  \node (1020) at (7*30:2) [my box] {1020};
  \node (1023) at (6*30:2) [my box] {1023};
  \node (1032) at (2*30:2) [my box] {1032};
  \node (1123) at (5*30:2) [my box] {1123};
  \node (1200) at (11*30:2) [my box] {1200};
  \node (1203) at (0*30:2) [my box] {1203};
  \node (1213) at (1*30:2) [my box] {1213};
  \node (1230) at (10*30:2) [my box] {1230};
  \node (1231) at (9*30:2) [my box] {1231};
  \node (1302) at (4*30:2) [my box] {1302};
  \node (1320) at (8*30:2) [my box] {1320};

  \draw [] (1000) to (1200);
  \draw [] (1123) to (1302);
  \draw [] (1230) to (1231);
  \draw [] (1200) to (1230);
  \draw [] (1000) to (1020);
  \draw [] (1002) to (1302);
  \draw [] (1002) to (1032);
  \draw [] (1231) to (1320);
  \draw [] (1020) to (1320);
  \draw [] (1020) to (1023);
  \draw [] (1200) to (1203);
  \draw [] (1000) to (1002);
  \draw [] (1032) to (1213);
  \draw [] (1123) to (1023);
  \draw [] (0000) to (1000);
  \draw [] (1203) to (1213);

\end{tikzpicture}
\end{center}

\begin{center}
\begin{tikzpicture}
    [scale=1.5,
     my box/.style={draw,minimum size=2em,inner sep=.1em,outer sep=.3em},
    ]

  \node (0000) at (0,1) [my box] {0000};
  \node (1000) at (0,0) [my box] {1000};

  \node (1002) at (-2,-1) [my box] {1002};
  \node (1020) at (0,-1) [my box] {1020};
  \node (1200) at (2,-1) [my box] {1200};

  \node (1302) at (-2.5,-2) [my box] {1302};
  \node (1032) at (-1.5,-2) [my box] {1023};
  \node (1023) at (-.5,-2) [my box] {1320};
  \node (1320) at (.5,-2) [my box] {1032};
  \node (1203) at (1.5,-2) [my box] {1230};
  \node (1230) at (2.5,-2) [my box] {1203};

  \node (1123) at (-2.5,-3) [my box] {1123};
  \node (1213) at (0,-3) [my box] {1231};
  \node (1231) at (2.5,-3) [my box] {1213};

  \draw [] (1000) to (1200);
  \draw [] (1123) to (1302);
  \draw [] (1230) to (1231);
  \draw [] (1200) to (1230);
  \draw [] (1000) to (1020);
  \draw [] (1002) to (1302);
  \draw [] (1002) to (1032);
  \draw [] (1231) to (1320);
  \draw [] (1020) to (1320);
  \draw [] (1020) to (1023);
  \draw [] (1200) to (1203);
  \draw [] (1000) to (1002);
  \draw [] (1032) to (1213);
  \draw [] (1123) to (1023);
  \draw [] (0000) to (1000);
  \draw [] (1203) to (1213);

\end{tikzpicture}
\end{center}

\begin{center}
\begin{tikzpicture}
    [scale=1.5,
     my box/.style={draw,minimum size=2em,inner sep=.1em,outer sep=.3em},
    ]


  \node (0000) at (0,1) [my box] {0000};
  \node (1000) at (0,0) [my box] {1000};

  \node (1002) at (-2,-1) [my box] {1002};
  \node (1020) at (0,-1) [my box] {1020};
  \node (1200) at (2,-1) [my box] {1200};

  \node (1302) at (-3,-2) [my box] {1302};
  \node (1023) at (-2,-2) [my box] {1023};
  \node (1203) at (-.5,-2) [my box] {1320};
  \node (1032) at (.5,-2) [my box] {1032};
  \node (1320) at (2,-2) [my box] {1230};
  \node (1230) at (3,-2) [my box] {1203};

  \node (1123) at (-3,-3) [my box] {1123};
  \node (1213) at (0,-3) [my box] {1231};
  \node (1231) at (3,-3) [my box] {1213};

  \draw [] (1000) to (1200);
  \draw [] (1123) to (1302);
  \draw [] (1230) to (1231);
  \draw [] (1200) to (1230);
  \draw [] (1000) to (1020);
  \draw [] (1002) to (1302);
  \draw [] (1002) to (1032);
  \draw [] (1231) to (1320);
  \draw [] (1020) to (1320);
  \draw [] (1020) to (1023);
  \draw [] (1200) to (1203);
  \draw [] (1000) to (1002);
  \draw [] (1032) to (1213);
  \draw [] (1123) to (1023);
  \draw [] (0000) to (1000);
  \draw [] (1203) to (1213);

\end{tikzpicture}
\end{center}

%------------------------------------------------------------------------------
\section{Subgraph Non-Relations Among 4-Vertex Graphs}

\url{https://hog.grinvin.org/ViewGraphInfo.action?id=26965}

\begin{center}
\begin{tikzpicture}
    [scale=1.2,xscale=1.7,
     my box/.style={draw,minimum size=2em,inner sep=.1em,outer sep=.3em},
    ]

  \node (complete) at (0,-2) [my box] {complete};
  \node (cycle across) at (1,-2) [my box] {cycle across};
  \node (4) at (2,-2) [my box] {p-1};
  \node (disconnected) at (4,-2) [my box] {disconnected};

  \node (cycle-1 hanging) at (0,0) [my box] {cycle-1 hanging};
  \node (cycle) at (1,0) [my box] {cycle};
  \node (5) at (5,0) [my box] {p-3=2edge};
  \node (7) at (4,0) [my box] {2-sep};
  \node (cycle-1 disc) at (2,1) [my box] {cycle-1 disc};
  \node (path) at (2.5,0) [my box] {path};
  \node (star) at (2,-1) [my box] {claw};

  \draw [] (star) to (path);
  \draw [] (star) to (cycle);
  \draw [] (cycle) to (cycle-1 hanging);
  \draw [] (5) to (7);
  \draw [] (7) to (star);
  \draw [] (cycle-1 disc) to (cycle);
  \draw [] (cycle-1 disc) to (path);
  \draw [] (star) to (cycle-1 disc);
  \draw [] (7) to (cycle-1 disc);

\end{tikzpicture}
\end{center}

%------------------------------------------------------------------------------
\section{Subgraph Relations Among 4-Vertex Graphs, 1-Edge Delta}

\url{https://hog.grinvin.org/ViewGraphInfo.action?id=26963}

\begin{center}
\begin{tikzpicture}
    [scale=1.2,xscale=1.7,
     my box/.style={draw,minimum size=2em,inner sep=.1em,outer sep=.3em},
    ]
  \node (complete) at (0, 0) [my box] {compl};
  \node (cycle across) at (1, 0) [my box,align=center] {cycle \\ across};
  \node (cycle) at (2, 1) [my box] {cycle};
  \node (cycle hanging) at (2,0) [my box,align=center] {cycle-1 \\ hanging};
  \node (path) at (3, 1) [my box] {path};
  \node (claw) at (3,0) [my box] {claw};
  \node (cycle-1 disc) at (3,-1) [my box,align=center] {cycle-1 \\ discon};
  \node (7) at (4,1) [my box] {2-sep};
  \node (5) at (4,0) [my box] {L};
  \node (4) at (5,0) [my box] {1-edge};
  \node (disconnected) at (6,0) [my box] {disc};

  \draw [->] (complete) to (cycle across);
  \draw [->] (cycle hanging) to (claw);
  \draw [->] (cycle hanging) to (path);
  \draw [->] (cycle hanging) to (cycle-1 disc);
  \draw [->] (path) to (5);
  \draw [->] (cycle) to (path);
  \draw [->] (4) to (disconnected);
  \draw [->] (5) to (4);
  \draw [->] (7) to (4);
  \draw [->] (cycle across) to (cycle hanging);
  \draw [->] (cycle-1 disc) to (5);
  \draw [->] (cycle across) to (cycle);
  \draw [->] (claw) to (5);
  \draw [->] (path) to (7);

\end{tikzpicture}
\end{center}

\begin{center}
\begin{tikzpicture}
    [scale=2,xscale=1.5,
     my box/.style={draw,minimum size=2em,inner sep=.1em,outer sep=.3em},
    ]
  \node (complete) at (0, 6) [my box] {complete};
  \node (cycle across) at (0, 5) [my box] {cycle across};
  \node (cycle) at (-1, 4) [my box] {cycle};
  \node (cycle hanging) at (0,4) [my box] {cycle-1 hanging};
  \node (path) at (-1, 3) [my box] {path};
  \node (claw) at (0,3) [my box] {claw};
  \node (cycle-1 disc) at (1,3) [my box] {cycle-1 discon};
  \node (7) at (-1,2) [my box] {2-sep};
  \node (5) at (0,2) [my box] {L};
  \node (4) at (0,1) [my box] {1-edge};
  \node (disconnected) at (0,0) [my box] {disconnected};

  \draw [->] (complete) to (cycle across);
  \draw [->] (cycle hanging) to (claw);
  \draw [->] (cycle hanging) to (path);
  \draw [->] (cycle hanging) to (cycle-1 disc);
  \draw [->] (path) to (5);
  \draw [->] (cycle) to (path);
  \draw [->] (4) to (disconnected);
  \draw [->] (5) to (4);
  \draw [->] (7) to (4);
  \draw [->] (cycle across) to (cycle hanging);
  \draw [->] (cycle-1 disc) to (5);
  \draw [->] (cycle across) to (cycle);
  \draw [->] (claw) to (5);
  \draw [->] (path) to (7);

\end{tikzpicture}
\end{center}

%------------------------------------------------------------------------------
\section{Subgraph Relations Among 4-Vertex Connected Graphs}

\url{https://hog.grinvin.org/ViewGraphInfo.action?id=748}

\begin{center}
\begin{tikzpicture}
    [scale=2,xscale=1.6,
     my box/.style={draw,minimum size=2em,inner sep=.1em,outer sep=.3em},
    ]
  \node (complete)            at (0,1) [my box] {complete};
  \node (cycle and across)    at (1,1) [my box] {K4 - 1};
  \node (cycle)               at (1.5,0) [my box] {cycle};
  \node (claw)                at (-.5,0) [my box] {claw};
  \node (cycle-1 and hanging) at (0,-1) [my box] {cycle-1 and hang};
  \node (path)                at (1,-1) [my box] {path};

  \draw [->,bend right=0] (complete) to (path);
  \draw [->] (complete) to (cycle and across);
  \draw [->] (complete) to (cycle);
  \draw [->] (complete) to (cycle-1 and hanging);
  \draw [->,bend left=0] (complete) to (claw);
  \draw [->] (cycle and across) to (cycle);
  \draw [->] (cycle and across) to (path);
  \draw [->] (cycle and across) to (cycle-1 and hanging);
  \draw [->] (cycle and across) to (claw);
  \draw [->] (cycle-1 and hanging) to (path);
  \draw [->] (cycle-1 and hanging) to (claw);
  \draw [->] (cycle) to (path);
\end{tikzpicture}
\end{center}

% \vspace{2ex}

\begin{center}
\begin{tikzpicture}
    [scale=2,xscale=1.8,
     my box/.style={draw,minimum size=2em,inner sep=.1em,outer sep=.3em},
    ]
  \node (complete)            at (0, 1) [my box] {complete};
  \node (cycle and across)    at (2, 1) [my box] {K4 - 1};
  \node (cycle)               at (0, -1) [my box] {cycle};
  \node (claw)                at (2,-1) [my box] {claw};
  \node (cycle-1 and hanging) at (.5, 0) [my box] {cycle-1 and hang};
  \node (path)                at (1.5, 0) [my box] {path};

  \draw [->,bend right=0] (complete) to (path);
  \draw [->] (complete) to (cycle and across);
  \draw [->] (complete) to (cycle);
  \draw [->] (complete) to (cycle-1 and hanging);
  \draw [->,bend left=0] (complete) to (claw);
  \draw [->] (cycle and across) to (cycle);
  \draw [->] (cycle and across) to (path);
  \draw [->] (cycle and across) to (cycle-1 and hanging);
  \draw [->] (cycle and across) to (claw);
  \draw [->] (cycle-1 and hanging) to (path);
  \draw [->] (cycle-1 and hanging) to (claw);
  \draw [->] (cycle) to (path);
\end{tikzpicture}
\end{center}

\begin{center}
\begin{tikzpicture}
    [scale=3.5,rotate=90-72,
     my box/.style={draw,minimum size=2em,inner sep=.1em,outer sep=.3em},
    ]
  \node (complete)            at (0,0) [my box] {complete};
  \node (claw)                at (0:1) [my box] {claw};
  \node (cycle)               at (2*72:1) [my box] {cycle};
  \node (cycle and across)    at (1*72:1) [my box] {cycle and across};
  \node (cycle-1 and hanging) at (4*72:1) [my box] {cycle-1 and hanging};
  \node (path)                at (3*72:1) [my box] {path};

  \draw [->,bend right=0] (complete) to (path);
  \draw [->] (complete) to (cycle and across);
  \draw [->] (complete) to (cycle);
  \draw [->] (complete) to (cycle-1 and hanging);
  \draw [->,bend left=0] (complete) to (claw);
  \draw [->] (cycle and across) to (cycle);
  \draw [->] (cycle and across) to (path);
  \draw [->] (cycle and across) to (cycle-1 and hanging);
  \draw [->] (cycle and across) to (claw);
  \draw [->] (cycle-1 and hanging) to (path);
  \draw [->] (cycle-1 and hanging) to (claw);
  \draw [->] (cycle) to (path);
\end{tikzpicture}
\end{center}

\begin{center}
\begin{tikzpicture}
    [scale=2,
     my box/.style={draw,minimum size=2em,inner sep=.1em,outer sep=.3em},
    ]
  \node (complete) at (1,2) [my box] {complete};
  \node (claw) at (3,0) [my box] {claw};
  \node (cycle) at (0,1) [my box] {cycle};
  \node (cycle and across) at (1.5, 1) [my box] {cycle and across};
  \node (cycle-1 and hanging) at (1,0) [my box] {cycle-1 and hanging};
  \node (path) at (-1,0) [my box] {path};

  \draw [->,bend right=20] (complete) to (path);
  \draw [->] (complete) to (cycle and across);
  \draw [->] (complete) to (cycle);
  \draw [->] (complete) to (cycle-1 and hanging);
  \draw [->,bend left=20] (complete) to (claw);
  \draw [->] (cycle and across) to (cycle);
  \draw [->] (cycle and across) to (path);
  \draw [->] (cycle and across) to (cycle-1 and hanging);
  \draw [->] (cycle and across) to (claw);
  \draw [->] (cycle-1 and hanging) to (path);
  \draw [->] (cycle-1 and hanging) to (claw);
  \draw [->] (cycle) to (path);
\end{tikzpicture}
\end{center}

%------------------------------------------------------------------------------
\section{Transpositions 4-Star = Claw}

\begin{center}
\begin{tikzpicture}
    [scale=1.6,
     my box/.style={draw,minimum size=2em,inner sep=.1em,outer sep=.3em},
    ]
  \newcommand\MyOuter{3.8}
  \newcommand\MyInner{2}
  \newcommand\MyOuterAngle{15}
  \newcommand\MyInnerAngle{15}
  \node (0-1-2-3) at (180+\MyOuterAngle:\MyOuter) [my box] {0,1,2,3};
  \node (0-1-3-2) at (300-\MyOuterAngle:\MyOuter) [my box] {0,1,3,2};
  \node (1-0-2-3) at (180+\MyInnerAngle:\MyInner) [my box] {1,0,2,3};
  \node (1-0-3-2) at (300+\MyOuterAngle:\MyOuter) [my box] {1,0,3,2};

  \node (0-2-1-3) at (60-\MyOuterAngle:\MyOuter) [my box] {0,2,1,3};
  \node (0-2-3-1) at (300-\MyInnerAngle:\MyInner) [my box] {0,2,3,1};
  \node (2-0-1-3) at (0+\MyOuterAngle:\MyOuter) [my box] {2,0,1,3};
  \node (2-0-3-1) at (300+\MyInnerAngle:\MyInner) [my box] {2,0,3,1};

  \node (0-3-1-2) at (120-\MyInnerAngle:\MyInner) [my box] {0,3,1,2};
  \node (0-3-2-1) at (120+\MyInnerAngle:\MyInner) [my box] {0,3,2,1};
  \node (3-0-1-2) at (0-\MyOuterAngle:\MyOuter) [my box] {3,0,1,2};
  \node (3-0-2-1) at (60-\MyInnerAngle:\MyInner) [my box] {3,0,2,1};

  \node (1-2-0-3) at (240-\MyInnerAngle:\MyInner) [my box] {1,2,0,3};
  \node (1-2-3-0) at (60+\MyInnerAngle:\MyInner) [my box] {1,2,3,0};
  \node (2-1-0-3) at (240-\MyOuterAngle:\MyOuter) [my box] {2,1,0,3};
  \node (2-1-3-0) at (180-\MyInnerAngle:\MyInner) [my box] {2,1,3,0};

  \node (1-3-0-2) at (240+\MyInnerAngle:\MyInner) [my box] {1,3,0,2};
  \node (1-3-2-0) at (120+\MyOuterAngle:\MyOuter) [my box] {1,3,2,0};
  \node (3-1-0-2) at (240+\MyOuterAngle:\MyOuter) [my box] {3,1,0,2};
  \node (3-1-2-0) at (180-\MyOuterAngle:\MyOuter) [my box] {3,1,2,0};

  \node (2-3-0-1) at (0-\MyInnerAngle:\MyInner) [my box] {2,3,0,1};
  \node (2-3-1-0) at (120-\MyOuterAngle:\MyOuter) [my box] {2,3,1,0};
  \node (3-2-0-1) at (0+\MyInnerAngle:\MyInner) [my box] {3,2,0,1};
  \node (3-2-1-0) at (60+\MyOuterAngle:\MyOuter) [my box] {3,2,1,0};

  \draw [] (2-0-3-1) to (3-0-2-1);
  \draw [] (2-0-3-1) to (1-0-3-2);
  \draw [] (0-2-1-3) to (3-2-1-0);
  \draw [] (2-1-3-0) to (0-1-3-2);
  \draw [] (3-2-0-1) to (0-2-3-1);
  \draw [] (3-2-1-0) to (2-3-1-0);
  \draw [] (1-3-0-2) to (0-3-1-2);
  \draw [] (0-1-2-3) to (2-1-0-3);
  \draw [] (0-3-2-1) to (1-3-2-0);
  \draw [] (0-1-2-3) to (1-0-2-3);
  \draw [] (2-1-3-0) to (1-2-3-0);
  \draw [] (2-0-1-3) to (3-0-1-2);
  \draw [] (0-1-2-3) to (3-1-2-0);
  \draw [] (1-2-0-3) to (0-2-1-3);
  \draw [] (0-2-3-1) to (2-0-3-1);
  \draw [] (1-2-0-3) to (2-1-0-3);
  \draw [] (0-2-1-3) to (2-0-1-3);
  \draw [] (3-2-1-0) to (1-2-3-0);
  \draw [] (3-0-1-2) to (1-0-3-2);
  \draw [] (2-3-1-0) to (1-3-2-0);
  \draw [] (1-2-0-3) to (3-2-0-1);
  \draw [] (2-3-1-0) to (0-3-1-2);
  \draw [] (1-3-0-2) to (3-1-0-2);
  \draw [] (2-3-0-1) to (0-3-2-1);
  \draw [] (1-0-2-3) to (3-0-2-1);
  \draw [] (1-0-2-3) to (2-0-1-3);
  \draw [] (3-2-0-1) to (2-3-0-1);
  \draw [] (2-1-0-3) to (3-1-0-2);
  \draw [] (1-3-0-2) to (2-3-0-1);
  \draw [] (3-1-2-0) to (1-3-2-0);
  \draw [] (3-0-2-1) to (0-3-2-1);
  \draw [] (0-1-3-2) to (3-1-0-2);
  \draw [] (0-1-3-2) to (1-0-3-2);
  \draw [] (3-0-1-2) to (0-3-1-2);
  \draw [] (0-2-3-1) to (1-2-3-0);
  \draw [] (3-1-2-0) to (2-1-3-0);

\end{tikzpicture}
\end{center}


%------------------------------------------------------------------------------
\section{Transpositions 4-Cycle}

\url{https://hog.grinvin.org/ViewGraphInfo.action?id=1292}

\begin{center}
\begin{tikzpicture}
    [scale=1.6,
     my box/.style={draw,minimum size=2em,inner sep=.1em,outer sep=.3em},
    ]
  \newcommand\MyOuter{3.8}
  \newcommand\MyInner{2}
  \newcommand\MyOuterAngle{15}
  \newcommand\MyInnerAngle{15}
  \node (0-1-2-3) at (180+\MyOuterAngle:\MyOuter) [my box] {0,1,2,3};
  \node (0-1-3-2) at (180-\MyOuterAngle:\MyOuter) [my box] {0,1,3,2};
  \node (1-0-2-3) at (180+\MyInnerAngle:\MyInner) [my box] {1,0,2,3};
  \node (1-0-3-2) at (180-\MyInnerAngle:\MyInner) [my box] {1,0,3,2};

  \node (0-2-1-3) at (240-\MyOuterAngle:\MyOuter) [my box] {0,2,1,3};
  \node (0-2-3-1) at (240-\MyInnerAngle:\MyInner) [my box] {0,2,3,1};
  \node (2-0-1-3) at (240+\MyOuterAngle:\MyOuter) [my box] {2,0,1,3};
  \node (2-0-3-1) at (240+\MyInnerAngle:\MyInner) [my box] {2,0,3,1};

  \node (0-3-1-2) at (120+\MyOuterAngle:\MyOuter) [my box] {0,3,1,2};
  \node (0-3-2-1) at (120+\MyInnerAngle:\MyInner) [my box] {0,3,2,1};
  \node (3-0-1-2) at (120-\MyOuterAngle:\MyOuter) [my box] {3,0,1,2};
  \node (3-0-2-1) at (120-\MyInnerAngle:\MyInner) [my box] {3,0,2,1};

  \node (1-2-0-3) at (300-\MyInnerAngle:\MyInner) [my box] {1,2,0,3};
  \node (1-2-3-0) at (300+\MyInnerAngle:\MyInner) [my box] {1,2,3,0};
  \node (2-1-0-3) at (300-\MyOuterAngle:\MyOuter) [my box] {2,1,0,3};
  \node (2-1-3-0) at (300+\MyOuterAngle:\MyOuter) [my box] {2,1,3,0};

  \node (1-3-0-2) at (60+\MyInnerAngle:\MyInner) [my box] {1,3,0,2};
  \node (1-3-2-0) at (60-\MyInnerAngle:\MyInner) [my box] {1,3,2,0};
  \node (3-1-0-2) at (60+\MyOuterAngle:\MyOuter) [my box] {3,1,0,2};
  \node (3-1-2-0) at (60-\MyOuterAngle:\MyOuter) [my box] {3,1,2,0};

  \node (2-3-0-1) at (0-\MyInnerAngle:\MyInner) [my box] {2,3,0,1};
  \node (2-3-1-0) at (0-\MyOuterAngle:\MyOuter) [my box] {2,3,1,0};
  \node (3-2-0-1) at (0+\MyInnerAngle:\MyInner) [my box] {3,2,0,1};
  \node (3-2-1-0) at (0+\MyOuterAngle:\MyOuter) [my box] {3,2,1,0};

  \draw [] (1-0-3-2) to (0-1-3-2);
  \draw [] (2-1-0-3) to (1-2-0-3);
  \draw [] (2-0-1-3) to (3-0-1-2);
  \draw [] (2-0-1-3) to (2-0-3-1);
  \draw [] (2-1-0-3) to (2-1-3-0);
  \draw [] (2-3-1-0) to (2-3-0-1);
  \draw [] (2-1-3-0) to (2-3-1-0);
  \draw [] (2-3-1-0) to (0-3-1-2);
  \draw [] (3-1-0-2) to (1-3-0-2);
  \draw [] (2-0-1-3) to (0-2-1-3);
  \draw [] (0-2-1-3) to (3-2-1-0);
  \draw [] (3-1-2-0) to (3-1-0-2);
  \draw [] (3-2-0-1) to (3-2-1-0);
  \draw [] (2-1-3-0) to (0-1-3-2);
  \draw [] (2-1-0-3) to (2-0-1-3);
  \draw [] (2-3-1-0) to (3-2-1-0);
  \draw [] (1-2-3-0) to (1-3-2-0);
  \draw [] (1-0-2-3) to (1-2-0-3);
  \draw [] (0-3-1-2) to (0-3-2-1);
  \draw [] (1-0-3-2) to (2-0-3-1);
  \draw [] (1-0-3-2) to (1-3-0-2);
  \draw [] (3-1-0-2) to (3-0-1-2);
  \draw [] (3-2-0-1) to (3-0-2-1);
  \draw [] (1-2-0-3) to (1-2-3-0);
  \draw [] (0-2-1-3) to (0-2-3-1);
  \draw [] (0-3-2-1) to (3-0-2-1);
  \draw [] (0-1-2-3) to (3-1-2-0);
  \draw [] (3-0-2-1) to (3-0-1-2);
  \draw [] (0-3-2-1) to (0-2-3-1);
  \draw [] (1-2-3-0) to (0-2-3-1);
  \draw [] (0-3-1-2) to (3-0-1-2);
  \draw [] (0-3-2-1) to (1-3-2-0);
  \draw [] (3-1-2-0) to (1-3-2-0);
  \draw [] (1-2-0-3) to (3-2-0-1);
  \draw [] (1-0-2-3) to (3-0-2-1);
  \draw [] (2-3-0-1) to (1-3-0-2);
  \draw [] (0-1-2-3) to (1-0-2-3);
  \draw [] (2-3-0-1) to (3-2-0-1);
  \draw [] (1-0-2-3) to (1-0-3-2);
  \draw [] (2-1-3-0) to (1-2-3-0);
  \draw [] (2-1-0-3) to (3-1-0-2);
  \draw [] (1-3-2-0) to (1-3-0-2);
  \draw [] (0-3-1-2) to (0-1-3-2);
  \draw [] (3-1-2-0) to (3-2-1-0);
  \draw [] (0-1-2-3) to (0-2-1-3);
  \draw [] (2-3-0-1) to (2-0-3-1);
  \draw [] (0-1-2-3) to (0-1-3-2);
  \draw [] (2-0-3-1) to (0-2-3-1);

\end{tikzpicture}
\end{center}


%------------------------------------------------------------------------------
\section{Transpositions 4-Row}

\url{https://hog.grinvin.org/ViewGraphInfo.action?id=1391}

4 by 1, row swap adjacent
\begin{center}
\begin{tikzpicture}
    [scale=1.8,
     my box/.style={draw,minimum size=2em,inner sep=.1em,outer sep=.3em},
    ]
  \newcommand\MyOuter{3}
  \newcommand\MyInner{2}
  \newcommand\MyOuterAngle{15}
  \newcommand\MyInnerAngle{15}
  \node (0-1-2-3) at (180+\MyOuterAngle:\MyOuter) [my box] {0,1,2,3};
  \node (0-1-3-2) at (180-\MyOuterAngle:\MyOuter) [my box] {0,1,3,2};
  \node (1-0-2-3) at (180+\MyInnerAngle:\MyInner) [my box] {1,0,2,3};
  \node (1-0-3-2) at (180-\MyInnerAngle:\MyInner) [my box] {1,0,3,2};

  \node (0-2-1-3) at (240-\MyOuterAngle:\MyOuter) [my box] {0,2,1,3};
  \node (0-2-3-1) at (240-\MyInnerAngle:\MyInner) [my box] {0,2,3,1};
  \node (2-0-1-3) at (240+\MyOuterAngle:\MyOuter) [my box] {2,0,1,3};
  \node (2-0-3-1) at (240+\MyInnerAngle:\MyInner) [my box] {2,0,3,1};

  \node (0-3-1-2) at (120+\MyOuterAngle:\MyOuter) [my box] {0,3,1,2};
  \node (0-3-2-1) at (120+\MyInnerAngle:\MyInner) [my box] {0,3,2,1};
  \node (3-0-1-2) at (120-\MyOuterAngle:\MyOuter) [my box] {3,0,1,2};
  \node (3-0-2-1) at (120-\MyInnerAngle:\MyInner) [my box] {3,0,2,1};

  \node (1-2-0-3) at (300-\MyInnerAngle:\MyInner) [my box] {1,2,0,3};
  \node (1-2-3-0) at (300+\MyInnerAngle:\MyInner) [my box] {1,2,3,0};
  \node (2-1-0-3) at (300-\MyOuterAngle:\MyOuter) [my box] {2,1,0,3};
  \node (2-1-3-0) at (300+\MyOuterAngle:\MyOuter) [my box] {2,1,3,0};

  \node (1-3-0-2) at (60+\MyInnerAngle:\MyInner) [my box] {1,3,0,2};
  \node (1-3-2-0) at (60-\MyInnerAngle:\MyInner) [my box] {1,3,2,0};
  \node (3-1-0-2) at (60+\MyOuterAngle:\MyOuter) [my box] {3,1,0,2};
  \node (3-1-2-0) at (60-\MyOuterAngle:\MyOuter) [my box] {3,1,2,0};

  \node (2-3-0-1) at (0-\MyInnerAngle:\MyInner) [my box] {2,3,0,1};
  \node (2-3-1-0) at (0-\MyOuterAngle:\MyOuter) [my box] {2,3,1,0};
  \node (3-2-0-1) at (0+\MyInnerAngle:\MyInner) [my box] {3,2,0,1};
  \node (3-2-1-0) at (0+\MyOuterAngle:\MyOuter) [my box] {3,2,1,0};

  \draw [] (2-3-1-0) to (2-1-3-0);
  \draw [] (2-0-3-1) to (2-0-1-3);
  \draw [] (2-3-1-0) to (3-2-1-0);
  \draw [] (0-1-3-2) to (0-3-1-2);
  \draw [] (3-1-0-2) to (3-0-1-2);
  \draw [] (3-2-1-0) to (3-1-2-0);
  \draw [] (3-2-0-1) to (3-2-1-0);
  \draw [] (0-3-2-1) to (3-0-2-1);
  \draw [] (3-0-1-2) to (3-0-2-1);
  \draw [] (2-1-3-0) to (1-2-3-0);
  \draw [] (1-3-2-0) to (3-1-2-0);
  \draw [] (1-0-2-3) to (1-2-0-3);
  \draw [] (0-1-2-3) to (0-1-3-2);
  \draw [] (2-1-0-3) to (1-2-0-3);
  \draw [] (3-2-0-1) to (3-0-2-1);
  \draw [] (1-3-0-2) to (1-0-3-2);
  \draw [] (2-0-1-3) to (0-2-1-3);
  \draw [] (0-1-2-3) to (1-0-2-3);
  \draw [] (1-3-0-2) to (3-1-0-2);
  \draw [] (2-0-3-1) to (0-2-3-1);
  \draw [] (1-3-2-0) to (1-3-0-2);
  \draw [] (1-0-2-3) to (1-0-3-2);
  \draw [] (0-3-1-2) to (3-0-1-2);
  \draw [] (2-0-1-3) to (2-1-0-3);
  \draw [] (3-1-2-0) to (3-1-0-2);
  \draw [] (2-1-0-3) to (2-1-3-0);
  \draw [] (1-2-3-0) to (1-2-0-3);
  \draw [] (0-1-3-2) to (1-0-3-2);
  \draw [] (1-2-3-0) to (1-3-2-0);
  \draw [] (0-3-2-1) to (0-2-3-1);
  \draw [] (0-1-2-3) to (0-2-1-3);
  \draw [] (2-3-0-1) to (2-3-1-0);
  \draw [] (2-3-0-1) to (3-2-0-1);
  \draw [] (0-3-1-2) to (0-3-2-1);
  \draw [] (2-0-3-1) to (2-3-0-1);
  \draw [] (0-2-1-3) to (0-2-3-1);

\end{tikzpicture}
\end{center}

\begin{center}
\begin{tikzpicture}
    [scale=1.3,xscale=1.2,
     my box/.style={draw,minimum size=2em,inner sep=.1em,outer sep=.3em},
    ]
  \node (0-1-2-3) at (1,1) [my box] {0,1,2,3};
  \node (1-0-2-3) at (0,1) [my box] {1,0,2,3};
  \node (0-1-3-2) at (2,0) [my box] {0,1,3,2};
  \node (1-0-3-2) at (-1,1) [my box] {1,0,3,2};

  \node (0-2-1-3) at (6,3) [my box] {0,2,1,3};
  \node (0-2-3-1) at (7,3) [my box] {0,2,3,1};
  \node (2-0-1-3) at (6,4) [my box] {2,0,1,3};
  \node (2-0-3-1) at (7,4) [my box] {2,0,3,1};

  \node (0-3-1-2) at (4,0) [my box] {0,3,1,2};
  \node (0-3-2-1) at (6,0) [my box] {0,3,2,1};
  \node (3-0-1-2) at (4,1) [my box] {3,0,1,2};
  \node (3-0-2-1) at (5,1) [my box] {3,0,2,1};

  \node (1-2-0-3) at (2,10) [my box] {1,2,0,3};
  \node (1-2-3-0) at (1,11) [my box] {1,2,3,0};
  \node (2-1-0-3) at (3,10) [my box] {2,1,0,3};
  \node (2-1-3-0) at (3,11) [my box] {2,1,3,0};

  \node (1-3-0-2) at (-1,4) [my box] {1,3,0,2};
  \node (1-3-2-0) at (-1,7) [my box] {1,3,2,0};
  \node (3-1-0-2) at (0,5)  [my box] {3,1,0,2};
  \node (3-1-2-0) at (0,6)  [my box] {3,1,2,0};

  \node (2-3-0-1) at (6,8)  [my box] {2,3,0,1};
  \node (2-3-1-0) at (5,11) [my box] {2,3,1,0};
  \node (3-2-0-1) at (5,9)  [my box] {3,2,0,1};
  \node (3-2-1-0) at (5,10) [my box] {3,2,1,0};

  \draw [] (2-3-1-0) to (2-1-3-0);
  \draw [] (2-0-3-1) to (2-3-0-1);
  \draw [] (0-3-2-1) to (0-2-3-1);
  \draw [] (3-0-1-2) to (3-0-2-1);
  \draw [] (0-1-2-3) to (0-1-3-2);
  \draw [] (2-3-0-1) to (2-3-1-0);
  \draw [] (2-0-1-3) to (2-1-0-3);
  \draw [] (1-2-3-0) to (1-2-0-3);
  \draw [] (2-0-3-1) to (2-0-1-3);
  \draw [] (2-3-1-0) to (3-2-1-0);
  \draw [] (2-3-0-1) to (3-2-0-1);
  \draw [] (3-1-2-0) to (3-1-0-2);
  \draw [] (2-0-1-3) to (0-2-1-3);
  \draw [] (1-3-0-2) to (3-1-0-2);
  \draw [] (0-1-3-2) to (0-3-1-2);
  \draw [] (2-1-0-3) to (1-2-0-3);
  \draw [] (0-2-1-3) to (0-2-3-1);
  \draw [] (0-3-2-1) to (3-0-2-1);
  \draw [] (1-2-3-0) to (1-3-2-0);
  \draw [] (1-0-2-3) to (1-2-0-3);
  \draw [] (3-2-0-1) to (3-2-1-0);
  \draw [] (1-3-0-2) to (1-0-3-2);
  \draw [] (0-3-1-2) to (0-3-2-1);
  \draw [] (3-2-0-1) to (3-0-2-1);
  \draw [] (1-3-2-0) to (3-1-2-0);
  \draw [] (0-1-3-2) to (1-0-3-2);
  \draw [] (2-1-0-3) to (2-1-3-0);
  \draw [] (0-1-2-3) to (1-0-2-3);
  \draw [] (3-2-1-0) to (3-1-2-0);
  \draw [] (1-0-2-3) to (1-0-3-2);
  \draw [] (0-1-2-3) to (0-2-1-3);
  \draw [] (2-1-3-0) to (1-2-3-0);
  \draw [] (0-3-1-2) to (3-0-1-2);
  \draw [] (2-0-3-1) to (0-2-3-1);
  \draw [] (1-3-2-0) to (1-3-0-2);
  \draw [] (3-1-0-2) to (3-0-1-2);

\end{tikzpicture}
\end{center}

%------------------------------------------------------------------------------
\section{Petersen Triangle Replaced}

Vertex transitive non-Hamiltonian.

\url{http://mathworld.wolfram.com/NonhamiltonianVertex-TransitiveGraph.html}

\begin{center}
\begin{tikzpicture}
    [scale=1.3,
     my box/.style={circle,draw,minimum size=2em,inner sep=.1em,outer sep=.2em},
     rotate=90
    ]
  \newcommand\MyOuter{4}
  \newcommand\MyInner{2.7}
  \newcommand\MyTriangle{.9}

  \node (1-1) at ($(0:\MyOuter)+(30:\MyTriangle)$) [my box] {1-1};
  \node (1-2) at (0:\MyOuter) [my box] {1-2};
  \node (1-3) at ($(0:\MyOuter)+(-30:\MyTriangle)$) [my box] {1-3};
  \node (2-1) at ($(72:\MyOuter)+(-30+72:\MyTriangle)$) [my box] {2-1};
  \node (2-2) at ($(72:\MyOuter)+(30+72:\MyTriangle)$) [my box] {2-2};
  \node (2-3) at (72:\MyOuter) [my box] {2-3};
  \node (3-1) at ($(2*72:\MyOuter)+(30+2*72:\MyTriangle)$) [my box] {3-1};
  \node (3-2) at ($(2*72:\MyOuter)+(-30+2*72:\MyTriangle)$) [my box] {3-2};
  \node (3-3) at (2*72:\MyOuter) [my box] {3-3};
  \node (4-1) at ($(3*72:\MyOuter)+(-30+3*72:\MyTriangle)$) [my box] {4-1};
  \node (4-2) at ($(3*72:\MyOuter)+(30+3*72:\MyTriangle)$) [my box] {4-2};
  \node (4-3) at (3*72:\MyOuter) [my box] {4-3};
  \node (5-1) at ($(4*72:\MyOuter)+(30+4*72:\MyTriangle)$) [my box] {5-1};
  \node (5-2) at (4*72:\MyOuter) [my box] {5-2};
  \node (5-3) at ($(4*72:\MyOuter)+(-30+4*72:\MyTriangle)$) [my box] {5-3};
  \node (6-1) at (0*72:\MyInner) [my box] {6-1};
  \node (6-2) at ($(0*72:\MyInner)+(180-30+0*72:\MyTriangle)$) [my box] {6-2};
  \node (6-3) at ($(0*72:\MyInner)+(180+30+0*72:\MyTriangle)$) [my box] {6-3};
  \node (7-1) at ($(1*72:\MyInner)+(180+30+1*72:\MyTriangle)$) [my box] {7-1};
  \node (7-2) at (1*72:\MyInner) [my box] {7-2};
  \node (7-3) at ($(1*72:\MyInner)+(180-30+1*72:\MyTriangle)$) [my box] {7-3};
  \node (8-1) at ($(2*72:\MyInner)+(180-30+2*72:\MyTriangle)$) [my box] {8-1};
  \node (8-2) at ($(2*72:\MyInner)+(180+30+2*72:\MyTriangle)$) [my box] {8-2};
  \node (8-3) at (2*72:\MyInner) [my box] {8-3};
  \node (9-1) at ($(3*72:\MyInner)+(180-30+3*72:\MyTriangle)$) [my box] {9-1};
  \node (9-2) at ($(3*72:\MyInner)+(180+30+3*72:\MyTriangle)$) [my box] {9-2};
  \node (9-3) at (3*72:\MyInner) [my box] {9-3};
  \node (10-1) at ($(4*72:\MyInner)+(180+30+4*72:\MyTriangle)$) [my box] {10-1};
  \node (10-2) at (4*72:\MyInner) [my box] {10-2};
  \node (10-3) at ($(4*72:\MyInner)+(180-30+4*72:\MyTriangle)$) [my box] {10-3};

  \draw [] (5-1) to (1-3);
  \draw [] (7-1) to (10-3);
  \draw [] (6-1) to (6-3);
  \draw [] (6-2) to (8-2);
  \draw [] (5-1) to (5-3);
  \draw [] (1-1) to (1-2);
  \draw [] (4-1) to (3-1);
  \draw [] (4-1) to (4-3);
  \draw [] (4-1) to (4-2);
  \draw [] (6-2) to (6-3);
  \draw [] (1-2) to (1-3);
  \draw [] (10-1) to (10-3);
  \draw [] (2-3) to (2-1);
  \draw [] (2-2) to (3-2);
  \draw [] (6-1) to (6-2);
  \draw [] (2-2) to (2-3);
  \draw [] (6-3) to (9-1);
  \draw [] (4-2) to (4-3);
  \draw [] (5-2) to (5-3);
  \draw [] (7-1) to (7-3);
  \draw [] (7-1) to (7-2);
  \draw [] (8-1) to (8-2);
  \draw [] (6-1) to (1-2);
  \draw [] (9-2) to (9-3);
  \draw [] (7-3) to (9-2);
  \draw [] (7-2) to (7-3);
  \draw [] (3-3) to (8-3);
  \draw [] (9-1) to (9-2);
  \draw [] (5-1) to (5-2);
  \draw [] (3-1) to (3-2);
  \draw [] (2-2) to (2-1);
  \draw [] (10-2) to (10-3);
  \draw [] (1-1) to (2-1);
  \draw [] (5-2) to (10-2);
  \draw [] (8-2) to (8-3);
  \draw [] (10-1) to (8-1);
  \draw [] (2-3) to (7-2);
  \draw [] (9-1) to (9-3);
  \draw [] (10-1) to (10-2);
  \draw [] (1-1) to (1-3);
  \draw [] (3-1) to (3-3);
  \draw [] (4-3) to (9-3);
  \draw [] (4-2) to (5-3);
  \draw [] (8-1) to (8-3);
  \draw [] (3-2) to (3-3);

\end{tikzpicture}
\end{center}

%------------------------------------------------------------------------------
\section{Petersen Line Graph}

Petersen 3-regular so line graph $3+3-2=4$ regular
\begin{center}
\begin{tikzpicture}
    [scale=1.3,
     my box/.style={circle,draw,minimum size=2em,inner sep=.15em,outer sep=.3em},
    ]
  \newcommand\MyOuter{4}
  \node (1-5)  at (18+3*72:\MyOuter) [my box] {1:5};
  \node (1-2)  at (18+0*72:\MyOuter) [my box] {1:2};
  \node (2-3)  at (18+2*72:\MyOuter) [my box] {2:3};
  \node (3-4)  at (18+4*72:\MyOuter) [my box] {3:4};
  \node (4-5)  at (18+1*72:\MyOuter) [my box] {4:5};
  \node (1-6)  at (18+4*72:2.1) [my box] {1:6};
  \node (5-10) at (18+2*72:2.1) [my box] {5:10};
  \node (4-9)  at (18+0*72:2.1) [my box] {4:9};
  \node (2-7)  at (18+1*72:2.1) [my box] {2:7};
  \node (3-8)  at (18+3*72:2.1) [my box] {3:8};
  \node (10-7) at (-18+2*72:.8) [my box] {10:7};
  \node (6-8)  at (-18+4*72:.8) [my box] {6:8};
  \node (6-9)  at (-18+0*72:.8) [my box] {6:9};
  \node (7-9)  at (-18+1*72:.8) [my box] {7:9};
  \node (8-10) at (-18+3*72:.8) [my box] {8:10};

  \draw [] (2-3) to (2-7);
  \draw [] (7-9) to (10-7);
  \draw [] (5-10) to (4-5);
  \draw [] (6-9) to (1-6);
  \draw [] (2-3) to (3-8);
  \draw [] (6-8) to (6-9);
  \draw [] (8-10) to (6-8);
  \draw [] (2-3) to (1-2);
  \draw [] (2-3) to (3-4);
  \draw [] (2-7) to (1-2);
  \draw [] (4-9) to (6-9);
  \draw [] (1-2) to (1-5);
  \draw [] (7-9) to (2-7);
  \draw [] (7-9) to (6-9);
  \draw [] (5-10) to (1-5);
  \draw [] (1-2) to (1-6);
  \draw [] (4-5) to (1-5);
  \draw [] (8-10) to (3-8);
  \draw [] (10-7) to (5-10);
  \draw [] (3-8) to (6-8);
  \draw [] (5-10) to (8-10);
  \draw [] (4-9) to (4-5);
  \draw [] (10-7) to (2-7);
  \draw [] (4-5) to (3-4);
  \draw [] (6-8) to (1-6);
  \draw [] (3-8) to (3-4);
  \draw [] (4-9) to (3-4);
  \draw [] (1-5) to (1-6);
  \draw [] (7-9) to (4-9);
  \draw [] (10-7) to (8-10);

\end{tikzpicture}
\end{center}

\begin{center}
\begin{tikzpicture}
    [scale=1.3,
     my box/.style={circle,draw,minimum size=2em,inner sep=.15em,outer sep=.3em},
    ]

  \node (1-5)  at (18+2*72:3.5) [my box] {1:5};
  \node (1-2)  at (18+1*72:3.5) [my box] {1:2};
  \node (2-3)  at (18+0*72:3.5) [my box] {2:3};
  \node (3-4)  at (18+4*72:3.5) [my box] {3:4};
  \node (4-5)  at (18+3*72:3.5) [my box] {4:5};
  \node (1-6)  at (-18+2*72:2) [my box] {1:6};
  \node (5-10) at (-18+3*72:2) [my box] {5:10};
  \node (4-9)  at (-18+4*72:2) [my box] {4:9};
  \node (2-7)  at (-18+1*72:2) [my box] {2:7};
  \node (3-8)  at (-18+0*72:2) [my box] {3:8};
  \node (10-7) at (-18+2*72:1) [my box] {10:7};
  \node (6-8)  at (-18+1*72:1) [my box] {6:8};
  \node (6-9)  at (-18+3*72:1) [my box] {6:9};
  \node (7-9)  at (-18+0*72:1) [my box] {7:9};
  \node (8-10) at (-18+4*72:1) [my box] {8:10};

  \draw [] (2-3) to (2-7);
  \draw [] (7-9) to (10-7);
  \draw [] (5-10) to (4-5);
  \draw [] (6-9) to (1-6);
  \draw [] (2-3) to (3-8);
  \draw [] (6-8) to (6-9);
  \draw [] (8-10) to (6-8);
  \draw [] (2-3) to (1-2);
  \draw [] (2-3) to (3-4);
  \draw [] (2-7) to (1-2);
  \draw [] (4-9) to (6-9);
  \draw [] (1-2) to (1-5);
  \draw [] (7-9) to (2-7);
  \draw [] (7-9) to (6-9);
  \draw [] (5-10) to (1-5);
  \draw [] (1-2) to (1-6);
  \draw [] (4-5) to (1-5);
  \draw [] (8-10) to (3-8);
  \draw [] (10-7) to (5-10);
  \draw [] (3-8) to (6-8);
  \draw [] (5-10) to (8-10);
  \draw [] (4-9) to (4-5);
  \draw [] (10-7) to (2-7);
  \draw [] (4-5) to (3-4);
  \draw [] (6-8) to (1-6);
  \draw [] (3-8) to (3-4);
  \draw [] (4-9) to (3-4);
  \draw [] (1-5) to (1-6);
  \draw [] (7-9) to (4-9);
  \draw [] (10-7) to (8-10);

\end{tikzpicture}
\end{center}


\begin{center}
\begin{tikzpicture}
    [scale=1.3,
     my box/.style={circle,draw,minimum size=2em,inner sep=.15em,outer sep=.3em},
    ]

  \node (1-5)  at (18+2*72:3.5) [my box] {1:5};
  \node (1-2)  at (18+1*72:3.5) [my box] {1:2};
  \node (2-3)  at (18+0*72:3.5) [my box] {2:3};
  \node (3-4)  at (18+4*72:3.5) [my box] {3:4};
  \node (4-5)  at (18+3*72:3.5) [my box] {4:5};
  \node (1-6)  at (-18+2*72:2) [my box] {1:6};
  \node (5-10) at (-18+3*72:2) [my box] {5:10};
  \node (4-9)  at (-18+4*72:2) [my box] {4:9};
  \node (2-7)  at (-18+1*72:2) [my box] {2:7};
  \node (3-8)  at (-18+0*72:2) [my box] {3:8};
  \node (10-7) at (18+3*72:1) [my box] {10:7};
  \node (6-8)  at (18+2*72:1) [my box] {6:8};
  \node (6-9)  at (18+4*72:1) [my box] {6:9};
  \node (7-9)  at (18+1*72:1) [my box] {7:9};
  \node (8-10) at (18+0*72:1) [my box] {8:10};

  \draw [] (2-3) to (2-7);
  \draw [] (7-9) to (10-7);
  \draw [] (5-10) to (4-5);
  \draw [] (6-9) to (1-6);
  \draw [] (2-3) to (3-8);
  \draw [] (6-8) to (6-9);
  \draw [] (8-10) to (6-8);
  \draw [] (2-3) to (1-2);
  \draw [] (2-3) to (3-4);
  \draw [] (2-7) to (1-2);
  \draw [] (4-9) to (6-9);
  \draw [] (1-2) to (1-5);
  \draw [] (7-9) to (2-7);
  \draw [] (7-9) to (6-9);
  \draw [] (5-10) to (1-5);
  \draw [] (1-2) to (1-6);
  \draw [] (4-5) to (1-5);
  \draw [] (8-10) to (3-8);
  \draw [] (10-7) to (5-10);
  \draw [] (3-8) to (6-8);
  \draw [] (5-10) to (8-10);
  \draw [] (4-9) to (4-5);
  \draw [] (10-7) to (2-7);
  \draw [] (4-5) to (3-4);
  \draw [] (6-8) to (1-6);
  \draw [] (3-8) to (3-4);
  \draw [] (4-9) to (3-4);
  \draw [] (1-5) to (1-6);
  \draw [] (7-9) to (4-9);
  \draw [] (10-7) to (8-10);

\end{tikzpicture}
\end{center}

%------------------------------------------------------------------------------
\section{Beineke All as Subgraphs, 12 Edges}

\url{https://hog.grinvin.org/ViewGraphInfo.action?id=748}

cf.\@ Introduction to Line Graphs
Emphasizing their construction, clique decompositions, and regularity

\begin{center}
\begin{tikzpicture}
    [scale=1,
     my box/.style={circle,draw,inner sep=.15em,outer sep=.3em},
    ]

  \node at (0,1) [my box,name=1] {1};
  \node at (0,2) [my box,name=3] {3};
  \node at (0,3) [my box,name=0] {0};
  \node at (0,4) [my box,name=2] {2};
  \node at (-2, .25) [my box,name=4] {4};
  \node at (2, .25) [my box,name=5] {5};

  \draw [] (0) to (4);
  \draw [] (0) to (2);
  \draw [] (2) to (5);
  \draw [] (1) to (4);
  \draw [] (0) to (5);
  \draw [] (3) to (5);
  \draw [] (4) to (5);
  \draw [] (0) to (3);
  \draw [] (3) to (1);
  \draw [] (3) to (4);
  \draw [] (2) to (4);
  \draw [] (1) to (5);
\end{tikzpicture}
\end{center}

\begin{center}
\begin{tikzpicture}
    [scale=2,
     my box/.style={circle,draw,inner sep=.15em,outer sep=.3em},
    ]

  \node at (4*72+18:1) [my box,name=0] {0};
  \node at (2*72+18:1) [my box,name=1] {1};
  \node at (0*72+18:1) [my box,name=2] {2};
  \node at (3*72+18:1) [my box,name=3] {3};
  \node at (0,0) [my box,name=4] {4};
  \node at (1*72+18:1) [my box,name=5] {5};

  \draw [] (0) to (4);
  \draw [] (0) to (2);
  \draw [] (2) to (5);
  \draw [] (1) to (4);
  \draw [] (0) to (5);
  \draw [] (3) to (5);
  \draw [] (4) to (5);
  \draw [] (0) to (3);
  \draw [] (3) to (1);
  \draw [] (3) to (4);
  \draw [] (2) to (4);
  \draw [] (1) to (5);
\end{tikzpicture}
\end{center}

\begin{center}
\begin{tikzpicture}
    [scale=2,
     my box/.style={circle,draw,inner sep=.15em,outer sep=.3em},
    ]

  \node at (0,0) [my box,name=0] {0};
  \node at (2,2) [my box,name=1] {1};
  \node at (2,0) [my box,name=2] {2};
  \node at (0,2) [my box,name=3] {3};
  \node at (1,1) [my box,name=4] {4};
  \node at (2,1) [my box,name=5] {5};

  \draw [] (0) to (4);
  \draw [] (0) to (2);
  \draw [] (2) to (5);
  \draw [] (1) to (4);
  \draw [] (0) to (5);
  \draw [] (3) to (5);
  \draw [] (4) to (5);
  \draw [] (0) to (3);
  \draw [] (3) to (1);
  \draw [] (3) to (4);
  \draw [] (2) to (4);
  \draw [] (1) to (5);
\end{tikzpicture}
\end{center}

\begin{center}
\begin{tikzpicture}
    [scale=2,
     my box/.style={circle,draw,inner sep=.15em,outer sep=.3em},
    ]

  \node at (0,0) [my box,name=0] {0};
  \node at (2,2) [my box,name=1] {1};
  \node at (2,0) [my box,name=2] {2};
  \node at (0,2) [my box,name=3] {3};
  \node at (.5,1) [my box,name=4] {4};
  \node at (1.5,1) [my box,name=5] {5};

  \draw [] (0) to (4);
  \draw [] (0) to (2);
  \draw [] (2) to (5);
  \draw [] (1) to (4);
  \draw [] (0) to (5);
  \draw [] (3) to (5);
  \draw [] (4) to (5);
  \draw [] (0) to (3);
  \draw [] (3) to (1);
  \draw [] (3) to (4);
  \draw [] (2) to (4);
  \draw [] (1) to (5);
\end{tikzpicture}
\end{center}

\begin{center}
\begin{tikzpicture}
    [scale=2,
     my box/.style={circle,draw,inner sep=.15em,outer sep=.3em},
    ]

  \node at (1,0) [my box,name=1] {1};
  \node at (2,0) [my box,name=3] {3};
  \node at (3,0) [my box,name=0] {0};
  \node at (4,0) [my box,name=2] {2};
  \node at (2.5,1) [my box,name=4] {4};
  \node at (2.5,-1) [my box,name=5] {5};

  \draw [] (0) to (4);
  \draw [] (0) to (2);
  \draw [] (2) to (5);
  \draw [] (1) to (4);
  \draw [] (0) to (5);
  \draw [] (3) to (5);
  \draw [] (4) to (5);
  \draw [] (0) to (3);
  \draw [] (3) to (1);
  \draw [] (3) to (4);
  \draw [] (2) to (4);
  \draw [] (1) to (5);
\end{tikzpicture}
\end{center}




%------------------------------------------------------------------------------
\section{Beineke Subgraph Relations}

\begin{center}
\begin{tikzpicture}
    [scale=1.5,
     my box/.style={circle,draw,inner sep=.15em,outer sep=.3em},
    ]

  \node at (0,0) [my box,name=1] {1};
  \node at (-1,0) [my box,name=2] {2};
  \node at (-1,-1) [my box,name=3] {3};
  \node at (1,0) [my box,name=4] {4};
  \node at (-1,1) [my box,name=5] {5};
  \node at (0,1) [my box,name=6] {6};
  \node at (1, -1) [my box,name=7] {7};
  \node at (1,1) [my box,name=8] {8};
  \node at (0,-1) [my box,name=9] {9};

  \draw [->] (6) to (8);
  \draw [->] (6) to (1);
  \draw [->] (2) to (1);
  \draw [->] (8) to (4);
  \draw [->] (4) to (1);
  \draw [->,bend left=0] (5) to (4);
  \draw [->] (9) to (4);
  \draw [->] (6) to (4);
  \draw [->,bend right=0] (7) to (4);
  \draw [->] (5) to (2);
  \draw [->] (3) to (1);
  \draw [->] (5) to (1);
  \draw [->] (6) to (5);
  \draw [->] (9) to (1);
  \draw [->] (3) to (2);
  \draw [->] (8) to (1);
  \draw [->] (9) to (7);
  \draw [->] (6) to (7);
  \draw [->] (7) to (1);
  \draw [->,bend left=0] (6) to (2);

\end{tikzpicture}
\end{center}

\begin{center}
\begin{tikzpicture}
    [scale=2,
     my box/.style={circle,draw,inner sep=.15em,outer sep=.3em},
    ]

  \node (1) at (0,0)           [my box] {1};
  \node (4) at (0*45+22.5*1:1) [my box] {4};
  \node (8) at (1*45+22.5*1:1) [my box] {8};
  \node (6) at (2*45+22.5*1:1) [my box] {6};
  \node (5) at (3*45+22.5*1:1) [my box] {5};
  \node (2) at (4*45+22.5*1:1) [my box] {2};
  \node (3) at (5*45+22.5*1:1) [my box] {3};
  \node (9) at (6*45+22.5*1:1) [my box] {9};
  \node (7) at (7*45+22.5*1:1) [my box] {7};

  \draw [->] (6) to (8);
  \draw [->] (6) to (1);
  \draw [->] (2) to (1);
  \draw [->] (8) to (4);
  \draw [->] (4) to (1);
  \draw [->,bend left=0] (5) to (4);
  \draw [->] (9) to (4);
  \draw [->] (6) to (4);
  \draw [->,bend right=0] (7) to (4);
  \draw [->] (5) to (2);
  \draw [->] (3) to (1);
  \draw [->] (5) to (1);
  \draw [->] (6) to (5);
  \draw [->] (9) to (1);
  \draw [->] (3) to (2);
  \draw [->] (8) to (1);
  \draw [->] (9) to (7);
  \draw [->] (6) to (7);
  \draw [->] (7) to (1);
  \draw [->,bend left=0] (6) to (2);

\end{tikzpicture}
\end{center}


%------------------------------------------------------------------------------
\section{Hanoi 2 Discs, 4 Spindles, Star}

\url{https://hog.grinvin.org/ViewGraphInfo.action?id=21152}

\begin{center}
\begin{tikzpicture}
    [scale=1.25,
     my box/.style={circle,draw,inner sep=.15em,outer sep=.3em},
     font=\scriptsize,
    ]
  % degree=2 count 9
  % degree=3 count 4
  % degree=4 count 3

  \node at (0,0)   [my box,name=00] {00};
  \node at (30:1)  [my box,name=01] {01};
  \node at (-90:1) [my box,name=02] {02};
  \node at (150:1) [my box,name=03] {03};

  \begin{scope}[shift={(210:2)}]
  \node at (0,0)   [my box,name=10] {10};
  \node at (210:1) [my box,name=11] {11};
  \node at (-30:1) [my box,name=12] {12};
  \node at (90:1) [my box,name=13] {13};
  \end{scope}

  \begin{scope}[shift={(90:2)}]
  \node at (0,0)   [my box,name=20] {20};
  \node at (-30:1) [my box,name=21] {21};
  \node at (90:1) [my box,name=22] {22};
  \node at (210:1) [my box,name=23] {23};
  \end{scope}

  \begin{scope}[shift={(-30:2)}]
  \node at (0,0)   [my box,name=30] {30};
  \node at (90:1) [my box,name=31] {31};
  \node at (210:1) [my box,name=32] {32};
  \node at (-30:1) [my box,name=33] {33};
  \end{scope}


  \draw [] (01) to (21);
  \draw [] (20) to (21);
  \draw [] (03) to (23);
  \draw [] (00) to (01);
  \draw [] (00) to (02);
  \draw [] (02) to (12);
  \draw [] (20) to (23);
  \draw [] (00) to (03);
  \draw [] (10) to (12);
  \draw [] (30) to (33);
  \draw [] (01) to (31);
  \draw [] (03) to (13);
  \draw [] (10) to (11);
  \draw [] (30) to (32);
  \draw [] (10) to (13);
  \draw [] (20) to (22);
  \draw [] (02) to (32);
  \draw [] (30) to (31);

\end{tikzpicture}
\end{center}

Each vertex is a configuration of discs on spindles for a variation by
Stockmeyer of the towers of Hanoi puzzle.  There are 4 spindles in a
star pattern and discs can only move between the centre spindle and
one of the 3 outer spindles (not among those outer spindles).

The present graph is for 2 discs.  The centre claw has big disc on the
centre spindle and the small disc moving to or from one of the outer
spindles.  When the small disc is not on the centre spindle the big
disc can move from there to one of the other outer spindles (where
further claws are the small disc moving).

The puzzle is to move both discs from one outer spindle to another outer
spindle.  This is a path between two of the degree\hyp{}1 vertices.

Paul K. Stockmeyer, ``Variations on the Four-Post Tower of Hanoi
Puzzle'', Congressus Numerantium, volume 102, 1994, pages 3-12,
\url{http://www.cs.wm.edu/~pkstoc/boca.ps}


%------------------------------------------------------------------------------
\section{Hanoi Graph 2 Discs, 4 Spindles, Linear}

\url{https://hog.grinvin.org/ViewGraphInfo.action?id=25143}

% 3, 10, 19, 34, 57, 88
% A160002
% my(n=2); 3^n+n-1 == 10
% my(n=3); 3^n+n-1 == 29

\begin{center}
\begin{tikzpicture}
  [scale=1.5,
   my box/.style={circle,draw,inner sep=.15em,outer sep=.3em,minimum size=1.6em},
  ]

  \node at (4,-1) [my box,name=0] {0};
  \node at (4,0) [my box,name=1] {1};
  \node at (4,1) [my box,name=2] {2};
  \node at (4,2) [my box,name=3] {3};

  \node at (1,1) [my box,name=4] {4};
  \node at (2,1) [my box,name=5] {5};
  \node at (3,1) [my box,name=6] {6};
  \node at (3,2) [my box,name=7] {7};

  \node at (0,1) [my box,name=8] {8};
  \node at (0,2) [my box,name=9] {9};
  \node at (1,2) [my box,name=10] {10};
  \node at (2,2) [my box,name=11] {11};

  \node at (-1,1) [my box,name=12] {12};
  \node at (-1,2) [my box,name=13] {13};
  \node at (-1,3) [my box,name=14] {14};
  \node at (-1,4) [my box,name=15] {15};


  \draw [] (14) to (15);
  \draw [] (9) to (10);
  \draw [] (0) to (1);
  \draw [] (9) to (8);
  \draw [] (4) to (5);
  \draw [] (3) to (7);
  \draw [] (1) to (2);
  \draw [] (7) to (11);
  \draw [] (6) to (7);
  \draw [] (4) to (8);
  \draw [] (12) to (8);
  \draw [] (9) to (13);
  \draw [] (2) to (6);
  \draw [] (13) to (14);
  \draw [] (2) to (3);
  \draw [] (6) to (5);
  \draw [] (12) to (13);
  \draw [] (10) to (11);

\end{tikzpicture}
\end{center}

Each vertex is a configuration of discs on spindles for a variation on the towers of Hanoi puzzle by Stockmeyer where discs can only move forward or backward between adjacent spindles along a row of 4 spindles.

The two degree-1 vertices are the 2 discs on the first or last
spindle.  The problem is to move the discs from one end to the other.
This is the graph diameter of 10 moves (OEIS A160002).

Paul K. Stockmeyer, ``Variations on the Four-Post Tower of Hanoi
Puzzle'', Congressus Numerantium, volume 102, 1994, pages 3-12,
\url{http://www.cs.wm.edu/~pkstoc/boca.ps}


% my(n=2); 3^n+n-1 == 10

Stockmeyer gives a straightforward recursive solution for the general case n discs along 4 spindles.  This establishes an upper bound of $3^n+n-1$ moves.  The n=2 case here is the last where that solution is optimal.

``The Four-in-a-Row Puzzle''


%------------------------------------------------------------------------------
\section{Hanoi Graph 2 Discs, 4 Spindles, Cyclic}

\url{https://hog.grinvin.org/ViewGraphInfo.action?id=25141}

\begin{center}
\begin{tikzpicture}
    [scale=2,
     my box/.style={circle,draw,inner sep=.15em,outer sep=.3em},
     font=\scriptsize,
    ]

  \node at (0,0) [my box,name=0] {0};
  \node at (0,1) [my box,name=1] {1};
  \node at (1,1) [my box,name=2] {2};
  \node at (1,0) [my box,name=3] {3};

  \node at (3,1) [my box,name=4] {4};
  \node at (3,0) [my box,name=5] {5};
  \node at (2,0) [my box,name=6] {6};
  \node at (2,1) [my box,name=7] {7};

  \node at (2,2) [my box,name=8] {8};
  \node at (2,3) [my box,name=9] {9};
  \node at (3,3) [my box,name=10] {10};
  \node at (3,2) [my box,name=11] {11};

  \node at (1,3) [my box,name=12] {12};
  \node at (1,2) [my box,name=13] {13};
  \node at (0,2) [my box,name=14] {14};
  \node at (0,3) [my box,name=15] {15};

  \draw [] (2) to (6);
  \draw [] (1) to (13);
  \draw [] (13) to (14);
  \draw [] (7) to (4);
  \draw [] (1) to (2);
  \draw [] (15) to (14);
  \draw [] (11) to (10);
  \draw [] (4) to (5);
  \draw [] (8) to (4);
  \draw [] (6) to (7);
  \draw [] (12) to (13);
  \draw [] (8) to (12);
  \draw [] (0) to (3);
  \draw [] (9) to (10);
  \draw [] (8) to (9);
  \draw [] (0) to (1);
  \draw [] (8) to (11);
  \draw [] (3) to (7);
  \draw [] (6) to (5);
  \draw [] (3) to (2);
  \draw [] (12) to (15);
  \draw [] (2) to (14);
  \draw [] (9) to (13);
  \draw [] (11) to (7);

\end{tikzpicture}
\end{center}

Each vertex is a configuration of discs on spindles for a variation on
the towers of Hanoi puzzle where discs can only move to an adjacent
spindle around a cycle of 4 spindles.

The degree-2 vertices are where the 2 discs are on the same spindle so
the only moves are the smaller disc back or forward.  The 4-cycle at
each of those vertices is the small disc going successively around the
4 spindles.  The cross connections between those cycles are where the
big disc moves (where permitted).


%------------------------------------------------------------------------------
\section{Hanoi 2 Discs 4 Spindles}

\url{https://hog.grinvin.org/ViewGraphInfo.action?id=22742}

\begin{center}
\begin{tikzpicture}
    [scale=.75,
     my box/.style={circle,draw,inner sep=.15em,outer sep=.3em},
     font=\scriptsize,
    ]
  \newcommand\MyOuter{3}

  \begin{scope}[shift={(0,\MyOuter)}]
    \node at (0,1) [my box,name=0] {0};
    \node at (-1,0) [my box,name=1] {1};
    \node at (1,0) [my box,name=2] {2};
    \node at (0,-1) [my box,name=3] {3};
  \end{scope}

  \begin{scope}[shift={(\MyOuter,0)}]
    \node at (0,-1) [my box,name=4] {4};
    \node at (1,0) [my box,name=5] {5};
    \node at (-1,0) [my box,name=6] {6};
    \node at (0,1) [my box,name=7] {7};
  \end{scope}

  \begin{scope}[shift={(-\MyOuter,0)}]
    \node at (0,-1) [my box,name=8] {8};
    \node at (1,0) [my box,name=9] {9};
    \node at (-1,0) [my box,name=10] {10};
    \node at (0,1) [my box,name=11] {11};
  \end{scope}

  \begin{scope}[shift={(0,-\MyOuter)}]
    \node at (0,1) [my box,name=12] {12};
    \node at (-1,0) [my box,name=13] {13};
    \node at (1,0) [my box,name=14] {14};
    \node at (0,-1) [my box,name=15] {15};
  \end{scope}

  \draw [] (9) to (11);
  \draw [] (4) to (6);
  \draw [] (3) to (11);
  \draw [] (8) to (9);
  \draw [] (15) to (14);
  \draw [] (12) to (15);
  \draw [] (5) to (6);
  \draw [] (12) to (14);
  \draw [] (15) to (13);
  \draw [] (8) to (12);
  \draw [] (8) to (11);
  \draw [] (1) to (13);
  \draw [] (9) to (13);
  \draw [] (6) to (14);
  \draw [] (4) to (7);
  \draw [] (5) to (7);
  \draw [] (7) to (11);
  \draw [] (0) to (2);
  \draw [] (10) to (9);
  \draw [] (1) to (9);
  \draw [] (3) to (7);
  \draw [] (4) to (12);
  \draw [] (6) to (7);
  \draw [] (0) to (1);
  \draw [] (1) to (2);
  \draw [] (4) to (5);
  \draw [] (12) to (13);
  \draw [] (10) to (11);
  \draw [] (13) to (14);
  \draw [] (4) to (8);
  \draw [] (8) to (10);
  \draw [] (2) to (6);
  \draw [] (2) to (3);
  \draw [] (0) to (3);
  \draw [] (2) to (14);
  \draw [] (1) to (3);

\end{tikzpicture}
\end{center}

Each vertex is a configuration of discs on spindles for a variation on
the towers of Hanoi puzzle with 2 discs on 4 spindles.

The degree-3 vertices are where the 2 discs are on the same spindle so
only the smaller disc can move.  The complete-4 clique at each of
those is the small disc moving among the 4 spindles.  The cross
connections between those subgraphs are where the big disc moves (to
one of the 2 spindles different from itself and the smaller disc).

\medskip

Cf.\ picture in Andreas M. Hinz, Sandi Klav\v{z}ar, Sara Sabrina
Zemlji\v{c}, ``Sierpinski Graphs as Spanning Subgraphs of Hanoi
Graphs'', Cent.\ Eur.\ J. Math., 11(6), 2013, 1153-1157.
DOI 10.2478/s11533-013-0227-7



%------------------------------------------------------------------------------
\section{Johnson 5,2}

\url{https://hog.grinvin.org/ViewGraphInfo.action?id=21154}

\begin{center}
\begin{tikzpicture}
    [scale=.74,
     my box/.style={circle,draw,inner sep=.15em,outer sep=.3em},
    ]
  \newcommand\MyOuter{9}
  \newcommand\MyInner{6.2}
  % \newcommand\MyInnerAngle{54+180}
  \newcommand\MyInnerAngle{54}

  \node at (90-0*72: \MyOuter) [my box,name=1-2] {1,2};
  \node at (90-1*72: \MyOuter) [my box,name=2-3] {2,3};
  \node at (90-2*72: \MyOuter) [my box,name=3-4] {3,4};
  \node at (90-3*72: \MyOuter) [my box,name=4-5] {4,5};
  \node at (90-4*72: \MyOuter) [my box,name=1-5] {1,5};

  \node at (\MyInnerAngle-0*72: \MyInner) [my box,name=1-3] {1,3};
  \node at (\MyInnerAngle-1*72: \MyInner) [my box,name=2-4] {2,4};
  \node at (\MyInnerAngle-2*72: \MyInner) [my box,name=3-5] {3,5};
  \node at (\MyInnerAngle-3*72: \MyInner) [my box,name=1-4] {1,4};
  \node at (\MyInnerAngle-4*72: \MyInner) [my box,name=2-5] {2,5};

  \draw (1-2) to (1-3);
  \draw (1-2) to (1-4);
  \draw (1-2) to (1-5);
  \draw (1-2) to (2-3);
  \draw (1-2) to (2-4);
  \draw (1-2) to (2-5);

  \draw (1-4) to (4-5);
  \draw (2-4) to (3-4);
  \draw (1-3) to (1-5);
  \draw (1-5) to (2-5);
  \draw (1-3) to (2-3);
  \draw (2-5) to (4-5);
  \draw (3-5) to (4-5);
  \draw (1-3) to (3-4);
  \draw (2-3) to (3-5);
  \draw (3-4) to (4-5);
  \draw (1-4) to (1-5);
  \draw (1-5) to (3-5);
  \draw (1-4) to (3-4);
  \draw (1-4) to (2-4);
  \draw (1-3) to (1-4);
  \draw (2-3) to (3-4);
  \draw (3-4) to (3-5);
  \draw (2-3) to (2-4);
  \draw (1-3) to (3-5);
  \draw (2-5) to (3-5);
  \draw (2-4) to (4-5);
  \draw (1-5) to (4-5);
  \draw (2-4) to (2-5);
  \draw (2-3) to (2-5);

\end{tikzpicture}
\end{center}

\begin{center}
\begin{tikzpicture}
    [scale=1,
     my box/.style={circle,draw,inner sep=.15em,outer sep=.3em},
     font=\scriptsize,
    ]

  \node at (-0*36: 5) [my box,name=1-2] {1,2};    % 3
  \node at (-1*36: 5) [my box,name=2-5] {2,5};    % 7
  \node at (-2*36: 5) [my box,name=2-4] {2,4};    % 6
  \node at (-3*36: 5) [my box,name=2-3] {2,3};    % 5
  \node at (-4*36: 5) [my box,name=3-5] {3,5};    % 8
  \node at (-5*36: 5) [my box,name=3-4] {3,4};    % 7
  \node at (-6*36: 5) [my box,name=1-3] {1,3};    % 4

  \node at (-7*36: 5) [my box,name=1-5] {1,5};    % 6
  \node at (-8*36: 5) [my box,name=1-4] {1,4};    % 5
  \node at (-9*36: 5) [my box,name=4-5] {4,5};    % 9

  \draw (1-2) to (1-3);
  \draw (1-2) to (1-4);
  \draw (1-2) to (1-5);
  \draw (1-2) to (2-3);
  \draw (1-2) to (2-4);
  \draw (1-2) to (2-5);

  \draw (1-4) to (4-5);
  \draw (2-4) to (3-4);
  \draw (1-3) to (1-5);
  \draw (1-5) to (2-5);
  \draw (1-3) to (2-3);
  \draw (2-5) to (4-5);
  \draw (3-5) to (4-5);
  \draw (1-3) to (3-4);
  \draw (2-3) to (3-5);
  \draw (3-4) to (4-5);
  \draw (1-4) to (1-5);
  \draw (1-5) to (3-5);
  \draw (1-4) to (3-4);
  \draw (1-4) to (2-4);
  \draw (1-3) to (1-4);
  \draw (2-3) to (3-4);
  \draw (3-4) to (3-5);
  \draw (2-3) to (2-4);
  \draw (1-3) to (3-5);
  \draw (2-5) to (3-5);
  \draw (2-4) to (4-5);
  \draw (1-5) to (4-5);
  \draw (2-4) to (2-5);
  \draw (2-3) to (2-5);

\end{tikzpicture}
\end{center}

\begin{center}
\begin{tikzpicture}
    [scale=.74,
     my box/.style={circle,draw,minimum width=2.5em,inner sep=.15em,outer sep=.3em},
    ]
  \newcommand\MyOuter{9}
  \newcommand\MyInner{5}
  \newcommand\MyInIn{5}
  % \newcommand\MyInnerAngle{54+180}
  \newcommand\MyInnerAngle{54}

  \node at (-0*36: \MyOuter) [my box,name=1-2] {1-2};
  \node at (-2*36: \MyOuter) [my box,name=2-3] {2-3};
  \node at (-4*36: \MyOuter) [my box,name=3-4] {3-4};
  \node at (-6*36: \MyOuter) [my box,name=4-5] {4-5};
  \node at (-8*36: \MyOuter) [my box,name=1-5] {1-5};

  \node at (-1*36: \MyInner) [my box,name=2-7] {2-7};
  \node at (-3*36: \MyInner) [my box,name=3-8] {3-8};
  \node at (-5*36: \MyInner) [my box,name=4-9] {4-9};
  \node at (-7*36: \MyInner) [my box,name=5-10] {5-10};
  \node at (-9*36: \MyInner) [my box,name=1-6] {1-6};

  \node at (-4*36: \MyInIn) [my box,name=10-7] {10-7};
  \node at (-6*36: \MyInIn) [my box,name=6-8] {6-8};
  \node at (-8*36: \MyInIn) [my box,name=7-9] {7-9};
  \node at (-10*36: \MyInIn) [my box,name=8-10] {8-10};
  \node at (-12*36: \MyInIn) [my box,name=6-9] {6-9};

  \draw [] (2-3) to (1-2);
  \draw [] (2-3) to (3-4);
  \draw [] (1-5) to (4-5);
  \draw [] (1-6) to (6-8);
  \draw [] (1-5) to (1-6);
  \draw [] (2-3) to (3-8);
  \draw [] (8-10) to (3-8);
  \draw [] (1-6) to (6-9);
  \draw [] (4-5) to (5-10);
  \draw [] (4-5) to (3-4);
  \draw [] (4-9) to (7-9);
  \draw [] (10-7) to (2-7);
  \draw [] (6-9) to (7-9);
  \draw [] (4-5) to (4-9);
  \draw [] (2-7) to (1-2);
  \draw [] (8-10) to (5-10);
  \draw [] (6-8) to (3-8);
  \draw [] (10-7) to (8-10);
  \draw [] (4-9) to (6-9);
  \draw [] (1-5) to (5-10);
  \draw [] (1-5) to (1-2);
  \draw [] (4-9) to (3-4);
  \draw [] (1-6) to (1-2);
  \draw [] (2-7) to (2-3);
  \draw [] (8-10) to (6-8);
  \draw [] (2-7) to (7-9);
  \draw [] (10-7) to (5-10);
  \draw [] (6-9) to (6-8);
  \draw [] (3-8) to (3-4);
  \draw [] (10-7) to (7-9);

\end{tikzpicture}
\end{center}

%------------------------------------------------------------------------------

\end{document}

% Local variables:
% compile-command: "latexmk -file-line-error -pdf pictures.tex"
% End:
