\magnification=1200
\input gooemacs
\input epsf
\font\titlefont=cmr8 at 16pt
\font\titlett=cmtt10 at 16pt
\font\seventtt=cmtt8 at 7pt
\font\metafnt=manfnt at 10pt
\font\gofont=cmss10
\def\metafont{{\metafnt/0123451}}
\def\\{{\char'134}}

\vfil
\centerline{\titlefont Go typesetting using \TeX}

\bigbreak
Here is a system for typesetting Go games and diagrams using \TeX.
This system may be used with either plain \TeX\ or La\TeX. It includes 
the \metafont\ sources for a new set of fonts, called GOOE, and a {\tt perl}
script called {\tt sgf2dg} which translates files in the common ``Smart Go
Format'' (SGF) files into (plain) \TeX.

The numerals on the Go stones in these fonts are instances of the Computer
Modern Fonts designed by Donald Knuth and his co-workers as part of the
\TeX\ system. The file {\tt romandg.mf} is the same as the file 
{\tt romand.mf} distributed with \TeX, with only trivial modifications 
to allow the fonts generated from it to be pasted onto Go stones. This file is
of course copyrighted by Donald Knuth. The remaining portions of the GOOE/{\tt
sgf2dg} system are published under the Gnu Public License, a copy of which is
distributed with this system in the file {\tt COPYING}. Although this system
is thus free software, we would appreciate acknowledgement if it is used to
publish a book.

For installation instructions, please see the file INSTALL that
is included with this package.

Another set of Go fonts was created by Hanna Kolodziejska around 1990, and
revised by Jan van der Steen. Those fonts, together with the latex document
style file {\tt go.sty} can be found at CTAN in the directory {\tt fonts/go},
or at the Go ftp sites (such as {\tt ftp://igs.nuri.net} in the {\tt
Go/printing} directory) under the name {\tt golatex}. Jan van der Steen's
utility {\tt sgf2misc} has the capability of generating La\TeX\ or postscript
files from SGF. It may be obtained from {\tt ftp://igs.nuri.net} in the {\tt
Go/prog} directory. Our work is independent of and different from
Kolodziejska's and van der Steen's.

{\tt sgf2dg} may be obtained from 
{\tt http://match.stanford.edu/bump/go.html}\hfil\break 
or {\tt ftp://match.stanford.edu/pub/sgf2dg-<version>.tar.gz}.  It is also
available from the Comprehensive Perl Archive Network (CPAN): go to {\tt
http://search.cpan.org/} and search for {\tt sgf2dg}.

\TeX\ is a trademark of the American Mathematical Society. \metafont\ is
a trademark of Addison Wesley Publishing Company.

\smallbreak
\line{\hglue2in Daniel Bump ({\tt bump@math.stanford.edu})\hfil}
\line{\hglue2in  Reid Augustin ({\tt reid@hellosix.com}), 1997, 1998\hfil}
\smallbreak\noindent

\vfil
\hbox to\hsize{
\vbox to 120 pt{\hsize= 84.0 pt\goo
\0??(\0??(\0??(\005+\006+\007+\0??>
\0??+\0??+\- !+\- @+\004+\- !+\001+
\0??+\0??+\- !+\- @+\0??+\003+\0??]
\0??+\0??+\- !+\- @+\- @+\- @+\002+
\0??+\0??+\- !+\- !+\- !+\- !+\0??]
\0??+\0??+\0??+\0??+\0??+\0??+\0??]
\0??+\0??+\0??+\0??+\0??+\0??+\0??]
\vfil}\hfil
\vbox to 120 pt{\hsize= 84.0 pt\goe
\0??(\0??(\0??(\005+\007+\008+\0??>
\0??+\0??+\- !+\- @+\006+\001+\002+
\0??+\0??+\- !+\- @+\0??+\003+\004+
\0??+\0??+\- !+\- @+\- @+\- @+\0??]
\0??+\0??+\- !+\- !+\- !+\- !+\0??]
\0??+\0??+\0??+\0??+\0??+\0??+\0??]
\0??+\0??+\0??+\0??+\0??+\0??+\0??]
\vfil}\hfil
\vbox to 120 pt{\hsize= 84.0 pt\goe
\0??(\0??(\0??(\0??(\008+\0??(\005+
\0??+\0??+\- !+\- @+\007+\001+\002+
\0??+\0??+\- !+\- @+\0??+\004+\003+
\0??+\0??+\- !+\- @+\- @+\- @+\006+
\0??+\0??+\- !+\- !+\- !+\- !+\0??]
\0??+\0??+\0??+\0??+\0??+\0??+\0??]
\0??+\0??+\0??+\0??+\0??+\0??+\0??]
\vfil}\hfil
\vbox to 120.0 pt{\hsize= 84.0 pt
{\goe
\0??(\0??(\0??(\004+\006+\007+\0??>
\0??+\0??+\- !+\- @+\003+\001+\005+
\0??+\0??+\- !+\- @+\0??+\002+\0??]
\0??+\0??+\- !+\- @+\- @+\- @+\0??]
\0??+\0??+\- !+\- !+\- !+\- !+\0??]
\0??+\0??+\0??+\0??+\0??+\0??+\0??]
\0??+\0??+\0??+\0??+\0??+\0??+\0??]}
\vfil
\break\noindent
Black \textstone{\goe\002=} is bad.\hfil\break
Black dies.\hfil\break
}}

\vfil\eject

\bigbreak
\hbox to\hsize{\noindent\vbox to 273.6 pt{\hsize=273.6 pt\bgoo
\0??<\0??(\0??(\0??(\0??(\0??(\0??(\0??(\0??(\0??(\0??(\0??(\0??(\0??(\0??(\0??(\0??(\0??(\0??>
\0??[\0??+\0??+\0??+\0??+\0??+\0??+\0??+\094+\0??+\096+\0??+\0??+\0??+\0??+\0??+\145+\144+\148+
\0??[\0??+\049+\002+\0??+\0??+\016+\0??+\015+\062+\026+\090+\093+\0??+\004+\146+\142+\143+\0??]
\0??[\0??+\0??+\050+\0??+\0??+\0??+\0??+\095+\0??*\059+\060+\089+\010+\124+\0??*\001+\147+\0??]
\0??[\054+\0??+\150+\0??+\0??+\042+\0??+\041+\0??+\061+\063+\099+\091+\092+\009+\0??+\0??+\0??]
\056+\053+\008+\149+\0??+\0??+\0??+\0??+\0??+\058+\111+\110+\108+\098+\012+\123+\0??+\0??+\0??]
\0??[\052+\051+\0??+\136+\0??+\044+\0??+\043+\121+\112+\109+\097+\102+\101+\013+\0??+\0??+\0??]
\138+\055+\133+\137+\135+\0??+\0??+\127+\0??+\0??+\115+\113+\104+\103+\100+\140+\141+\0??+\0??]
\0??[\134+\084+\0??+\0??+\0??+\0??+\0??+\086+\088+\119+\114+\106+\105+\116+\029+\0??+\0??+\0??]
\0??[\0??+\0??+\045+\057+\047+\0??+\087+\085+\125+\120+\118+\107+\117+\122+\0??*\0??+\0??+\0??]
\0??[\0??+\014+\046+\048+\065+\0??+\0??+\0??+\0??+\0??+\0??+\126+\0??+\028+\0??+\011+\0??+\0??]
\0??[\082+\078+\0??+\0??+\080+\0??+\0??+\0??+\0??+\0??+\0??+\0??+\0??+\0??+\032+\033+\0??+\0??]
\0??[\077+\068+\067+\073+\066+\0??+\0??+\0??+\0??+\0??+\0??+\0??+\0??+\0??+\0??+\031+\0??+\0??]
\0??[\075+\069+\070+\071+\0??+\0??+\0??+\0??+\0??+\0??+\040+\0??+\0??+\0??+\018+\0??+\0??+\0??]
\0??[\076+\074+\072+\081+\0??+\0??+\064+\0??+\0??+\0??+\0??+\0??+\022+\0??+\0??+\006+\0??+\0??]
\0??[\079+\005+\083+\007+\129+\0??+\027+\038+\0??*\0??+\0??+\023+\019+\0??+\024+\017+\020+\0??]
\0??[\0??+\0??+\0??+\0??+\0??+\035+\039+\128+\030+\0??+\0??+\132+\139+\0??+\003+\025+\021+\0??]
\0??[\0??+\0??+\0??+\0??+\037+\036+\034+\0??+\0??+\0??+\0??+\130+\131+\0??+\0??+\0??+\0??+\0??]
\0??,\0??)\0??)\0??)\0??)\0??)\0??)\0??)\0??)\0??)\0??)\0??)\0??)\0??)\0??)\0??)\0??)\0??)\0??.
}\hfil\vbox to 273.6 pt{\hsize=109 pt\parindent=0pt\vglue8pt
Left: A classic game with big fonts.
\medskip
\line{Ito Showa 6-Dan (W)\hfil}
\line{Shusaku, 4-Dan (B)\hfil}
\line{October 15, 1844\hfil}\vfil}}
% \bigbreak
% \medbreak 
% {\bf Above:} a classic game using fonts in a size larger than default. Ito
% Showa, 6-Dan (W) played Shusaku, 4-Dan (B) on October 15, 1844.

%\vfil\eject

\bigbreak
\centerline{\titlett sgf2dg}
\bigbreak

Go games are commonly stored in the ``Smart Go Format.'' For example,
games played on the internet on IGS or NNGS are stored in this format.
Tools such as {\tt xmgt}, {\tt cgoban} or {\tt xgoban} can be used to generate
SGF files using a mouse.

The program {\tt sgf2dg} takes a game in Smart Go Format and
translates it into a \TeX\ file. As long as you have the GOOE
fonts and the macros in {\tt gooemacs.tex} you can {\tt tex} the
resulting file, or edit it and incorporate it into a longer
document. {\tt sgf2dg} is a {\tt perl} script, and you must have
{\tt perl} (version 5.001 or later) installed to run it.  If you
do not have an up-to-date perl, you can obtain it from the
Comprehensive Perl Archive Network (CPAN) at {\tt
http://cpan.org} or from {\tt
http://language.perl.com/info/software.html}.

Included with GOOE/{\tt sgf2dg} is an SGF file titled
{\tt genan-shuwa.sgf}. It is the record of a famous game. The commentary
is not intended to be profound but is included to show how {\tt sgf2dg}
treats comments. After running 

\medbreak
\line{\qquad \tt \$ sgf2dg genan-shuwa.sgf\hfil}
\medbreak\noindent
(or {\tt C> perl sgf2dg genan-shuwa} from the DOS prompt under Windows)
and

\medbreak
\line{\qquad \tt \$ tex genan-shuwa\hfil}

\medbreak\noindent
we obtain files called {\tt genan-shuwa.tex} and {\tt genan-shuwa.dvi}. Here
is the {\it unedited} result of this experiment:

\vfil\eject
{\setGoFonts at 9.5pt    % make fonts slightly smaller so two diagrams fit on a page
%
% Begin material pasted in from genan-shuwa.tex
%
%%%%%%%%%%%%%%%%%%%%%%%%%%%%%%%%%%%%%%%%%%%%%%%%%%%%%%%%%%%%%%%%%%%%%%
%
% This file was created by sgf2dg version 4.198 with the following command line:
%
% /usr/bin/sgf2dg -coords genan-shuwa.sgf -converter TeX -iv -floatControl rllrll
%
% sgf2dg was created by Reid Augustin.  The go fonts, TeX
% macros and TeX programming were designed by Daniel Bump.
%
% More information about the sgf2dg package can be found at:
%
%               http://match.stanford.edu/bump/go.html
%
%%%%%%%%%%%%%%%%%%%%%%%%%%%%%%%%%%%%%%%%%%%%%%%%%%%%%%%%%%%%%%%%%%%%%%
%
% \magnification=1200   % Commented out for inclusion in manual.tex
% goWhiteInk changes from black to white ink, but it's not supported by
%   all output drivers (notably pdftex).  Dg2TeX only uses it for long
%   labels on black stones, so it may not matter...
\def\goWhiteInk#1{\special{color push rgb 1 1 1} {#1} \special{color pop}}%
% goLap is used to overlap a long label on a stone or intersection
\def\goLap#1#2{\setbox0=\hbox{#1} \rlap{#1} \raise 2\goTextAdj\hbox to \wd0{\hss\eightpoint{#2}\hss}}%
% goLapWhite overlaps like goLap, but also changes to white ink for the label
\def\goLapWhite#1#2{\setbox0=\hbox{#1}\rlap{#1}\raise 2\goTextAdj\hbox to \wd0{\hss\eightpoint\goWhiteInk{#2}\hss}}%
% rc places right-hand side coordinates
\def\rc#1{\raise \goTextAdj\hbox to \goIntWd{\kern \goTextAdj\hss\rm#1\hss}}%
% bc places bottom coordinates
\def\bc#1{\hbox to \goIntWd{\hss#1\hss}}%
\lineskip=0pt
\parindent=0pt
% \raggedbottom         % commented out for inclusion in manual.tex allow pages to end short (if next diagram doesn't fit)
% \input gooemacs       % commented out for inclusion in manual.tex (we already did this, don't do it again)
\gool
\newbox\boardBox        % a box to put the board into
\newbox\floatBox
\newdimen\floatWd
\newdimen\floatHt
\newdimen\ftextWd       % width of text alongside float
\newif\iffloatRight     % controls whether to float on left or right
\floatRighttrue         % starting default
\def\floatLeft#1#2{     % text on the right side, float on the left
    \floatRightfalse \float{#1}{#2}
}
\def\floatRight#1#2{    % text on the left side, float on the right
    \floatRighttrue \float{#1}{#2}
}
% from http://www.tug.org/utilities/plain/cseq.html#vss-rp:
\def\hcropmark(#1,#2,#3){% (x,y,width)   line from x,y to width
     \vbox to 0pt{%
          \kern #2\hbox{%
               \kern #1\vrule height 0.1pt width #3%
          }%
          \vss% \vss is often used in a \vbox to 0pt{}.
     }%
     \ifvmode\nointerlineskip\fi%
}
\def\vcropmark(#1,#2,#3){% (x,y,height)   line from x,y to height
     \vbox to 0pt{%
          \kern #2\hbox{%
               \kern #1\vrule height #3 width 0.1pt%
          }%
          \vss%
     }%
     \ifvmode\nointerlineskip\fi%
}
\def\fbox(#1,#2,#3,#4){ %   (x, y, width, height)
    \begingroup
    \hcropmark(#1,#2,#3)
    \dimen0=#2
    \advance\dimen0 #4
    \hcropmark(#1,\dimen0,#3)
    \vcropmark(#1,#2,#4)
    \dimen0=#1
    \advance\dimen0 #3
    \vcropmark(\dimen0,#2,#4)
    \endgroup
}
\def\fbox(#1,#2,#3,#4){}%   (x, y, width, height)  % comment this line out to show outlines around floats
% the float macro
\def\float#1#2{%
    \setbox\floatBox=\vbox{#1}          % insert float into box
    \floatWd=\wd\floatBox               % width of float, add gap between text and float
    \floatHt=\ht\floatBox
    \advance\floatHt \dp\floatBox       % height plus depth of float
    \vskip 0pt plus \floatHt \penalty-60 \vskip 0pt plus -\floatHt% make sure there's enough vertical space for the full diagram
    \ftextWd=\hsize                     % width of text alongside float
    \global\advance\ftextWd -\floatWd   % total width - (float width)
    \iffloatRight \fbox(\ftextWd,0pt,\floatWd,\floatHt)%
    \else         \fbox(0pt,0pt,\floatWd,\floatHt)%
    \fi
    \vbox to 0pt{%
       \iffloatRight
            \moveright\ftextWd% move right to where float should be
       \fi
       \hbox{\tolerance=10000\hbadness=10000%
            \box\floatBox       % place the float - leave boxed to prevent page-breaks in the middle
       }
       \vss
    }
    \advance\floatHt \baselineskip     % add padding below float
    \setbox\floatBox=\vbox{%                % box for text
        \tolerance=10000\hbadness=10000
        \advance\floatWd 3em                % add gap to float width
        \global\advance\ftextWd -3em        % and remove it from text width
        \iffloatRight \hangindent -\floatWd % indent right side
        \else         \hangindent  \floatWd % indent left side
        \fi
        \hangafter=\floatHt
        \advance \hangafter \baselineskip    % round up to nearest line
        \advance \hangafter -1               % round up to nearest line
        \divide\hangafter -\baselineskip
        \vskip\goIntHt \vskip -\goTextAdj \noindent#2%            % place the text
    }
    \dimen255=\ht\floatBox \advance\dimen255 \dp\floatBox  % text height
    \unvbox\floatBox
    \ifdim\dimen255 < \floatHt       % is text shorter than float height?
        \advance\floatHt -\dimen255  % float height minus text height
        \kern\floatHt                % get down to the bottom of the float
    \fi
}
%%%%%%%%%%%%%%%%%%%%%%%%%%%%%%%%%%%
%  Start of Diagram 1: 1-50
%%%%%%%%%%%%%%%%%%%%%%%%%%%%%%%%%%%
\floatRight{\setbox\boardBox\vbox{\goo
\hbox{\0??<\0??(\0??(\0??(\0??(\0??(\0??(\0??(\0??(\0??(\0??(\0??(\0??(\0??(\0??(\0??(\0??(\0??(\0??>\rc{19}}
\hbox{\0??[\0??+\0??+\0??+\0??+\0??+\0??+\0??+\0??+\0??+\0??+\0??+\0??+\0??+\0??+\0??+\0??+\0??+\0??]\rc{18}}
\hbox{\0??[\0??+\0??+\045+\003+\019+\018+\0??+\0??+\0??+\0??+\0??+\0??+\0??+\002+\008+\010+\0??+\0??]\rc{17}}
\hbox{\0??[\015+\013+\044*\0??+\0??+\020+\0??+\0??+\0??*\0??+\0??+\0??+\006+\005+\0??*\001+\0??+\0??]\rc{16}}
\hbox{\0??[\0??+\014+\012+\021+\028+\0??+\0??+\0??+\0??+\0??+\0??+\0??+\0??+\007+\0??+\009+\0??+\0??]\rc{15}}
\hbox{\0??[\0??+\0??+\0??+\029+\030+\0??+\0??+\0??+\0??+\0??+\0??+\016+\0??+\017+\004+\0??+\0??+\0??]\rc{14}}
\hbox{\0??[\0??+\050+\0??+\031+\032+\0??+\0??+\0??+\0??+\0??+\0??+\0??+\0??+\0??+\0??+\0??+\0??+\0??]\rc{13}}
\hbox{\0??[\0??+\049+\026+\033+\034+\0??+\0??+\0??+\0??+\0??+\0??+\0??+\0??+\0??+\0??+\0??+\0??+\0??]\rc{12}}
\hbox{\0??[\0??+\025+\024+\035+\039+\043+\0??+\0??+\0??+\0??+\0??+\0??+\0??+\0??+\0??+\011+\0??+\0??]\rc{11}}
\hbox{\0??[\0??+\023+\036*\038+\042+\0??+\0??+\0??+\0??*\0??+\0??+\0??+\0??+\0??+\0??*\0??+\0??+\0??]\rc{10}}
\hbox{\0??[\048+\046+\037+\041+\0??+\0??+\0??+\0??+\0??+\0??+\0??+\0??+\0??+\0??+\0??+\0??+\0??+\0??]\rc{9}}
\hbox{\0??[\0??+\0??+\040+\047+\0??+\0??+\0??+\0??+\0??+\0??+\0??+\0??+\0??+\0??+\0??+\0??+\0??+\0??]\rc{8}}
\hbox{\0??[\0??+\027+\0??+\0??+\0??+\0??+\0??+\0??+\0??+\0??+\0??+\0??+\0??+\0??+\0??+\0??+\0??+\0??]\rc{7}}
\hbox{\0??[\0??+\0??+\0??+\0??+\0??+\0??+\0??+\0??+\0??+\0??+\0??+\0??+\0??+\0??+\0??+\0??+\0??+\0??]\rc{6}}
\hbox{\0??[\0??+\0??+\022+\0??+\0??+\0??+\0??+\0??+\0??+\0??+\0??+\0??+\0??+\0??+\0??+\0??+\0??+\0??]\rc{5}}
\hbox{\0??[\0??+\0??+\0??*\0??+\0??+\0??+\0??+\0??+\0??*\0??+\0??+\0??+\0??+\0??+\0??*\0??+\0??+\0??]\rc{4}}
\hbox{\0??[\0??+\0??+\0??+\0??+\0??+\0??+\0??+\0??+\0??+\0??+\0??+\0??+\0??+\0??+\0??+\0??+\0??+\0??]\rc{3}}
\hbox{\0??[\0??+\0??+\0??+\0??+\0??+\0??+\0??+\0??+\0??+\0??+\0??+\0??+\0??+\0??+\0??+\0??+\0??+\0??]\rc{2}}
\hbox{\0??,\0??)\0??)\0??)\0??)\0??)\0??)\0??)\0??)\0??)\0??)\0??)\0??)\0??)\0??)\0??)\0??)\0??)\0??.\rc{1}}
\vskip 4pt
\hbox{\bc A\bc B\bc C\bc D\bc E\bc F\bc G\bc H\bc J\bc K\bc L\bc M\bc N\bc O\bc P\bc Q\bc R\bc S\bc T}}\smallskip
\nobreak\vbox{\hsize=\wd\boardBox
\box\boardBox\rm
{\centerline{{\bf Diagram 1}: 1-50}}
}}
{\noindent
May 16-18, 1842\hfil\break
{\bf White:} Genan \hfil\break
{\bf Black:} Shuwa \hfil\break
\hfil\break
\hfil\break
}
%%%%%%%%%%%%%%%%%%%%%%%%%%%%%%%%%%%
%  Start of Diagram 2: 51-100
%%%%%%%%%%%%%%%%%%%%%%%%%%%%%%%%%%%
\floatLeft{\setbox\boardBox\vbox{\goo
\hbox{\0??<\0??(\0??(\0??(\0??(\0??(\0??(\0??(\0??(\0??(\0??(\0??(\0??(\0??(\0??(\0??(\0??(\0??(\0??>\rc{19}}
\hbox{\0??[\0??+\0??+\0??+\0??+\0??+\0??+\0??+\0??+\0??+\0??+\0??+\0??+\0??+\0??+\0??+\0??+\098+\0??]\rc{18}}
\hbox{\0??[\0??+\0??+\- @+\- @+\- @+\- !+\084+\0??+\0??+\0??+\0??+\0??+\0??+\- !+\- !+\- !+\097+\0??]\rc{17}}
\hbox{\0??[\- @+\- @+\- !*\057+\0??+\- !+\083+\0??+\082*\090+\0??+\088+\- !+\- @+\0??*\- @+\0??+\0??]\rc{16}}
\hbox{\0??[\0??+\- !+\- !+\- @+\- !+\0??+\0??+\079+\094+\093+\089+\085+\087+\- @+\0??+\- @+\0??+\0??]\rc{15}}
\hbox{\0??[\0??+\056+\0??+\- @+\- !+\0??+\081+\080+\066+\091+\086+\- !+\0??+\- @+\- !+\0??+\0??+\0??]\rc{14}}
\hbox{\0??[\055+\- !+\051+\- @+\- !+\0??+\071+\0??+\095+\092+\0??+\0??+\0??+\0??+\0??+\0??+\0??+\0??]\rc{13}}
\hbox{\0??[\0??+\- @+\- !+\- @+\- !+\0??+\0??+\0??+\067+\096+\0??+\0??+\0??+\0??+\0??+\074+\099+\0??]\rc{12}}
\hbox{\0??[\0??+\- @+\- !+\- @+\- @+\- @+\063+\0??+\0??+\0??+\0??+\0??+\0??+\070+\076+\- @+\078+\0??]\rc{11}}
\hbox{\0??[\0??+\- @+\- !*\- !+\- !+\054+\062+\064+\0??*\068+\0??+\0??+\0??+\0??+\075*\072+\073+\0??]\rc{10}}
\hbox{\0??[\- !+\- !+\- @+\- @+\053+\0??+\0??+\0??+\0??+\0??+\0??+\0??+\0??+\0??+\0??+\077+\0??+\0??]\rc{9}}
\hbox{\0??[\0??+\0??+\- !+\- @+\052+\061+\0??+\065+\0??+\0??+\0??+\0??+\0??+\0??+\0??+\0??+\0??+\0??]\rc{8}}
\hbox{\0??[\0??+\- @+\058+\059+\0??+\0??+\0??+\0??+\0??+\0??+\0??+\0??+\0??+\0??+\0??+\0??+\0??+\0??]\rc{7}}
\hbox{\0??[\0??+\0??+\060+\0??+\0??+\069+\0??+\0??+\0??+\0??+\0??+\0??+\0??+\0??+\0??+\0??+\0??+\0??]\rc{6}}
\hbox{\0??[\0??+\0??+\- !+\0??+\0??+\0??+\0??+\0??+\0??+\0??+\0??+\0??+\0??+\0??+\0??+\0??+\0??+\0??]\rc{5}}
\hbox{\0??[\0??+\0??+\0??*\0??+\0??+\0??+\0??+\0??+\0??*\0??+\0??+\0??+\0??+\0??+\0??*\0??+\0??+\0??]\rc{4}}
\hbox{\0??[\0??+\0??+\0??+\0??+\0??+\0??+\0??+\0??+\0??+\0??+\0??+\0??+\0??+\0??+\0??+\0??+\0??+\0??]\rc{3}}
\hbox{\0??[\0??+\0??+\0??+\0??+\0??+\0??+\0??+\0??+\0??+\0??+\0??+\0??+\0??+\0??+\0??+\0??+\0??+\0??]\rc{2}}
\hbox{\0??,\0??)\0??)\0??)\0??)\0??)\0??)\0??)\0??)\0??)\0??)\0??)\0??)\0??)\0??)\0??)\0??)\0??)\0??.\rc{1}}
\vskip 4pt
\hbox{\bc A\bc B\bc C\bc D\bc E\bc F\bc G\bc H\bc J\bc K\bc L\bc M\bc N\bc O\bc P\bc Q\bc R\bc S\bc T}}\smallskip
\nobreak\vbox{\hsize=\wd\boardBox
\box\boardBox\rm
{\centerline{{\bf Diagram 2}: 51-100}}
}}
{\noindent
\nobreak\vbox{\hsize=\ftextWd \baselineskip=\goIntHt \advance\baselineskip 3pt\noindent\iffloatRight\hfil\fi
 \textstone{\goo\100=} at  \textstone{\goo\072=}\iffloatRight\else\hfil\fi\break\vskip -8pt}
}
%%%%%%%%%%%%%%%%%%%%%%%%%%%%%%%%%%%
%  Start of Diagram 3: 101-150
%%%%%%%%%%%%%%%%%%%%%%%%%%%%%%%%%%%
\floatLeft{\setbox\boardBox\vbox{\goo
\hbox{\0??<\0??(\0??(\0??(\0??(\0??(\0??(\0??(\0??(\0??(\0??(\0??(\0??(\0??(\0??(\0??(\0??(\0??(\0??>\rc{19}}
\hbox{\0??[\0??+\0??+\0??+\0??+\0??+\0??+\0??+\133+\140+\0??+\121+\114+\0??+\0??+\0??+\0??+\- !+\0??]\rc{18}}
\hbox{\0??[\0??+\0??+\- @+\- @+\- @+\- !+\- !+\127+\128+\113+\102+\107+\108+\- !+\- !+\- !+\- @+\0??]\rc{17}}
\hbox{\0??[\- @+\- @+\- !*\- @+\0??+\- !+\- @+\134+\- !*\- !+\101+\- !+\- !+\- @+\0??*\- @+\0??+\0??]\rc{16}}
\hbox{\0??[\0??+\- !+\- !+\- @+\- !+\0??+\139+\- @+\- !+\- @+\- @+\- @+\- @+\- @+\0??+\- @+\0??+\0??]\rc{15}}
\hbox{\116[\110+\- !+\0??+\- @+\- !+\0??+\- @+\- !+\- !+\- @+\- !+\- !+\0??+\- @+\- !+\0??+\0??+\0??]\rc{14}}
\hbox{\124[\- @+\- !+\- @+\- @+\- !+\0??+\- @+\0??+\- @+\- !+\0??+\0??+\0??+\0??+\0??+\0??+\0??+\0??]\rc{13}}
\hbox{\111[\104+\- @+\- !+\- @+\- !+\0??+\0??+\0??+\- @+\- !+\0??+\142+\0??+\0??+\0??+\- !+\- @+\0??]\rc{12}}
\hbox{\0??[\105+\- @+\- !+\- @+\- @+\- @+\- @+\0??+\131+\119+\130+\0??+\0??+\- !+\- !+\103+\- !+\143]\rc{11}}
\hbox{\0??[\0??+\- @+\- !*\- !+\- !+\- !+\- !+\- !+\148*\- !+\125+\136+\0??+\144+\- @*\- t+\- @+\0??]\rc{10}}
\hbox{\0??[\- !+\- !+\- @+\- @+\- @+\0??+\0??+\0??+\0??+\0??+\137+\146+\0??+\145+\0??+\- @+\0??+\0??]\rc{9}}
\hbox{\0??[\0??+\117+\- !+\- @+\- !+\- @+\0??+\- @+\0??+\0??+\149+\0??+\0??+\150+\0??+\0??+\0??+\0??]\rc{8}}
\hbox{\0??[\118+\- @+\- !+\- @+\0??+\0??+\0??+\0??+\0??+\0??+\0??+\0??+\0??+\0??+\0??+\0??+\0??+\0??]\rc{7}}
\hbox{\0??[\0??+\0??+\- !+\0??+\0??+\- @+\0??+\0??+\0??+\0??+\0??+\0??+\0??+\0??+\0??+\0??+\0??+\0??]\rc{6}}
\hbox{\0??[\0??+\0??+\- !+\0??+\0??+\0??+\0??+\0??+\0??+\0??+\0??+\0??+\0??+\0??+\0??+\0??+\0??+\0??]\rc{5}}
\hbox{\0??[\0??+\0??+\0??*\0??+\0??+\0??+\0??+\0??+\0??*\0??+\0??+\0??+\0??+\0??+\0??*\0??+\0??+\0??]\rc{4}}
\hbox{\0??[\0??+\0??+\0??+\0??+\0??+\0??+\0??+\0??+\0??+\0??+\0??+\0??+\0??+\0??+\147+\0??+\0??+\0??]\rc{3}}
\hbox{\0??[\0??+\0??+\0??+\0??+\0??+\0??+\0??+\0??+\0??+\0??+\0??+\0??+\0??+\0??+\0??+\0??+\0??+\0??]\rc{2}}
\hbox{\0??,\0??)\0??)\0??)\0??)\0??)\0??)\0??)\0??)\0??)\0??)\0??)\0??)\0??)\0??)\0??)\0??)\0??)\0??.\rc{1}}
\vskip 4pt
\hbox{\bc A\bc B\bc C\bc D\bc E\bc F\bc G\bc H\bc J\bc K\bc L\bc M\bc N\bc O\bc P\bc Q\bc R\bc S\bc T}}\smallskip
\nobreak\vbox{\hsize=\wd\boardBox
\box\boardBox\rm
{\centerline{{\bf Diagram 3}: 101-150}}
}}
{\noindent
\nobreak\vbox{\hsize=\ftextWd \baselineskip=\goIntHt \advance\baselineskip 3pt\noindent\iffloatRight\hfil\fi
 \textstone{\goo\106=},  \textstone{\goo\112=},  \textstone{\goo\120=},  \textstone{\goo\126=},  \textstone{\goo\132=},  \textstone{\goo\138=} at  \textstone{\goo\- tt}\iffloatRight\else\hfil\fi\break\vskip -8pt}
\nobreak\vbox{\hsize=\ftextWd \baselineskip=\goIntHt \advance\baselineskip 3pt\noindent\iffloatRight\hfil\fi
 \textstone{\goo\109=},  \textstone{\goo\115=},  \textstone{\goo\123=},  \textstone{\goo\129=},  \textstone{\goo\135=},  \textstone{\goo\141=} at  \textstone{\goo\103=}\iffloatRight\else\hfil\fi\break\vskip -8pt}
\nobreak\vbox{\hsize=\ftextWd \baselineskip=\goIntHt \advance\baselineskip 3pt\noindent\iffloatRight\hfil\fi
 \textstone{\goo\122=} at  \textstone{\goo\107=}\iffloatRight\else\hfil\fi\break\vskip -8pt}
124: Now the situation becomes complicated beyond the point of mortal understanding, since in addition to the ko on the right, White is trying to revive his dead stones on the left. If they live, White does not need to worry about eyes in the center.\hfil\break
}
%%%%%%%%%%%%%%%%%%%%%%%%%%%%%%%%%%%
%  Start of Diagram 4: 151-200
%%%%%%%%%%%%%%%%%%%%%%%%%%%%%%%%%%%
\floatRight{\setbox\boardBox\vbox{\goo
\hbox{\0??<\0??(\0??(\0??(\0??(\0??(\0??(\0??(\0??(\0??(\0??(\0??(\0??(\0??(\0??(\0??(\0??(\0??(\0??>\rc{19}}
\hbox{\196[\193+\0??+\0??+\0??+\0??+\0??+\0??+\- @+\- !+\0??+\- @+\- !+\0??+\0??+\0??+\0??+\- !+\0??]\rc{18}}
\hbox{\197[\186+\187+\- @+\- @+\- @+\- !+\- !+\- @+\- !+\- @+\- !+\- !+\- !+\- !+\- !+\- !+\- @+\0??]\rc{17}}
\hbox{\192[\- @+\- @+\- !*\- @+\0??+\- !+\- @+\- !+\- !*\- !+\- @+\- !+\- !+\- @+\0??*\- @+\0??+\0??]\rc{16}}
\hbox{\0??[\0??+\- !+\- !+\- @+\- !+\0??+\- @+\- @+\- !+\- @+\- @+\- @+\- @+\- @+\0??+\- @+\0??+\0??]\rc{15}}
\hbox{\- ![\- !+\- !+\0??+\- @+\- !+\0??+\- @+\- !+\- !+\- @+\- !+\- !+\0??+\- @+\- !+\0??+\0??+\0??]\rc{14}}
\hbox{\- ![\- @+\- !+\- @+\- @+\- !+\0??+\- @+\0??+\- @+\- !+\0??+\0??+\0??+\0??+\0??+\0??+\0??+\0??]\rc{13}}
\hbox{\- @[\154+\- @+\- !+\- @+\- !+\0??+\0??+\0??+\- @+\- !+\0??+\- !+\0??+\0??+\0??+\- !+\- @+\0??]\rc{12}}
\hbox{\188[\- @+\- @+\- !+\- @+\- @+\- @+\- @+\0??+\- @+\- @+\- !+\0??+\0??+\- !+\- !+\- @+\0??+\- @]\rc{11}}
\hbox{\0??[\0??+\- @+\- !*\- !+\- !+\- !+\- !+\- !+\- !*\- !+\- @+\- !+\0??+\- !+\- @*\0??+\- @+\0??]\rc{10}}
\hbox{\0??[\- !+\- !+\- @+\- @+\- @+\0??+\0??+\0??+\0??+\0??+\- @+\- !+\156+\- @+\175+\- @+\0??+\0??]\rc{9}}
\hbox{\0??[\0??+\- @+\- !+\- @+\- !+\- @+\0??+\- @+\0??+\0??+\- @+\0??+\0??+\- !+\199+\0??+\0??+\0??]\rc{8}}
\hbox{\0??[\- !+\- @+\- !+\- @+\0??+\0??+\0??+\0??+\0??+\0??+\173+\0??+\176+\0??+\183+\0??+\0??+\0??]\rc{7}}
\hbox{\0??[\0??+\0??+\- !+\0??+\0??+\- @+\0??+\0??+\0??+\179+\172+\157+\0??+\184+\185+\0??+\0??+\0??]\rc{6}}
\hbox{\0??[\0??+\0??+\- !+\0??+\0??+\0??+\0??+\0??+\180+\178+\177+\0??+\0??+\198+\0??+\0??+\0??+\0??]\rc{5}}
\hbox{\0??[\0??+\0??+\0??*\0??+\0??+\155+\160+\0??+\0??*\0??+\164+\153+\167+\0??+\0??*\0??+\0??+\0??]\rc{4}}
\hbox{\0??[\0??+\152+\0??+\151+\159+\158+\161+\0??+\174+\168+\0??+\162+\163+\0??+\- @+\0??+\0??+\0??]\rc{3}}
\hbox{\0??[\194+\191+\195+\0??+\0??+\165+\0??+\189+\171+\190+\170+\182+\166+\169+\0??+\0??+\0??+\0??]\rc{2}}
\hbox{\0??,\0??)\0??)\0??)\0??)\0??)\0??)\0??)\0??)\0??)\0??)\0??)\0??)\181)\0??)\0??)\0??)\0??)\0??.\rc{1}}
\vskip 4pt
\hbox{\bc A\bc B\bc C\bc D\bc E\bc F\bc G\bc H\bc J\bc K\bc L\bc M\bc N\bc O\bc P\bc Q\bc R\bc S\bc T}}\smallskip
\nobreak\vbox{\hsize=\wd\boardBox
\box\boardBox\rm
{\centerline{{\bf Diagram 4}: 151-200}}
}}
{\noindent
\nobreak\vbox{\hsize=\ftextWd \baselineskip=\goIntHt \advance\baselineskip 3pt\noindent\iffloatRight\hfil\fi
 \textstone{\goo\200=} at  \textstone{\goo\186=}\iffloatRight\else\hfil\fi\break\vskip -8pt}
165: As is traditional, the captured stone is removed from the board.\hfil\break
188: At long last, White takes the chance to eliminate the ko\hfil\break
aji.\hfil\break
}
%%%%%%%%%%%%%%%%%%%%%%%%%%%%%%%%%%%
%  Start of Diagram 5: 201-250
%%%%%%%%%%%%%%%%%%%%%%%%%%%%%%%%%%%
\floatLeft{\setbox\boardBox\vbox{\goo
\hbox{\0??<\0??(\201(\0??(\0??(\215(\0??(\0??(\0??(\0??(\0??(\0??(\0??(\0??(\0??(\0??(\0??(\0??(\0??>\rc{19}}
\hbox{\- ![\- @+\0??+\0??+\211+\206+\213+\214+\- @+\- !+\0??+\- @+\- !+\0??+\0??+\0??+\0??+\- !+\0??]\rc{18}}
\hbox{\207[\- t+\- @+\- @+\- @+\- @+\- !+\- !+\- @+\- !+\- @+\- !+\- !+\- !+\- !+\- !+\- !+\- @+\218]\rc{17}}
\hbox{\- ![\- @+\- @+\- !*\- @+\0??+\- !+\- @+\- !+\- !*\- !+\- @+\- !+\- !+\- @+\0??*\- @+\216+\0??]\rc{16}}
\hbox{\0??[\0??+\- !+\- !+\- @+\- !+\0??+\- @+\- @+\- !+\- @+\- @+\- @+\- @+\- @+\0??+\- @+\217+\249]\rc{15}}
\hbox{\- ![\- !+\- !+\0??+\- @+\- !+\0??+\- @+\- !+\- !+\- @+\- !+\- !+\0??+\- @+\- !+\232+\233+\0??]\rc{14}}
\hbox{\- ![\0??+\- !+\- @+\- @+\- !+\0??+\- @+\0??+\- @+\- !+\0??+\0??+\0??+\203+\0??+\202+\205+\0??]\rc{13}}
\hbox{\0??[\- !+\- @+\- !+\- @+\- !+\0??+\0??+\0??+\- @+\- !+\0??+\- !+\0??+\0??+\204+\- !+\- @+\0??]\rc{12}}
\hbox{\- ![\- @+\- @+\- !+\- @+\- @+\- @+\- @+\0??+\- @+\- @+\- !+\0??+\0??+\- !+\- !+\- @+\0??+\- @]\rc{11}}
\hbox{\0??[\0??+\- @+\- !*\- !+\- !+\- !+\- !+\- !+\- !*\- !+\- @+\- !+\0??+\- !+\- @*\0??+\- @+\0??]\rc{10}}
\hbox{\0??[\- !+\- !+\- @+\- @+\- @+\0??+\0??+\0??+\250+\231+\- @+\- !+\- !+\- @+\- @+\- @+\0??+\0??]\rc{9}}
\hbox{\0??[\0??+\- @+\- !+\- @+\- !+\- @+\0??+\- @+\0??+\0??+\- @+\0??+\0??+\- !+\- @+\0??+\0??+\0??]\rc{8}}
\hbox{\0??[\- !+\- @+\- !+\- @+\0??+\0??+\0??+\0??+\235+\0??+\- @+\0??+\- !+\0??+\- @+\0??+\0??+\0??]\rc{7}}
\hbox{\0??[\0??+\0??+\- !+\0??+\227+\- @+\0??+\225+\234+\- @+\0??+\- @+\212+\- !+\- @+\0??+\243+\0??]\rc{6}}
\hbox{\0??[\0??+\222+\- !+\228+\226+\0??+\229+\0??+\- !+\- !+\- @+\0??+\0??+\- !+\239+\238+\244+\245]\rc{5}}
\hbox{\0??[\220+\219+\221*\0??+\0??+\- @+\- !+\223+\224*\0??+\- !+\- @+\- @+\208+\241*\0??+\0??+\0??]\rc{4}}
\hbox{\0??[\0??+\- !+\0??+\- @+\- @+\0??+\- @+\0??+\- !+\- !+\0??+\- !+\- @+\209+\- @+\240+\0??+\0??]\rc{3}}
\hbox{\0??[\- !+\- @+\- @+\0??+\0??+\- @+\0??+\- @+\- @+\- !+\- !+\- !+\- !+\- @+\0??+\242+\0??+\0??]\rc{2}}
\hbox{\0??,\248)\246)\247)\0??)\0??)\0??)\0??)\0??)\0??)\0??)\0??)\0??)\- @)\237)\236)\0??)\0??)\0??.\rc{1}}
\vskip 4pt
\hbox{\bc A\bc B\bc C\bc D\bc E\bc F\bc G\bc H\bc J\bc K\bc L\bc M\bc N\bc O\bc P\bc Q\bc R\bc S\bc T}}\smallskip
\nobreak\vbox{\hsize=\wd\boardBox
\box\boardBox\rm
{\centerline{{\bf Diagram 5}: 201-250}}
}}
{\noindent
\nobreak\vbox{\hsize=\ftextWd \baselineskip=\goIntHt \advance\baselineskip 3pt\noindent\iffloatRight\hfil\fi
 \textstone{\goo\210=} at  \textstone{\goo\- tt}\iffloatRight\else\hfil\fi\break\vskip -8pt}
\nobreak\vbox{\hsize=\ftextWd \baselineskip=\goIntHt \advance\baselineskip 3pt\noindent\iffloatRight\hfil\fi
 \textstone{\goo\230=} at  \textstone{\goo\207=}\iffloatRight\else\hfil\fi\break\vskip -8pt}
245: White is dead in the corner. It is unknown whether Genan knew the invasion was doomed, or whether he made a misreading. According to one theory, Genan knew that he was slightly behind and invaded hoping for a Black mistake. Of course, the failed invasion made Black's lead larger, and after a few more moves, Black won by 6 points.\hfil\break
}
%%%%%%%%%%%%%%%%%%%%%%%%%%%%%%%%%%%
%  Start of Diagram 6: 251-270
%%%%%%%%%%%%%%%%%%%%%%%%%%%%%%%%%%%
\floatLeft{\setbox\boardBox\vbox{\goo
\hbox{\265<\264(\- @(\0??(\0??(\- @(\0??(\0??(\0??(\0??(\0??(\0??(\0??(\0??(\0??(\0??(\0??(\0??(\0??>\rc{19}}
\hbox{\- ![\- @+\269+\0??+\- @+\0??+\- @+\- !+\- @+\- !+\0??+\- @+\- !+\0??+\0??+\0??+\0??+\- !+\0??]\rc{18}}
\hbox{\- ![\- !+\- @+\- @+\- @+\- @+\- !+\- !+\- @+\- !+\- @+\- !+\- !+\- !+\- !+\- !+\- !+\262+\- !]\rc{17}}
\hbox{\- ![\- @+\- @+\- !*\- @+\0??+\- !+\- @+\- !+\- !*\- !+\- @+\- !+\- !+\- @+\0??*\- @+\- !+\261]\rc{16}}
\hbox{\270[\0??+\- !+\- !+\- @+\- !+\0??+\- @+\- @+\- !+\- @+\- @+\- @+\- @+\- @+\0??+\- @+\- @+\- @]\rc{15}}
\hbox{\- ![\- !+\- !+\0??+\- @+\- !+\0??+\- @+\- !+\- !+\- @+\- !+\- !+\0??+\- @+\- !+\- !+\- @+\0??]\rc{14}}
\hbox{\- ![\0??+\- !+\- @+\- @+\- !+\0??+\- @+\0??+\- @+\- !+\0??+\0??+\0??+\- @+\0??+\- !+\- @+\0??]\rc{13}}
\hbox{\0??[\- !+\- @+\- !+\- @+\- !+\0??+\0??+\0??+\- @+\- !+\0??+\- !+\0??+\0??+\- !+\- !+\- @+\0??]\rc{12}}
\hbox{\- ![\- @+\- @+\- !+\- @+\- @+\- @+\- @+\0??+\- @+\- @+\- !+\0??+\0??+\- !+\- !+\- @+\0??+\- @]\rc{11}}
\hbox{\0??[\0??+\- @+\- !*\- !+\- !+\- !+\- !+\- !+\- !*\- !+\- @+\- !+\0??+\- !+\- @*\0??+\- @+\0??]\rc{10}}
\hbox{\0??[\- !+\- !+\- @+\- @+\- @+\252+\253+\0??+\- !+\- @+\- @+\- !+\- !+\- @+\- @+\- @+\0??+\0??]\rc{9}}
\hbox{\0??[\0??+\- @+\- !+\- @+\- !+\- @+\0??+\- @+\251+\0??+\- @+\0??+\0??+\- !+\- @+\0??+\0??+\0??]\rc{8}}
\hbox{\0??[\- !+\- @+\- !+\- @+\255+\0??+\0??+\0??+\- @+\0??+\- @+\0??+\- !+\0??+\- @+\0??+\0??+\0??]\rc{7}}
\hbox{\0??[\0??+\0??+\- !+\254+\- @+\- @+\0??+\- @+\- !+\- @+\0??+\- @+\- !+\- !+\- @+\0??+\- @+\0??]\rc{6}}
\hbox{\0??[\0??+\- !+\- !+\- !+\- !+\0??+\- @+\0??+\- !+\- !+\- @+\257+\256+\- !+\- @+\- !+\- !+\- @]\rc{5}}
\hbox{\0??[\- !+\- @+\- @*\0??+\0??+\- @+\0??+\- @+\- !*\0??+\- !+\- @+\- @+\- !+\- @*\267+\0??+\266]\rc{4}}
\hbox{\0??[\0??+\- !+\0??+\- @+\- @+\0??+\- @+\0??+\- !+\- !+\0??+\- !+\- @+\- @+\- @+\- !+\0??+\0??]\rc{3}}
\hbox{\0??[\- !+\- @+\- @+\0??+\0??+\- @+\0??+\- @+\- @+\- !+\- !+\- !+\- !+\- @+\0??+\- !+\0??+\0??]\rc{2}}
\hbox{\0??,\- !)\- !)\- @)\0??)\0??)\0??)\0??)\259)\258)\260)\0??)\263)\- @)\- @)\- !)\0??)\0??)\0??.\rc{1}}
\vskip 4pt
\hbox{\bc A\bc B\bc C\bc D\bc E\bc F\bc G\bc H\bc J\bc K\bc L\bc M\bc N\bc O\bc P\bc Q\bc R\bc S\bc T}}\smallskip
\nobreak\vbox{\hsize=\wd\boardBox
\box\boardBox\rm
{\centerline{{\bf Diagram 6}: 251-270}}
}}
{\noindent
\nobreak\vbox{\hsize=\ftextWd \baselineskip=\goIntHt \advance\baselineskip 3pt\noindent\iffloatRight\hfil\fi
 \textstone{\goo\268=} at  \textstone{\goo\264=}\iffloatRight\else\hfil\fi\break\vskip -8pt}
}
% \bye % Final \bye commented out for inclusion in manual.tex
%
% End material pasted in from genan-shuwa.tex
%
}%

This is the end of the included \TeX\ pasted from {\tt genan-shuwa.tex}.
\vfil

\hbox to\hsize{\vbox to 144pt{\hsize=86.4pt{\bgoe
\0??+\- !+\- @+\0??+\- @+\0??]
\- !+\- @+\- @+\- @+\0??+\0??]
\0??+\- !+\- @+\0??+\0??+\0??]
\- !+\0??+\- @+\0??+\0??+\0??]
\- !+\- !+\- @+\0??+\- @+\0??]
\0??+\- !+\- @+\- !+\- !+\- @]
\- @+\- !+\- @+\002+\003+\001]
\- @+\- @+\- @+\- !+\0??+\006]
\- !+\- @+\0??+\- !+\005+\004]
\- @)\- @)\- !)\008)\0??)\007.
}\vglue6pt 
\centerline{\bf Diagram 7}
}\hfil
\vbox to 144pt{\hsize=3.75in
For completeness, let us explain why White is dead in the corner. The
following sequence was not played, and is beyond the reading of most
players. It is certain that both players read this out and saw the tesuji of
\textstone{\goe\008=}. What is unknown is whether Genan perhaps saw this
tesuji {\it before} he played at 236, and hoped that Shuwa would miss it. If
\textstone{\goe\008=} is used to capture \textstone{\goe\007=}, then White 
obtains a ko. Perhaps Genan was aware of this, but felt that he was behind,
and hoping for a swindle invaded anyway, then gave up the plan?
\vfil}}
\vglue-.4in
\hangindent=1.666666667in\hangafter-6 {\bf Diagram 7} was not generated 
directly from the game file {\tt genan-shuwa.sgf} since it shows moves not
played in the game. These moves are included in {\tt genan-shuwa.sgf} as
a {\it variation}. Most Smart Go Format editors support variations, which
are discussed below in a separate section. The variation diagram as 
produced by {\tt sgf2dg} is labelled ``{\bf Variation 1} on Move 245.''
We did not wish to use it unedited. Rather, we reran {\tt sgf2dg}
with the options {\tt -n -break 246,254 \hbox{-l 14} \hbox{-t 10} -bigFonts
-simple}, producing output which we edited into the current document to
make {\bf Diagram~7}.

\vfil\eject
\centerline{\titlefont Command Line Options}
\bigbreak

We now describe the options which come with {\tt sgf2dg}. First, 
{\tt sgf2dg -h} (or {\tt sgf2dg -h|more}) prints a help message,
then exits. (Under Windows, use the command {\tt perl sgf2dg -h}
instead, or more generally 
{\tt perl sgf2dg} [{\it options}] [{\it filename}].)

\bigbreak
{\tt\parskip=0pt\parindent=5pt\obeylines\obeyspaces\hsize=400pt
\def\[{{\char'173}}
\def\]{{\char'175}}
sgf2dg [options] [file.sgf]
\medskip
\ -h | -help                 print this message and exit
\ -v | -version              print version number and exit
\ -verbose                   print diagnostic information
\ -i | -in                   input file name (STDIN for standard input)
\ -o | -out                  output file name (STDOUT for standard output)
\ -t | -top                  top line in diagram
\ -b | -bottom               bottom line in diagram
\ -l | -left                 leftmost line in diagram
\ -r | -right                rightmost line in diagram
\ -crop                      auto-crop diagram
\ -break | -breakList        a list of first move in each diagram
\ -m | -movesPerDiagram      number of moves per diagram
\ -d | -doubleDigits         label stones modulo 100
\ -n | -newNumbers           begin each diagram with number 1
\ -rv | -relativeVarNums     start each variation from 1
\ -av | -absoluteVarNums     move numbers in variation == main line numbers
\ -cv | -correlativeVarNums  start main variations from 1 
\ -rl | -repeatLast          repeat last move as first in next diagram
\ -ic | -ignoreComments      ignore SGF comments
\ -il | -ignoreLetters       ignore SGF letters
\ -im | -ignoreMarks         ignore SGF marks
\ -iv | -ignoreVariations    ignore SGF variations
\ -ip | -ignorePass          ignore SGF pass moves
\ -ia | -ignoreAll           ignore SGF letters, marks, variations, passes
\ -firstDiagram              first diagram to print
\ -lastDiagram               last diagram to print
\ -placeHandi                place handicap stones on board (old style)
\ -coords                    print coordinates
\ -cs | -coordStyle style    coordinate style: normal, sgf, or numeric
\ -c | -convert | -converter name of a Diagram converter (see below)
\ -simple                    use a very simple TeX format
\ -mag number                TeX \\magnification - default is 1000
\ -twoColumn                 use two-column format
\ -bigFonts                  use fonts magnified 1.2 times
\ -texComments               \\, \[ and \] in comments not modified
\ -floatControl string       'string' controls float (diagram) placement
}
\bigbreak
The help message continues with some more explanation of each option.
We will discuss the options in more detail now.

{\tt -verbose} is useful for tracking down sgf2dg and converter decisions
about diagram breaks.

The {\tt -i} and {\tt -o} options are not needed with normal usage.
The usual usage is {\tt sgf2dg {\rm [{\it options}]} filename} instead of
{\tt sgf2dg {\rm [{\it options}]} -i filename}. If
a file named {\tt filename.sgf} or just {\tt filename} exists, it
will create a \TeX\ file called {\tt filename.tex}. 

In its default usage {\tt sgf2dg} produces full-board diagrams.
You may specify the top, bottom, left and right edges with the
{\tt -t}, {\tt -b}, {\tt -l} and {\tt -r} tags. For example,
{\tt \hbox{-l 14} \hbox{-t 10}} produces a diagram in which
has its left edge at 14, its right edge at 19 (the default),
the top edge at 10, and the bottom edge at 19 (the default).
These settings were used in {\bf Diagram~7}.

{\tt -crop} trims empty lines from the diagrams.  At least two lines around
each stone are not cropped, and if the board edge is within three lines of
a stone, it is not cropped.

Generally one does not want an entire game in a single diagram. {\tt
-breakList} and {\tt -movesPerDiagram} are mechanisms for controlling the
number of moves in the diagrams.

The {\tt -breakList <{\it list\/}>} option, where {\tt<{\it list\/}>} is a list of
moves separated by commas, will produce a sequence of diagrams broken at the
given moves.

If {\tt -movesPerDiagram} {\tt<{\it N\/}>} is used, where {\tt<{\it N\/}>}
is an integer, then {\tt sgf2dg} will break the diagram after each
{\tt<}{\it N\/}{\tt>} moves. If neither {\tt -breakList} nor {\tt
-movesPerDiagram} is specified, the {\tt -movesPerDiagram} option is
understood, with {\tt<{\it N\/}>} set by default to 50. (It can be disabled
by setting {\tt -movesPerDiagram 10000}.) On the other hand if {\tt
-breakList} is used, then {\tt<{\it N\/}>} defaults to infinity. The
options {\tt -breakList} and {\tt -movesPerDiagram} can be used together.

{\tt -doubleDigits} and {\tt -newNumbers} are alternative schemes for
keeping the numbers on the stones small. If {\tt -doubleDigits} is set,
then the stones are numbered modulo $100$, or more precisely, the first
move in the diagram is reduced modulo $100$ to a number from $1$ to
$99$.

If {\tt -newNumbers} is set, each diagram begins numbering afresh with the
numeral $1$. Both schemes emulate strategies commonly used in commenting
games.

There are circumstances where {\tt sgf2dg} will automatically break
a diagram at a point not specified by the user. When a stone is placed
on a spot where there was previously a stone (as, for example, in a
ko fight), {\tt sgf2dg} marks the first stone if it is unnumbered
(i.e.\ was played on a previous diagram). Then it adds a comment
of the form ``\textstone{\goo\067=} at \textstone{\goo\055=}'' or
``\textstone{\goo\067=} at \textstone{\goo\- ;;}.'' 
In order to prevent ambiguities, there should not be more than one marked
stone of the same color referenced in this way on a single diagram. To prevent
this, {\tt sgf2dg} may break a diagram at a point not specified by the user.

There are three alternative schemes for numbering in variation diagrams.
In the default numbering scheme, each variation is numbered starting
with a stone numbered ``1.'' We call this scheme {\it relative} variation
numbering. You can specify it using {\tt -rv} or {\tt -relativeVarNums}, but
you don't have to, since it is the default.

In the second scheme, called {\it absolute} variation numbering, each
variation is numbered relative to the beginning of the game. Thus
if the variation occurs at move 120, the first stone in the variation
diagram is given the number 120. This option is invoked with {\tt -av} or 
{\tt -absoluteVarNums}.

In the third scheme, called {\it correlative} variation numbering, each
variation is numbered from the first move to deviate from the main line.
Thus all variations beginning with the same move (branching from the
main line) are numbered consistently. This option is invoked with
the {\tt -cv} or {\tt -correlativeVarNums} option.

{\tt -repeatlast} causes the last move in each diagram to duplicate the
first move in the next. This emulates a common style of Go writing.

{\tt -ic} or {\tt ignoreComments} causes SGF comments to be ignored.

{\tt -il} or {\tt ignoreLetters} causes SGF letters to be ignored.

{\tt -im} or {\tt ignoreMarks} causes SGF marks to be ignored.

{\tt -im} or {\tt -ignoreVariations} causes variations to be ignore.
(Variations are discussed below in a later section.)

{\tt -ip} or {\tt ignorePass} causes SGF passes to be ignored.
If this option is not used, a pass causes a remark in the comments.

{\tt -ia} or {\tt ignoreAll} causes all SGF letters, marks, variations
and passes to be ignored.

{\tt -firstDiagram <{\it Diagram Number}>} allows you to specify
the first diagram to print;

{\tt -lastDiagram <{\it Diagram Number}>} allows you to specify
the last diagram to print.

{\tt -placeHandi} should not be necessary except when converting rather
old style SGF files.  Newer SGF formats require explicit handicap placement
using the normal AddBlack (AB) property.  If your diagram is missing its
handicap stones, this option may fix it.

{\tt -converter} selects the output converter.  See the next section
for more details on converters.  The default converter is {\tt
Games::Go::Sgf2Dg::Dg2Tex} which creates source for \TeX.  Converters get
'{\tt Games::Go::Sgf2Dg::Dg2}' prepended, so enter only the part after {\tt Games::Go::Sgf2Dg::Dg2}.
The default is thus equivilent to {\tt -converter TeX}.

The remaining options apply only to the {\tt Dg2TeX} converter.

{\tt -simple (Dg2TeX)} avoids \TeX\ complexity, producing the diagram with
the comments below it. This option is useful if you intend to edit
the \TeX\ and don't want a fancy format forced on you.

{\tt -mag number (Dg2TeX)} changes the default {\tt \\magnification} from
1000 to {\tt number}.

{\tt -twoColumn (Dg2TeX)} implements a two-column format, making use of the alternative
macros in {\tt gotcmacs.tex}. Two-column formats are used in some Go books and
magazines, for example John Power's beautiful {\it Invincible: The Games of
Shusaku}. The Go font is reduced by 1.2 in size so that two $19\times 19$
boards may fit side-by-side on the page. The text font is not reduced.

{\tt -bigFonts (Dg2TeX)} uses {\tt \\bgoo} and {\tt \\bgoe}, using fonts which
are 1.2 times larger.

\vfil\eject  % prevent a bad page break here
{\tt -texComments (Dg2TeX)}: Certain characters, when found in comments, are normally
remapped as follows:

\medbreak
\centerline{
\vbox{
\baselineskip=15pt
\halign{\hfil\quad#\quad\hfil&\hfil\quad#\quad\hfil&\quad#\hfil\cr
    $\backslash$ &$\rightarrow$  &\$$\backslash$backslash\$  \cr
    $\lbrace$    &$\rightarrow$  &\$$\backslash$lbrace\$  \cr
    $\rbrace$    &$\rightarrow$  &\$$\backslash$rbrace\$  \cr
    \$           &$\rightarrow$  &$\backslash$\$  \cr
    \&           &$\rightarrow$  &$\backslash$\&  \cr
    \#           &$\rightarrow$  &$\backslash$\#  \cr
    $\wedge$     &$\rightarrow$  &\$$\backslash$wedge\$  \cr
    \_           &$\rightarrow$  &$\backslash$\_  \cr
    \%           &$\rightarrow$  &$\backslash$\%  \cr
    $\sim$       &$\rightarrow$  &\$$\backslash$sim\$  \cr
    $<$          &$\rightarrow$  &\$$<$\$  \cr
    $>$          &$\rightarrow$  &\$$>$\$  \cr
    $|$          &$\rightarrow$  &\$$|$\$  \cr
}}}
\medbreak

This is done because either {\it these characters do not exist in \TeX\
roman font}, so it is impossible to print them without changing fonts, or
because the characters have special meaning to \TeX\ (see {\it The \TeX\
Book} page 38). Instead of using the characters directly, {\tt Dg2TeX}
makes plausible substitutions.  When {\tt texComments} is specified, the
mappings are suppressed so you can embed normal \TeX\ source (like {\tt
\char'173\\bf boldface\char'175}) directly inside the comments.

{\tt -floatControl (Dg2TeX)} is followed by a control string.  In normal
mode (not {\tt simple} and not {\tt twoColumn}), {\tt Dg2TeX} floats the
board diagram to the left or right of the text and if the text is extensive
enough, it flows around the diagram.  The control string is a sequence of
letters that control placement of the diagram float.
\medbreak
\centerline{
\vbox{
\halign{\hfil#\qquad&#\hfil\cr
         Letter   &Diagram placement\cr
              l   &left (text to the right)\cr
              r   &right\cr
              a   &alternately on the left and right\cr
      any other   &random\cr
}}}
\medbreak
The first letter is for the first diagram, second letter is for the second
diagram, and so on. When there is only one letter left, that letter
controls all remaining diagrams.  The default control string is '{\tt rx}'
which places the first diagram on the right, and the rest are placed
randomly.

\vfil\eject
\centerline{\titlefont Output Converters}
\vglue.18in
{\tt sgf2dg} is designed to use plugin output converters.  Each
converter produces diagrams in a specific output format.  The output
converters available with the current distribution are:

\medbreak
\centerline{
\vbox{
\halign{#\qquad&#\hfil\cr
{\tt Games::Go::Sgf2Dg::Dg2TeX}     &\TeX\ source (default)\cr
{\tt Games::Go::Sgf2Dg::Dg2Mp}      &MetaPost embedded in \TeX\ (Encapsulated PostScript)\cr
{\tt Games::Go::Sgf2Dg::Dg2ASCII}   &simple ASCII diagrams\cr
{\tt Games::Go::Sgf2Dg::Dg2Ps}      &PostScript\cr
{\tt Games::Go::Sgf2Dg::Dg2PDF}     &Portable Document Format (PDF)\cr
{\tt Games::Go::Sgf2Dg::Dg2Tk}      &Perl/Tk NoteBook/Canvas\cr
{\tt Games::Go::Sgf2Dg::Dg2TkPs}    &PostScript via Dg2Tk (Dg2Ps is nicer)\cr
{\tt Games::Go::Sgf2Dg::Dg2SL}      &Sensei's Library (thanks to Marcel Gruenauer\cr
}}}
\medbreak

Converters are quite easy to write - it take just a few hours if
you are already conversant with the conversion target.  If you would
like to create a converter plugin module, the easiest way is to make
a copy of an existing converter (Dg2ASCII.pm for example) and
modify it.  Once it's working, please be sure to send us a copy so
we can add it to the distribution.

Converters are always prepended with 'Games::Go::Sgf2Dg::Dg2', so to select
the ASCII converter instead of the default \TeX\ converter, use:
\vglue.12in
    {\tt sgf2dg ... -converter ASCII ...}
\vglue.12in
Converter names are case sensitive.

The default output filename suffix is determined by the converter name: the
name is lower-cased to become the suffix, so the ASCII converter produces
{\tt <filename>.ascii} from {\tt <filename>.sgf}.

You can select different default converters by changing the name of
the {\tt sgf2dg} script (or better, make symbolic links, or copies if
your system can't handle links).  The converter name is extracted
from the name with this regular expression:
\vglue.12in
    {\tt m/sgf2(.*)/}
\vglue.12in
Anything after {\tt sgf2} is assumed to be the name of a converter
module.  For example, let's create a link to the script:
\vglue.12in
    {\tt \$ cd /usr/local/bin}

    {\tt \$ ln -s sgf2dg sgf2Xyz}
\vglue.12in
Executing:
\vglue.12in
    {\tt \$ sgf2Xyz foo.sgf [ options ]}
\vglue.12in
\noindent
attempts to use {\tt Games::Go::Sgf2Dg::Dg2Xyz} as the converter.  The
converter name extracted from the script name is case sensitive.

% Rewritten to avoid a bad page break

The three extracted names {\tt tex}, {\tt diagram} and {\tt dg}
are treated specially. These three names always attempt to use {\tt
Games::Go::Sgf2Dg::Dg2TeX} as the converter.

% Note that three extracted names are treated specially:
% \vglue.12in
%     {\tt tex}
% 
%     {\tt diagram}
% 
%     {\tt dg}
% \vglue.12in
% These three names (when extracted from the script name) always
% attempt to use {\tt Games::Go::Sgf2Dg::Dg2TeX} as the converter.

\vfil\eject
\centerline{\titlefont Converter Options}
\vglue.18in

Converter options are prepended with the converter name so that
option {\tt xyz} for converter {\tt Games::Go::Sgf2Dg::Dg2Abc} is written on
the command line as:
\vglue.12in
    {\tt \$ sgf2dg ... -Abc-xyz ...}
\vglue.12in
Converter options that take arguments must be quoted so that the
shell passes the option and any arguments as a single ARGV.  For
example, if the {\tt xyz} option for converter {\tt Dg2Abc} takes {\tt foo} and
{\tt bar} as additional arguments, the command line would be:
\vglue.12in
    {\tt \$ sgf2dg ... "-Abc-xyz foo bar" ...}
\vglue.12in

\noindent
or a more realistic example of changing the background color:

\vglue.12in
    {\tt \$ sgf2dg genan-shuwa -converter Tk "-Tk-bg \#d2f1b4bc8c8b"}
\vglue.12in
Since {\tt sgf2dg} is a super-set replacement for the Sgf2TeX package,
\TeX\ holds the default position for converters.  Because of this
historically priviledged position, the {\tt Dg2TeX} options do not need to
be prepended with {\tt -TeX-}.

See the {\tt man} or {\tt perldoc} pages for converter-specific
options.  eg:
\vglue.12in
    {\tt perldoc Games::Go::Sgf2Dg::Dg2PDF}.
\vglue.12in
Of the converters available with this release, {\tt Dg2ASCII} takes
no additional options.  {\tt Dg2Tk} and therefore {\tt Dg2TkPs} take
no additional options directly, but arguments are assumed to be
standard {\tt Tk} and {\tt Tk::Canvas} options and are passed to
the {\tt Tk::Canvas} widgets.  This explains why the background
color example above works.

\vfil\eject
\centerline{\titlefont Handling Variations using {\titlett sgfsplit}}
\vglue.18in
If your SGF file contains variations, {\tt sgf2dg} will print them
and generate labels for them according to a logical scheme.

We also offer an alternative. The {\tt C} program {\tt sgfsplit} will 
create from the file a sequence of other SGF files each containing
one variation. The normal usage is {\tt sgfsplit} [{\it filename}]. This will
create from the file [{\it filename}] or [{\it filename}]{\tt .sgf} a sequence of
files called [{\it filename}]{\tt .0.sgf}, [{\it filename}]{\tt .1.sgf}, etc. The first
file contains the main line, the remaining files, variations. Also
created is an executable shell script called [{\it filename}]{\tt .sgf2dg}
which contains suggested commands for invoking {\tt sgf2dg}. In particular,
breakpoints are specified for each variation. You may wish to edit this file
before executing it to fine tune the board sizes. If you invoke {\tt sgfsplit}
in the form {\tt sgfsplit} [{\it filename}] [{\it extra parameters}] the
extra parameters (such as {\tt -simple}) are passed along in 
[{\it filename}]{\tt .sgf2dg}.

One may ask why {\tt sgfsplit} is needed since {\tt sgf2dg} has built-in
support for variations. One possible reason is that by splitting up the
file into variational components, one may supply different values of
the {\tt -t}, {\tt -b}, {\tt -l} and {\tt -r} parameters. This requires
editing the {\tt [filename].sgf2dg} file before executing it.

\vglue1in
\centerline{\titlefont Small Boards}
\vglue.25in
\hbox to\hsize{\vbox to 108.0 pt{\hsize= 108.0 pt\goo
\0??<\0??(\048(\043(\046(\0??(\042(\0??(\0??>
\0??[\049+\017+\018+\0??+\022+\041+\040+\0??]
\0??[\019+\044+\021+\016+\009+\001+\0??+\0??]
\0??[\023+\003+\024+\014+\008+\015+\011+\045]
\0??[\020+\025+\050+\0??*\0??+\013+\010+\012]
\0??[\026+\0??+\029+\028+\032+\006+\005+\0??]
\0??[\0??+\004+\027+\031+\036+\002+\007+\0??]
\0??[\0??+\030+\0??+\037+\034+\033+\035+\0??]
\0??,\0??)\0??)\0??)\039)\038)\0??)\0??)\0??.
\hfil\break\vglue10pt
\hbox{\rm \textstone{\goo\047=} at \textstone{\goo\021=}}\hfil\break}\hfil
\vbox to 108.9pt{\hsize=1.8in\rm\tolerance=10000\vfil
The most commonly used small board sizes are $9\times9$ and $13\times13$.
If {\tt sgf2dg} is used on a game of any of these sizes, an
appropriate small board is used. The board size is set using the SGF
property~{\tt SZ}.\vfil}\hfil
\vbox to 108.0 pt{\hsize= 108.0 pt\goo
\0??<\062(\- !(\0??(\- !(\0??(\- !(\0??(\0??>
\064[\- @+\- @+\- !+\0??+\- !+\- @+\- !+\0??]
\0??[\- @+\053+\- @+\- !+\- @+\- @+\056+\054]
\063[\- @+\- @+\- !+\- !+\- !+\- @+\- @+\- @]
\061[\- !+\- @+\- !+\0??*\052+\- @+\- !+\- !]
\065[\- !+\051+\- @+\- !+\- !+\- !+\- @+\055]
\060[\066+\- !+\- @+\- @+\- !+\- !+\- @+\0??]
\059[\057+\- !+\067+\- @+\- !+\- @+\- @+\0??]
\068,\058)\069)\070)\- @)\- !)\0??)\0??)\0??.
\hfil\break}}\vskip 40pt

The FF4 specification for SGF files allows {\tt SZ[X:Y]} where {\tt X} and
{\tt Y} are arbitrary numbers.  {\tt sgf2dg} supports this syntax.

\vfil\eject
\centerline{\titlefont\TeX\ Macros}

\bigbreak
Using the {\tt sgf2dg} translator one need not worry about writing \TeX\
code. Nevertheless with some understanding of how \TeX\ works, GOOE/{\tt
sgf2dg} becomes a more powerful system. One method is to use {\tt sgf2dg
-simple} to create \TeX\ files, which may then be edited and merged into a
larger document. For this, some understanding of how \TeX\ works is useful.
(The unedited output of {\tt sgf2dg -simple} is not beautiful---this 
option is perhaps {\it only} useful if you intend to edit its output.
But if this is your intention, {\tt sgf2dg -simple} will produce a file
as uncluttered as possible.)

The ultimate source for \TeX nical enlightenment is of course
{\it The \TeX\ Book} by Donald Knuth (Addison-Wesley, 1984). We will try
to explain a few things which may be helpful to the user.

The file {\tt gooemacs.tex} contains macros needed to typeset Go games. You
may either paste these into the beginning of a file, or input them. 
The most important macros defined in {\tt gooemacs.tex} are {\tt \\goo} and
{\tt\\goe}. These toggle Go modes in which the black stones have odd numbers
({\tt \\goo}) or even numbers ({\tt \\goe}).

{\tt gotcmacs.tex} is an alternative to {\tt gooemacs.tex}, implementing
a two-column format. The macros in {\tt gotcmacs.tex} are used if you run
{\tt sgf2dg} in the {\tt -twoColumn} mode.

Here is the \TeX\ code needed to produce an empty Go board, with some text
alongside, as on the following page:

\vfil
{\seventtt\parindent=0pt\baselineskip=9pt\obeylines
\def\[{{\char'173}}
\def\]{{\char'175}}
\\hbox to \\hsize \[\\vbox\[\\hsize=228pt\\goo
\\0??<\\0??(\\0??(\\0??(\\0??(\\0??(\\0??(\\0??(\\0??(\\0??(\\0??(\\0??(\\0??(\\0??(\\0??(\\0??(\\0??(\\0??(\\0??>
\\0??[\\0??+\\0??+\\0??+\\0??+\\0??+\\0??+\\0??+\\0??+\\0??+\\0??+\\0??+\\0??+\\0??+\\0??+\\0??+\\0??+\\0??+\\0??]
\\0??[\\0??+\\0??+\\0??+\\0??+\\0??+\\0??+\\0??+\\0??+\\0??+\\0??+\\0??+\\0??+\\0??+\\0??+\\0??+\\0??+\\0??+\\0??]
\\0??[\\0??+\\0??+\\0??*\\0??+\\0??+\\0??+\\0??+\\0??+\\0??*\\0??+\\0??+\\0??+\\0??+\\0??+\\0??*\\0??+\\0??+\\0??]
\\0??[\\0??+\\0??+\\0??+\\0??+\\0??+\\0??+\\0??+\\0??+\\0??+\\0??+\\0??+\\0??+\\0??+\\0??+\\0??+\\0??+\\0??+\\0??]
\\0??[\\0??+\\0??+\\0??+\\0??+\\0??+\\0??+\\0??+\\0??+\\0??+\\0??+\\0??+\\0??+\\0??+\\0??+\\0??+\\0??+\\0??+\\0??]
\\0??[\\0??+\\0??+\\0??+\\0??+\\0??+\\0??+\\0??+\\0??+\\0??+\\0??+\\0??+\\0??+\\0??+\\0??+\\0??+\\0??+\\0??+\\0??]
\\0??[\\0??+\\0??+\\0??+\\0??+\\0??+\\0??+\\0??+\\0??+\\0??+\\0??+\\0??+\\0??+\\0??+\\0??+\\0??+\\0??+\\0??+\\0??]
\\0??[\\0??+\\0??+\\0??+\\0??+\\0??+\\0??+\\0??+\\0??+\\0??+\\0??+\\0??+\\0??+\\0??+\\0??+\\0??+\\0??+\\0??+\\0??]
\\0??[\\0??+\\0??+\\0??*\\0??+\\0??+\\0??+\\0??+\\0??+\\0??*\\0??+\\0??+\\0??+\\0??+\\0??+\\0??*\\0??+\\0??+\\0??]
\\0??[\\0??+\\0??+\\0??+\\0??+\\0??+\\0??+\\0??+\\0??+\\0??+\\0??+\\0??+\\0??+\\0??+\\0??+\\0??+\\0??+\\0??+\\0??]
\\0??[\\0??+\\0??+\\0??+\\0??+\\0??+\\0??+\\0??+\\0??+\\0??+\\0??+\\0??+\\0??+\\0??+\\0??+\\0??+\\0??+\\0??+\\0??]
\\0??[\\0??+\\0??+\\0??+\\0??+\\0??+\\0??+\\0??+\\0??+\\0??+\\0??+\\0??+\\0??+\\0??+\\0??+\\0??+\\0??+\\0??+\\0??]
\\0??[\\0??+\\0??+\\0??+\\0??+\\0??+\\0??+\\0??+\\0??+\\0??+\\0??+\\0??+\\0??+\\0??+\\0??+\\0??+\\0??+\\0??+\\0??]
\\0??[\\0??+\\0??+\\0??+\\0??+\\0??+\\0??+\\0??+\\0??+\\0??+\\0??+\\0??+\\0??+\\0??+\\0??+\\0??+\\0??+\\0??+\\0??]
\\0??[\\0??+\\0??+\\0??*\\0??+\\0??+\\0??+\\0??+\\0??+\\0??*\\0??+\\0??+\\0??+\\0??+\\0??+\\0??*\\0??+\\0??+\\0??]
\\0??[\\0??+\\0??+\\0??+\\0??+\\0??+\\0??+\\0??+\\0??+\\0??+\\0??+\\0??+\\0??+\\0??+\\0??+\\0??+\\0??+\\0??+\\0??]
\\0??[\\0??+\\0??+\\0??+\\0??+\\0??+\\0??+\\0??+\\0??+\\0??+\\0??+\\0??+\\0??+\\0??+\\0??+\\0??+\\0??+\\0??+\\0??]
\\0??,\\0??)\\0??)\\0??)\\0??)\\0??)\\0??)\\0??)\\0??)\\0??)\\0??)\\0??)\\0??)\\0??)\\0??)\\0??)\\0??)\\0??)\\0??.\]
\\hfil
\\vbox to 228pt\[\\hsize=1.8in\\tolerance=10000\\vglue6pt
\\noindent\[\\bf Example 1 (left).\] Empty Go board. Blah blah blah etc.
\\vfil\]\]
}
\vfil\eject

\hbox to \hsize {\vbox{\hsize=228pt\goo
\0??<\0??(\0??(\0??(\0??(\0??(\0??(\0??(\0??(\0??(\0??(\0??(\0??(\0??(\0??(\0??(\0??(\0??(\0??>
\0??[\0??+\0??+\0??+\0??+\0??+\0??+\0??+\0??+\0??+\0??+\0??+\0??+\0??+\0??+\0??+\0??+\0??+\0??]
\0??[\0??+\0??+\0??+\0??+\0??+\0??+\0??+\0??+\0??+\0??+\0??+\0??+\0??+\0??+\0??+\0??+\0??+\0??]
\0??[\0??+\0??+\0??*\0??+\0??+\0??+\0??+\0??+\0??*\0??+\0??+\0??+\0??+\0??+\0??*\0??+\0??+\0??]
\0??[\0??+\0??+\0??+\0??+\0??+\0??+\0??+\0??+\0??+\0??+\0??+\0??+\0??+\0??+\0??+\0??+\0??+\0??]
\0??[\0??+\0??+\0??+\0??+\0??+\0??+\0??+\0??+\0??+\0??+\0??+\0??+\0??+\0??+\0??+\0??+\0??+\0??]
\0??[\0??+\0??+\0??+\0??+\0??+\0??+\0??+\0??+\0??+\0??+\0??+\0??+\0??+\0??+\0??+\0??+\0??+\0??]
\0??[\0??+\0??+\0??+\0??+\0??+\0??+\0??+\0??+\0??+\0??+\0??+\0??+\0??+\0??+\0??+\0??+\0??+\0??]
\0??[\0??+\0??+\0??+\0??+\0??+\0??+\0??+\0??+\0??+\0??+\0??+\0??+\0??+\0??+\0??+\0??+\0??+\0??]
\0??[\0??+\0??+\0??*\0??+\0??+\0??+\0??+\0??+\0??*\0??+\0??+\0??+\0??+\0??+\0??*\0??+\0??+\0??]
\0??[\0??+\0??+\0??+\0??+\0??+\0??+\0??+\0??+\0??+\0??+\0??+\0??+\0??+\0??+\0??+\0??+\0??+\0??]
\0??[\0??+\0??+\0??+\0??+\0??+\0??+\0??+\0??+\0??+\0??+\0??+\0??+\0??+\0??+\0??+\0??+\0??+\0??]
\0??[\0??+\0??+\0??+\0??+\0??+\0??+\0??+\0??+\0??+\0??+\0??+\0??+\0??+\0??+\0??+\0??+\0??+\0??]
\0??[\0??+\0??+\0??+\0??+\0??+\0??+\0??+\0??+\0??+\0??+\0??+\0??+\0??+\0??+\0??+\0??+\0??+\0??]
\0??[\0??+\0??+\0??+\0??+\0??+\0??+\0??+\0??+\0??+\0??+\0??+\0??+\0??+\0??+\0??+\0??+\0??+\0??]
\0??[\0??+\0??+\0??*\0??+\0??+\0??+\0??+\0??+\0??*\0??+\0??+\0??+\0??+\0??+\0??*\0??+\0??+\0??]
\0??[\0??+\0??+\0??+\0??+\0??+\0??+\0??+\0??+\0??+\0??+\0??+\0??+\0??+\0??+\0??+\0??+\0??+\0??]
\0??[\0??+\0??+\0??+\0??+\0??+\0??+\0??+\0??+\0??+\0??+\0??+\0??+\0??+\0??+\0??+\0??+\0??+\0??]
\0??,\0??)\0??)\0??)\0??)\0??)\0??)\0??)\0??)\0??)\0??)\0??)\0??)\0??)\0??)\0??)\0??)\0??)\0??.}
\hfil
\vbox to 228pt{\hsize=1.8in\tolerance=10000\vglue6pt
\noindent{\bf Example 1 (left).} Empty Go board. We use the macro
{\tt \\goo} instead of {\tt \\goe} because we intend to fill
this with stones in which the black ones are odd numbered. Cut the relevant
lines from this file {\tt manual.tex} and use them as a template to which
stones may be added as in the succeeding examples.

\bigbreak\noindent
{\bf Example 2 (below).} If you want a bigger font within the same 
document, you may use {\tt\\bgoo} and {\tt\\bgoe} in place of {\tt\\goo} and
{\tt\\goe} to get the fonts in 14 point.
\vfil}}

\vglue.25in
\hbox to\hsize{\hfil\vbox{\hsize=273.6pt\bgoo
\0??<\0??(\0??(\0??(\0??(\0??(\0??(\0??(\0??(\0??(\0??(\0??(\0??(\0??(\0??(\0??(\0??(\0??(\0??>
\0??[\0??+\0??+\0??+\0??+\0??+\0??+\0??+\0??+\0??+\0??+\0??+\0??+\0??+\0??+\0??+\0??+\0??+\0??]
\0??[\0??+\0??+\0??+\0??+\0??+\0??+\0??+\0??+\0??+\0??+\0??+\0??+\0??+\0??+\0??+\0??+\0??+\0??]
\0??[\0??+\0??+\0??*\0??+\0??+\0??+\0??+\0??+\0??*\0??+\0??+\0??+\0??+\0??+\0??*\0??+\0??+\0??]
\0??[\0??+\0??+\0??+\0??+\0??+\0??+\0??+\0??+\0??+\0??+\0??+\0??+\0??+\0??+\0??+\0??+\0??+\0??]
\0??[\0??+\0??+\0??+\0??+\0??+\0??+\0??+\0??+\0??+\0??+\0??+\0??+\0??+\0??+\0??+\0??+\0??+\0??]
\0??[\0??+\0??+\0??+\0??+\0??+\0??+\0??+\0??+\0??+\0??+\0??+\0??+\0??+\0??+\0??+\0??+\0??+\0??]
\0??[\0??+\0??+\0??+\0??+\0??+\0??+\0??+\0??+\0??+\0??+\0??+\0??+\0??+\0??+\0??+\0??+\0??+\0??]
\0??[\0??+\0??+\0??+\0??+\0??+\0??+\0??+\0??+\0??+\0??+\0??+\0??+\0??+\0??+\0??+\0??+\0??+\0??]
\0??[\0??+\0??+\0??*\0??+\0??+\0??+\0??+\0??+\0??*\0??+\0??+\0??+\0??+\0??+\0??*\0??+\0??+\0??]
\0??[\0??+\0??+\0??+\0??+\0??+\0??+\0??+\0??+\0??+\0??+\0??+\0??+\0??+\0??+\0??+\0??+\0??+\0??]
\0??[\0??+\0??+\0??+\0??+\0??+\0??+\0??+\0??+\0??+\0??+\0??+\0??+\0??+\0??+\0??+\0??+\0??+\0??]
\0??[\0??+\0??+\0??+\0??+\0??+\0??+\0??+\0??+\0??+\0??+\0??+\0??+\0??+\0??+\0??+\0??+\0??+\0??]
\0??[\0??+\0??+\0??+\0??+\0??+\0??+\0??+\0??+\0??+\0??+\0??+\0??+\0??+\0??+\0??+\0??+\0??+\0??]
\0??[\0??+\0??+\0??+\0??+\0??+\0??+\0??+\0??+\0??+\0??+\0??+\0??+\0??+\0??+\0??+\0??+\0??+\0??]
\0??[\0??+\0??+\0??*\0??+\0??+\0??+\0??+\0??+\0??*\0??+\0??+\0??+\0??+\0??+\0??*\0??+\0??+\0??]
\0??[\0??+\0??+\0??+\0??+\0??+\0??+\0??+\0??+\0??+\0??+\0??+\0??+\0??+\0??+\0??+\0??+\0??+\0??]
\0??[\0??+\0??+\0??+\0??+\0??+\0??+\0??+\0??+\0??+\0??+\0??+\0??+\0??+\0??+\0??+\0??+\0??+\0??]
\0??,\0??)\0??)\0??)\0??)\0??)\0??)\0??)\0??)\0??)\0??)\0??)\0??)\0??)\0??)\0??)\0??)\0??)\0??.
}\hfil}

\vfil\eject

The \TeX\ code given above for creating an empty board should be parsed as follows:

\vfil
{\tt\parindent=.5in\obeylines\offinterlineskip
\def\[{{\char'173}}
\def\]{{\char'175}}
\\hbox to \\hsize\[\\vbox\[\\hsize=228pt\\goo{\rm [Game Diagram]}\]
\\hfil
\\vbox to 228pt\[\\hsize=1.8in\\tolerance=10000\\vglue6pt{\rm [Text]}\\vfil\]\]
}
\vfil

In \TeX, there are two types of ``boxes'' which can contain typography. A {\it
vertical box} can consist of smaller boxes stacked together vertically, and a
{\it horizontal box} can consist of smaller boxes stacked together
horizontally. A horizontal box can be created using the {\tt\\hbox} macro,
so 

\vfil
{\tt\\hbox\char'173{\rm [First Box]\qquad [Second Box] ...}\char'175}
\vfil\noindent
creates a horizontal box containing a series of other boxes. or 
{\tt\\hbox to $\left<\hbox{\rm dimension}\right>$\char'173 ...\char'175}
creates a horizontal
box of specified width. Similarly
{\tt\\vbox to $\left<\hbox{\rm dimension}\right>$\char'173 ...\char'175} makes
a vertical box of specified height, and by putting the command
{\tt\\hsize=$\left<\hbox{\rm another dimension}\right>$} inside the braces we
can also control the width of the vertical box.

The above diagram consists of a horizontal box containing two vertical boxes,
separated by an {\tt\\hfil}. The latter is expandable horizontal ``glue'' or
space filling material which can grow horizontally to fill the gaps between
the boxes. The first vertical box consists of the game diagram, and since each
intersection on the Go board is a box 12 points high and 12 points wide, its
dimension is $12\times19=228$ points. (1 inch equals 72.27 point.)

The second vertical box is specified as

\vfil
\centerline{\noindent{\tt\\vbox to 228pt\char'173\\hsize=1.8in\\vglue6pt
{\rm [Text] }\\vfil\char'175}.}
\vfil

Thus we've forced it to have height 228 points---to match the known height of
the first vertical box---and width 1.8in. The width is chosen somewhat 
arbitrarily to make a small separation between the boxes. Plain \TeX\ sets
the width of a line to 6.5in, but since the first line of {\tt manual.tex} is
{\tt\\magnification=1200}, the true {\tt hsize} is actually 
$6.5{\rm in}\div1.2=391.462$pt. Thus the {\tt\\hfil} separating the two
vertical boxes will expand to fill the gap of $391.462-228-108.405=55.057$pt or
about .915 inch. Inside the second vertical box we find three items 
stacked vertically. The first ({\tt\\vglue6pt}) consists of inflexible vertical
glue filling 6 points; the second consists of the text which goes alongside
the diagram; and the third ({\tt\\vfil}) consists of flexible vertical glue to
fill the space down to the bottom of the box.  The text is thus moved down 6
points in the box. This has the effect of lining it up with the top of the go
board.

\vfil\eject
\hbox to \hsize {\vbox{\hsize=228pt\goo
\0??<\0??(\0??(\0??(\0??(\0??(\0??(\0??(\0??(\0??(\0??(\0??(\0??(\0??(\0??(\0??(\0??(\0??(\0??>
\0??[\0??+\0??+\0??+\0??+\0??+\0??+\0??+\0??+\0??+\0??+\0??+\0??+\0??+\0??+\0??+\0??+\0??+\0??]
\0??[\0??+\0??+\0??+\0??+\007+\0??+\0??+\0??+\0??+\0??+\0??+\0??+\0??+\0??+\0??+\0??+\0??+\0??]
\0??[\0??+\0??+\004+\0??+\0??+\0??+\0??+\0??+\009+\0??+\0??+\0??+\033+\0??+\001+\0??+\0??+\0??]
\0??[\0??+\0??+\0??+\0??+\0??+\0??+\0??+\0??+\0??+\0??+\0??+\0??+\0??+\0??+\0??+\031+\0??+\0??]
\0??[\0??+\008+\0??+\0??+\0??+\0??+\0??+\0??+\0??+\0??+\0??+\0??+\0??+\0??+\032+\028+\0??+\0??]
\0??[\0??+\0??+\0??+\0??+\0??+\0??+\0??+\0??+\0??+\0??+\0??+\0??+\0??+\0??+\0??+\0??+\0??+\0??]
\0??[\0??+\0??+\0??+\0??+\0??+\0??+\0??+\0??+\0??+\0??+\0??+\0??+\0??+\0??+\0??+\0??+\0??+\0??]
\0??[\0??+\0??+\0??+\0??+\0??+\0??+\0??+\0??+\0??+\0??+\0??+\0??+\0??+\0??+\034+\0??+\0??+\0??]
\0??[\0??+\0??+\0??*\0??+\0??+\0??+\0??+\0??+\0??*\0??+\0??+\0??+\0??+\0??+\0??*\0??+\0??+\0??]
\0??[\0??+\0??+\0??+\0??+\0??+\0??+\0??+\0??+\0??+\0??+\0??+\0??+\0??+\0??+\005+\0??+\0??+\0??]
\0??[\0??+\0??+\0??+\0??+\0??+\0??+\0??+\0??+\0??+\0??+\0??+\0??+\029+\0??+\0??+\0??+\0??+\0??]
\0??[\0??+\0??+\0??+\0??+\0??+\0??+\0??+\0??+\0??+\0??+\0??+\022+\030+\024+\0??+\0??+\0??+\0??]
\0??[\0??+\0??+\0??+\0??+\0??+\0??+\0??+\0??+\0??+\0??+\0??+\020+\015+\0??+\0??+\0??+\0??+\0??]
\0??[\0??+\0??+\0??+\0??+\0??+\0??+\0??+\0??+\0??+\023+\021+\013+\012+\016+\010+\0??+\0??+\0??]
\0??[\0??+\0??+\002+\0??+\0??+\0??+\0??+\0??+\025+\!a  \006+\018+\011+\014+\0??*\0??+\0??+\0??]
\0??[\0??+\0??+\0??+\0??+\0??+\0??+\026+\0??+\0??+\0??+\0??+\0??+\019+\017+\003+\0??+\0??+\0??]
\0??[\0??+\0??+\0??+\0??+\0??+\0??+\0??+\027+\0??+\0??+\0??+\0??+\0??+\0??+\0??+\0??+\0??+\0??]
\0??,\0??)\0??)\0??)\0??)\0??)\0??)\0??)\0??)\0??)\0??)\0??)\0??)\0??)\0??)\0??)\0??)\0??)\0??.}
\hfil
\vbox to 228pt{\hsize=1.5in\tolerance=10000\vglue6pt
\noindent{\bf Example 3.} We begin by replacing some of the stones {\tt\\0??} in
the previous file with {\tt\\001}, {\tt\\002}, ... to obtain numbered
Go stones \textstone{\goo\001=}, \textstone{\goo\002=}, .... In this example, 
\textstone{\goo\025=} should be at `{\it a\/}.'
\vfil}}


\hbox to \hsize {\vbox{\hsize=228pt\goo
\0??<\0??(\0??(\0??(\0??(\0??(\0??(\0??(\0??(\0??(\0??(\0??(\0??(\0??(\0??(\0??(\0??(\0??(\0??>
\0??[\0??+\0??+\0??+\0??+\0??+\0??+\0??+\0??+\0??+\0??+\0??+\0??+\0??+\0??+\0??+\0??+\0??+\0??]
\0??[\0??+\0??+\0??+\0??+\- @+\0??+\0??+\0??+\0??+\0??+\0??+\0??+\0??+\0??+\0??+\0??+\0??+\0??]
\0??[\0??+\0??+\- !+\0??+\0??+\0??+\0??+\0??+\- @+\0??+\0??+\0??+\- :+\0??+\- :+\0??+\0??+\0??]
\0??[\0??+\0??+\0??+\0??+\0??+\0??+\0??+\0??+\0??+\0??+\0??+\0??+\0??+\0??+\0??+\- :+\0??+\0??]
\0??[\0??+\- !+\0??+\0??+\0??+\0??+\0??+\0??+\0??+\0??+\0??+\046+\0??+\0??+\- ;+\- ;+\0??+\0??]
\0??[\0??+\0??+\0??+\0??+\0??+\0??+\0??+\0??+\0??+\0??+\045+\044+\0??+\0??+\0??+\0??+\0??+\0??]
\0??[\0??+\0??+\0??+\0??+\0??+\0??+\0??+\0??+\0??+\0??+\0??+\043+\042+\038+\0??+\0??+\0??+\0??]
\0??[\0??+\0??+\0??+\0??+\0??+\0??+\0??+\0??+\0??+\0??+\0??+\0??+\039+\037+\- !+\0??+\0??+\0??]
\0??[\0??+\0??+\0??*\0??+\0??+\0??+\0??+\0??+\0??*\0??+\0??+\0??+\0??+\0??+\040+\0??+\0??+\0??]
\0??[\0??+\0??+\0??+\0??+\0??+\0??+\0??+\0??+\0??+\0??+\0??+\0??+\0??+\041+\- @+\0??+\0??+\0??]
\0??[\0??+\0??+\0??+\0??+\0??+\0??+\0??+\0??+\0??+\0??+\047+\0??+\- @+\0??+\0??+\0??+\0??+\0??]
\0??[\0??+\0??+\0??+\0??+\0??+\0??+\0??+\0??+\0??+\0??+\0??+\- !+\- !+\- !+\0??+\035+\0??+\0??]
\0??[\0??+\0??+\0??+\0??+\0??+\0??+\0??+\0??+\0??+\0??+\0??+\- !+\- @+\0??+\0??+\0??+\0??+\0??]
\0??[\0??+\0??+\0??+\0??+\0??+\0??+\0??+\0??+\0??+\- @+\- @+\- @+\- !+\- !+\- !+\0??+\0??+\0??]
\0??[\0??+\0??+\- !+\0??+\0??+\0??+\0??+\0??+\- @+\0??+\- !+\- !+\- @+\- !+\0??*\036+\0??+\0??]
\0??[\0??+\0??+\0??+\0??+\0??+\0??+\- !+\0??+\0??+\0??+\0??+\0??+\- @+\- @+\- @+\048+\0??+\0??]
\0??[\0??+\0??+\0??+\0??+\0??+\0??+\0??+\- @+\0??+\0??+\0??+\052+\0??+\0??+\051+\049+\050+\0??]
\0??,\0??)\0??)\0??)\0??)\0??)\0??)\0??)\0??)\0??)\0??)\0??)\0??)\0??)\0??)\0??)\0??)\053)\0??.
}
\hfil
\vbox to 228pt{\hsize=1.5in\tolerance=10000\vglue6pt
\noindent{\bf Example 4.} Continuing from the above. A normal black 
stone such as \textstone{\goo\- @@} is represented in the \TeX\ file
as {\tt\\- @+}, and a normal white stone such as 
\textstone{\goo\- !!} is represented by {\tt\\- !+}.  There are a number of
special symbols available used for marking the board, including the
triangled stones seen in this diagram.  See below for a complete list.
\vfil}}

\vfil\eject

\hbox to\hsize{\vbox to 86.88889pt{\hsize=3in\vglue6pt 
{\bf Example 5.} Diagrams are handled similarly, except that we only use a
portion of the template from Example~1. In this example, since White
moves first, we use {\tt\\goe} instead of {\tt\\goo}. Since the box
containing the diagram on the right has height $7\times12=84$ pt, we give the
box containing the text on the left the same height.
\vfil}\hfil
\vbox to 86.88889pt{\hsize=132pt\goe
\0??<\0??(\0??(\0??(\0??(\0??(\0??(\0??(\0??(\0??(\0??
\0??[\0??+\0??+\0??+\0??+\0??+\0??+\0??+\0??+\0??+\0??
\0??[\0??+\004+\002+\001+\!  a\0??+\0??+\0??+\0??+\0??
\0??[\0??+\- !+\003+\- @+\0??+\0??+\0??+\0??+\0??*\0??
\0??[\0??+\0??+\0??+\0??+\0??+\0??+\0??+\0??+\0??+\0??
\0??[\0??+\0??+\- @+\0??+\0??+\0??+\0??+\0??+\0??+\0??
\0??[\0??+\0??+\0??+\0??+\0??+\0??+\0??+\0??+\0??+\0??
\vfil}}

\vglue.2in

\vglue.18in
\centerline{\titlefont Letters on Stones}
\vglue.18in

Beginning with version~3.1 {\tt sgf2dg} supports stones labelled with
italic letters. If you have an older version of {\tt sgf2dg} you
will have to replace the file {\tt gooemacs.tex}. You will also
have to install a few new metafont files, namely {\tt gobl.mf}, 
{\tt gowl.mf}, {\tt paramsc.mf} and {\tt goxl.mf}. After executing
the macro {\tt \\gool}, you can use the macros {\tt\\401=} through
{\tt\\426=} to obtain the black labelled stones \textstone{\goo\401=}
through \textstone{\goo\426=}, and macros {\tt\\501=} through
{\tt\\526=} to obtain the white labelled stones \textstone{\goo\501=}
through \textstone{\goo\526=}. 

Thus you can produce diagrams such as this one:

\smallskip
\vbox to 228.0 pt{\hsize= 228.0 pt\goo\gool
\0??<\0??(\0??(\0??(\0??(\0??(\0??(\0??(\0??(\0??(\0??(\0??(\- !(\- @(\0??(\0??(\0??(\0??(\0??>
\0??[\0??+\0??+\0??+\0??+\0??+\0??+\0??+\0??+\0??+\0??+\- !+\- !+\- @+\0??+\0??+\0??+\0??+\0??]
\0??[\0??+\0??+\0??+\- !+\0??+\- !+\- !+\- !+\0??+\- !+\0??+\- @+\0??+\- @+\- @+\0??+\- @+\0??]
\0??[\0??+\- !+\0??*\0??+\0??+\- !+\- @+\- @+\- !+\0??+\- !+\0??+\- @+\- @+\- !+\- @+\- @+\- @]
\0??[\0??+\0??+\- !+\0??+\0??+\0??+\- !+\- @+\- @+\- !+\0??+\- @+\0??+\- @+\- !+\- !+\- !+\- @]
\0??[\0??+\0??+\0??+\- !+\0??+\- !+\0??+\- !+\- @+\0??+\- !+\- @+\0??+\- @+\- !+\- !+\- @+\- @]
\0??[\0??+\- !+\- !+\0??+\- !+\0??+\0??+\0??+\- @+\0??+\- @+\0??+\0??+\- @+\- !+\0??+\- !+\- !]
\0??[\0??+\0??+\- !+\- @+\0??+\- @+\0??+\- @+\0??+\- @+\0??+\- @+\- @+\- !+\0??+\- !+\- !+\- !]
\0??[\0??+\- !+\- @+\- @+\- @+\0??+\0??+\0??+\- !+\- @+\- !+\- @+\- !+\0??+\- !+\- @+\- @+\- !]
\0??[\0??+\- !+\- !+\- !+\- !+\- !+\- !+\- !+\0??*\- !+\- !+\- @+\- @+\- !+\- !+\- @+\0??+\- @]
\0??[\0??+\- !+\- @+\- @+\- @+\- @+\- @+\0??+\- !+\- !+\- !+\- !+\- !+\- @+\- @+\- @+\0??+\0??]
\0??[\0??+\- !+\- @+\0??+\0??+\0??+\0??+\- @+\- !+\0??+\- !+\- @+\- !+\- @+\- @+\0??+\0??+\0??]
\0??[\0??+\- !+\- !+\- !+\- !+\0??+\- @+\- !+\- !+\- !+\0??+\- @+\- !+\- @+\0??+\0??+\0??+\0??]
\0??[\0??+\- !+\- @+\- @+\- !+\0??+\- @+\0??+\0??+\0??+\- !+\- @+\- @+\- @+\- @+\0??+\0??+\0??]
\0??[\- !+\- !+\- !+\- @+\0??+\- @+\0??+\- @+\0??+\- !+\- @+\- @+\- @+\501=\0??+\0??+\0??+\0??]
\0??[\- @+\- !+\- @+\- @+\0??+\0??+\0??+\0??+\- @+\- @+\502=\0??+\0??+\501=\0??*\- @+\0??+\0??]
\- @[\0??+\- @+\- !+\- @+\0??+\0??+\0??+\- @+\0??+\502=\502=\- @+\0??+\501=\- @+\0??+\0??+\0??]
\0??[\0??+\- @+\- !+\- !+\- @+\0??+\0??+\0??+\0??+\- @+\- @+\0??+\- @+\501=\501=\- @+\0??+\0??]
\0??,\0??)\0??)\0??)\0??)\0??)\0??)\0??)\0??)\0??)\0??)\0??)\0??)\0??)\0??)\0??)\0??)\0??)\0??.
}

\vglue.2in

\vfil\eject
\centerline{\titlefont Special Symbols}
\vglue.18in
Special symbols include the characters for making the board (9 types of
intersection plus hoshi points), circle, square, triangle, and X marks for
empty intersections, and the same four marks for black and white stones.
Here is a complete list of the marks, the character assignment in
gooegb.mf, and the appropriate {\tt \\goo} macro invocation.

\medbreak
\centerline{
\vbox{
\baselineskip=15pt
\halign{\hfil\quad#\quad\hfil&\hfil\quad#\quad\hfil&\hfil\quad#\quad\hfil&\quad#\hfil\cr
Symbol                 &{\tt \\gooegb} &{\tt \\goo}     &Description\cr
\textstone{\goo\0??<}  &{\tt <}        &{\tt \\0??<}    &Top left corner\cr
\textstone{\goo\0??(}  &{\tt (}        &{\tt \\0??(}    &Top edge\cr
\textstone{\goo\0??>}  &{\tt >}        &{\tt \\0??>}    &Top right corner\cr
\textstone{\goo\0??[}  &{\tt [}        &{\tt \\0??[}    &Left edge\cr
\textstone{\goo\0??+}  &{\tt +}        &{\tt \\0??+}    &Middle of the board\cr
\textstone{\goo\0??*}  &{\tt *}        &{\tt \\0??*}    &Hoshi point\cr
\textstone{\goo\0??]}  &{\tt ]}        &{\tt \\0??]}    &Right edge\cr
\textstone{\goo\0??,}  &{\tt ,}        &{\tt \\0??,}    &Bottom left corner\cr
\textstone{\goo\0??)}  &{\tt )}        &{\tt \\0??)}    &Bottom edge\cr
\textstone{\goo\0??.}  &{\tt .}        &{\tt \\0??.}    &Bottom right corner\cr
\textstone{\goo\0??@}  &{\tt @}        &{\tt \\0??@}    &Black stone\cr
\textstone{\goo\0??!}  &{\tt !}        &{\tt \\0??!}    &White stone\cr
\textstone{\goo\0??1}  &{\tt 1}        &{\tt \\0??1}    &Circle\cr
\textstone{\goo\0??C}  &{\tt C}        &{\tt \\0??C}    &Black Stone, circled\cr
\textstone{\goo\0??c}  &{\tt c}        &{\tt \\0??c}    &White Stone, circled\cr
\textstone{\goo\0??2}  &{\tt 2}        &{\tt \\0??2}    &Square\cr
\textstone{\goo\0??S}  &{\tt S}        &{\tt \\0??S}    &Black Stone, squared\cr
\textstone{\goo\0??s}  &{\tt s}        &{\tt \\0??s}    &White Stone, squared\cr
\textstone{\goo\0??3}  &{\tt 3}        &{\tt \\0??3}    &Triangle\cr
\textstone{\goo\0??:}  &{\tt :}        &{\tt \\0??:}    &Black Stone, triangled\cr
\textstone{\goo\0??T}  &{\tt T}        &{\tt \\0??T}    &Black Stone, triangled (copy of :)\cr
\textstone{\goo\0??;}  &{\tt ;}        &{\tt \\0??;}    &White Stone, triangled\cr
\textstone{\goo\0??t}  &{\tt t}        &{\tt \\0??t}    &White Stone, triangled (copy of ;)\cr
\textstone{\goo\0??4}  &{\tt 4}        &{\tt \\0??4}    &X\cr
\textstone{\goo\0??X}  &{\tt X}        &{\tt \\0??X}    &Black Stone, Xed\cr
\textstone{\goo\0??x}  &{\tt x}        &{\tt \\0??x}    &White Stone, Xed\cr
\textstone{\goo\0??-}  &{\tt -}        &{\tt \\0??-}    &Blob: erase area under label (use white ink)\cr
}}}
\medbreak

\def\goWhiteInk#1{\special{color push rgb 1 1 1} {#1} \special{color pop}}
\def\goLap#1#2{\setbox0=\hbox{#1}\rlap{#1}\raise 3.5pt\hbox to \wd0{\hss\eightpoint{#2}\hss}}
\def\goLapWhite#1#2{\setbox0=\hbox{#1}\rlap{#1}\raise 3.5pt\hbox to \wd0{\hss\eightpoint\goWhiteInk{#2}\hss}}

\vfil\eject
\centerline{\titlefont Overlapping Symbols}
\vglue.18in

\TeX\ provides a method of overlapping font characters so that you can put
any font symbol on your board and stones.  For example, you may wish to
mark a particularly sharp move: \textstone{\goLap{\gooegb !}{$\sharp$}}.

The {\tt Dg2TeX} converter writes the following three macros into each output
file to assist with overlapping: {\tt \\goLap\#1\#2}, {\tt
\\goWhiteInk\#1}, and {\tt \\goLapWhite\#1\#2}.  You may copy them
and change them to suite your needs.

{\tt \\goLap} takes two arguments and draws the second argument on top of
the first one.  For black symbols on white stones, this is trivial.

Overlapping white symbols onto black stones is more problematic as not all
output drivers support changing ink colors.  For those that do, the {\tt
\\goWhiteInk} and {\tt \\goLapWhite} macros are useful.  {\tt \\goWhiteInk}
uses {\tt \\special color} to change the ink to white while it prints its
argument.  {\tt \\goLapWhite} does the same overlap that {\tt \\goLap}
does, but it uses {\tt \\goLapWhite} when printing the second (overlapping)
argument so it can be seen against black stones.  Since not all output
drivers honor the {\tt \\special color}, you will have to experiment to see
if your tool works with this kind of overlapping ({\tt pdftex} does not
support {\tt \\special color} at this time).

Invoking {\tt \\goLapWhite} as follows.

\medbreak
{\tt\parindent=.5in\obeylines\offinterlineskip
\def\[{{\char'173}}
\def\]{{\char'175}}
\\textstone\[\\goLapWhite\[\\goo \\0??@\]\[\$\\clubsuit\$\]\]
}
\medbreak

\noindent produces this club black move: \textstone{\goLapWhite{\goo \0??@}{$\clubsuit$}}

Drawing white on black tends to suffer from the black overwhelming the white so
that a flat black move: \textstone{\goLapWhite{\goo \0??@}{$\flat$}} really
requires larger magnification to fully appreciate.  Also, any white overlap
that spills out onto the empty board will not be visible as it is also
white.

The {\tt \\textstone} macro is defined in {\tt gooemacs.tex}.  It adjusts
the stone vertically to the proper position for inclusion in the text.

\vfil\eject
\centerline{\titlefont Font Design Notes}
\vglue.18in
Since you have the \metafont\ sources to the GOOE fonts, you are free to
change the stones, and their numerals. The principal configurable files are
{\tt paramsa.mf} and {\tt paramsb.mf}. If you change the fonts, you will need
to reinstall them as described earlier in this manual.  If you reinstall the
fonts, you should be aware that {\tt MakeTeXPK} generates files with names
such as {\tt gooa.360pk} etc.\ and caches them somewhere in your system. You
will need to find and delete these cached font files, or else your changes
will not show up.

The numerals on the go stones are instances of the Computer Modern Fonts
which were created for \TeX\ by Donald Knuth. The file {\tt romandg.mf}
is the same as the file {\tt romand.mf} from the \TeX\ distribution,
with only trivial modifications to allow us to paste the numerals onto
the stones.

The Computer Modern Font family depends on certain parameters, which we set in
the file {\tt paramsb.mf}. From the single \metafont\ file {\tt romand.mf},
all of the numerals in the various fonts distributed with
\TeX\ are generated, except the italic ones, which use {\tt itald.mf} instead.
Only the parameters are changed to obtain these varied fonts.

If you wish to modify the numerals on the stones, you should edit {\tt
paramsa.mf} and {\tt paramsb.mf}, but leave {\tt romandg.mf} untouched. To get
some idea of the variety of fonts which can be created from {\tt romand.mf},
consult Knuth, {\it Computer Modern Typefaces}, Volume E in the Computers and
Typesetting series, Addison-Wesley 1986. In the section titled ``Font
Specimens'' at the back of the book, the 75 standard typefaces in the Computer
Modern family are presented.

In Go typography, such as may be found in any Go book or
magazine, one does not find modern numerals, such as these from the Computer
Modern font {\tt cmr10}: 0123456789. Instead, one finds simplified {\it sans
serif\/} fonts. These are more readable in the context of a Go diagram. We
obtained good results basing the numerals on the {\tt cmss10} font: {\gofont
0123456789}.  We changed four settings only, namely the font parameters {\tt
u} and {\tt fig\char'137height}, which control the width and height of the
digits, the parameter {\tt notch\char'137cut}, which we found had to be
increased to correct the shape of the 4 when the font was compressed
horizontally, and {\tt fine}, increasing which improves the appearance
of \textstone{\goo\002=} and possibly other numerals by fattening the
arc at its base. Consult the initial section of Knuth, {\it loc.\ cit.},
titled ``Introduction to the Parameters'' for more information about the font
parameters. The parameters must satisfy certain inequalities which are
described there. In particular, {\tt fine} cannot be increased past 
{\tt 17/36pt} unless {\tt thin\char'137join} is also increased.

Apart from Knuth's parameters in {\tt paramsb.mf}, a number of configurable
parameters are set in {\tt paramsa.mf}. The parameters {\tt oneleft}, 
{\tt twoleft}, etc.\ control the horizontal spacing of the stones. If 
{\tt setback} is set to a small nonzero number (probably no larger than 0.1)
the stones will be slightly smaller than 12 points in diameter, and a portion
of the board will be visible between them. This requires modification to {\tt
gooemacs.tex}.  The changes needed are documented in the comments to that
file. 

If the parameter {\tt smudge} is set $>0$ in {\tt paramsa.mf}, the numerals on
the Black stones are expanded by {\tt smudge} pixels in the vertical
direction. This can enhance readability at the cost of some crispness.
The default value {\tt smudge=1} optimizes readability on older
300dpi laser printers.

It would be very desirable to have Type 1 versions of these fonts,
since these are required for use with {\tt pdftex}.

% \vfil\eject
% 
% \centerline{\titlefont Problems with {\titlett xdvik}}
% \vglue.25in
% The popular X-Window \TeX\ previewer {\tt xdvi} sometimes has
% difficulty displaying the {\tt sgf2dg} fonts correctly. The
% symptom is that the stones (which should touch) seem slightly
% ``shaved'' so that the rounded edges are cut off. A screen
% shot illustrating this problem may be found at the web site
% at {\tt http://match.stanford.edu/bump/go.html}. 
% 
% There are two forks to the {\tt xdvi} development tree. One
% version, maintained by co-workers of Karl Berry, is usually
% called {\tt xdvik}. (The version of {\tt xdvi} distributed
% with current TeTeX appears to be {\tt xdvik}.) It is this
% version of {\tt xdvi} which exhibits this bug. Hopefully this
% will be corrected at a future time, but in the meantime if
% your {\tt xdvi} exhibits this problem, you can download
% the other version of {\tt xdvi} from Paul Vojta's site at
% {\tt http://math.berkeley.edu/\char'176 vojta/xdvi.html}.

\vfil\eject
\centerline{\titlefont MetaPost}
\medbreak
John Hobby's {\bf MetaPost} is a program adapted from Donald Knuth's
\metafont\ which produces encapsulated postscript graphics from a text
file. MetaPost is standard with many \TeX\ distributions, such as Te\TeX,
Web2c, Mik\TeX\ and N\TeX. It may be called {\tt mpost} or {\tt mp} on
your system. Look for the manual, called {\tt mpman.ps}.

There are two ways that you can use MetaPost to make Go diagrams.
First, you can run {\tt sgf2dg -converter Mp}. For example
\medskip
\centerline{\noindent{\tt sgf2dg -converter Mp genan-shuwa}}
\medskip
will produce two files, called {\tt genan-shuwa.mp} and
{\tt genan-shuwa.tex}. Before you run {\tt tex}, you must run
MetaPost. We will assume that MetaPost is installed on your
system as {\tt mpost} though on some systems it may be installed
as {\tt mp}. Now issue the commands
\medskip
\medbreak
{\tt\obeylines\obeyspaces
mpost genan-shuwa.mp
tex genan-shuwa
dvips -o genan-shuwa.ps genan-shuwa.dvi
}
\medskip
These will first produces the files {\tt genan-shuwa.1,
genan-shuwa.2, ... , genan-shuwa.30}. 
The second command then produces the file {\tt genan-shuwa.dvi};
finally the {\tt dvips} invocation produces {\tt genan-shuwa.ps}.
You can print the postscript file or preview it with ghostview.

The \TeX\ file for this manual 
includes the \metafont\ postscript files as follows:
\medbreak
{\tt \\epsffile\char'173genan-shuwa.1\char'175}
\medbreak\noindent
Note {\tt \\input epsf} at the beginning of the {\tt manual.tex},
which includes the {\tt epsf} package. If you want to use
{\tt pdftex}, you will have to make pdf versions of these
files using {\tt mps2eps} and {\tt epstopdf}, which you
may include using the {\tt miniltx} and {\tt graphicx} packages.

Unfortunately you probably cannot successfully preview
{\tt genan-shuwa.dvi}. The reason for this is that the
files {\tt genan-shuwa.1} etc.\ are not true encapsulated
postscript files---they do not contain the required font
information. They work perfectly well with {\tt dvips}
but not other utilities such as {\tt xdvi}. This issue
is discussed at the web page of Jon Edvardsson, who
describes the postscript files produced by MetaPost as
{\it MetaPost postscript}:

\medbreak
\centerline{\noindent{\tt http://www.ida.liu.se/\~{}joned/download/mps2eps/}}
\medbreak

A utility called {\tt mps2eps} by Jon Edvardsson which
cleverly filters a MetaPost post\-script file using
{\tt dvips} and {\tt tex} to produce a true encapsulated
postscript file. A copy of his program is included in
the {\tt sgf2mpost} directory of the distribution. If
it is installed, then you can issue the shell command

\medbreak
{\tt\obeylines\obeyspaces
for i in \`{}ls genan-shuwa.[1-9]*\`{}; do
echo \"{}processing \$i\"{};
mps2eps \$i;
mv \$i.eps \$i; 
done
}
\medbreak

\noindent
This replaces the MetaPost postscript files by true
encapsulated postscript file. If you
do this after you run {\tt mpost} and before you run {\tt tex},
then the {\tt .dvi} file that you produce can be previewed
using {\tt xdvi}.

\vglue.3in

\centerline{\titlefont SGF2MPOST}
\medbreak

{\bf Sgf2mpost} is a tool for producing Go diagrams from SGF files.
It produces first a MetaPost file, which you can run through {\tt mpost}.
The go diagrams in this document were prepared using {\tt sgf2mpost}. 
The particular commands used to prepare this document are contained in
{\tt sgf2mpost/Makefile}. For example, Figure~2 is prepared with the commands:
\medbreak
{\parindent=0pt\obeylines
{\tt sgf2mpost -i vars.sgf -o figure2.mp -r 9 -b 7 -S 0.85 -F cmbx10}
{\tt mpost figure2.mp}
}
\medbreak

This creates a file called {\tt figure2.1}. The sgf2mpost command line options
are:

\medbreak
{\tt \obeylines\parindent = 0pt
-i\ <input file>
-o\ <output file>
-s\ <start move>\ \ \ \ \ \ \ first NUMBERED move number or board location
-e\ <end move>\ \ \ \ \ \ \ \ \ last move number or board location displayed
-n\ <num>\ \ \ \ \ \ \ \ \ \ \ \ \ \ first numbered stone gets this numeral
-l\ <left column>\ \ \ \ \ \ limit portion of board printed
-r\ <right column>
-t\ <top row>
-b\ <bottom row>
-L\ <location>:<label>\ (e.g.\ -L a:K3) mark the location with this label
-T\ <location>\ \ \ \ \ \ \ \ \ mark this stone with a triangle
-S\ <scale factor>\ \ \ \ \ scaling factor (default 1.0
-F\ <font name>\ \ \ \ \ \ \ \ (e.g. \"{}cmr8\"{}) TeX font for numbers
-I\ <font name>\ \ \ \ \ \ \ \ (e.g. \"{}cmti8\"{}) TeX font for letters
-B\ <font name>\ \ \ \ \ \ \ \ TeX font for numbers > 99
-h\ \ \ \ \ \ \ \ \ \ \ \ \ \ \ \ \ \ \ \ print this message
-d\ \ \ \ \ \ \ \ \ \ \ \ \ \ \ \ \ \ \ \ print debugging traces
-v\ \ \ \ \ \ \ \ \ \ \ \ \ \ \ \ \ \ \ \ print version
}
\medbreak

The first two options are mandatory: you must specify the sgf input file
and the MetaPost output file. We suggest that you give the output file
the suffix {\tt .mp}. Running it through {\tt mpost} will produce an
encapsulated postscript file with the same prefix, and the suffix~{\tt~.1}.

If you have {\tt dvips} (included in most \TeX\ distributions), you may
include the encapsulated postscript file in your your \TeX\ document. Start
your \TeX\ file with the macro {\tt \\input epsf}. Then include the diagram
with the command {\tt \\epsffile$\{${\it filename}$\}$}. Use this document
as a model.

You can use any standard \TeX\ font in your diagrams. (Avoid cminch.) 
We use a variety of different fonts below to demonstrate the
possibilities. See Knuth, {\it Computer Modern Typefaces} for a
complete list.

To build {\tt sgf2mpost}, change to the {\tt sgf2mpost} directory
and issue the command {\tt make}.

\vfil\eject
\centerline{\titlefont SGF2MPOST Examples}
\medbreak

Over the next few pages we will give examples of how {\tt sgf2mpost}
can be used. Particularly, we will give examples using different
fonts. The {\tt sgf2mpost/Makefile} contains instructions for
making the figures in this document. Different fonts are used
for different figures. The encapsulated postscript files are
made from the {\tt sf2mpost} directory with the command
{\tt make eps}. Study the Makefile to see how the different figures
were made.

Figure~1 show a game played by GNU Go on NNGS against an opponent who chose to
give GNU a 9 stone handicap. GNU Go is black, taking 9 stones.

\bigbreak
\centerline{\hglue-20pt\epsffile{figure1.1.eps}}
\medbreak
\centerline{{\bf Figure 1}. GNU Go 3.3.12 (B) vs.\ 9 stone handicap. Moves 1--56.}

\smallskip
After 108, W resigned, having lost two corners. Up to
move 56, B made a number of strange moves, none of them
disasterous.

\smallskip\noindent
B 8. Assuming B wants to play in this area, $a$ is better.

\smallskip\noindent
B 28. B should pull out 24.

\smallskip\noindent
B 32. Better to play at 45.

\smallskip\noindent
B 34. Capturing is better, since the game move gives W a useful
forcing move at 35.

\smallskip\noindent
B 38. Good.

\smallskip\noindent
W 47. Should be at 48. W does not take the chance B has
given him, and does not get a second one.

\smallskip\noindent
B 56. Good. After this W collapsed in the upper left. We
will examine GNU Go's analysis of this move. GNU Go
values this move as worth 35.55 points; the second
highest move is 'b' valued 29.00 points, and the
third most highly valued move is 'c' worth 22.05 points.

% The reasons given for 56:
% 
% \smallskip
% {\parindent=0pt\tt\obeylines
% \ \ E14: owl attack/defend for F12
% \ \ E14: owl attack/defend for G16
% \ \ E14: 21.35 - change in territory
% \ \ E14: 0.70 - strategic effect on D13
% \ \ E14: 2.59 - strategic effect on D10
% \ \ E14: 3.56 - strategic effect on H16
% \ \ E14: 6.92 - strategic effect on F19
% \ \ E14:   0.00 - total followup value, added 0.00 as territorial followup
% \ \ E14: 0.41 - connects strings (connect value 4, shape factor 0.082)
% Move generation values E14 to 35.54
% }
% \smallskip

\vglue-5pt
\hbox{
\vbox{\hsize=3in\vglue10pt\hbox{\epsffile{figure2.1.eps}}
\hbox{\hsize=3in{\bf Figure 2.} An OWL variation.}}\hglue.65in
\vbox{\hsize=200pt\tolerance=10000
GNU Go's OWL code takes 639 variations to conclude that G16 can be attacked at
E14, and 409 variations to conclude that it can be defended at G13. A human
player would look at far fewer variations but probably come to the
same conclusion. A typical variation considered by GNU Go is given at
left.

B 60. Strange shape. Simply extending to 61 would be the ordinary move.

W 63. W should simply live at 65 or 66.

W 71. Unfortunately W does not really have sente.

B 80. Good.

B 90. The top is more urgent.

W 91. W should live at 92.

B 94. Better at 99 for a simple resolution.

B 106. Better at S19 to avoid ko.

W 107. If W plays at S19, he gets a ko.

After 108, W resigns.
}}

\bigbreak
\hbox{\vbox{\hsize=1.15in\vglue 2in\noindent{\bf Figure 3.} 57--108.
B 70 connects.
}
\vbox{\epsffile{figure3.1.eps}}}
\medbreak

\bigbreak
\centerline{\hglue-20pt\epsffile{figure4.1.eps}}
\medbreak
\centerline{{\bf Figure 4}. 1-155. 93 at 75, 96 at 90, 99 at 75 and 144 at 31.}

\bigbreak
Figure 4 shows another game. GNU Go played poorly until move 156,
then came to life.

{\parskip=8pt
B 26. Must be at B15.

B 42. Even in isolation this is bad. B must connect at K5.

B 60. Bad tenuki.

B 80. Having played well on the bottom, B must continue at L2.

B 92. Pointless.

B 102. Due to an OWL mistake, B thinks the corner is alive. It can of
course be attacked at S1. B sees that R1 threatens O1, and due to
a missing pattern believes there is another half eye around O5.

B 120 and 126. Pointless moves.

By move 155 W must think he has a won game.
}

\bigbreak
\centerline{\hglue-20pt\epsffile{figure5.1.eps}}
\medbreak
\centerline{{\bf Figure 5}. 156--168. B defends the center group by attacking.}

\bigbreak

\centerline{\hglue-20pt\epsffile{figure6.1.eps}}
\medbreak
\centerline{{\bf Figure 6}. 169--186. The 158--159 exchange fills a crucial
liberty for W.}

\centerline{\hglue-20pt\epsffile{figure7.1.eps}}
\medbreak
\centerline{{\bf Figure 7}. 187--202. Ignoring 187, B fights and wins a ko.}
\bigbreak

\centerline{\hglue-20pt\epsffile{figure8.1.eps}}
\medbreak
\centerline{{\bf Figure 8}. 203--218. 207 connects. W resigns.}

\centerline{\hglue-20pt\epsffile{figure9.1.eps}}
\medbreak
\centerline{{\bf Figure 9}. From another game. After 1, GNU Go will
capture one of the four marked strings.}
\bigbreak

{\hfil\epsffile{figure10.1.eps}\hfil\epsffile{figure10a.1.eps}}
\medbreak
\centerline{{\bf Figure 10}. A 13x13 game. GNU Go 3.3.12 is white. 
Black passes at moves 3 and 5.}

\vfil\eject

\vglue1in
\centerline{\titlefont No Warranty}
\medbreak

With the exception of the files {\tt romandg.mf} and {\tt itallg.mf}, 
which are part of the Computer Modern Font sources and therefore copyrighted
by Donald Knuth, all \metafont\ sources and \TeX\ macros included herewith, as
well as the computer programs {\tt sgf2dg} and {\tt sgfsplit} are
published under the GNU General Public License, version 2 (GPL). 
A copy of this license is contained in the file {\tt COPYING}. If for some
reason your distribution is missing this file, consult the Free Software
foundation at:
\smallskip
\centerline{\tt http://www.gnu.org/licenses/gpl.html}
\smallskip\noindent
or write to them at The Free Software Foundation, Inc., 59 Temple Place ---
Suite 330, Boston, MA 02111-1307, USA.

The GPL permits free redistribution of software published under its terms,
provided the source code to that software, and the source to any published
software derived from it is always provided or made available. Software
published under the GPL comes with {\bf absolutely no warranty}. Therefore
{\bf nothing in this release, including but not limited to the \metafont\
sources and the programs {\tt sgf2dg} and {\tt sgfsplit} comes with any
warranty whatsoever.}

\vfil\eject
\bigbreak
\centerline{\titlefont Test Pattern}
\bigbreak

The purpose of this page is to confirm that the fonts are working
correctly. Testing, \textstone{\goo\001=}, \textstone{\goo\002=}, 
\textstone{\goo\003=}, ...

\bigbreak
\gooa
\line{\char1\char2\char3\char4\char5\char6\char7\char8\char9
\char10\char11\char12\char13\char14\char15\char16\char17\char18
\char19\char20\char21\char22\char23\char24\char25\hfil}
\line{\char26\char27\char28\char29\char30\char31\char32\char33\char34
\char35\char36\char37\char38\char39\char40\char41\char42\char43
\char44\char45\char46\char47\char48\char49\char50\hfil}
\line{\char51\char52\char53\char54\char55\char56\char57\char58\char59
\char60\char61\char62\char63\char64\char65\char66\char67\char68
\char69\char70\char71\char72\char73\char74\char75\hfil}
\line{\char76\char77\char78\char79\char80\char81\char82\char83\char84
\char85\char86\char87\char88\char89\char90\char91\char92\char93
\char94\char95\char96\char97\char98\char99\goob\char0}
\goob
\line{\char1\char2\char3\char4\char5\char6\char7\char8
\char9\char10\char11\char12\char13\char14\char15\char16\char17
\char18\char19\char20\char21\char22\char23\char24\char25\hfil}
\line{\char26\char27\char28\char29\char30\char31\char32\char33\char34
\char35\char36\char37\char38\char39\char40\char41\char42\char43
\char44\char45\char46\char47\char48\char49\char50\hfil}
\line{\char51\char52\char53\char54\char55\char56\char57\char58\char59
\char60\char61\char62\char63\char64\char65\char66\char67\char68
\char69\char70\char71\char72\char73\char74\char75\hfil}
\line{\char76\char77\char78\char79\char80\char81\char82\char83\char84
\char85\char86\char87\char88\char89\char90\char91\char92\char93
\char94\char95\char96\char97\char98\char99\gooc\char0}
\gooc
\line{\char1\char2\char3\char4\char5\char6\char7\char8\char9
\char10\char11\char12\char13\char14\char15\char16\char17\char18
\char19\char20\char21\char22\char23\char24\char25\hfil}
\line{\char26\char27\char28\char29\char30\char31\char32\char33\char34
\char35\char36\char37\char38\char39\char40\char41\char42\char43
\char44\char45\char46\char47\char48\char49\char50\hfil}
\line{\char51\char52\char53\char54\char55\char56\char57\char58\char59
\char60\char61\char62\char63\char64\char65\char66\char67\char68
\char69\char70\char71\char72\char73\char74\char75\hfil}
\line{\char76\char77\char78\char79\char80\char81\char82\char83\char84
\char85\char86\char87\char88\char89\char90\char91\char92\char93
\char94\char95\char96\char97\char98\char99\good\char0}
\good
\line{\char1\char2\char3\char4\char5\char6\char7\char8\char9
\char10\char11\char12\char13\char14\char15\char16\char17\char18
\char19\char20\char21\char22\char23\char24\char25\hfil}
\line{\char26\char27\char28\char29\char30\char31\char32\char33\char34
\char35\char36\char37\char38\char39\char40\char41\char42\char43
\char44\char45\char46\char47\char48\char49\char50\hfil}
\line{\char51\char52\char53\char54\char55\char56\char57\char58\char59
\char60\char61\char62\char63\char64\char65\char66\char67\char68
\char69\char70\char71\char72\char73\char74\char75\hfil}
\line{\char76\char77\char78\char79\char80\char81\char82\char83\char84
\char85\char86\char87\char88\char89\char90\char91\char92\char93
\char94\char95\char96\char97\char98\char99\gooegb;}

\medbreak

\line{\gool\401=\402=\403=\404=\405=\406=\407=\408=\409=\410=\411=\412=\413=\414=\415=\416=\417=\418=\419=\420=\421=\422=\423=\424=\425=\426=\hfil}
\line{\gool\501=\502=\503=\504=\505=\506=\507=\508=\509=\510=\511=\512=\513=\514=\515=\516=\517=\518=\519=\520=\521=\522=\523=\524=\525=\526=\hfil}

\medbreak


\goea
\line{\char1\char2\char3\char4\char5\char6\char7\char8\char9
\char10\char11\char12\char13\char14\char15\char16\char17\char18
\char19\char20\char21\char22\char23\char24\char25\hfil}
\line{\char26\char27\char28\char29\char30\char31\char32\char33\char34
\char35\char36\char37\char38\char39\char40\char41\char42\char43
\char44\char45\char46\char47\char48\char49\char50\hfil}
\line{\char51\char52\char53\char54\char55\char56\char57\char58\char59
\char60\char61\char62\char63\char64\char65\char66\char67\char68
\char69\char70\char71\char72\char73\char74\char75\hfil}
\line{\char76\char77\char78\char79\char80\char81\char82\char83\char84
\char85\char86\char87\char88\char89\char90\char91\char92\char93
\char94\char95\char96\char97\char98\char99\goeb\char0}
\goeb
\line{\char1\char2\char3\char4\char5\char6\char7\char8
\char9\char10\char11\char12\char13\char14\char15\char16\char17
\char18\char19\char20\char21\char22\char23\char24\char25\hfil}
\line{\char26\char27\char28\char29\char30\char31\char32\char33\char34
\char35\char36\char37\char38\char39\char40\char41\char42\char43
\char44\char45\char46\char47\char48\char49\char50\hfil}
\line{\char51\char52\char53\char54\char55\char56\char57\char58\char59
\char60\char61\char62\char63\char64\char65\char66\char67\char68
\char69\char70\char71\char72\char73\char74\char75\hfil}
\line{\char76\char77\char78\char79\char80\char81\char82\char83\char84
\char85\char86\char87\char88\char89\char90\char91\char92\char93
\char94\char95\char96\char97\char98\char99\goec\char0}
\goec
\line{\char1\char2\char3\char4\char5\char6\char7\char8\char9
\char10\char11\char12\char13\char14\char15\char16\char17\char18
\char19\char20\char21\char22\char23\char24\char25\hfil}
\line{\char26\char27\char28\char29\char30\char31\char32\char33\char34
\char35\char36\char37\char38\char39\char40\char41\char42\char43
\char44\char45\char46\char47\char48\char49\char50\hfil}
\line{\char51\char52\char53\char54\char55\char56\char57\char58\char59
\char60\char61\char62\char63\char64\char65\char66\char67\char68
\char69\char70\char71\char72\char73\char74\char75\hfil}
\line{\char76\char77\char78\char79\char80\char81\char82\char83\char84
\char85\char86\char87\char88\char89\char90\char91\char92\char93
\char94\char95\char96\char97\char98\char99\goed\char0}
\goed
\line{\char1\char2\char3\char4\char5\char6\char7\char8\char9
\char10\char11\char12\char13\char14\char15\char16\char17\char18
\char19\char20\char21\char22\char23\char24\char25\hfil}
\line{\char26\char27\char28\char29\char30\char31\char32\char33\char34
\char35\char36\char37\char38\char39\char40\char41\char42\char43
\char44\char45\char46\char47\char48\char49\char50\hfil}
\line{\char51\char52\char53\char54\char55\char56\char57\char58\char59
\char60\char61\char62\char63\char64\char65\char66\char67\char68
\char69\char70\char71\char72\char73\char74\char75\hfil}
\line{\char76\char77\char78\char79\char80\char81\char82\char83\char84
\char85\char86\char87\char88\char89\char90\char91\char92\char93
\char94\char95\char96\char97\char98\char99\gooegb:}



%\bye
% Make the small-font test page below by commenting out the above \bye.

\vfil\eject
{\centerline{\titlefont Small Fonts}
\bigbreak

\rm
The purpose of this page is to test the fonts in a smaller size. These
fonts are used in the two-column format. They may be smaller than you may ever
need. If you need fonts at this size on a 300dpi device, you may wish to
consider setting {\tt smudge} to 1 or greater in {\tt paramsa.mf}. (This
is the current default.) On the other hand if you have a high resolution
device {\tt smudge:=0} might give a crisper font. If you change 
{\tt paramsa.mf} or {\tt paramsb.mf} you must reinstall the fonts.

\setGoFonts at 8.333pt % set smaller fonts
\offinterlineskip
\bigbreak
\gooa
\line{\char1\char2\char3\char4\char5\char6\char7\char8\char9
\char10\char11\char12\char13\char14\char15\char16\char17\char18
\char19\char20\char21\char22\char23\char24\char25\hfil}
\line{\char26\char27\char28\char29\char30\char31\char32\char33\char34
\char35\char36\char37\char38\char39\char40\char41\char42\char43
\char44\char45\char46\char47\char48\char49\char50\hfil}
\line{\char51\char52\char53\char54\char55\char56\char57\char58\char59
\char60\char61\char62\char63\char64\char65\char66\char67\char68
\char69\char70\char71\char72\char73\char74\char75\hfil}
\line{\char76\char77\char78\char79\char80\char81\char82\char83\char84
\char85\char86\char87\char88\char89\char90\char91\char92\char93
\char94\char95\char96\char97\char98\char99\goob\char0}
\goob
\line{\char1\char2\char3\char4\char5\char6\char7\char8
\char9\char10\char11\char12\char13\char14\char15\char16\char17
\char18\char19\char20\char21\char22\char23\char24\char25\hfil}
\line{\char26\char27\char28\char29\char30\char31\char32\char33\char34
\char35\char36\char37\char38\char39\char40\char41\char42\char43
\char44\char45\char46\char47\char48\char49\char50\hfil}
\line{\char51\char52\char53\char54\char55\char56\char57\char58\char59
\char60\char61\char62\char63\char64\char65\char66\char67\char68
\char69\char70\char71\char72\char73\char74\char75\hfil}
\line{\char76\char77\char78\char79\char80\char81\char82\char83\char84
\char85\char86\char87\char88\char89\char90\char91\char92\char93
\char94\char95\char96\char97\char98\char99\gooc\char0}
\gooc
\line{\char1\char2\char3\char4\char5\char6\char7\char8\char9
\char10\char11\char12\char13\char14\char15\char16\char17\char18
\char19\char20\char21\char22\char23\char24\char25\hfil}
\line{\char26\char27\char28\char29\char30\char31\char32\char33\char34
\char35\char36\char37\char38\char39\char40\char41\char42\char43
\char44\char45\char46\char47\char48\char49\char50\hfil}
\line{\char51\char52\char53\char54\char55\char56\char57\char58\char59
\char60\char61\char62\char63\char64\char65\char66\char67\char68
\char69\char70\char71\char72\char73\char74\char75\hfil}
\line{\char76\char77\char78\char79\char80\char81\char82\char83\char84
\char85\char86\char87\char88\char89\char90\char91\char92\char93
\char94\char95\char96\char97\char98\char99\good\char0}
\good
\line{\char1\char2\char3\char4\char5\char6\char7\char8\char9
\char10\char11\char12\char13\char14\char15\char16\char17\char18
\char19\char20\char21\char22\char23\char24\char25\hfil}
\line{\char26\char27\char28\char29\char30\char31\char32\char33\char34
\char35\char36\char37\char38\char39\char40\char41\char42\char43
\char44\char45\char46\char47\char48\char49\char50\hfil}
\line{\char51\char52\char53\char54\char55\char56\char57\char58\char59
\char60\char61\char62\char63\char64\char65\char66\char67\char68
\char69\char70\char71\char72\char73\char74\char75\hfil}
\line{\char76\char77\char78\char79\char80\char81\char82\char83\char84
\char85\char86\char87\char88\char89\char90\char91\char92\char93
\char94\char95\char96\char97\char98\char99\gooegb;\hfil}

\medbreak
\line{\gool\401=\402=\403=\404=\405=\406=\407=\408=\409=\410=\411=\412=\413=\414=\415=\416=\417=\418=\419=\420=\421=\422=\423=\424=\425=\426=\hfil}
\line{\gool\501=\502=\503=\504=\505=\506=\507=\508=\509=\510=\511=\512=\513=\514=\515=\516=\517=\518=\519=\520=\521=\522=\523=\524=\525=\526=\hfil}

\medbreak

\goea
\line{\char1\char2\char3\char4\char5\char6\char7\char8\char9
\char10\char11\char12\char13\char14\char15\char16\char17\char18
\char19\char20\char21\char22\char23\char24\char25\hfil}
\line{\char26\char27\char28\char29\char30\char31\char32\char33\char34
\char35\char36\char37\char38\char39\char40\char41\char42\char43
\char44\char45\char46\char47\char48\char49\char50\hfil}
\line{\char51\char52\char53\char54\char55\char56\char57\char58\char59
\char60\char61\char62\char63\char64\char65\char66\char67\char68
\char69\char70\char71\char72\char73\char74\char75\hfil}
\line{\char76\char77\char78\char79\char80\char81\char82\char83\char84
\char85\char86\char87\char88\char89\char90\char91\char92\char93
\char94\char95\char96\char97\char98\char99\goeb\char0}
\goeb
\line{\char1\char2\char3\char4\char5\char6\char7\char8
\char9\char10\char11\char12\char13\char14\char15\char16\char17
\char18\char19\char20\char21\char22\char23\char24\char25\hfil}
\line{\char26\char27\char28\char29\char30\char31\char32\char33\char34
\char35\char36\char37\char38\char39\char40\char41\char42\char43
\char44\char45\char46\char47\char48\char49\char50\hfil}
\line{\char51\char52\char53\char54\char55\char56\char57\char58\char59
\char60\char61\char62\char63\char64\char65\char66\char67\char68
\char69\char70\char71\char72\char73\char74\char75\hfil}
\line{\char76\char77\char78\char79\char80\char81\char82\char83\char84
\char85\char86\char87\char88\char89\char90\char91\char92\char93
\char94\char95\char96\char97\char98\char99\goec\char0}
\goec
\line{\char1\char2\char3\char4\char5\char6\char7\char8\char9
\char10\char11\char12\char13\char14\char15\char16\char17\char18
\char19\char20\char21\char22\char23\char24\char25\hfil}
\line{\char26\char27\char28\char29\char30\char31\char32\char33\char34
\char35\char36\char37\char38\char39\char40\char41\char42\char43
\char44\char45\char46\char47\char48\char49\char50\hfil}
\line{\char51\char52\char53\char54\char55\char56\char57\char58\char59
\char60\char61\char62\char63\char64\char65\char66\char67\char68
\char69\char70\char71\char72\char73\char74\char75\hfil}
\line{\char76\char77\char78\char79\char80\char81\char82\char83\char84
\char85\char86\char87\char88\char89\char90\char91\char92\char93
\char94\char95\char96\char97\char98\char99\goed\char0}
\goed
\line{\char1\char2\char3\char4\char5\char6\char7\char8\char9
\char10\char11\char12\char13\char14\char15\char16\char17\char18
\char19\char20\char21\char22\char23\char24\char25\hfil}
\line{\char26\char27\char28\char29\char30\char31\char32\char33\char34
\char35\char36\char37\char38\char39\char40\char41\char42\char43
\char44\char45\char46\char47\char48\char49\char50\hfil}
\line{\char51\char52\char53\char54\char55\char56\char57\char58\char59
\char60\char61\char62\char63\char64\char65\char66\char67\char68
\char69\char70\char71\char72\char73\char74\char75\hfil}
\line{\char76\char77\char78\char79\char80\char81\char82\char83\char84
\char85\char86\char87\char88\char89\char90\char91\char92\char93
\char94\char95\char96\char97\char98\char99\gooegb:}
}
\bye
