\documentclass{report}

\usepackage[a4paper]{geometry}
\usepackage{a4wide}
\usepackage{makeidx}
\makeindex
\usepackage{graphicx}
\usepackage{titlesec}
%%header

\begin{document}

\titleformat{\chapter}[display]
{}{}{0pc}
{
\Huge\bfseries
\ifnum \thechapter>0
  \thechapter\space\space\space
\fi
}

\author{Dmitry Karasik}
\title{Prima - the perl graphic toolkit}
\maketitle
\thispagestyle{empty}
\clearpage

\tableofcontents
\sloppy

\chapter{Introduction}

\subsubsection{Preface}

Prima is an extensible Perl toolkit for multi-platform GUI development.
Platforms supported include Linux, Windows NT/9x/2K, and UNIX/X11
workstations (FreeBSD, IRIX, SunOS, Solaris and others).
The toolkit contains a rich set of standard widgets and has emphasis on 2D
image processing tasks. A Perl program using PRIMA looks and behaves identically
on X11 and Win32.

The Prima project was started in 1997 in Protein Laboratory, Copenhagen,
by Anton Berezin, Dmitry Karasik, and Vadim Belman.

This document describes programming with Prima graphic toolkit, and is a
collection of manual pages of Prima application program interface ( API ),
written by D.Karasik, except Prima::IniFile and Prima::ScrollBar, written
by A.Berezin.

\subsubsection{Requirements}

Prima supports perl versions 5.004 and above. The recommended perl
versions are 5.005 and above. In UNIX(tm) environments, Prima can use
the following graphic libraries: libjpeg, libgif, libtiff, libpng, libXpm, libwebp.

\subsubsection{Installation}

The toolkit can be downloaded from \texttt{http://www.prima.eu.org} in
source and binary forms. Before installing, check the content of README file
in the distribution. The installation from the source is performed by
executing commands
\begin{verbatim}
  perl Makefile.PL
  make
  make test
  make install
\end{verbatim}

There is a mailing list dedicated for various Prima-related discussions,
prima@prima.eu.org. This list is also a proper place to send bug reports to.
To subscribe to the list, send mail to \texttt{<majordomo@prima.eu.org>} and
include \texttt{subscribe prima <optional address>} in the body of your message.

\subsubsection{Authors}

Dmitry Karasik,
Anton Berezin,
Vadim Belman

\subsubsection{Credits}

David Scott,
Kai Fiebach,
Johannes Blankenstein,
Teo Sankaro,
Mike Castle,
H.Merijn Brand,
Richard Morgan
Kevin Ryde
Chris Marshall
Slaven Rezic
Waldemar Biernacki
Andreas Hernitscheck
David Mertens
Teo Sankaro
Gabor Szabo
Fabio D'Alfonso
Rob "Sisyphus"
Chris Marshall
Reini Urban
Nadim Khemir
Vikas N Kumar
Upasana Shukla
Sergey Romanov
Mathieu Arnold
Petr Pisar
Judy Hawkins
Myra Nelson
Sean Healy
Ali Yassen
Maximilian Lika

-- thank you for your help.

\subsubsection{Copyright}

(c) 1997-2003 The Protein Laboratory, University of Copenhagen
(c) 1997-2019 Dmitry Karasik

