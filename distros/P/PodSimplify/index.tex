% This LaTeX-file was created by <(null)> Wed Feb 14 21:39:30 1996
% LyX 0.8 (C) 1995 by Matthias Ettrich
\nonstopmode
\documentclass[10pt,letterpaper,oneside,onecolumn]{report}
\usepackage{lyx}
\pagestyle{plain}
\setcounter{secnumdepth}{3}
\setcounter{tocdepth}{3}
\lyxletterstyle
\begin{document}

\noindent This
is
a
working
example
of
how
the
POD
\ttfamily X\(<\)\(>\)
\rmfamily command
works.

\begin{lyxcode}
Foo
X\(<\)bar\(>\).


\end{lyxcode}
Produces
the
output:

\begin{lyxcode}
Foo
bar.


\end{lyxcode}
With
the
index
statement
``bar''.
\lyxline{\normalsize}

\begin{lyxcode}
Foo
X\(<\)I\(<\)bar\(>\)\(>\).

\end{lyxcode}
Produces
the
output:

\begin{lyxcode}
Foo
\em bar\em .

\end{lyxcode}
\noindent With
the
index
statement
``bar''.
\lyxline{\normalsize}

\begin{lyxcode}
Foo
X\(<\)functions/seek\(>\).

\end{lyxcode}
Produces
the
output:

\begin{lyxcode}
Foo
seek.

\end{lyxcode}
With
the
index
statement
``functions/seek''
\lyxline{\normalsize}

\begin{lyxcode}
Foo
X\(<\)seek;func/lseek;func/fseek\(>\).

\end{lyxcode}
Produces
the
output:

\begin{lyxcode}
Foo
seek.

\end{lyxcode}
With
the
index
statements
``func/lseek''
\&
``func/fseek''.
\lyxline{\normalsize}

\begin{lyxcode}
Foo
X\(<\)I\(<\)-J\(>\)\(>\).

\end{lyxcode}
Produces
the
output:

\begin{lyxcode}
Foo
\em -J\em .

\end{lyxcode}
With
the
index
statement
``-J''.
\lyxline{\normalsize}

\end{document}
