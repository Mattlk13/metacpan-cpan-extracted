% Template for a long, formal document.  Be sure to run latex twice, 
% or the table of contents will be correct.  Modify the definition and
% content sections.


%%%%%%%%%%%%%%%% Add definitions here %%%%%%%%%%%%%%%%%%%%%%%
\def\title{Decompression Theory}
\def\version{1.0}
\def\author{Jaap Voets}
\def\date{08-05-2006}
%%%%%%%%%%%%%%%% End definition section %%%%%%%%%%%%%%%%%%%%%

\documentclass[11pt]{report}
%\usepackage{a4}
\usepackage[pdftex]{graphicx}
\usepackage{xspace}
\usepackage{url}
%\usepackage{fancyhdr}
\usepackage[usenames]{color}
\usepackage{eurosym}
%\newcommand{\blinck}{Blinck International BV\xspace}

%%%% Headers %%%%
%\pagestyle{fancyplain}
%\lhead[{\includegraphics[scale=0.05]{images/blincklogo.png}}]{\small \title}
%\rhead[{\small \title}]{\includegraphics[scale=0.05]{images/blincklogo.png}}
%\fancyfoot{} % Clear out the existing page number
%\lfoot{{\slshape Version \version, \date}} 
%\rfoot{\thepage}
%\renewcommand{\headrulewidth}{0.4pt}
%\renewcommand{\footrulewidth}{0.4pt}

%%%% Title page %%%%
\begin{document}
\thispagestyle{empty}
%\includegraphics[scale=0.3]{images/blincklogo.png} \vfill
\begin{flushright}
\LARGE \title\\ \normalsize \vspace{0.3in} \author \vfill
\end{flushright}
\begin{flushleft}
\scriptsize \hrule \smallskip \date\hfill Version \version\\
\end{flushleft}\normalsize


%%%%%%%%%%%%%%%%%%%%%%%%%%% Begin Content %%%%%%%%%%%%%%%%%%%%%%%%%%%%
\chapter{Decompression theory}
This document accompanies the perl module Deco.pm. It will give you a basic background on the Haldanian model used in the module.
Many thanks go to the authors of www.deepocean.net who created the text from which I borrowed heavily.

\section{History}

 In modern diving, tables and schedules are common for estimating no-decompression limits, decompression profiles  and saturation levels. 
 Use of a diving computer during the dive is most common nowadays. Tables and computers are based on decompression theory, which describes inert 
 gas uptake and saturation of bodily tissue when breathing compressed air (or other gas mixtures). The development of this theory was started in 
 1908 by John Scott Haldane c.s. Haldane, an English physiologist, described the Nitrogen saturation process by using a body model which comprises 
 several hypothetical tissue 'compartments'.\\
 
A compartment can be characterized by a variable called 'half-time', which is a measure for inert gas 
 uptake. Theory was further developed during the years '50 and '60 by U.S. Navy. The concept of 'M-values' was developed by Robert D. Workman of 
 the U.S. Navy Experimental Diving Unit (NEDU). In the early '70s Schreiner applied the theory to changing pressure (ascending/descending). 
 Recently B�hlmann improved the theory and developed ZH-12L and ZH-16L model, which are quite popular in current diving computers. 
 At this moment a lot is still unknown about the exact processes which take place during saturation and decompression. 
 Most of the theory presented has been found empirically, i.e. by performing tests on human subjects in decompression chambers and from 
 decompression accident statistics. Recently, a more physical approach resulted in bubble theories. These theories physically describe what 
 is happening during decompression.
 
\section{Inert gas saturation and supersaturation}
When you breathe a breathing gas that contains an inert gas (gasses which do not take part in the oxidative metabolism and are not 'used' by the body) 
like Nitrogen (N2) or Helium (He2), this gas is dissolved in the blood by means of gas exchange in the lungs. Blood takes the dissolved gas to the 
rest of the bodily tissue. Tissue takes up dissolved gas from the blood. Gas keeps on dissolving in blood and tissue until the partial pressure 
of the dissolved gas is equal to the partial pressure of the gas breathed in, throughout the entire body. This is called saturation. 
Rates of saturation vary with different parts of the body. 
The nervous system and spine get saturated very fast (fast tissues), whereas fat and bones saturate very slowly 
(slow tissues).\\

When staying at sea level for a long time, like most of us do, and breathing air, again like most of us do, the entire body is saturated with 
Nitrogen, which makes up the air for 78\%. Since at sea level air pressure is roughly 1 bar (we can neglect barometric air pressure variations, 
which are expressed in milli-bar), the partial pressure of the dissolved Nitrogen thoughout the entire body is 1 bar * 78\%=0.78 bar (actually it 
is a bit less, as we will see later, but for the moment this will do). If a diver dives to 20 meter, he breathes air at 3 bar. Partial Nitrogen 
pressure in the air he breathes is 3 bar * 78\% = 2.34 bar. If the diver sits down and wait for quite a long time, the diver's body gets saturated 
with Nitrogen at a partial pressure of 2.34 bar (the fastest tissues saturate in 25 minutes, the slowest take two and a half day to saturate). 
So far, so good (at least if our diver has enough air supply).\\

If the diver goes back to the surface, however, he arrives with his body saturated 
with Nitrogen at a partial 2.34 bar pressure, whereas the air he breathes at the surface has a partial Nitrogen pressure of 0.78 bar. The body 
is supersaturated. Dissolved Nitrogen in the tissue and blood will go back to the free gas phase, in order to equalize the pressures. 
The Nitrogen forms micro bubbles, which are transported by the blood and removed from the body by the respiratory system. However, if to much 
Nitrogen goes back to the free phase, micro bubbles grow and form bubbles that may block veins and arteries. The diver gets bent and will develop 
decompression sickness (DCS) symptoms.\\

A certain amount of supersaturation is allowed, without getting bent (at least with low risk of getting bent). In fact, supersaturation 
(a pressure gradient) is needed in order to decompress (get rid of the excess Nitrogen). The amount of allowed supersaturation is different 
for various types of tissue. This is the reason the body is divided in hypothetical tissue 'compartments' in most decompression models. 
Each compartment is characterized by its half-time. This is the period the tissue takes a partial inert gas pressure which is half way between 
the partial pressure before and after a pressure change of the environment. Haldane suggested two compartments, recent theories (like ZH-16L) 
use up to 16 compartments. Decompression theory deals with two items:\\

\begin{itemize}
\item Modeling the inert gas absorption in the bodily tissue
\item Estimate limits of supersaturation for each tissue, beyond which decompression sickness (DCS) symptoms develop
\end{itemize}
If these items are known, one can fill in tables, estimate no-decompression times, calculate decompression profiles, plan dives, etc. 

\section{Calculating inert gas absorption by the tissue}

In this section we will derive the equations which describe inert gas uptake by bodily tissue. If you get frightened by a bit of mathematics, 
please skip the derivation, but have a look at the end result in the high lighted boxes (equation (\ref{haldane}) and (\ref{schreiner})). 
The rate a particular tissue (compartment) takes up inert gas (i.e. the rate of change in partial pressure of that gas in the tissue) is 
proportional to the partial pressure difference between the gas in the lungs and the dissolved gas in the tissue. We can express this 
mathematically by:
\begin{equation}  	
\frac{d}{{dt}}P_t \left( t \right) =  k  [ P_{alv}  \left( t \right)  - P_t  \left( t \right) ] 
\end{equation}  

where
\begin{itemize}
\item $P_t\left(t\right)$ Partial pressure of the gas in the particular tissue ($bar$)
\item $P_{alv}\left(t\right)$ Partial pressure of the gas in the breathing mix. To be precise: 
Gas exchange takes place in the lungs (alveoli). Hence, we have to consider the gas alveolar partial pressure. 
This pressure may be changing with time, if the diver changes depth ($bar$)
\item $k$ A constant depending on the type of tissue ($min^{-1}$)
\item $t$ Time ($min$)
\end{itemize}

This is a differential equation which is quite familiar in physics and apply to many processes like diffusion and 
heat transfer. Solving this equation requires following steps:

\begin{enumerate}
\item Write the equation to the familiar form (a.k.a. the inhomogenous differential equation):

\begin{equation}  	
\frac{d}{{dt}}P \left( t \right) +  k  P_t \left( t \right) = k P_{alv}  \left( t \right)
\label{haldane01}
\end{equation}  


\item Solve the homogenous equation (\ref{haldane02}) by trying $P_{th}\left(t\right) = C_0e^{-\lambda t}$ and solve for $\lambda$.
\begin{equation}
\frac{d}{{dt}}P_{th} \left( t \right) +  k  P_{th} \left( t \right) = 0
\label{haldane02}
\end{equation}

The 'h' in $P_{th}$ denotes that we are dealing with the homogenous equation. Substituting the solution in (\ref{haldane02}) 
results in $\lambda = k$. So the homogenous solution of the equation (\ref{haldane02}) is:
\begin{equation}
P_{th}  \left( t \right)   =  C_0 e^{-kt}
\label{haldane03}
\end{equation}

\item Find the particular solution for the inhomogenous equation (\ref{haldane01}) and solve constants using boundary conditions. In order to solve this equation, 
we have to know more about the partial gas pressure Palv(t). Two useful situations described in literature are a 
situation in which $P_{alv}\left(t\right)$ is constant (corresponding to remaining at a certain depth) and a 
situation in which $P_{alv}\left(t\right)$ varies linearly with time (corresponding to ascending/descending with 
constant speed). We will have a look at both situations.

\subsection{constant ambient pressure}

We look at the situation in which the alveolar partial pressure of the gas remains constant: $P_{alv}\left(t\right) = P_{alv0}$.
This corresponds to a diving situation in which the diver remains at a certain depth. Equation (\ref{haldane01}) becomes:

\begin{equation}
\frac{d}{{dt}}P_{t} \left( t \right) +  k  P_{t} \left( t \right) = k P_{alv0}
\label{haldane04}
\end{equation}

We 'try' the solution:
  
\begin{equation}
P_{t}  \left( t \right)   =  C_0 e^{-kt} + C_1
\label{haldane05}
\end{equation}	

If we subsitute solution (\ref{haldane05}) in equation (\ref{haldane04}) the e's cancel out and we are left with $C_1 = P_{alv0}$. 
Now we have to think of a boundary condition, in order to find $C_0$. We assume some partial pressure in the tissue 
$P_t\left(0\right) = P_{t0}$ at $t=0$. If we substitute this into equation (\ref{haldane05}) we find that 
$C_0 = [P_{t0} - P_{alv0} ]$. So we are left with the following equation for the partial pressure in a 
specific type of tissue (characterized by the constant $k$):

  
\begin{equation}
P_{t}  \left( t \right)   =  P_{alv0} + [P _{t0} - P_{alv0}]  e^{-kt} 
\label{haldane06}
\end{equation}

where

\begin{itemize}
\item $P_t\left(t\right)$ Partial pressure of the gas in the tissue ($bar$)
\item $P_{t0}$ Initial partial pressure of the gas in the tissue at t=0 ($bar$)
\item $P_{alv0}$ Constant partial pressure of the gas in the breathing mix in the alveoli ($bar$)
\item $k$ A constant depending on the type of tissue ($min^{-1}$)
\item $t$ Time ($min$)
\end{itemize}

This equation is known in literature as the {\bf Haldane equation}. We can rewrite it a bit so that it corresponds to 
a form which is familiar in decompression theory literature:
  	
\begin{equation}
P_{t}  \left( t \right)   =  P_{t0} + [ P_{alv0} - P _{t0} ]  [ 1 - e^{-kt} ]
\label{haldane}
\end{equation}

	
\subsection{linearly varying ambient pressure}

Very few divers plunge into the deep and remain at a certain depth for a long time. For that reason we will 
look at the situation in which the diver ascends or descends with constant speed. This means the partial 
pressure of the gas he breathes varies linearly with time. Going back to equation (\ref{haldane01}) this means $P_{alv}$ can be 
writen as $P_{alv} = P_{alv0} + Rt P_{alv0}$ is the initial partial pressure of the gas in the breathing mixture at t=0, 
and R is the change rate (in bar/minute) of the partial pressure of this gas in the alveoli. Note: R is positive 
for descending (pressure increase) and negative for ascending (pressure decrease). Substituting this in (\ref{haldane01}) gives us:
  
\begin{equation}
\frac{d}{{dt}}P_{t} \left( t \right) +  k  P_{t} \left( t \right) = k P_{alv0} + k R t
\label{schreiner01}
\end{equation}	

We 'try' the solution:
  	
\begin{equation}
P_{t}  \left( t \right)   =  C_0 e^{-kt} + C_1 t + C_2
\label{schreiner02}
\end{equation}

Substituting solution (\ref{schreiner02}) in equation (\ref{schreiner01}) leaves us with:

\begin{equation}
[ k C_1 - k R ]  t +  [ C_1 + k C_2 - k P_{alv0} ]   =  0
\label{schreiner03}
\end{equation}

To find a solution for $C_1$ and $C_2$ that hold for every t we have to make both parts between the square brackets 
in (\ref{schreiner03}) equal to 0. This results in $C_1 = R$ and $C_2 = P_{alv0} - \frac{R}{k}$. In this way we find:
	
\begin{equation}
P_{t}  \left( t \right)   =  C_0 e^{-kt} +R t + P_{alv0} - \frac{R}{k}
\label{schreiner05}
\end{equation}

Again we use as boundary condition $P_t\left(0\right) = P_{t0}$ at t=0 in order to find $C_0$. 
Substituting this in (\ref{schreiner05}) 
we find $C_0 = P_{t0} - P_{alv0} + R/k.$ So for the ultimate solution we find:


\begin{equation}
P_{t}  \left( t \right)   = P_{alv0} + R [t +  - \frac{R}{k}] - [ P_{alv0} - P_{t0}  - \frac{R}{k} ] e^{-kt}
\label{schreiner}
\end{equation}

where 

\begin{itemize}
\item $P_t\left(t\right)$    Partial pressure of the gas in the tissue (bar)
\item $P_{t0}$    Initial partial pressure of the gas in the tissue at t=0 (bar)
\item $P_{alv0}$    Initial (alveolar) partial pressure of the gas in the breathing mix at t=0 (bar)
\item $k$    A constant depending on the type of tissue
\item $R$    Rate of change of the partial inert gas pressure in the breathing mix in the alveoli (bar/min) R=Q Ramb, 
    in which Q is the fraction of the inert gas and Ramb is the rate of change of the ambient pressure.
\item $t$    Time (min)
\end{itemize}

This solution was first proposed by Schreiner and hence known as the {\bf Schreiner equation}. 
If we set the rate of change R to 0 (remaining at constant depth), the equation transforms in the Haldane 
equation (\ref{haldane}). The Schreiner equation is excellent for application in a simulation as used in diving computers. 
With the same frequency of measuring of the environment pressure and performing the calculation, applying the 
Schreiner equation gives a more precise approximation of the actual pressure profile in the bodily tissue than 
the Haldane equation. 
\end{enumerate}

\section{Half times}
So we see an exponential behavior. When we look at the first situation (constant depth) we have a tissue with in 
initial partial pressure $P_{t0}$. Eventually the partial pressure of gas in the tissue will reach the partial 
pressure of the gas in the breathing mixture $P_{alv0}$. We can calculate how long it takes for the partial pressure 
to get half way in between, i.e. $e^{-k\tau} = \frac{1}{2}$. The variable $\tau$ (tau) is called the 'half-time' and is 
usually used for characterizing tissue (compartments). Rewriting: $-k\tau = \ln\left(\frac{1}{2}\right) = -\ln\left(2\right)$. 
So the relation between $k$ and the half-time $\tau$ is:
 
\begin{equation}
\tau = \frac{\ln\left(2\right)}{k}
\label{halftime1}
\end{equation}

and
\begin{equation}
k = \frac{\ln\left(2\right)}{\tau}
\label{halftime2}
\end{equation}

\section{The alveolar partial pressure}

So far we did not worry about the values of $P_{alv}$. We will have a closer look at this alveolar partial pressure 
of the inert gas and how it is related to the ambient pressure. The pressure of the air (or gas mixture) 
the diver breathes is equal to the ambient pressure Pamb surrounding the diver. The ambient pressure depends on the 
water depth and the atmospheric pressure at the water surface. To be precise: it is equal to the atmospheric 
pressure (1 bar at sea level) increased with 1 bar for every ten meters depth. The partial pressure of the inert gas 
in the alveoli depends on several factors:

\begin{itemize}
\item The partial pressure (fraction Q) of the inert gas in the air or gas mixture breathed in
\item The water vapor pressure. The dry air breathed in is humidified completely by the upper airways 
(nose, larynx, trachea). Water vapor dilutes the breathing gas. A constant vapor pressure at 37 degrees Celsius 
of 0.0627 bar (47 mm Hg) has to be subtracted from the ambient pressure
\item Oxygen O$_2$ is removed from the breathing gas by respiratory gas exchange in the lungs
\item Carbon Dioxide CO$_2$ is added to the breathing gas by gas exchange in the lungs. Since the partial pressure 
of CO$_2$ in dry air (and in common breathing mixtures) is negligible, the partial pressure of the CO$_2$ in the lungs 
will be equal to the arterial partial pressure. This pressure is 0.0534 bar (40 mm Hg).
\end{itemize}

The process of Oxygen consumption and Carbon Dioxide production is characterized by the respiratory quotient RQ, 
the volume ratio of Carbon Dioxide production to the Oxygen consumption. Under normal steady state conditions the 
lungs take up about 250 ml of Oxygen, while producing about 200 ml of Carbon Dioxide per minute, resulting in an 
RQ value of about 200/250=0.8. Depending on physical exertion and nutrition RQ values range from 0.7 to 1.0. 
Schreiner uses RQ=0.8 , US Navy uses RQ=0.9 and B�hlmann uses 1.0.\\

The alveolar ventilation equation gives us the partial pressure of the inert gas with respect to the ambient pressure:
\begin{equation}  	
P_{alv}  =   [ P_{amb} - P_{H2O} - P_{CO2} + \Delta P_{O2} ]  Q
\end{equation} 

or after rewriting
\begin{equation}  
P_{alv}  =   [ P_{amb} - P_{H2O} + \frac{1 - RQ}{RQ}  P_{CO2} ] Q
\label{alveolar}
\end{equation} 

where 
\begin{itemize}
\item $P_{alv}$    Partial pressure of the gas in the alveoli (bar)
\item $P_{amb}$    Ambient pressure, i.e. the pressure of the breathing gas(bar)
\item $P_{H2O}$    Water vapor pressure, at 37 degrees Celsius 0.0627 bar (47 mm Hg)
\item $P_{CO2}$    Carbon Dioxide pressure, we can use 0.0534 bar (40 mm Hg) 
\item $\Delta P_{O2}$    Decrease in partial Oxygen pressure due to gas exchange in the lungs 
\item $RQ$    Respiratory quotient: ratio of Carbon Dioxide production to Oxygen consumption
\item $Q$    Fraction of inert gas in the breathing gas. For example N2 fraction in dry air is 0.78
\end{itemize}

The Schreiner RQ value is the most conservative of the three RQ values. Under equal circumstances using the 
Schreiner value results in the highest calculated partial alveolar pressure and hence the highest partial 
pressure in the tissue compartments. This leads to shorter no decompression times and hence to less risk for DCS. 

\section{Example}
We will have a look at our diver who plunges to 30 m and stays there for a while. 
The diver breathes compressed air and did not dive for quite a while before this dive. 
So at the start of the dive, all his tissue is saturated with Nitrogen at a level that corresponds to sea level. 
We neglect the period of descending. In particular, we will look at two types of tissue in his body with a half 
time of 4 minutes (the fastest tissue) resp. 30 minutes (medium fast tissue). The ambient pressure at 30 meters is 
4 bar. Equation (\ref{alveolar}) gives us a partial alveolar N2 pressure of 3.08 bar at 30 meters and 0.736 bar at 
sea level, using the RQ=0.9 value of the US Navy. Substituting these values in equation (\ref{haldane}) result in (14), 
predicting the partial pressure in the tissues. This pressure is shown in figure 1:

\begin{equation}
P_{t4}  \left( t \right)   =  3.08 + [ 0.736 - 3.08 ]  [ 1 - e^{- \frac{\ln(2)}{4}t} ]
\end{equation}

\includegraphics[scale=0.6]{images/tissue4.png} \vfill
After 10 minutes at 30 meters, this diver comes up to 20 meters, will stay there for another 5, then goes on to 10 meters 
for another 5 minutes and finally surfaces. As you can see in figure \ref{fig_tissue4} the pressure 
increases with the predicted exponential form, untile we decrease the ambient pressure. A distinct
 ``knack'' in de the graph marks the start of the off-gassing phase.

\section{Supersaturation limits and M-Values}

So we are now able to calculate inert gas levels and the amount of supersaturation in all tissue
 compartments of the diver. As we stated a certain amount of supersaturation is allowed, without
  developing DCS symptoms. In this section we will define limits applying to supersaturation levels. 
  As we will see these limits depend on:
\begin{itemize}
\item Type (half-time) of the tissue
\item Ambient pressure, i.e. the pressure of the breathing gas (depending on depth and atmospheric pressure)
\end{itemize}

\subsection{Limits according to Haldane}

In 1908 Haldane presented the first model for decompression. He noticed that divers could surface from a depth of 10 meter, 
without developing DCS. He concluded that the pressure in the tissue can exceed the ambient pressure by a factor of 2. 
(Actually the factor the partial pressure of the Nitrogen in the body exceeds the ambient pressure is 0.78*2=1.56, as Workman concluded)

Haldane used this ratio to construct the first decompression tables. Up to 1960 ratio's were used. Different ratio's were defined by various scientists. 
In that period most of the US Navy decompression tables were calculated using this method.  


\subsection{Workman M-values}

At longer and deeper dives, the ratio limits did not provide enough safety. Further research into supersaturation limits was performed by 
Robert D. Workman around 1965. Workman performed research for the U.S. Navy Experimental Diving Unit (NEDU). He found that each tissue compartment had a 
different partial pressure limit, above which DCS symptoms develop. He called this limiting pressure M. He found a linear relationship between this M-value and depth. 
Hence he defined this relationship as: 
\begin{equation}
M  =  M_0 + \Delta M d
\label{workman}
\end{equation}

where

\begin{itemize}
\item $M$    Partial pressure limit, for each tissue compartment (bar)
\item $M_0$    The partial pressure limit at sea level (zero depth), defined for each tissue compartment (bar)
\item $\Delta M$   Increase of M per meter depth, defined for each compartment (bar/m)
\item $d$    Depth (m) 
\end{itemize}

\section{No-decompression times}

A number of tables, like the PADI RDP express no-decompression times. These are maximum times a diver can stay at a certain depth being able to go to the surface 
without the need for decompression stops. Based on equation (\ref{haldane}) we can calculate this time for a particular tissue compartment:

\begin{equation}  
t_{no\_deco}  =   - \frac{1}{k} \ln \left ( \frac{ P_{no\_deco} - P_{alv0} } { P_{t0} - P_{alv0} } \right )
\label{no_deco_time}
\end{equation}

Ofcourse we have to calculate this time for every tissue and take the minimum value as limit for 
the diver to remain at the depth. 

\chapter{Oxygen}

Oxygen is the gas that keeps you alive. The human body needs oxygen to generate energy. We can't do very long 
without it. Oxygen however can also kill you. In certain cases, it becomes poisenous. In recreational diving with
air you will hardly run into the oxygen limits. If you stick to not going deeper than the advised maximum depth
of 40 meters and keeping within the no-decompression limits.

\section{CNS}
There are 2 ways in which oxygen becomes toxic. The first one is directly related to the depth and gas you are breathing.
When the partial pressure of oxgyen (the $pO_2$) exceeds 1.6 bar, you are at direct risk of convulsing.
This would be pretty harmless if you weren't submerged. A diver convulsing is bound to lose his regulator and drown.
So to avoid problems it is advised to keep the $pO_2$ below 1.4 and only use 1.6 when in rest during a decompression stop.


\section{Pulmonary Oxygen Toxicity}

When the lungs are exposed to partial oxygen pressures of 0.5 bar or more, they become irritated. This version of
oxygen toxicity is not life threatening, but a temporary drop in the functioning of your breathing apparatus.
To track the exposure to high oxygen levels, the \bf{Oxygen Tolerance Units} have been invented. These are defined as:
\begin{equation}  
OTU  =   t * \left ( \frac{pO_2 - 0.5} { 0.5 } \right ) ^ { - \frac{5}{6} }     , when pO_2 > 0.5
\label{otu}
\end{equation}

where

\begin{itemize}
\item $pO_2$ partial Oxygen pressure in bar
\item $t$    time in minutes exposed to the $pO_2$
\end{itemize}

Depending on how many days in a row you are diving there are limits on the number of OTU's you can acquire.

\section{CNS}

%%%%%%%%%%%%%%%%%%%%%%%%%%% End Content %%%%%%%%%%%%%%%%%%%%%%%%%%%%%%
\end{document}
