\documentclass{article}
%\usepackage{graphpap}
\usepackage{amsmath}
\usepackage{multirow}
\begin{document}

\section{Basic}
Math: $a=b$
\begin{picture}(100,100)
\put(0,0){\line(1,0){100}}
\end{picture}
%\end{document}

\section{Pictures}
\subsection{Lines:} using integers, scaling with floats, \LaTeX counters, TeX counters.

All should be the same.
\par
\begin{picture}(100,100)
\put(0,0){\line(1,0){100}}
\put(0,0){\line(1,1){50}}
\put(0,0){\line(0,1){100}}
\end{picture}
{\unitlength=100pt\relax
\begin{picture}(1.0,1.0)
\put(0,0){\line(1,0){1.0}}
\put(0,0){\line(1,1){0.5}}
\put(0,0){\line(0,1){1.0}}
\end{picture}
}

\newcounter{foo}
\setcounter{foo}{50}
\begin{picture}(100,100)
\put(0,0){\line(1,0){100}}
%\put(0,0){\line(1,1){100}}
\put(0,0){\line(1,1){\value{foo}}}
\put(0,0){\line(0,1){100}}
\end{picture}
\newcount\ffoo\ffoo=50\relax
\begin{picture}(100,100)
\put(0,0){\line(1,0){100}}
%\put(0,0){\line(1,1){100}}
\put(0,0){\line(1,1){\ffoo}}
\put(0,0){\line(0,1){100}}
\end{picture}
%\end{document}

\subsection{Vectors:}

\begin{picture}(100,100)(-100,-100)
\put(0,0){\vector(-1, 0){100}}
\put(0,0){\vector(-1,-1){100}}
\put(0,0){\vector( 0,-1){100}}
\end{picture}

\subsection{Boxes;}

\begin{picture}(100,100)
\put(0,0){\bf A}
\put(15,15){\framebox(15,15){A}}
\put(30,30){\framebox(15,15){\bf A}}
\put(45,45){\framebox(15,15){A}}
\put(60,30){\frame{\shortstack{A\\B\\\bf C\\D}}}
\put(75,20){\framebox(15,80){\shortstack{A\\\bf B\\C\\D}}}
\end{picture}

\begin{picture}(100,40)
  \put(0,0){\makebox(10,10){x}}
  \put(20,0){\framebox(10,10){x}}
  \put(40,0){\dashbox{2}(10,10){x}}
  \put(60,0){\frame{x}}
\end{picture}

\subsection{Box positioning}

Xg
vs. $Xg$
vs. \begin{picture}(12,12)\put(0,0){Xg}\end{picture}
vs. \begin{picture}(12,12)\put(0,0){$Xg$}\end{picture}
vs. \begin{picture}(12,12)\put(0,0){\makebox(12,12)[bl]{Xg}}\end{picture}
vs. \begin{picture}(12,12)\put(0,0){\makebox(12,12)[bl]{$Xg$}}\end{picture}

\begin{picture}(120,120)
  \put(  0,  0){\line(1,0){120}}
  \put(  0, 20){\line(1,0){120}}
  \put(  0, 40){\line(1,0){120}}
  \put(  0, 60){\line(1,0){120}}
  \put(  0, 80){\line(1,0){120}}
  \put(  0,100){\line(1,0){120}}
  \put(  0,120){\line(1,0){120}}
  \put(  0,  0){\line(0,1){120}}
  \put( 20,  0){\line(0,1){120}}
  \put( 40,  0){\line(0,1){120}}
  \put( 60,  0){\line(0,1){120}}
  \put( 80,  0){\line(0,1){120}}
  \put(100,  0){\line(0,1){120}}
  \put(120,  0){\line(0,1){120}}
  \put(  0, 20){Xg}
  \put( 60, 20){$Xg$}
  \put( 40, 20){\makebox(0,0){Xg}}
  \put(100, 20){\makebox(0,0){$Xg$}}
  \put(  0, 60){\makebox(20,20)[tl]{Xg}}
  \put( 20, 60){\makebox(20,20)[t]{Xg}}
  \put( 40, 60){\makebox(20,20)[tr]{Xg}}
  \put(  0, 80){\makebox(20,20)[l]{Xg}}
  \put( 20, 80){\makebox(20,20)[]{Xg}}
  \put( 40, 80){\makebox(20,20)[r]{Xg}}
  \put(  0,100){\makebox(20,20)[bl]{Xg}}
  \put( 20,100){\makebox(20,20)[b]{Xg}}
  \put( 40,100){\makebox(20,20)[br]{Xg}}
  \put( 60, 60){\makebox(20,20)[tl]{$Xg$}}
  \put( 80, 60){\makebox(20,20)[t]{$Xg$}}
  \put(100, 60){\makebox(20,20)[tr]{$Xg$}}
  \put( 60, 80){\makebox(20,20)[l]{$Xg$}}
  \put( 80, 80){\makebox(20,20)[]{$Xg$}}
  \put(100, 80){\makebox(20,20)[r]{$Xg$}}
  \put( 60,100){\makebox(20,20)[bl]{$Xg$}}
  \put( 80,100){\makebox(20,20)[b]{$Xg$}}
  \put(100,100){\makebox(20,20)[br]{$Xg$}}
\end{picture}

\subsection{Circles:}

\begin{picture}(100,100)
\put(20,20){\circle{30}\vector(0,1){15}\circle*{5}}
\put(70,70){\oval(60,30)}
\put(70,20){\oval(45,30)}
\end{picture}

\vskip 1em

\subsection{Curves:}

\begin{picture}(100,100)
\qbezier(40,80)(80,120)(100,80)
\qbezier[50](40,89)(80,129)(100,89)
\end{picture}

\subsection{Repeats:}

\begin{picture}(100,100)
\multiput(0,0)(10,10){11}{a}
\end{picture}

\begin{picture}(100,100)
\put(10,10){\circle{5}}
\put(20,20){\framebox(10,10){\circle{5}}}
\put(30,30){\framebox(10,10)[l]{\circle{5}}}
\put(40,40){\framebox(10,10)[r]{\circle{5}}}
\put(50,50){\framebox(10,10)[t]{\circle{5}}}
\put(60,60){\framebox(10,10)[b]{\circle{5}}}
\end{picture}

\section{User Examples}
\begin{picture}(100,100)(-30,0)\thicklines
	\put(11,11){\vector(1,1){33}}
	\put(44,44){\line(1,1){23.5}}
	\put(75.5,71){\vector(1,0){24.5}}
	\put(100,71){\line(1,0){20}}
	\put(0,0){O}
	\put(33,43){$\vec{r}$}
	\put(93,78){$\vec{p}$}
	\put(71,71){\circle{10}}
	\put(11,11){\circle*{3}}
	\put(66.5,68){$\times$}
	\put(58,75){$\vec{L}$}
\end{picture}

Impressive, eh?


\begin{figure}[!ht] \centering
\begin{picture}(70,65)
\put(12,57){\makebox(0,0){$F(x,y)$}}

\put(30,40){\line(1,0){40}}
\put(30,20){\line(1,0){40}}
\put(30,0){\line(1,0){40}}

\put(30,0){\line(0,1){40}}
\put(50,0){\line(0,1){40}}
\put(70,0){\line(0,1){40}}

\put(50,57){\makebox(0,0){$y$}}
\put(40,47){\makebox(0,0){\footnotesize $0$}}
\put(60,47){\makebox(0,0){\footnotesize $1$}}

\put(14,20){\makebox(0,0){$x$}}
\put(22,30){\makebox(0,0){\footnotesize $0$}}
\put(22,10){\makebox(0,0){\footnotesize $1$}}

\put(40,30){\makebox(0,0){$m_0$}}
\put(60,30){\makebox(0,0){$m_1$}}
\put(40,10){\makebox(0,0){$m_2$}}
\put(60,10){\makebox(0,0){$m_3$}}
\end{picture}
\caption{Mapping of two-variable minterms on a Karnaugh map.}
\label{fg:kmap2}
\end{figure}

\section{DLMF Examples}
\begin{figure}[h]
  \setlength{\unitlength}{0.035in}
  \centering
  \begin{picture}(152,38)(-1,-1)
    \put(0,31){\makebox(2,6){\small 1}}
    \put(0,26){\framebox(2,6){$s$}}
    \put(3,31){\makebox(8,6){\small 8}}
    \put(3,26){\framebox(8,6){$E$}}
    \put(12,31){\makebox(23,6){\small 23 bits}}
    \put(12,26){\framebox(23,6){$f$}}
    \put(133,31){$N=32$,}
    \put(135,27){$p=24$}
%
    \put(0,18){\makebox(2,6){\small 1}}
    \put(0,13){\framebox(2,6){$s$}}
    \put(3,18){\makebox(11,6){\small 11}}
    \put(3,13){\framebox(11,6){$E$}}
    \put(15,18){\makebox(52,6){\small 52 bits}}
    \put(15,13){\framebox(52,6){$f$}}
    \put(133,17){$N=64$,}
    \put(135,13){$p=53$}
%
    \put(0,5){\makebox(2,6){\small 1}}
    \put(0,0){\framebox(2,6){$s$}}
    \put(3,5){\makebox(15,6){\small 15}}
    \put(3,0){\framebox(15,6){$E$}}
    \put(19,5){\makebox(112,6){\small 112 bits}}
    \put(19,0){\framebox(112,6){$f$}}
    \put(133,4){$N=128$,}
    \put(135,0){$p=113$}
  \end{picture}
\caption[{Representation of data in the binary interchange formats 
  for binary32, binary64 and binary128.}]
  {Floating-point arithmetic. Representation of data in the binary 
  interchange formats for binary32, binary64 and binary128
  (previously single, double and quad precision).}
\end{figure}



\begin{table}[h]
    \caption{Cubature formulas for disk and
square.}
\setlength{\unitlength}{0.35in}%
\divide\tabcolsep2\relax
%\renewcommand{\arraystretch}{2}%
\renewcommand{\arraystretch}{1.2}%
\makeatletter
\def\dodots{\@ifnextchar({\dodot}{}}%)
\def\dodot(#1,#2){\put(#1,#2){\circle*{0.15}}\dodots}
\def\doaxes{%
   \multiput(0,0)(0.1,0){10}{\line(1,0){0.05}}
   \multiput(0,0)(0,0.1){10}{\line(0,1){0.05}}
   \multiput(0,0)(-0.1,0){10}{\line(-1,0){0.05}}
   \multiput(0,0)(0,-0.1){10}{\line(0,-1){0.05}}
}
\makeatother
\newcommand{\circleframe}[1]{
%\raisebox{-0.6in}{
%\vrule height 1em depth 1in width 0pt\relax
\begin{picture}(2.4,3.0)(-1.2,-1.55)%
   \doaxes
   \put(0,0){\circle{2}}
   \dodots#1
\end{picture}}%}
\newcommand{\squareframe}[1]{
%\raisebox{-0.6in}{
%\vbox to 1in {
%\vrule height 1em depth 1in width 0pt\relax
\begin{picture}(2.4,3.0)(-1.2,-1.55)%
   \doaxes
   \put(-1,1){\line(1,0){2}}
   \put(-1,1){\line(0,-1){2}}
   \put(1,-1){\line(-1,0){2}}
   \put(1,-1){\line(0,1){2}}
   \dodots#1
\end{picture}}%}
\centering\begin{tabular}{@{}c@{}lcc}
\hline\hline
Diagram & $(x_j,y_j)$ & $w_j$ & $R$ \\\hline
%
\multirow{5}{*}{
\circleframe{%
(0,0)%
(1,0)(0,1)%
(-1,0)(0,-1)}
}
&&&\\
& $(0,0)$     & $\tfrac{1}{2}$ & $O(h^4)$ \\
& $(\pm h,0)$ & $\tfrac{1}{8}$ & \\
& $(0,\pm h)$ & $\tfrac{1}{8}$ & \\
&&&\\ \hline%
%
\multirow{5}{*}{
\circleframe{%
(0.5,0.5)(-0.5,0.5)(0.5,-0.5)(-0.5,-0.5)}
}
&&&\\
&&&\\
& $(\pm \tfrac{1}{2}h,\pm \tfrac{1}{2} h)$  & $\tfrac{1}{4}$ & $O(h^4)$ \\
&&&\\
&&&\\\hline%
%
\multirow{5}{*}{
\circleframe{%
(0,0)%
(-1,0)(1,0)(0,-1)(0,1)%
(0.5,0.5)(-0.5,0.5)(0.5,-0.5)(-0.5,-0.5)}
}
&&&\\
& $(0,0)$     & $\tfrac{1}{6}$ & $O(h^8)$\\
& $(\pm h,0)$, $(0,\pm h)$ & $\tfrac{1}{24}$ & \\
& $(\pm\tfrac{1}{2}h,\pm\tfrac{1}{2}h)$ & $\tfrac{1}{6}$ & \\
&&&\\\hline%
%
\multirow{5}{*}{
\circleframe{%
(0,0)%
(0.8165,0)(-0.8165,0)%
(0.408,0.707)(-0.408,0.707)(0.408,-0.707)(-0.408,-0.707)}
}
&&&\\
&  $(0,0)$     & $\tfrac{1}{4}$ & $O(h^6)$\\
&  $(\pm\tfrac{1}{3}\sqrt{6}h,0)$ & $\tfrac{1}{8}$ &\\
&  $(\pm\tfrac{1}{6}\sqrt{6}h,\pm\tfrac{1}{2}\sqrt{2}h)$ & $\tfrac{1}{8}$ &\\
&&&\\\hline%
\multirow{5}{*}{
\squareframe{%
(0,0)%
(1,0)(-1,0)(0,1)(0,-1)%
(1,1)(-1,1)(1,-1)(-1,-1)}
}
&&&\\
& $(0,0)$     & $\tfrac{4}{9}$ & $O(h^4)$\\
& $(\pm h,0)$, $(0,\pm h)$ & $\tfrac{1}{9}$ & \\
& $(\pm h,\pm h)$ & $\tfrac{1}{36}$ & \\
&&&\\\hline%
\multirow{5}{*}{
\squareframe{%
(0.577,0.577)(-0.577,0.577)(0.577,-0.577)(-0.577,-0.577)}
}
&&&\\
&&&\\
& $(\pm\tfrac{1}{3}\sqrt{3}h,\pm\tfrac{1}{3}\sqrt{3}h)$ & $\tfrac{1}{4}$ & $O(h^4)$ \\
&&&\\
&&&\\\hline%
\multirow{5}{*}{
\squareframe{%
(0,0)%
(0.7746,0)(-0.7746,0)(0,0.7746)(0,-0.7746)%
(0.7746,0.7746)(-0.7746,0.7746)(0.7746,-0.7746)(-0.7746,-0.7746)}
}
&&&\\
& $(0,0)$     & $\tfrac{16}{81}$ & $O(h^6)$\\
& $(\pm\sqrt{\tfrac{3}{5}}h,0)$, $(0,\pm\sqrt{\tfrac{3}{5}}h)$ & $\tfrac{10}{81}$ & \\
& $(\pm\sqrt{\tfrac{3}{5}}h, \pm\sqrt{\tfrac{3}{5}}h)$ & $\tfrac{25}{324}$ & \\
&&&\\\hline
\end{tabular}
\end{table}

\end{document}
