\documentclass{article}
\usepackage{xcolor}
\usepackage{cancel}
\usepackage{siunitx}
\usepackage{float}
\usepackage{multirow}
\usepackage{booktabs}

\DeclareSIQualifier\polymer{pol}
\DeclareSIQualifier\catalyst{cat}
\DeclareSIQualifier\isotropic{i} %?

\DeclareSIUnit\abbra{ a }
\DeclareSIUnit\abbrb{ \abbra }  % abbreviation
\DeclareSIUnit\abbrc{ \abbrb }  % abbreviation twice
\DeclareSIUnit\abbrd{ \abbre }  % before defined
\DeclareSIUnit\abbre{ e }

\def\mabbra{a}
\DeclareSIUnit\mabbrb{\mabbra}

\DeclareSIPrefix\killer{\kilo}{3}

\usepackage{amsmath}
\begin{document}
{
\color{red}
Some text \\
\SI{4}{\metre\per\sievert} \\
More text \\
\SI[color = blue]{4}{\metre\per\sievert} \\
Still red here!
\numlist[color = blue]{1;2;3;4}\\
Still red here!\\
}

\si{m^2.s}\\
\si{\mu m^2}\\

\SI{0.094}{\pi . \milli\meter . \milli\radian}\\

\SI{0.094}{\frac{1}{3} . \milli\meter . \milli\radian}\\

\SI{0.094}{\pi \per \milli\meter . \milli\radian\tothe{3}}\\
\section{Numbers}
\subsection{General}
%\num{\sqrt{2}}\\
%\num{{a word}}\\
\num{12345,67890}\\
\num{1+-2i}\\
\num{.3e45}\\
\num{1.654 x 2.34 x 3.430}\\
\num{\pi}\\
\num{2\pi}\\
\num{\pi/3}\\

\num{123}\\
\num{1234}\\
\num{12345}\\
\num{0.123}\\
\num{0,1234}\\
\num{.12345}\\
\num{3.45d-4}\\
\num{-e10}\\

\num{123e4}\\
%\num{123e4(3)}\\
\num{123(3)e4}\\

\num{123\pm2}\\
\num{123\pm2i}\\
\num{123+234i}\\
%\num{123+234}\\
%\num{123e2+234e3i}\\
\num{123+234ie3}\\
\num{123(1)+234(1)ie3}\\
%\num{123e2(1)+234e3i(1)}\\
%\num{+234(1)ie3}\\              %comes out weird?
\num{+3i}\\
\num{+3ie4}\\

Pretty nonsensical stuff?
\num{1.\pi e+3}\\
\def\dig{1234}\\
\num{\dig.\dig}\\
\def\odd{\xi}\\
\def\odder{\odd}\\
\num[input-symbols=\xi]{3\xi}\\
\num[input-symbols=\xi]{3\odd}\\
%\num[input-symbols=\odd]{3\odd}\\
\num[input-symbols=\odd, input-protect-tokens=\odd]{3\odd}\\
\num[input-symbols=\odd, input-protect-tokens=\odd]{3\odder}\\
\num[input-symbols=\odder, input-protect-tokens=\odder]{3\odder}\\
%\num[input-symbols=\odder, input-protect-tokens=\odd]{3\odder}\\
%\num[input-symbols=\odd, input-protect-tokens=\odder]{3\odder}\\

\num{1.23(1)}\\
%\num{1.23(0.01)}
\num{1.23\pm0.01}\\
%\num{1.23(1.0)}\\
%\num{1.23(\pm1)}\\
\num{1.23(\pi)}\\

\subsection{Parsing numbers}

\subsubsection{input-digits, input-decimal-markers, input-signs, input-exponent-markers}

\subsubsection{input-symbols, input-ignore}
\subsubsection{input-comparators}

\num{< 10} \\
%\SI{>> 5}{\metre} \\
\num{\le 0.12}\\

\subsubsection{input-open-uncertainty, input-close-uncertainty, input-uncertainty-signs}
\num{9.99(9)}\\
\num{9.99 +- 0.09}\\
\num{9.99 \pm 0.09}\\
\num{123 +- 4.5}\\
\num{12.3 +- 6}\\

\subsubsection{input-complex-roots}
\num{9.99 + 88.8i} \\
\num{9.99 + i88.8}\\

\subsubsection{input-protect-tokens}

\subsubsection{parse-numbers}
\num[parse-numbers = false]{\sqrt{2}}\\

\subsection{Post-processing numbers}
\subsubsection{round-mode, round-precision}
{
\num{1.23456} \\
\num{14.23} \\
\num{0.12345(9)} \\
\sisetup{
  round-mode = places,
  round-precision = 3
}%
\num{1.23456} \\
\num{14.23} \\
\num{0.12345(9)} \\
\sisetup{
  round-mode = figures,
  round-precision = 3
}%
\num{1.23456} \\
\num{14.23} \\
\num{0.12345(9)}\\
}
\subsubsection{round-integer-to-decimal}
{
\num[round-mode = figures]{1} \\
\num[round-mode = places]{1} \\
\sisetup{round-integer-to-decimal}
\num[round-mode = figures]{1} \\
\num[round-mode = places]{1}\\
}

\subsubsection{round-minimum}
{
\sisetup{round-mode = places}%
\num{0.0055} \\
\num{0.0045} \\
\sisetup{round-minimum = 0.01}%
\num{0.0055} \\
\num{0.0045}\\
}

\subsubsection{round-half}
{
\sisetup{round-mode = places, round-half = up}%
\num{0.055} \\
\num{0.045} \\
\sisetup{round-half = even}%
\num{0.055} \\
\num{0.045}
}

\subsubsection{add-decimal-zero, add-integer-zero}
{
\num{123.} \\
\num{456} \\
\num{.789} \\
\sisetup{
  add-decimal-zero = false,
  add-integer-zero = false,
}%
\num{123.} \\
\num{456} \\
\num{.789}\\
}

\subsubsection{minimum-integer-digits}
\num{123} \\
\num[minimum-integer-digits = 1]{123} \\
\num[minimum-integer-digits = 2]{123} \\
\num[minimum-integer-digits = 3]{123} \\
\num[minimum-integer-digits = 4]{123}\\

\subsubsection{explicit-sign, retain-explicit-sign}
\num{+345} \\
\num[retain-explicit-plus]{+345} \\
\num[explicit-sign = -]{345}\\
\num[explicit-sign = -]{+345}\\

\subsubsection{retain-unity-mantissa, retain-zero-exponent}
\num{1e4} \\
\num[retain-unity-mantissa = false]{1e4} \\
\num{444e0} \\
\num[retain-zero-exponent = true]{444e0}\\

\subsubsection{scientific-notation, fixed-exponent}
{
\num{0.001}\\
\num{0.0100} \\
\num{1200}\\
\sisetup{scientific-notation = true}%
\num{0.001}\\
\num{0.0100} \\
\num{1200}\\
\sisetup{scientific-notation = engineering}%
\num{0.001}\\
\num{0.0100} \\
\num{1200}\\
\sisetup{
fixed-exponent = 2,
scientific-notation = fixed,
}%
\num{0.001}\\
\num{0.0100} \\
\num{1200}\\
}

\subsubsection{omit-uncertainty}
\num{0.01(2)} \\
\num[omit-uncertainty]{0.01(2)}\\

\subsection{Printing numbers}
\subsubsection{group-digits, group-four-digits,group-seperator}
\num{12345.67890}\\
\num[group-digits= false]{12345.67890}\\
\num[group-digits= decimal]{12345.67890} \\
\num[group-digits= integer]{12345.67890}\\

\num[group-digits= false]{12345.67890}\\
\num[group-digits= decimal]{12345.67890} \\
\num[group-digits= integer]{12345.67890}\\

\num{1234567890.1234567890}\\
\num[group-four-digits]{1234567890.1234567890}\\

\num{12345} \\
\num[group-separator = {,}]{12345} \\
\num[group-separator = \text{~}]{12345}\\

\subsubsection{group-minimum-digits}
\num{1234} \\
\num[group-minimum-digits = 4]{1234} \\
\num{1234.5678} \\
\num[group-minimum-digits = 4]{1234.5678}\\

\subsubsection{output-complex-root,output-decimal-marker,copy-complex-root,copy-decimal-marker}
\num{1.23} \\
\num[output-decimal-marker = {,}]{1.23} \\
\num{1+2i} \\
\num[output-complex-root = \text{\ensuremath{i}}]{1+2i} \\
\num[output-complex-root = j]{1+2i} \\
\num[copy-complex-root]{1+2j} \\
\num[copy-decimal-marker]{555,555}\\

\subsubsection{complex-root-position}
\num{67-0.9i} \\
\num[complex-root-position = before-number]{67-0.9i} \\
\num[complex-root-position = after-number]{67-0.9i}\\

\subsubsection{exponent-base, exponent-product}
\num[exponent-product = \times]{1e2} \\
\num[exponent-product = \cdot]{1e2} \\
\num[exponent-base = 2]{1e2}\\

\subsubsection{output-exponent-marker}
\num[output-exponent-marker = \text{e}]{1e2} \\
\num[output-exponent-marker = \ensuremath{\mathrm{E}}]{1e2}\\

\subsubsection{separate-uncertainty,uncertainty-separator,output-open-uncertainty,output-close-uncertain
ty}
{
\num{1.234(5)} \\
\num{1.234\pm 0.005} \\
\num[separate-uncertainty = true]{1.234(5)} \\
\num[separate-uncertainty = true]{1.234\pm 0.005} \\
\sisetup{
  output-open-uncertainty = [,
  output-close-uncertainty = ],
  uncertainty-separator= {\,}
}
\num{1.234(5)}\\
}
\num{8.2(13)} \\
\num{8.2\pm1.3} \\
\num[separate-uncertainty]{8.2(13)}\\
\num[separate-uncertainty]{8.2\pm1.3} \\

\num{1.234(5) x \pi} \\
\num[separate-uncertainty = true]{1.234(5) x \pi} \\

\num{1.2 +- 0.001}\\
\num[separate-uncertainty]{1.2 +- 0.001}\\

\subsubsection{bracket-numbers, open-bracket, close-bracket}
{
\num{1 e10} \\
\num{2i e10} \\
\num{1+2i e10} \\
\num[bracket-numbers = false]{1+2i e10} \\
\sisetup{
  open-bracket = \{,
  close-bracket = \},
}
\num{1+2i e10}\\
}

\subsubsection{negative-color}
\num{-15673} \\
\num[negative-color = red]{-15673}

\subsubsection{bracket-negative-numbers}
\num{-15673} \\
\num[bracket-negative-numbers]{-15673} \\
%\SI{-10}{\metre} \\
%\SI[bracket-negative-numbers]{-10}{\metre}

\subsection{Multi-part Numbers}

\subsubsection{input-product,input-quotient}
\num{1 x 2 x 3} \\
\num{1e4 x 2(3) x 3/4} \\
\num[input-product=*]{4 * 5 * 6} \\
\num{ 1 / 2e4 } \\
\num{ 1e2 / 3e4 }\\

\subsubsection{output-product, output-quotient}
\num[output-product = \cdot]{4.87 x 5.321 x 6.90545} \\
\num[output-quotient = \text{ div }]{1 / 2}\\

\subsubsection{quotient-mode}
\num{1 / 2e4} \\
\num[quotient-mode = fraction]{1 / 2e4}\\

\subsubsection{fraction-function}
{
\sisetup{quotient-mode= fraction}
\num{1 / 1}\\
\num[fraction-function= \dfrac]{1 / 2}\\
%\num[fraction-function= \sfrac]{1 / 3}\\
\num[fraction-function= \tfrac]{1 / 4}\\
}

\subsection{Lists and ranges of numbers}
\subsubsection{list-final-separator,list-pair-separator,list-separator}
\numlist{0.1;0.2;0.3}\\
\numlist[list-separator = {; }]{0.1;0.2;0.3}\\
\numlist[list-final-separator = {, }]{0.1;0.2;0.3} \\
\numlist[
list-separator
= { and },
list-final-separator = { and finally }
]{0.1;0.2;0.3} \\
\numlist{0.1;0.2} \\
\numlist[list-pair-separator = {, and }]{0.1;0.2}\\

\subsection{range-phrase}

\numrange{5}{100} \\
\numrange[range-phrase = --]{5}{100}\\

\subsection{Angles}
\subsubsection{number-angle-product}
\ang{2.67} \\
\ang[number-angle-product = \,]{2.67}
\subsubsection{arc-separator}
\ang{6;7;6.5} \\
\ang[arc-separator = \,]{6;7;6.5}
\subsubsection{add-arc-degree-zero,add-arc-minute-zero,add-arc-second-zero}
\ang{-1;;} \\
\ang{;-2;} \\
\ang{;;-3} \\
{
\sisetup{add-arc-degree-zero}
\ang{-1;;} \\
\ang{;-2;} \\
\ang{;;-3} \\
}
{
\sisetup{add-arc-minute-zero}
\ang{-1;;} \\
\ang{;-2;} \\
\ang{;;-3} \\
}
{
\sisetup{add-arc-second-zero}
\ang{-1;;} \\
\ang{;-2;} \\
\ang{;;-3}\\
}
\ang[add-arc-minute-zero,add-arc-second-zero]{45.697}\\

\subsubsection{angle-symbol-over-decimal}
\ang{45.697} \\
\ang{6;7;6.5} \\
\ang[angle-symbol-over-decimal]{45.697} \\
\ang[angle-symbol-over-decimal]{6;7;6.5}

% \ang{10}\\
% \ang{12.3}\\
% \ang{4,5}\\
% \ang{1;2;3}\\
% \ang{;;1}\\
% \ang{+10;;}\\
% \ang{-0;1;}\\


\section{Units}
\def\showunit#1{#1 & \texttt{$\backslash$#1} & \si{\csname#1\endcsname}}
\begin{table}
\caption{SI base units}
\centering
\begin{tabular}{lll}\hline\hline
Unit & Macro & Symbol \\\hline
\showunit{ampere}\\
\showunit{candela}\\
\showunit{kelvin}\\
\showunit{kilogram}\\
\showunit{metre}\\
\showunit{mole}\\
\showunit{second}\\
\end{tabular}
\end{table}

\begin{table}
\caption{Coherent derived units}
\centering
\begin{tabular}{llllll}\hline\hline
Unit & Macro & Symbol &
Unit & Macro & Symbol \\\hline
\showunit{becquerel}     & \showunit{newton} \\
\showunit{degreeCelsius} & \showunit{ohm} \\
\showunit{coulomb}       & \showunit{pascal} \\
\showunit{farad}         & \showunit{radian} \\
\showunit{gray}          & \showunit{siemens} \\
\showunit{hertz}         & \showunit{sievert} \\
\showunit{henry}         & \showunit{steradian} \\
\showunit{joule}         & \showunit{tesla} \\
\showunit{katal}         & \showunit{volt} \\
\showunit{lumen}         & \showunit{watt} \\
\showunit{lux}           & \showunit{weber} \\
\end{tabular}
\end{table}

\begin{table}
\caption{Non-SI units}
\centering
\begin{tabular}{lll}\hline\hline
Unit & Macro & Symbol \\\hline
\showunit{day}\\
\showunit{degree}\\
\showunit{hectare}\\
\showunit{hour} \\
\showunit{litre}\\
\showunit{liter}\\
\showunit{arcminute}\\
\showunit{minute}\\
\showunit{arcsecond}\\
\showunit{tonne}\\
\end{tabular}
\end{table}

\begin{table}
\caption{Expermental Non-SI units}
\centering
\begin{tabular}{lll}\hline\hline
Unit & Macro & Symbol \\\hline
\showunit{astronomicalunit}\\
\showunit{atomicmassunit}\\
\showunit{bohr}\\
\showunit{clight}\\
\showunit{dalton}\\
\showunit{electronmass}\\
\showunit{electronvolt}\\
\showunit{elementarycharge}\\
\showunit{hartree}\\
\showunit{planckbar}\\
\end{tabular}
\end{table}

\begin{table}
\caption{Other non-SI units}
\centering
\begin{tabular}{lll}\hline\hline
Unit & Macro & Symbol \\\hline
\showunit{angstrom}\\
\showunit{bar}\\
\showunit{barn}\\
\showunit{bel}\\
\showunit{decibel}\\
\showunit{knot}\\
\showunit{mmHg}\\
\showunit{nauticalmile}\\
\showunit{neper}\\
\end{tabular}
\end{table}

\def\showprefix#1{#1 & \texttt{$\backslash$#1} & \si{\csname#1\endcsname} & \si[prefixes-as-symbols=false]{\csname#1\endcsname}}
\begin{table}
\caption{Other non-SI units}
\centering
\begin{tabular}{llllllll}\hline\hline
Unit & Macro & Symbol & Power
Unit & Macro & Symbol & Power \\\hline
\showprefix{yocto} & \showprefix{deca}\\
\showprefix{zepto} & \showprefix{hecto}\\
\showprefix{atto}  & \showprefix{kilo}\\
\showprefix{femto} & \showprefix{mega}\\
\showprefix{pico}  & \showprefix{giga}\\
\showprefix{nano}  & \showprefix{tera}\\
\showprefix{micro} & \showprefix{peta}\\
\showprefix{milli} & \showprefix{exa}\\
\showprefix{centi} & \showprefix{zetta}\\
\showprefix{deci}  & \showprefix{yotta}\\
\end{tabular}
\end{table}


\begin{table}
\caption{Abbreviated units}
\centering
\begin{tabular}{lll}\hline\hline
Unit & Macro & Symbol \\\hline
\showunit{fg}\\
\showunit{pg}\\
\showunit{ng}\\
\showunit{ug}\\
\showunit{mg}\\
\showunit{g}\\
\showunit{kg}\\
\showunit{amu}\\\hline
\showunit{pm}\\
\showunit{nm}\\
\showunit{um}\\
\showunit{mm}\\
\showunit{cm}\\
\showunit{dm}\\
\showunit{m}\\
\showunit{km}\\\hline
\showunit{as}\\
\showunit{fs}\\
\showunit{ps}\\
\showunit{ns}\\
\showunit{us}\\
\showunit{ms}\\
\showunit{s}\\\hline
\showunit{fmol}\\
\showunit{pmol}\\
\showunit{nmol}\\
\showunit{umol}\\
\showunit{mmol}\\
\showunit{mol}\\
\showunit{kmol}\\\hline
\showunit{pA}\\
\showunit{nA}\\
\showunit{uA}\\
\showunit{mA}\\
\showunit{A}\\
\showunit{kA}\\\hline
\showunit{ul}\\
\showunit{ml}\\
\showunit{l}\\
\showunit{hl}\\
\showunit{uL}\\
\showunit{mL}\\
\showunit{L}\\
\showunit{hL}\\\hline
\showunit{mHz}\\
\showunit{Hz}\\
\showunit{kHz}\\
\showunit{MHz}\\
\showunit{GHz}\\
\showunit{THz}\\\hline
\showunit{mN}\\
\showunit{N}\\
\showunit{kN}\\
\showunit{MN}\\\hline
\showunit{Pa}\\
\showunit{kPa}\\
\showunit{MPa}\\
\showunit{GPa}\\\hline
\showunit{mohm}\\
\showunit{kohm}\\
\showunit{Mohm}\\\hline
\showunit{pV}\\
\showunit{nV}\\
\showunit{uV}\\
\showunit{mV}\\
\showunit{V}\\
\showunit{kV}\\\hline
\showunit{W}\\
\showunit{uW}\\
\showunit{mW}\\
\showunit{kW}\\
\showunit{MW}\\
\showunit{GW}\\\hline
\showunit{J}\\
\showunit{kJ}\\\hline
\showunit{eV}\\
\showunit{meV}\\
\showunit{keV}\\
\showunit{MeV}\\
\showunit{GeV}\\
\showunit{TeV}\\\hline
\showunit{kWh}\\\hline
\showunit{F}\\
\showunit{fF}\\
\showunit{pF}\\\hline
\showunit{K}\\\hline
\showunit{dB}\\\hline
\end{tabular}
\end{table}

\begin{table}
\caption{Binary prefixes}
\centering
\begin{tabular}{llllllll}\hline\hline
Unit & Macro & Symbol & Power \\\hline
\showprefix{kibi} \\
\showprefix{mebi} \\
\showprefix{gibi} \\
\showprefix{tebi} \\
\showprefix{pebi} \\
\showprefix{exbi} \\
\showprefix{zebi} \\
\showprefix{yobi} \\
\end{tabular}
\end{table}

%\subsection{Creating Units}
%\subsection{Loading additional units}

\subsection{Using units}
% Note that \kilogram is defined as \kilo\gram, and acts sorta like a macro
\si{\kilogram}
\si{\kilo\gram}
% Not allowed.
%\si{\kilo\kilogram}
%\si{\kilo\kilo\gram}
{
% and note that regular macros ARE expanded!
\def\killermeater{\kilo\meter}
\si{\killermeater}
}
{
% BUT, it prefers its OWN definitions!
\def\gram{Grr}
\si{\kilo\gram}
}

% However, it's happy with various indirect definitions.
% Ie. units, prefixes (at least) act like regular macros within \si.
\si{\abbra}\\
\si{\abbrb}\\
\si{\abbrc}\\
\si{\abbrd}\\
\si{\abbre}\\

\si{\mabbra}\\
\si{\mabbrb}\\

\si{\killer\meter}\\

% Where can highlight go?
\si{\highlight{red}\kilo\gram\metre\per\second} \\
\si{\kilo\highlight{red}\gram\metre\per\second} \\
\si{\kilo\gram\highlight{red}\metre\per\second} \\
\si{\kilo\gram\metre\highlight{red}\per\second} \\
\si{\kilo\gram\metre\per\highlight{red}\second} \\

\si{\cancel\kilo\gram\metre\per\second} \\
\si{\kilo\cancel\gram\metre\per\second} \\
\si{\kilo\gram\cancel\metre\per\second} \\
\si{\kilo\gram\metre\cancel\per\second} \\
\si{\kilo\gram\metre\per\cancel\second} \\

\subsubsection{forbid-literal-units, inter-unit-product}
\si{\farad\squared\lumen\candela} \\
\si[inter-unit-product = \ensuremath{{}\cdot{}}]
{\farad\squared\lumen\candela}\\

\subsubsection{per-mode, per-symbol, bracket-unit-denominator}
\si{\joule\per\mole\per\kelvin} \\
\si{\metre\per\second\squared} \\
\si[per-mode=fraction]{\joule\per\mole\per\kelvin} \\
\si[per-mode=fraction]{\joule\raiseto{-1}\mole\per\kelvin} \\
\si[per-mode=fraction]{\metre\per\second\squared}\\
\si{\ampere\per\mole\second} \\
\si[per-mode = reciprocal-positive-first]{\ampere\per\mole\second}

{
\sisetup{per-mode = symbol}%
\si{\joule\per\mole\per\kelvin} \\
\si{\metre\per\second\squared} \\
\si[per-symbol = \text{~div~}]{\joule\per\mole\per\kelvin} \\
\si[bracket-unit-denominator = false]{\joule\per\mole\per\kelvin}\\
}
\si[per-mode=repeated-symbol]{\joule\per\mole\per\kelvin}\\
{
\sisetup{per-mode = symbol-or-fraction}%
\( \si{\joule\per\mole\per\kelvin} \)
\[ \si{\joule\per\mole\per\kelvin} \]
\si{\joule\per\mole\per\kelvin} \\
\(
\displaystyle
\si{\joule\per\mole\per\kelvin}
\)
\[
\textstyle
\si{\joule\per\mole\per\kelvin}
\]
}

\subsubsection{sticky-per}
\si{\pascal\per\gray\henry} \\
\si[sticky-per]{\pascal\per\gray\henry}\\

\subsubsection{power-font}
\si{\metre\per\second\squared} \\
\si[power-font = unit]{\metre\per\second\squared}\\

\subsubsection{literal-superscript-as-power}
\si{m.s^{2}} \\
\si[literal-superscript-as-power = false]{m.s^{2}}\\

\subsubsection{qualifier-mode, qualifier-phrase}
\si{\kilogram\polymer\squared\per\mole\catalyst\per\hour} \\
\si[qualifier-mode = brackets]
{\kilogram\polymer\squared\per\mole\catalyst\per\hour} \\
\si[qualifier-mode = subscript]
{\kilogram\polymer\squared\per\mole\catalyst\per\hour} \\
\si[qualifier-mode = space]
{\kilogram\polymer\squared\per\mole\catalyst\per\hour} \\
\si[qualifier-mode = text]
{\deci\bel\isotropic}\\

\si[qualifier-mode = phrase]
{\kilogram\polymer\squared\per\mole\catalyst\per\hour} \\
\si[qualifier-mode = phrase, qualifier-phrase = { by }]
{\kilogram\polymer\squared\per\mole\catalyst\per\hour} \\

\subsubsection{prefixes-as-symbols}
\si{\milli\litre\per\mole\deci\ampere} \\
%%\SI{10}{\kilo\gram\squared\deci\second} \\
\si[prefixes-as-symbols=false]{\milli\litre\per\mole\deci\ampere}\\
\si[prefixes-as-symbols=false]{\kilo\gram\squared\deci\second}\\
\si{\mega\gram\squared\deci\second}\\
\si[prefixes-as-symbols=false]{\mega\gram\squared\deci\second}\\
\si{\micro\gram\squared\deci\second}\\
\si[prefixes-as-symbols=false]{\micro\gram\squared\deci\second}\\
\si{\per\mega\gram\squared\deci\second}\\
\si[prefixes-as-symbols=false]{\per\mega\gram\squared\deci\second}\\
\si{\per\micro\gram\squared\deci\second}\\
\si[prefixes-as-symbols=false]{\per\micro\gram\squared\deci\second}\\
%%\SI[prefixes-as-symbols=false]{10}{\kilo\gram\squared\deci\second}\\

\subsubsection{parse-units}
%%\SI{300}{\MHz} \\
%%\SI[parse-units = false]{300}{\MHz}\\

\subsection{Numbers with units}
\subsubsection{allow-number-unit-breaks}
\subsubsection{number-unit-product}
\SI{2.67}{\farad} \\
\SI[number-unit-product = \ ]{2.67}{\farad} \\
\SI[number-unit-product = ]{2.67}{\farad}\\

\subsubsection{multi-part-units}
{
\sisetup{separate-uncertainty}%
\SI{12.3(4)}{\kilo\gram} \\
\SI[multi-part-units = brackets]{12.3(4)}{\kilo\gram} \\
\SI[multi-part-units = repeat]{12.3(4)}{\kilo\gram}\\
\SI[multi-part-units = single]{12.3(4)}{\kilo\gram}\\
}

{
\sisetup{separate-uncertainty,bracket-numbers = false}%
\num{1.234(5)e-4} \\
\SI{1.234(5)e-4}{\metre}\\
}
\subsubsection{product-units}
\SI{2 x 3 x 4}{\metre} \\
\SI[product-units = brackets]{2 x 3 x 4}{\metre}\\
\SI[product-units = brackets-power]{2 x 3 x 4}{\metre}\\
\SI[product-units = power]{2 x 3 x 4}{\metre}\\
\SI[product-units = repeat]{2 x 3 x 4}{\metre}\\
\SI[product-units = single]{2 x 3 x 4}{\metre}\\

\subsubsection{list-units,range-units}
\SIlist{2;4;6;8}{\tesla} \\
\SIlist[list-units = brackets]{2;4;6;8}{\tesla} \\
\SIlist[list-units = repeat]{2;4;6;8}{\tesla} \\
\SIlist[list-units = single]{2;4;6;8}{\tesla} \\
\SIrange{2}{4}{\degreeCelsius} \\
\SIrange[range-units = brackets]{2}{4}{\degreeCelsius} \\
\SIrange[range-units = repeat]{2}{4}{\degreeCelsius} \\
\SIrange[range-units = single]{2}{4}{\degreeCelsius}\\

\subsubsection{exponent-to-prefix}
\SI{1700}{\g} \\
\SI{1.7e3}{\g} \\
{
\sisetup{exponent-to-prefix}%
\SI{1700}{\g} \\
\SI{1.7e3}{\g} \\
}
{
\sisetup{fixed-exponent = 3, scientific-notation = fixed}%
\SI{1700}{\g} \\
\SI{1.7e3}{\g}\\
}


\section{Tabular material}
% \begin{tabular}{S}
% 2.3456 \\
% \end{tabular}
% \end{document}


\begin{table}[H]
\caption{Standard behaviour of the \texttt{S} column type.}
\label{tab:S:standard}
\centering
\begin{tabular}{S}
\toprule
{Some Values} \\
\midrule
2.3456 \\
34.2345 \\
-6.7835 \\
90.473 \\
5642.5 \\
1.2e3 \\
e4 \\
\bottomrule
\end{tabular}
\end{table}


%\end{document}



\begin{table}[H]
\caption{Detection of surrounding material in an \texttt{S} column.}
\label{tab:S:extras}
\centering
\begin{tabular}{S[color=orange]}
\toprule
{Some Values} \\
\midrule
12.34 \\
\color{purple} 975,31 \\
44.268 \textsuperscript{\emph{a}} \\
\bottomrule
\end{tabular}
\end{table}

\begin{table}[H]
\caption{Controlling complex alignment with the tablenum macro.}
\label{tab:tablenum}
\centering
\begin{tabular}{lr}
\toprule
Heading & Heading \\
\midrule
Info & More info \\
Info & More info \\
\multicolumn{2}{c}{\tablenum[table-format = 4.4]{12,34}} \\
\multicolumn{2}{c}{\tablenum[table-format = 4.4]{333.5567}} \\
\multicolumn{2}{c}{\tablenum[table-format = 4.4]{4563.21}} \\
\bottomrule
\end{tabular}
\hfil
\begin{tabular}{lr}
\toprule
Heading & Heading \\
\midrule
\multirow{2}*{\tablenum{88,999}} & aaa \\
& bbb \\
\multirow{2}*{\tablenum{33,435}} & ccc \\
& ddd \\
\bottomrule
\end{tabular}
\end{table}

\begin{table}[H]
\centering
\caption{Units in tables.}
\label{tab:s:demo}
\begin{tabular}{s}
\toprule
\multicolumn{1}{c}{Unit} \\
\midrule
\metre\squared\per\second \\
\pascal \\
m.s^{-1} \\
\bottomrule
\end{tabular}
\end{table}

\begin{table}[H]
\centering
\caption{The \texttt{s} column processes everything.}
\label{tab:s:processing}
\sisetup{color = orange}
\begin{tabular}{ss}
\toprule
{Unit}
& \multicolumn{1}{c}{Unit}\\
\midrule
{\si{m^3}} & \multicolumn{1}{c}{\si{m^3}} \\
\kilogram
& \kilogram \\
\bottomrule
\end{tabular}
\end{table}

\subsubsection{table-parse-only}
\begin{table}[H]
\centering
\caption{Parsing without aligning in an \texttt{S} column.}
\label{tab:S:parse}
\begin{tabular}{SS[table-parse-only]}
\toprule
{Decimal-centred} &
{Simple centring} \\
\midrule
12.345 & 12.345 \\
6,78   & 6,78   \\
-88.8(9) & -88.8(9) \\
4.5e3    & 4.5e3 \\
\bottomrule
\end{tabular}
\end{table}

\subsubsection{table-number-alignment}
\begin{table}[H]
\caption{Aligning the \texttt{S} column.}
\label{tab:S:align}
\centering
\sisetup{
table-figures-integer = 2,
table-figures-decimal = 4
}
\begin{tabular}{
S
S[table-number-alignment = center]
S[table-number-alignment = left]
S[table-number-alignment = right]
}
\toprule
{Some Values} & {Some Values} & {Some Values} & {Some Values} \\
\midrule
2.3456 & 2.3456 & 2.3456 & 2.3456\\
34.2345 & 34.2345 & 34.2345 & 34.2345\\
56.7835 & 56.7835& 56.7835 & 56.7835\\
90.473 & 90.473 & 90.473 & 90.473\\
\bottomrule
\end{tabular}
\end{table}

\subsubsection{table-figures-decimal, table-figures-exponent,table-figures-integer,table-figures-
uncertainty}

\begin{table}[H]
\caption{Reserving space in \texttt{S} columns.}
\label{tab:S:space}
\sisetup{
table-number-alignment = center,
table-figures-integer = 2
}
\centering
\begin{tabular}{
S
S[table-number-alignment = right]
S[table-figures-uncertainty = 1]
S[separate-uncertainty, table-figures-uncertainty = 1]
S[table-sign-mantissa]
S[table-figures-exponent = 1]
}
\toprule
{Values}
& {Values}
& {Values}
& {Values}
& {Values}
& {Values} \\
\midrule
2.3 & 2.3 & 2.3(5) & 2.3(5) & 2.3 & 2.3e8\\
34.23 & 34.23 & 34.23(4) & 34.23(4) & 34.23 & 34.23\\
56.78 & 56.78 & 56.78(3) & 56.78(3) & -56.78 & 56.78e3\\
3,76 & 3,76 & 3,76(2) & 3.76(2) & +-3.76 & e6\\
\bottomrule
\end{tabular}
\end{table}

\subsubsection{table-comparator}
\begin{table}[H]
\caption{Reserving space for comparators in \texttt{S} columns.}
\label{tab:S:comparators}
\sisetup{
table-number-alignment = center,
table-figures-integer = 2,
table-figures-decimal = 2,
table-figures-exponent = 2,
}
\centering
\begin{tabular}{
S
S[table-comparator = true]}
\toprule
{Values}
& {Values} \\
\midrule
2 .3  & < 2.3e8\\
34.23 & = 34.23 \\
56.78 & >= 56.78e3\\
3,76  & \gg e6 \\
\bottomrule
\end{tabular}
\end{table}

\subsubsection{table-format}
\begin{table}[H]
\caption{Using the \texttt{table-format} option.}
\label{tab:S:format}
\centering
\begin{tabular}{
S
S[table-format = 2.2]
S[table-format = 2.2(1)]
S[table-format = +2.2]
S[table-format = 2.2e1]
}
\toprule
{Values}
& {Values}
& {Values}
& {Values}
& {Values} \\
\midrule
2.3 & 2.3 & 2.3(5) & 2.3 & 2.3e8 \\
34.23 & 34.23 & 34.23(4) & 34.23 & 34.23 \\
56.78 & 56.78 & 56.78(3) & -56.78 & 56.78e3 \\
3,76 & 3,76 & 3.76(2) & +-3.76 & e6 \\
\bottomrule
\end{tabular}
\end{table}


\subsubsection{table-space-text-pre, table-space-text-post}
\begin{table}[H]
\caption{Text before and after numbers.}
\label{tab:S:ends}
\centering
\sisetup{
table-number-alignment = center,
table-figures-integer = 2,
table-figures-decimal = 4,
table-space-text-pre
= now~,
table-space-text-post =
\textsuperscript{\emph{a}}
}
\begin{tabular}{S}
\toprule
{Values} \\
\midrule
2.3456 \\
34.2345 \textsuperscript{\emph{a}}\\
56.7835 \\
now~ 90.473 \\
\bottomrule
\end{tabular}
\end{table}

\subsubsection{table-align-comparator, table-align-exponent, table-align-uncertainty}
\begin{table}[H]
\centering
\caption{The \texttt{table-align-exponent} option}
\label{tab:align:exp}
\sisetup{table-format = 1.3e2, table-number-alignment = center}
\begin{tabular}{SS[table-align-exponent = false]}
\toprule
{Header} & {Header} \\
\midrule
1.2e3 & 1.2e3 \\
1.234e56 & 1.234e56 \\
\bottomrule
\end{tabular}
\end{table}

\begin{table}[H]
\centering
\caption{The \texttt{table-align-uncertainty} option}
\label{tab:align:uncert}
\sisetup{
separate-uncertainty,
table-format = 1.3(1),
}
\begin{tabular}{SS[table-align-uncertainty = false]}
\toprule
{Header} & {Header} \\
\midrule
1.2(1) & 1.2(3) \\
1.234(5) & 1.234(5) \\
\bottomrule
\end{tabular}
\end{table}

\begin{table}[H]
\centering
\caption{The \texttt{table-align-comparator} option}
\label{tab:align:comp}
\sisetup{table-format = >2.2}
\begin{tabular}{SS[table-align-comparator = false]}
\toprule
{Header} & {Header} \\
\midrule
> 1.2 & > 1.2 \\
< 12.34 & < 12.34 \\
\bottomrule
\end{tabular}
\end{table}

\subsubsection{table-omit-exponent}
\begin{table}[H]
\centering
\caption{The \texttt{table-omit-exponent} option}
\label{tab:exp:omit}
\begin{tabular}{
S[table-format = 1.1e1]
S[fixed-exponent = 3, table-format = 2.1, table-omit-exponent]
}
\toprule
{Header} & {Header / \num{e3}} \\
\midrule
1.2e3 & 1.2e3 \\
3e2 & 3e2 \\
1.0e4 & 1.0e4 \\
\bottomrule
\end{tabular}
\end{table}

\subsubsection{table-align-text-pre,table-align-text-post}
% \begin{table}[H]
% \caption{Closing notes up to text.}
% \label{tab:S:notes}
% \newrobustcmd\NoteMark[1]{%
% \textsuperscript{\emph{#1}}%
% }
% \centering
% \sisetup{
% table-number-alignment = center,
% table-figures-integer = 2,
% table-figures-decimal = 4,
% table-space-text-pre
% = \NoteMark{a}
% }
% \begin{tabular}{
% S
% S[table-align-text-pre = false]
% }
% \toprule
% {Values}
% & {Values} \\
% \midrule
% 2.3456 &
% 2.3456 \\
% \NoteMark{a} 4.234 & \NoteMark{a} 4.234 \\
% \NoteMark{b}
% .78
% & \NoteMark{b}
% .78 \\
% \NoteMark{d} 88
% & \NoteMark{d} 88
% \\
% \bottomrule
% \end{tabular}
% \hfil
% \sisetup{table-space-text-post = \NoteMark{a}}
% \begin{tabular}{
% S
% S[table-align-text-post = false]
% }
% \toprule
% {Values}
% & {Values} \\
% \midrule
% 2.3456
% & 2.3456 \\
% 34.234 \NoteMark{a} & 34.234 \NoteMark{a} \\
% 56.78
% \NoteMark{b} & 56.78 \NoteMark{b} \\
% 90.4
% \NoteMark{c} & 90.4
% \NoteMark{c} \\
% 88
% \NoteMark{d} & 88
% \NoteMark{d} \\
% \bottomrule
% \end{tabular}
% \end{table}

\subsubsection{table-auto-round}
\begin{table}[H]
\centering
\caption{The \texttt{table-auto-round} option.}
\label{tab:S:auto}
\sisetup{
table-number-alignment = center,
table-figures-integer = 1,
table-figures-decimal = 3
}
% Notice the overfull hbox which results with
% the first column
\begin{tabular}{
S
S[table-auto-round]
}
\toprule
{Header} & {Header} \\
\midrule
1.2 & 1.2 \\
1.2345 & 1.2345 \\
\bottomrule
\end{tabular}
\end{table}

\subsubsection{parse-numbers}
\begin{table}[H]
\caption{Aligning without parsing.}
\label{tab:S:nonparsed}
\sisetup{
parse-numbers = false,
table-figures-integer = 2,
table-figures-decimal = 3
}
\centering
\begin{tabular}{
S
S[table-number-alignment = center]
S[table-number-alignment = right]
S[table-number-alignment = left]
}
\toprule
{Some values}
& {Some values}
& {Some values}
& {Some values} \\
\midrule
2.35 &
2.35 &
2.35 &
2.35 \\
34.234 &
34.234 &
34.234 & 34.234 \\
56.783 &
56.783 &
56.783 & 56.783 \\
3,762 &
3,762 &
3,762 &
3.762 \\
\sqrt{2} & \sqrt{2} & \sqrt{2} & \sqrt{2} \\
\bottomrule
\end{tabular}
\end{table}

\subsubsection{table-text-alignment}
\begin{table}[H]
\caption{Aligning text in \texttt{S} columns.}
\label{tab:S:text}
\sisetup{
table-number-alignment = center,
table-figures-integer = 4,
table-figures-decimal = 4
}
\centering
\begin{tabular}{
S
S[table-text-alignment = left]
S[table-text-alignment = right]
}
\toprule
{Values}
& {Values}
& {Values} \\
\midrule
992.435 & 992.435 & 992.435 \\
7734.2344 & 7734.2344 & 7734.2344 \\
56.7834 &
56.7834 &
56.7834 \\
3,7462 &
3,7462 &
3,7462 \\
\bottomrule
\end{tabular}
\end{table}

\subsubsection{table-unit-alignment}
\begin{table}[H]
\centering
\caption{Alignment options in \texttt{s} columns.}
\label{tab:s:align}
\begin{tabular}
{
s[table-unit-alignment = right]
s
s[table-unit-alignment = left]
}
\toprule
{Right-aligned} &
{Centred text} &
{Left-aligned} \\
\midrule
\metre\per\second & \metre\per\second & \metre\per\second \\
\kilogram
& \kilogram
& \kilogram
\\
\bottomrule
\end{tabular}
\end{table}

\subsubsection{table-alignment}

\subsubsection{table-column-width}
\begin{table}[H]
\centering
\caption{Fixed-width columns.}
\label{tab:width:fixed}
\begin{tabular}
{
s
s[table-column-width = 2 cm]
S
S[table-column-width = 2 cm]
}
\toprule
{Flexible} &
{Fixed}
&
{Flexible} &
{Fixed}
\\
\midrule
\metre\per\second & \metre\per\second & 1.23 & 1.23 \\
\kilogram\candela & \kilogram\candela & 45.6 & 45.6 \\
\bottomrule
\end{tabular}
\end{table}

% \begin{table}[H]
% \centering
% \caption{Right-aligning under a heading.}
% \label{tab:width:special}
% \settowidth\mylength{Long header}
% \sisetup{
% table-format
% = 4
% ,
% table-number-alignment = center
% ,
% table-column-width
% = \mylength ,
% input-decimal-markers =
% ,
% input-symbols
% = .
% ,
% }
% \begin{tabular}{S}
% \toprule
% {Long header} \\
% \midrule
% 12.33 \\
% 2
% \\
% 1234
% \\
% \bottomrule
% \end{tabular}
% \end{table}

\end{document}













\si{kg.m.s^{-1}}\\
\si{\kilogram\metre\per\second} \\
\si[per-mode=symbol]{\kilogram\metre\per\second} \\
\si[per-mode=symbol]{\kilogram\metre\per\ampere\per\second}

\numlist{10;20;30}\\
\SIlist{0.13;0.67;0.80}{\milli\metre} \\
\numrange{10}{20}\\
\SIrange{0.13}{0.67}{\milli\metre}\\


\numlist{10;30;50;70}

\numrange{10}{30}


\si{kg.m/s^2} \\
\si{g_{polymer}~mol_{cat}.s^{-1}}

\si{\kilo\gram\metre\per\square\second}
\si{\gram\per\cubic\centi\metre}
\si{\square\volt\cubic\lumen\per\farad}
\si{\metre\squared\per\gray\cubic\lux}
\si{\henry\second}

\SI[mode=text]{1.23}{J.mol^{-1}.K^{-1}}\\
\SI{.23e7}{\candela}\\
\SI[per-mode=symbol]{1.99}[\$]{\per\kilogram}\\
\SI[per-mode=fraction]{1,345}{\coulomb\per\mole}\\

\SIlist{10;30;45}{\metre}

\si{\square\becquerel} \\
\si{\joule\squared\per\lumen} \\
\si{\cubic\lux\volt\tesla\cubed}

\si{\henry\tothe{5}} \\
\si{\raiseto{4.5}\radian}

\si{\joule\per\mole\per\kelvin} \\
\si{\joule\per\mole\kelvin} \\
\si{\per\henry\tothe{5}} \\
\si{\per\square\becquerel}

\si{\kilogram\of{metal}} \\
\SI[qualifier-mode = brackets]
{0.1}{\milli\mole\of{cat}\per\kilogram\of{prod}}

\si[per-mode = fraction]
{\cancel\kilogram\metre\per\cancel\kilogram\per\second} \\
\si{\highlight{red}\kilogram\metre\per\second} \\
\si[unit-color = purple]
{\highlight{red}\kilogram\metre\per\second}


\si{\kilo} \\
\si{\micro} \\
\si[prefixes-as-symbols = false]{\kilo}

\si{\kilo\gram\micro} \\
\SI{10}{\micro}


\begin{table}
\caption{Standard behaviour of the \texttt{S} column type.}
\label{tab:S:standard}
\centering
\begin{tabular}{S}
\toprule
{Some Values} \\
\midrule
2.3456 \\
34.2345 \\
-6.7835 \\
90.473 \\
5642.5
\\
1.2e3 \\
e4 \\
\bottomrule
\end{tabular}
\end{table}

\begin{table}
\caption{Detection of surrounding material in an \texttt{S}
column.}
\label{tab:S:extras}
\centering
\begin{tabular}{S[color=orange]}
\toprule
{Some Values} \\
\midrule
12.34 \\
\color{purple} 975,31 \\
44.268 \textsuperscript{\emph{a}} \\
\bottomrule
\end{tabular}
\end{table}

\begin{table}
\caption{Controlling complex alignment with the tablenum macro.}
\label{tab:tablenum}
\centering
\begin{tabular}{lr}
\toprule
Heading & Heading \\
\midrule
Info & More info \\
Info & More info \\
\multicolumn{2}{c}{\tablenum[table-format = 4.4]{12,34}}
\\
\multicolumn{2}{c}{\tablenum[table-format = 4.4]{333.5567}} \\
\multicolumn{2}{c}{\tablenum[table-format = 4.4]{4563.21}}
\bottomrule
\end{tabular}
\hfil
\begin{tabular}{lr}
\toprule
Heading & Heading \\
\midrule
\multirow{2}*{\tablenum{88,999}} & aaa \\
& bbb \\
\multirow{2}*{\tablenum{33,435}} & ccc \\
& ddd \\
\bottomrule
\end{tabular}
\end{table}

\begin{table}
\centering
\caption{Units in tables.}
\label{tab:s:demo}
\begin{tabular}{s}
\toprule
\multicolumn{1}{c}{Unit} \\
\midrule
\metre\squared\per\second \\
\pascal \\
m.s^{-1} \\
\bottomrule
\end{tabular}
\end{table}


\end{document}
